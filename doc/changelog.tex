% CHANGELOG.TEX
%
% The documentation in this file is part of PyXPlot
% <http://www.pyxplot.org.uk>
%
% Copyright (C) 2006-7 Dominic Ford <coders@pyxplot.org.uk>
%               2008   Ross Church
%
% $Id$
%
% PyXPlot is free software; you can redistribute it and/or modify it under the
% terms of the GNU General Public License as published by the Free Software
% Foundation; either version 2 of the License, or (at your option) any later
% version.
%
% You should have received a copy of the GNU General Public License along with
% PyXPlot; if not, write to the Free Software Foundation, Inc., 51 Franklin
% Street, Fifth Floor, Boston, MA  02110-1301, USA

% ----------------------------------------------------------------------------

% LaTeX source for the PyXPlot Users' Guide

\chapter{ChangeLog}
\index{ChangeLog}

\subsection*{2008 Jan ??: PyXPlot 0.7.0}

\subsubsection*{Summary:}

Third PyXPlot beta-release. The code has undergone significant streamlining,
and now runs approximately twice as fast as version 0.6.3 when handling large
datafiles. Memory usage has also been radically reduced. Two new data
processing commands have been introduced. The {\tt tabulate} command can be
used to produce textual datafiles, allowing the user to read data in from
files, apply some analysis, and then write the processed data back to file. The
{\tt histogram} command can be used to estimate the frequency densities of sets
of data points, either by binning them into a bar chart, or by fitting a
functional form to their frequency density.

\subsubsection*{Details -- New and Extended Commands:}

\begin{itemize}
\item {\tt tabulate}
\item {\tt histogram}
\item {\tt set label} and {\tt text} commands extended to allow a colour to be
specified.
\end{itemize}

\subsubsection*{Details -- API changes}

\begin{itemize}
\item {\tt diff\_dx()} and {\tt int\_dx()} functions -- the function to be
differentiated or integrated must now be placed in quote marks.
\end{itemize}

\subsubsection*{Details -- Change of System Requirements:}

\begin{itemize}
\item Requirement of PyX version 0.9 updated to PyX version 0.10. Note that new versions of the PyX graphics library are not generally backwardly compatible.
\end{itemize}

\subsection*{2007 Feb 26: PyXPlot 0.6.3}

\subsubsection*{Summary:}

Second PyXPlot beta-release. The most significant change is the introduction of
a new command-line parser, with greatly improved handling of complex
expressions and much more meaningful syntax error messages. Multi-platform
compatibility has also been massively improved, and dependencies loosened.  A
small number of new commands have been added; most notable among them are the
{\tt jpeg} and {\tt eps} commands, which embed images in multiplots.

\subsubsection*{Details -- New and Extended Commands:}

\begin{itemize}
\item {\tt jpeg}
\item {\tt eps}
\item {\tt set xtics} and {\tt set mxtics}
\item {\tt text} and {\tt set label} commands extended to allow text rotation.
\item {\tt set log} command extended to allow the use of logarithms with bases other than 10.
\item {\tt set preamble}
\item {\tt set term enlarge | noenlarge}
\item {\tt set term pdf}
\item {\tt set term x11\_persist}
\end{itemize}

\subsubsection*{Details -- Eased System Requirements:}

\begin{itemize}
\item Requirement on Python 2.4 minimum eased to version 2.3 minimum.
\item Requirements on scipy and readline eased; PyXPlot will now work in reduced form when they are absent, though they are still strongly recommended.
\item dvips and ghostscript are no longer required.
\end{itemize}

\subsubsection*{Details -- Removed Commands:}

Due to a general refinement of PyXPlot's API, some of the less sensible pieces
of syntax from Version~0.5 are no longer supported. The author apologises for
any inconvenience caused.

\begin{itemize}
\item The {\tt delete\_arrow}, {\tt delete\_text}, {\tt move\_text}, {\tt undelete\_arrow} and {\tt undelete\_text} commands have been removed from the PyXPlot API. The {\tt move}, {\tt delete} and {\tt undelete} commands should now be used to act upon all types of multiplot objects.
\item The {\tt set terminal} command no longer accepts the {\tt enhanced} and {\tt noenhanced} modifiers. The {\tt postscript} and {\tt eps} terminals should be used instead.
\item The {\tt select} modifier, used after the {\tt plot}, {\tt replot}, {\tt fit} and {\tt spline} command can now only be used once; to specify multiple {\tt select} criteria, use the {\tt and} logical operator.
\end{itemize}

\subsection*{2006 Sep 09: PyXPlot 0.5.8}

First beta-release.
