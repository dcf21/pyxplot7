\chapter{ChangeLog}
\index{ChangeLog}

\subsection*{2007 Feb 26: PyXPlot 0.6.3}

\subsubsection*{Summary:}

Second PyXPlot beta-release. The most significant change is the introduction of
a new command-line parser, with greatly improved handling of complex
expressions and much more meaningful syntax error messages. Multi-platform
compatibility has also been massively improved, and dependencies loosened.  A
small number of new commands have been added; most notable among them are the
{\tt jpeg} and {\tt eps} commands, which embed images in multiplots.

\subsubsection*{Details -- New and Extended Commands:}

\begin{itemize}
\item {\tt jpeg}
\item {\tt eps}
\item {\tt set xtics} and {\tt set mxtics}
\item {\tt text} and {\tt set label} commands extended to allow text rotation.
\item {\tt set log} command extended to allow the use of logarithms with bases other than 10.
\item {\tt set preamble}
\item {\tt set term enlarge | noenlarge}
\item {\tt set term pdf}
\item {\tt set term x11\_persist}
\end{itemize}

\subsubsection*{Details -- Eased System Requirements:}

\begin{itemize}
\item Requirement on Python 2.4 minimum eased to version 2.3 minimum.
\item Requirements on scipy and readline eased; PyXPlot will now work in reduced form when they are absent, though they are still strongly recommended.
\item dvips and ghostscript are no longer required.
\end{itemize}

\subsubsection*{Details -- Removed Commands:}

Due to a general refinement of PyXPlot's API, some of the less sensible pieces
of syntax from Version~0.5 are no longer supported. The author apologises for
any inconvenience caused.

\begin{itemize}
\item The {\tt delete\_arrow}, {\tt delete\_text}, {\tt move\_text}, {\tt undelete\_arrow} and {\tt undelete\_text} commands have been removed from the PyXPlot API. The {\tt move}, {\tt delete} and {\tt undelete} commands should now be used to act upon all types of multiplot objects.
\item The {\tt set terminal} command no longer accepts the {\tt enhanced} and {\tt noenhanced} modifiers. The {\tt postscript} and {\tt eps} terminals should be used instead.
\item The {\tt select} modifier, used after the {\tt plot}, {\tt replot} {\tt fit} and {\tt spline} command can now only be used once; to specify multiple {\tt select} criteria, use the {\tt and} logical operator.
\end{itemize}

\subsection*{2006 Sep 09: PyXPlot 0.5.8}

First beta-release.
