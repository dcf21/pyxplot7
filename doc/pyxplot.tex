% PYXPLOT.TEX
%
% The code in this file is part of PyXPlot
% <http://www.pyxplot.org.uk>
%
% Copyright (C) 2006-7 Dominic Ford <coders@pyxplot.org.uk>
%
% $Id: pyxplot.tex,v 1.46 2007/02/21 09:35:12 dcf21 Exp $
%
% PyXPlot is free software; you can redistribute it and/or modify it under the
% terms of the GNU General Public License as published by the Free Software
% Foundation; either version 2 of the License, or (at your option) any later
% version.
%
% You should have received a copy of the GNU General Public License along with
% PyXPlot; if not, write to the Free Software Foundation, Inc., 51 Franklin
% Street, Fifth Floor, Boston, MA  02110-1301, USA

% ----------------------------------------------------------------------------

% LaTeX source for the PyXPlot Users' Guide

\documentclass[a4paper,onecolumn,11pt]{book}
\usepackage[dvips]{graphicx}
\usepackage{amssymb,amsmath,url,lscape,longtable,fancyvrb,makeidx,wasysym}
\makeindex
\def\version{0.6.3}
\def\reldate{February 2007}
\begin{document}

\begin{titlepage}
\normalsize
\vspace*{4.0cm}
\begin{center}
{\Huge \bf PyXPlot Users' Guide}\\
\end{center}
\medskip
\medskip
\begin{center}
{\LARGE \bf A Commandline Plotting Package, \\ \vspace{2mm} with Interface similar to that of Gnuplot, \\ \vspace{2mm} which produces \\ \vspace{2mm} Publication-Quality Output. \\}
\end{center}
\medskip
\medskip
\begin{center}
{\Large Version \version \\}
\end{center}
\bigskip
\vskip 2.5cm
\begin{center}
{Dominic Ford\\
Trinity College\\
Cambridge\\
CB2 1TQ\\
UK\\
Email: \noindent \texttt{coders@pyxplot.org.uk}}
\end{center}

\vfill
\bigskip
\begin{center}
{\Large \reldate \\}
\end{center}
\vfill
\end{titlepage}

\pagenumbering{roman}

\tableofcontents

\chapter{Introduction}
\pagenumbering{arabic}

\label{introduction}

\section{Overview}

PyXPlot is a commandline graphing package, which, for ease of use, has an
interface based heavily upon that of gnuplot -- perhaps UNIX's most widely-used
plotting package. Despite the shared interface, however, PyXPlot is intended to
significantly improve upon the quality of gnuplot's output, producing
publication-quality figures. The commandline interface has also been extended,
providing a wealth of new features, and short-cuts for some operations which
were felt to be excessively cumbersome in the original.

The motivation behind PyXPlot's creation was the apparent lack of a free
plotting package which combined both high-quality output and a simple
interface.  Some -- pgplot for one -- provided very attractive output, but
required a program to be written each time a plot was to be produced -- a
potentially time consuming task. Others, gnuplot being the prime example, were
quick and simple to use, but produced less attractive results.

PyXPlot attempts to fill that gap, offering the best of both worlds. Though the
interface is based upon that of gnuplot, text is now rendered with all of the
beauty and flexibility of the \LaTeX\ typesetting environment; the
``multiplot'' environment is made massively more flexible, making it easy to
produce galleries of plots; and the range of possible output formats is
extended -- to name but a few of the enhancements. A number of examples of the
results of which PyXPlot is capable can be seen in Section~\ref{gallery}.

As well as the ease of use and flexibility of gnuplot's commandline interface
-- it can be used either interactively, read a list of commands from a file, or
receive instructions through a UNIX pipe from another process -- I believe it
to bring another distinct advantage. It insists upon data being written to a
datafile on disk before being plotted. Packages which allow, or more often
require, plotting to be done from within a programming language can encourage
the calculation of data and its plotting to occur in the same program. I
believe this to be a dangerous temptation, as the storage of raw datapoints to
disk can then often be seen as a secondary priority. Months later, when the
need arises to replot the same data in a different form, or to compare it with
newer data, remembering how to use a hurriedly written program can prove
tricky, but remembering how to plot a simple datafile less so.

The similarity of the interface to that of gnuplot is such that simple scripts
written for gnuplot should work with PyXPlot with minimal modification; gnuplot
users should be able to get started very quickly.  However, as PyXPlot remains
work in progress, it supports only a subset of the functionality and
configurability of gnuplot, and some features may be found to be missing.
These will be discussed further in Section~\ref{missing_features}. A
description of those features which have been added to the interface can be
found in Chapter~\ref{new_features}.

A brief overview of gnuplot's interface is provided for novice users in
Chapter~\ref{gnuplot_intro}. However, the attention of past gnuplot users is
drawn to one of the key changes to the interface -- namely that all textual
labels on plots are now printed using the \LaTeX\ typesetting environment. This
does unfortunately introduce some incompatibility with the original, since some
strings which were valid before are no longer valid. For example:

\begin{verbatim}set xlabel 'x^2'\end{verbatim}

\noindent would have been valid in gnuplot, but now needs to be written in
\LaTeX\ mathmode as:\footnote{As in gnuplot, all textual labels in PyXPlot % DCF21 COMPLAINT
should be enclosed in either single or double quotes. If one were to want to
render a string containing apostrophes, it would be necessary to enclose the
string in double quotes, to prevent confusion between the apostrophe in the
\LaTeX, and the closing quote at the end of the line. However, to allow for
those wanting to render \LaTeX\ strings containing both single and double quote
characters -- for example, ``\texttt{J$\backslash$"org's Data}'' -- PyXPlot
recognises the backslash character to be an escape character when followed by
either ' or " in a \LaTeX\ string. This is the \textit{only} case in which
PyXPlot considers $\backslash$ an escape character. Consequently, in the
example above, the ``\texttt{$\backslash$"}'' would need to be double escaped:
``\texttt{J$\backslash\backslash$"org's Data}''.\index{escape
characters}\index{quote characters}\index{special characters}}

\begin{verbatim}set xlabel '$x^2$'\end{verbatim}

\noindent It is the view of the author, however, that the nuisance of this
incompatibility is far outweighed by the power that \LaTeX\ brings. For users
with no prior knowledge of \LaTeX\ the author recommends Tobias Oetiker's
excellent introduction, \textit{The Not So Short Guide to \LaTeX
$2\epsilon$}\footnote{Download from:\\
\url{http://www.ctan.org/tex-archive/info/lshort/english/lshort.pdf}}.

\section{System Requirements}

PyXPlot works on many UNIX-like operating systems. The authors have tested it
under Linux, SunOS and MacOS X, and believe that it should work on other
similar systems. It requires that the following software packages (not
included) be installed:\index{system requirements}

\vspace{0.5cm}
\begin{tabular}{ll}
python  & (Version 2.3 or later) \\
latex   & (Used for all textual labels) \\
convert & (ImageMagick; needed for the gif, png and jpg terminals) \\
\end{tabular}
\vspace{0.5cm}

The following package is not required for installation, but many PyXPlot
features are disabled when it is not present, including the \texttt{fit} and
\texttt{spline} commands and the integration of functions. It is very strongly
recommended:

\vspace{0.5cm}
\begin{tabular}{ll} 
scipy   & (Python Scientific Library) \\
\end{tabular}
\vspace{0.5cm}

The following package is not required for installation, but it is not possible
to use the X11 terminal, i.e. to display plots on screen, without it:

\vspace{0.5cm}
\begin{tabular}{ll}
gv      & (Ghostview; used for the X11 terminal) \\
\end{tabular}
\vspace{0.5cm}

Debian/Ubuntu users can find the above software in the packages \texttt{tetex-extra},
\texttt{gv}, \texttt{imagemagick}, \texttt{python2.3},
\texttt{python2.3-scipy}.\index{Debian Linux}\index{installation!under Debian}

\section{Installation}
\index{installation}

\subsection{Installation as User}

The following steps describe the installation of PyXPlot from a
\texttt{.tar.gz} archive by a user without superuser (i.e. root) access to his
machine. It is assumed that the packages listed above have already been
installed; if they are not, you need to contact your system administrator.

\begin{itemize}
\item Unpack the distributed .tar.gz:

\begin{verbatim}
tar xvfz pyxplot_0.6.2.tar.gz
cd pyxplot
\end{verbatim}

\item Run the installation script:

\begin{verbatim}
./configure
make
\end{verbatim}

\item Finally, start PyXPlot:

\begin{verbatim}
./pyxplot
\end{verbatim}

\end{itemize}

\subsection{System-wide Installation}

Having completed the steps described above, PyXPlot may be installed
system-wide by a superuser with the following additional step:

\begin{verbatim}
make install
\end{verbatim}

By default, the PyXPlot executable installs to \texttt{/usr/local/bin/pyxplot}.
If desired, this installation path may be modified in the file
\texttt{Makefile.skel}, by changing the variable \texttt{USRDIR} in the first
line to an alternative desired installation location.

PyXPlot may now be started by any system user, simply by typing:

\begin{verbatim}
pyxplot
\end{verbatim}

\section{Credits}

Before proceeding any further, the author would like to express his gratitude
to those people who have contributed to PyXPlot -- first and foremost, to
J\"org Lehmann and Andr\'e Wobst, for writing the PyX graphics library for
python, upon which this software is heavily built. Thanks must also go to Ross
Church for his many useful comments and suggestions during its development.

\section{Legal Blurb}

This manual and the software which it describes are both copyright (C)
Dominic Ford 2006-7. They are both distributed under the GNU General Public
License (GPL) Version~2, a copy of which is provided in the \texttt{COPYING}
file in this distribution.\index{General Public License} Alternatively, it may
be downloaded from:\\ \url{http://www.gnu.org/copyleft/gpl.html}.

\chapter{First Steps With PyXPlot}
\label{gnuplot_intro}

In this chapter, I shall provide a brief overview of the basic operation of
PyXPlot, essentially covering those areas of syntax which are borrowed directly
from gnuplot. Users who are already familiar with gnuplot may wish to skim or
skip this chapter, though Section~\ref{missing_features}, detailing which
parts of gnuplot's interface are and are not supported in PyXPlot, may be of
interest. In the following chapter, I shall go on to describe PyXPlot's
extensions of gnuplot's interface.

Describing gnuplot's interface in its entirety is a substantial task, and what
follows is only an overview; novice users can find many excellent tutorials on
the web which will greatly supplement what is provided below.

\section{Getting Started}

The simplest way to start PyXPlot is simply to type ``\texttt{pyxplot}'' at a
shell prompt to start an interactive session. A PyXPlot commandline prompt will
appear, into which commands can be typed. PyXPlot can be exited either by
typing ``\texttt{exit}'', ``\texttt{quit}'', or by pressing CTRL-D.\index{exit
command@\texttt{exit} command}\index{quit command@\texttt{quit} command}

Alternatively, a list of commands to be executed may be stored in a command
script, and executed by passing the filename of the command script to PyXPlot
on the shell commandline, for example:\index{command line syntax}

\begin{verbatim}
pyxplot foo
\end{verbatim}

\noindent In this case, PyXPlot would exit immediately after finishing
executing the commands from the file \texttt{foo}. Several filenames may be
passed on the commandline, to be executed in sequence:

\begin{verbatim}
pyxplot foo1 foo2 foo3
\end{verbatim}

\noindent Wildcards can also be used; the following would execute all command
scripts in the presenting working directory whose filenames end with a
\texttt{.plot} suffix:

\begin{verbatim}
pyxplot *.plot
\end{verbatim}

It is possible to use PyXPlot both interactively, and from command scripts, in
the same session. One way to do this is to pass the magic filename `--' on the
commandline:

\begin{verbatim}
pyxplot foo1 - foo2
\end{verbatim}

\noindent This magic filename represents an interactive session, which
commences after the execution of \texttt{foo1}, and should be terminated in the
usual way after use, with the ``\texttt{exit}'' or ``\texttt{quit}'' commands.
Afterwards, the command script \texttt{foo2} would execute.

From within an interactive session, it is possible to run a command script
using the \texttt{load}\index{load command@\texttt{load} command} command:

\begin{verbatim}
pyxplot> load 'foo'
\end{verbatim}

\noindent This example would have the same effect as typing the contents of the
file \texttt{foo} into the present session.

A related command is ``\texttt{save}''\index{save command@\texttt{save}
command}, which stores a history of the commands executed in the present
interactive session to file.

All command files can include comment lines, which should begin with a hash
character, for example:\index{comment lines}\index{command scripts!comment
lines}

\begin{verbatim}
# This is a comment
\end{verbatim}

Long commands may be split over multiple lines in the script by terminating
each line of it with a backslash character, whereupon the following line will
be appended to the end of it.

\section{First Plots}
\label{first_plots}

The basic workhorse command of PyXPlot is the \texttt{plot} command, which is
used to produce all plots. The following simple example would plot the function
$\sin(x)$:\index{plot command@\texttt{plot} command}

\begin{verbatim}
plot sin(x)
\end{verbatim}

\noindent It is also possible to plot data from files. The following would plot
data from a file `\texttt{datafile}', taking the $x$-coordinate of each point
from the first column of the datafile, and the $y$-coordinate from the second.
The datafile is assumed to be in plain text format, with columns separated by
whitespace and/or commas\footnote{If the filename of a datafile ends with a
\texttt{.gz} suffix, it is assuming to be gzipped plaintext, and is decoded
accordingly.}:

\begin{verbatim}
plot 'datafile'
\end{verbatim}

Several items can be plotted on the same graph by separating them by commas:

\begin{verbatim}
plot 'datafile', sin(x), cos(x)
\end{verbatim}

\noindent It is possible to define one's own variables and functions, and then
plot them:

\begin{verbatim}
a = 2
b = 1
c = 1.5
f(x) = a*(x**2) + b*x + c
plot f(x)
\end{verbatim}

\noindent To unset a variable or function once it has been set, the following
syntax should be
used:\index{variables!unsetting}\index{functions!unsetting}\index{unsetting
variables}

\begin{verbatim}
a =
f(x) =
\end{verbatim}

Labels can be applied to the two axes of the plot, and a title put at the top:

\begin{verbatim}
set xlabel 'This is the X axis'
set ylabel 'This is the Y axis'
set title 'A Plot of sin(x)'
plot sin(x)
\end{verbatim}

\noindent All such text labels are displayed using \LaTeX, and so any \LaTeX\ 
commands can be used, for example to put equations on axes:

\begin{verbatim}
set xlabel '$\frac{x^2}{c^2}$'
\end{verbatim}


\noindent As a caveat, however, this does mean that care needs to be taken to
escape any of \LaTeX's reserved characters -- i.e.:
$\backslash$~\&~\%~\#~\{~\}~\$~\_~\^{} or $\sim$.

Having set labels and titles, they may be removed thus:

\begin{verbatim}
set xlabel ''
set ylabel ''
set title ''
\end{verbatim}

\noindent These are two other ways of removing the title from a plot:

\begin{verbatim}
set notitle
unset title
\end{verbatim}

The \texttt{unset} command\index{unset command@\texttt{unset} command} may be
followed by essentially any word that can follow the \texttt{set} command, such
as \texttt{xlabel} or \texttt{title}, to return that setting to its default
configuration. The \texttt{reset} command\index{reset command@\texttt{reset}
command} restores all configurable parameters to their default states.

\section{Operators and Functions}

As has already been seen above, some functions such as $\sin(x)$ are
pre-defined within PyXPlot. A list of frequently-used functions which are
predefined in PyXPlot is given in Table~\ref{functions_table}\footnote{Users
with some programming experience may be interested to know that all of the
functions in the core and math modules of python are recognised.}. A list of
operators recognised by PyXPlot is given in Table~\ref{operators_table}.

\begin{table}
\begin{longtable}{|lp{8cm}|}
\hline
acos($x$)&
Return the arc cosine (measured in radians) of $x$.\\
asin($x$)&
Return the arc sine (measured in radians) of $x$.\\
atan($x$)&
Return the arc tangent (measured in radians) of $x$.\\
atan2($y,x$)&
Return the arc tangent (measured in radians) of $y/x$. Unlike $\mathrm{atan}(y/x)$, the signs of both $x$ and $y$ are considered.\\
ceil($x$)&
Return the ceiling of $x$ as a float. This is the smallest integral value $\geq x$.\\
cos($x$)&
Return the cosine of $x$ (measured in radians).\\
cosh($x$)&
Return the hyperbolic cosine of $x$.\\
degrees($x$)&
Convert angle $x$ from radians to degrees.\\
exp($x$)&
Return $e$ raised to the power of $x$.\\
fabs($x$)&
Return the absolute value of the float $x$.\\
floor($x$)&
Return the floor of $x$ as a float. This is the largest integral value $\leq x$.\\
fmod($x,y$)&
Return fmod(x, y), according to platform C.  x \% y may differ.\\
frexp($x$)&
Return the mantissa and exponent of $x$, as pair $(m,e)$. $m$ is a float and $e$ is an int, such that $x = m \times 2^e$. If $x$ is 0, $m$ and $e$ are both 0.  Else $0.5 \leq \mathrm{abs}(m) < 1.0$.\\
hypot($x,y$)&
Return the Euclidean distance, $\sqrt{x^2 + y^2}$.\\
ldexp($x, i$)&
Return $x \times 2^i$. \\
log($x[,base]$)&
Return the logarithm of $x$ to the given base. If the base not specified, returns the natural logarithm (base $e$) of $x$.\\
log10($x$)&
Return the base 10 logarithm of $x$.\\
modf($x$)&
Return the fractional and integer parts of $x$.  Both results carry the sign of $x$.  The integer part is returned as a real.\\
pow($x,y$)&
Return $x^y$.\\
radians($x$)&
Converts angle $x$ from degrees to radians.\\
sin($x$)&
Return the sine of $x$ (measured in radians).\\
sinh($x$)&
Return the hyperbolic sine of $x$.\\
sqrt($x$)&
Return the square root of $x$.\\
tan($x$)&
Return the tangent of $x$ (measured in radians).\\
tanh($x$)&
Return the hyperbolic tangent of $x$.\\
\hline
\end{longtable}
\caption{A list of mathematical functions which are pre-defined within PyXPlot.}
\label{functions_table}
\end{table}

\begin{table}
\begin{longtable}{|lp{8cm}|}
\hline
\texttt{+} & Algebraic sum \\
\texttt{-} & Algebraic subtraction \\
\texttt{*} & Algebraic multiplication \\
\texttt{**} & Algebraic exponentiation \\
\texttt{/} & Algebraic division \\
\texttt{\%} & Modulo operator \\
\texttt{<<} & Left binary shift \\
\texttt{>>} & Right binary shift \\
\texttt{\&} & Binary and \\
\texttt{|} & Binary or \\
\texttt{\^{}} & Logical exclusive or \\
\texttt{<} & Magnitude comparison \\
\texttt{>} & Magnitude comparison \\
\texttt{<=} & Magnitude comparison \\
\texttt{>=} & Magnitude comparison \\
\texttt{==} & Equality comparison \\
\texttt{!=} & Equality comparison \\
\texttt{<>} & Alias for \texttt{!=} \\
\texttt{and} & Logical and \\
\texttt{or} & Logical or \\
\hline
\end{longtable}
\caption{A list of mathematical operators which PyXPlot recognises.}
\label{operators_table}
\end{table}

\section{Plotting Datafiles}
\label{plot_datafiles}

In the simple example of the previous section, we plotted the first column of a
datafile against the second. It is also possible to plot any arbitrary column
of a datafile against any other; the syntax for doing this is:

\begin{verbatim}
plot 'datafile' using 3:5
\end{verbatim}

\noindent This example would plot the fifth column of the file
\texttt{datafile} against the third. As mentioned above, columns in datafiles
can be separated using whitespace and/or commas, which means that PyXPlot is
compatible both with the format used by gnuplot, and also with
comma-separated-value (CSV)\index{csv files} files which many
spreadsheets\index{spreadsheets, importing data from} produce. Algebraic
expressions may also be used in place of column numbers, for example:

\begin{verbatim}
plot 'datafile' using (3+$1+$2):(2+$3)
\end{verbatim}

\noindent In algebraic expressions, column numbers should be prefixed by dollar
signs, to distinguish them from numerical constants. The example above would
plot the sum of the values in the first two columns of the datafile, plus
three, on the horizontal axis, against two plus the value in the third column
on the vertical axis. A more advanced example might be:

\begin{verbatim}
plot 'datafile' using 3.0:$($2)
\end{verbatim}

\noindent This would place all of the datapoints on the line $x=3$, drawing
their vertical positions from the value of some column $n$ in the datafile,
where the value of $n$ is itself read from the second column of the datafile.

Later, in Section~\ref{horizontal_datafiles}, I shall discuss how to plot rows
of datafiles against one another, in horizontally arranged datafiles.

It is also possible to plot data from only a range of lines within a datafile.
When PyXPlot reads a datafile, it looks for any blank lines in the file. It
divides the datafile up into ``data blocks'', each being separated by single
blank lines. The first datablock is numbered 0, the next 1, and so on.
\index{datafile format}

When two or more blank lines are found together, the datafile is divided up
into ``index blocks''. Each index block may be made up of a series of data
blocks. To clarify this, a labelled example datafile is shown in
figure~\ref{sample_datafile}.

\begin{figure}
\begin{tabular}{p{2.2cm}l}
\hline
\texttt{0.0 \ 0.0} & Start of index 0, data block 0. \\
\texttt{1.0 \ 1.0} & \\
\texttt{2.0 \ 2.0} & \\
\texttt{3.0 \ 3.0} & \\
                   & A single blank line marks the start of a new data block. \\
\texttt{0.0 \ 5.0} & Start of index 0, data block 1. \\
\texttt{1.0 \ 4.0} & \\
\texttt{2.0 \ 2.0} & \\
                   & A double blank line marks the start of a new index. \\
                   & ... \\
\texttt{0.0 \ 1.0} & Start of index 1, data block 0. \\
\texttt{1.0 \ 1.0} & \\
                   & A single blank line marks the start of a new data block. \\
\texttt{0.0 \ 5.0} & Start of index 1, data block 1. \\
                   & $<$etc$>$ \\
\hline
\end{tabular}
\caption{An Example PyXPlot Datafile -- the datafile is shown in the two left-hand columns, and commands are shown to the right.}
\label{sample_datafile}
\end{figure}

By default, when a datafile is plotted, all data blocks in all index blocks are
plotted. To plot only the data from one index block, the following syntax may
be used:

\begin{verbatim}
plot 'datafile' index 1
\end{verbatim}

\noindent To achieve the default behaviour of plotting all index blocks, the
\texttt{index} modifier should be followed by a negative number.\index{index
modifier@\texttt{index} modifier}

It is also possible to specify which lines and/or data blocks to plot from
within each index. For this purpose the \texttt{every} modifier is used, which
takes six values, separated by colons:\index{every modifier@\texttt{every}
modifier}

\begin{verbatim}
plot 'datafile' every a:b:c:d:e:f
\end{verbatim}

The values have the following meanings:

\begin{longtable}{p{1.2cm}p{10.5cm}}
$a$ & Plot data only from every $a\,$th line in datafile. \\
$b$ & Plot only data from every $b\,$th block within each index block. \\
$c$ & Plot only from line $c$ onwards within each block. \\
$d$ & Plot only data from block $d$ onwards within each index block. \\
$e$ & Plot only up to the $e\,$th line within each block. \\
$f$ & Plot only up to the $f\,$th block within each index block. \\
\end{longtable}

\noindent Any or all of these values can be omitted, and so the following would
both be valid statements:

\begin{verbatim}
plot 'datafile' index 1 every 2:3
plot 'datafile' index 1 every :::3
\end{verbatim}

\noindent The first would plot only every other data point from every third
data block; the second from the third line onwards within each data block.

A final modifier for selecting which parts of a datafile are plotted is
\texttt{select}, which plots only those data points which satisfy some given
criterion. This is an extension of gnuplot's original syntax, and is described
in Section~\ref{select_modifier}.

\newpage % Why is the necessary???

\section{Directing Where Output Goes}
\label{directing_output}

By default, when PyXPlot is used interactively, all plots are displayed on the
screen. It is also possible to produce postscript output, to be read into other
programs or embedded into \LaTeX\ documents, as well as a variety of other
graphic formats. The \texttt{set terminal} command\index{set terminal
command@\texttt{set terminal} command}\footnote{gnuplot users should note that
the syntax of the \texttt{set terminal} command in PyXPlot is rather different;
see Section~\ref{set_terminal2}.} is used to specify the output format that is
required, and the \texttt{set output} command\index{set output
command@\texttt{set output} command} the file to which output should be
directed. For example,

\begin{verbatim}
set terminal postscript
set output 'myplot.eps'
plot sin(x)
\end{verbatim}

\noindent would produce a postscript plot of $\sin(x)$ to the file
\texttt{myplot.eps}.

The \texttt{set terminal} command can also be used to configure further aspects
of the output file format. For example, the following would produce
black-and-white and colour output respectively:

\begin{verbatim}
set terminal monochrome
set terminal colour
\end{verbatim}

\noindent The former is useful for preparing plots for black-and-white
publications, the latter for preparing plots for colourful presentations.

Both encapsulated and non-encapsulated postscript can be produced. The former
is recommended for producing figures to embed into documents, the latter for
plots which are to be printed without further processing. The
\texttt{postscript} terminal produces the latter; the \texttt{eps} terminal
should be used to produce the former.  Similarly the \texttt{pdf} terminal
produces pdf files:

\begin{verbatim}
set terminal postscript
set terminal eps
set terminal pdf
\end{verbatim}

It is also possible to produce plots in the gif, png and jpeg graphic formats,
as follows:

\begin{verbatim}
set terminal gif
set terminal png
set terminal jpg
\end{verbatim}

More than one of the above keywords can be combined on a single line, for
example:

\begin{verbatim}
set terminal postscript colour
set terminal gif monochrome
\end{verbatim}

To return to the default state of displaying plots on screen, the \texttt{x11}
terminal should be selected:

\begin{verbatim}
set terminal x11
\end{verbatim}

For more details of the \texttt{set terminal} command, including how to produce
transparent gifs and pngs, see Section~\ref{set_terminal2}.

We finally note that, after changing terminals, the \texttt{replot} command is
especially useful; it repeats the last \texttt{plot} command.\index{replot
command@\texttt{replot} command}. If any plot items are placed after it, they
are added to the last plot.

\section{Data Styles}

By default, data from files is plotted with points, and functions are plotted
with lines. However, either kinds of data can be plotted in a variety of ways.
To plot a function with points, for example, the following syntax is
used\footnote{Note that when a plot command contains both
\texttt{using}/\texttt{every} modifiers, and the \texttt{with} modifier, the
latter must come last.}\index{with modifier@\texttt{with} modifier}:

\begin{verbatim}
plot sin(x) with points
\end{verbatim}

\noindent The number of points displayed (i.e. the number of samples of the
function) can be set as follows\index{set samples command@\texttt{set samples}
command}:

\begin{verbatim}
set samples 100
\end{verbatim}

Likewise, datafiles can be plotted with lines:

\begin{verbatim}
plot 'datafile' with lines
\end{verbatim}

A variety of other styles are available. \texttt{linespoints} combines both the
\texttt{points} and \texttt{lines} styles, drawing lines through points.
Errorbars can also be drawn, as follows:

\begin{verbatim}
plot 'datafile' with yerrorbars
\end{verbatim}

\noindent In this case, three columns of data need to be specified: the $x$-
and $y$-coordinates of each datapoint, plus the size of the vertical errorbar
on that datapoint. By default, the first three columns of the datafile are
used, but once again (see Section~\ref{plot_datafiles}), the \texttt{using}
modifier can be used:

\begin{verbatim}
plot 'datafile' using 2:3:7 with yerrorbars
\end{verbatim}

More details of the errorbars plot style can be found in
Section~\ref{errorbars}. Other plots styles supported by PyXPlot are listed in
Section~\ref{missing_features}, and their details can be found in many gnuplot
tutorials. Bar charts will be discussed further in Section~\ref{barcharts}.

The modifiers ``\texttt{pointtype}''\index{pointtype
modifier@\texttt{pointtype} modifier} and ``\texttt{linetype}''\index{linetype
modifier@\texttt{linetype} modifier}\label{pointtype_modifier}, which can be
abbreviated to ``\texttt{pt}'' and ``\texttt{lt}'' respectively, can also be
placed after the \texttt{with} modifier. Each should be followed by an integer.
The former specifies what shape of points should be used to plot the dataset,
and the latter, whether a line should be continuous, dotted, dash-dotted, etc.
Different integers correspond to different styles.

The default plotting style refered to above can also be changed.  For example:

\begin{verbatim}
set style data lines
\end{verbatim}

\noindent The default style for plotting data from files is then changed to
lines.  Similarly the ``{\tt set style function}''\index{set style
command@\texttt{set style} command} command changes the default style for
plotting functions.

\section{Setting Axis Ranges}

In Section~\ref{first_plots}, the \texttt{set xlabel} configuration command was
previously introduced for placing text labels on axes. In this section, the
configuration of axes is extended to setting their ranges.

By default, PyXPlot automatically scales axes to some sensible range which
contains all of the plotted data. However, it is also possible for the user to
override this and set his own range.\index{axes!setting ranges} This can be
done directly from the plot command, for example:

\begin{verbatim}
plot [-1:1][-2:2] sin(x)
\end{verbatim}
\label{plot_ranges}

\noindent The ranges are specified immediately after the \texttt{plot}
statement, with the syntax \texttt{[minimum:maximum]}.\footnote{An alternative
valid syntax is to replace the colon with the word `to': \texttt{[minimum to
maximum]}.} The first specified range applies to the $x$-axis, and the second
to the $y$-axis.\footnote{As will be discussed in
Section~\ref{ranges_multiaxes}, if further ranges are specified, they apply to
the $x2$-axis, then the $y2$-axis, and so forth.} Any of the values can be
omitted, for example:

\begin{verbatim}
plot [:][-2:2] sin(x)
\end{verbatim}

\noindent would only set a range on the $y$-axis.

Alternatively, ranges can be set before the \texttt{plot} statement, using the
\texttt{set xrange}\index{set xrange command@\texttt{set xrange} command}
statement, for example:

\begin{verbatim}
set xrange [-2:2]
set y2range [a:b]
\end{verbatim}

Having done so, a range may subsequently be turned off, and an axis returned to
its default autoscaling behaviour, using the \texttt{set autoscale}\index{set
autoscale command@\texttt{set autoscale} command} command, which takes a list
of axes to which it is to apply. If no list is supplied, then the command is
applied to all axes.

\begin{verbatim}
set autoscale x y
set autoscale
\end{verbatim}

Axes can be set to have logarithmic scales using the \texttt{set
logscale}\index{set logscale command@\texttt{set logscale} command} command,
which also takes a list of axes to which it should apply. Its converse is
\texttt{set nologscale}\index{set nologscale command@\texttt{set nologscale}
command}:

\begin{verbatim}
set logscale
set nologscale y x x2
\end{verbatim}

Further discussion of the configuration of axes can be found in
Section~\ref{axis_extensions}.

\section{Function Fitting}
\index{fit command@\texttt{fit} command}
\label{fit_command}

It is possible to fit functional forms to data points in datafiles using the
\texttt{fit} command. A simple example might be:\footnote{In gnuplot, this example would have been written \texttt{fit f(x) ...}, rather than \texttt{fit f() ...}. This syntax is supported in PyXPlot, but deprecated.}

\begin{verbatim}
f(x) = a*x+b
fit f() 'datafile' index 1 using 2:3 via a,b
\end{verbatim}

The coefficients to be varied are listed after the keyword ``\texttt{via}'';
the keywords \texttt{index}, \texttt{every} and \texttt{using} have the same
meanings as in the plot command.\footnote{The \texttt{select} keyword, to be
introduced in Section~\ref{select_modifier} can also be used.}

This is useful for producing best-fit lines\index{best fit
lines}\footnote{Another way of producing best-fit lines is a to use a cubic
spline; more details in given in Section~\ref{spline_command}}, and also has
applications for estimating the gradients of datasets.  The syntax is
essentially identical to that used by gnuplot, though a few points are worth
noting:

\begin{itemize}
\item When fitting a function of $n$ variables, at least $n+1$ columns (or
rows -- see Section~\ref{horizontal_datafiles}) must be specified after the
\texttt{using} modifier. By default, the first $n+1$ columns are used. These
correspond to the values of each of the $n$ inputs to the function, plus
finally the value which the output from the function is aiming to match.
\item If an additional column is specified, then this is taken to contain the
standard error in the value that the output from the function is aiming to
match, and can be used to weight the datapoints which are input into the
\texttt{fit} command. 
\item By default, the starting values for each of the fitting parameters is
$1.0$. However, if the variables to be used in the fitting process are already
set before the \texttt{fit} command is called, these initial values are used
instead. For example, the following would use the initial values
$\{a=100,b=50\}$:
\begin{verbatim}
f(x) = a*x+b
a = 100
b = 50
fit f() 'datafile' index 1 using 2:3 via a,b
\end{verbatim}

\item As with all numerical fitting procedures, the \texttt{fit} command comes
with caveats. It uses a generic fitting algorithm, and may not work well with
poorly behaved or ill-constrained problems. It works best when all of the
values it is attempting to fit are of order unity. For example, in a problem
where $a$ was of order $10^{10}$, the following might fail:
\begin{verbatim}
f(x) = a*x
fit f() 'datafile' via a
\end{verbatim}
However, better results might be achieved if $a$ were artificially made of
order unity, as in the following script:
\begin{verbatim}
f(x) = 1e10*a*x
fit f() 'datafile' via a
\end{verbatim}

\item A series of ranges may be specified after the \texttt{fit} command, using
the same syntax as in the \texttt{plot} command, as described in
Section~\ref{plot_ranges}. If ranges are specified then only datapoints falling
within these ranges are used in the fitting process; the ranges refer to each
of the $n$ variables of the fitted function in order.

\item For those interested in the mathematical details, the workings of the
\texttt{fit} command is discussed in more detail in Chapter~\ref{fit_math}.

\end{itemize}

At the end of the fitting process, the best-fitting values of each parameter
are output to the terminal, along with an estimate of the uncertainty in each.
Additionally, the Hessian, covariance and correlation matrices are output in
both human-readable and machine-readable formats, allowing a more complete
assessment of the probability distribution of the parameters.

\section{Interactive Help}

In addition to this Users' Guide, PyXPlot also has a \texttt{help} command,
which provides a hierarchical source of information. Typing `help' alone gives a
brief introduction to the help system, as well as a list of topics on which
help is available. To display help on any given topic, type `help' followed by
the name of the topic. For example:

\begin{verbatim}
help commands
\end{verbatim}

\noindent provides information on PyXPlot's commands. Some topics have
subtopics, which are listed at the end of each page. To view them, add further
words to the end of your help request -- an example might be:

\begin{verbatim}
help commands help
\end{verbatim}

\noindent which would display help on the \texttt{help} command itself.

\section{Differences Between PyXPlot and Gnuplot}
\label{missing_features}

The commands supported by PyXPlot are only a subset of those available in
gnuplot, although most of its functionality is present. Features which are
supported by this version include:

\begin{itemize}
\item Allocation of user-defined variables and functions.
\item The \texttt{print}, \texttt{help}, \texttt{exit} and \texttt{quit} commands.
\item The \texttt{reset} and \texttt{clear} commands.
\item The \texttt{!} command, to execute the remainder of the line as a shell command, e.g. \texttt{!ls}.
\item The \texttt{cd} and \texttt{pwd} commands, to change and display the current working directory.
\item The use of ` ` back-quotes to substitute the output of a shell command.\footnote{It should be noted that back-quotes can only be used outside quotes. For example, \texttt{set xlabel '`ls`'} will not work. The best way to do this would be: \texttt{set xlabel `echo "'" ; ls ; echo "'"`}.}
\item Set plot titles, axis labels, axis ranges, pointsize, linestyles, etc.
\item Fitting of functions to data via the \texttt{fit} command.
\item Basic 2d plotting and replotting of functions and datafiles, with the following styles: \texttt{lines}, \texttt{points}, \texttt{linespoints}, \texttt{dots}, \texttt{boxes}, \texttt{steps}, \texttt{fsteps}, \texttt{histeps}, \texttt{impulses}, \texttt{csplines}, \texttt{acsplines} and errorbars of all flavours (see Section \ref{errorbars} for details of changes to errorbars).
\item Automatic and manual selection of linestyles, linetypes, linewidths, pointtypes and pointsizes.
\item Use of dual axes. Note: Operation here differs slightly from original gnuplot; dual axes are displayed whenever they are defined, there is no need to \texttt{set xtics nomirror}. See the details in the following section.
\item Placing arrows and textual labels on plots.
\item Putting grids on plots (colour can be set, but not linestyle).
\item Setting plot aspect ratios with \texttt{set size ratio} or \texttt{set size square}.
\item Multiplot (which is very significantly improved over gnuplot; see Section~\ref{multiplot})
\item Setting major/minor tics with the \texttt{set xtics} and \texttt{set mxtics} commands.
\end{itemize}

Gnuplot features which PyXPlot does not presently support include:

\begin{itemize}
\item Parametric function plotting.
\item Three-dimensional plotting (i.e. the \texttt{splot} command).
\end{itemize}

\chapter{Extensions of Gnuplot's Interface}
\label{new_features}

A large number of new functions are available in PyXPlot which were not
originally present in gnuplot. This chapter describes these extensions.  From
here onwards I shall presume that the user is familiar with the basic operation
of gnuplot, and shall concentrate on the differences between PyXPlot's
interface and that of gnuplot. In addition to having read the previous chapter,
novice users may also find it of use to consult one of the many gnuplot
tutorials which are to be found on the web before proceeding.

\section{The Commandline Environment}

PyXPlot uses the Gnu Readline commandline environment, which means that the up
and down arrow keys can be used to repeat previously executed commands. Each
user's command history is stored in his homespace in a history file called
`\texttt{.pyxplot\_history}', allowing PyXPlot to remember command histories
between sessions. Additionally, a \texttt{save} command is provided, allowing
the user to save his command history from the present session to a text file;
this has the following syntax:

\begin{verbatim}
save 'output_filename'
\end{verbatim}

From the shell commandline, the PyXPlot accepts the following switches which
modify its behaviour:\index{command line syntax}

\begin{longtable}{p{3.5cm}p{8.5cm}}
\texttt{-h --help} & Display a short help message listing the available commandline switches.\\
\texttt{-v --version} & Display the current version number of PyXPlot.\\
\texttt{-q --quiet} & Turn off the display of the welcome message on startup. \\
\texttt{-V --verbose} & Display the welcome message on startup, as happens by default. \\
\texttt{-c --colour} & Use colour highlighting\footnote{This will only function on terminals which support colour output.} to display output in green, warning messages in amber, and error messages in red.\footnote{The author apologies to those members of the population who are red/green colourblind, but draws their attention to the following sentence.} These colours can be changed in the \texttt{terminal} section of the configuration file; see Section~\ref{configuration} for more details. \\
\texttt{-m --monochrome} & Do not use colour highlighting, as happens by default. \\
\end{longtable}

\section{Formatting and Terminals}
\label{set_terminal2}

In this section I shall outline the new and modified commands for controlling
the graphic output format of PyXPlot.

The widths of plots may be set be means of two commands -- \texttt{set
size}\index{set size command@\texttt{set size} command} and \texttt{set
width}\index{set width command@\texttt{set width} command}. Both are
equivalent, and should be followed by the desired width measured in
centimetres, for example:

\begin{verbatim}
set width 20
\end{verbatim}

The \texttt{set size} command can also be used to set the aspect ratio of plots
by following it with the keyword \texttt{ratio}\index{set size ratio
command@\texttt{set size ratio} command}. The number which follows should be
the desired ratio of height to width. The following, for example would produce
plots three times as high as they are wide:

\begin{verbatim}
set size ratio 3.0
\end{verbatim}

The command \texttt{set size noratio} returns to PyXPlot's default aspect ratio
of the golden ratio, i.e. $\left((1+\sqrt{5})/2\right)^{-1}$, which matches
that of a sheet of A4 paper\footnote{Of less practical significance, it has
been in use since the time of the Pythagoreans, and is seen repeatedly in the
architecture of the Parthenon.}.  The special command \texttt{set size
square}\index{set size square command@\texttt{set size square} command} sets
the aspect ratio to unity.

If the \texttt{enlarge} modifier is used with the {\tt set terminal} command
then the whole plot is enlarged or, in the case of large plots, shrunk to the
current paper size (minus a small margin).  The aspect ratio of the plot is
preserved.  This is perhaps most useful when preparing a plot to send to a
printer with the postscript terminal.

In Section~\ref{directing_output} I described how the \texttt{set terminal}
command\index{set terminal command@\texttt{set terminal} command} can be used
to produce plots in various graphic formats. In addition, I here describe how
the way in which plots are displayed on the screen can be changed. The default
terminal, \texttt{X11}, is used to send output to screen.

By default, each time a new plot is generated, if the previous plot is still
open on the display, the X11 terminal will replace it with the new one, thus
keeping only one plot window open at a time. This has the advantage that the
desktop does not become flooded with plot windows.

If this behaviour is not desired, old plots can be kept visible when plotting
further graphs by using the the \texttt{X11\_multiwindow} terminal: \index{X11
terminal}\index{multiple windows}

\begin{verbatim} 
set terminal X11_singlewindow
plot sin(x)
plot cos(x)  <-- first plot window disappears
\end{verbatim}

\noindent c.f.:

\begin{verbatim} 
set terminal X11_multiwindow
plot sin(x)
plot cos(x)  <-- first plot window remains
\end{verbatim}

As an additional option, the \texttt{X11\_persist} terminal keeps plot windows
open after PyXPlot exits; the above two terminals close all plot windows upon
exit.

\index{set terminal command@\texttt{set terminal} command} As there are many
changes to the options accepted by the \texttt{set terminal} command in
comparison to those understood by gnuplot, the syntax of PyXPlot's command is
given below, followed by a list of the recognised settings:

\begin{verbatim} 
set terminal { X11_singlewindow | X11_multiwindow | X11_persist |
               postscript | eps | gif | png | jpg }
             { colour | color | monochrome }
             { portrait | landscape }
             { invert | noinvert }
             { transparent | solid }
             { enlarge | noenlarge }
\end{verbatim}

\begin{longtable}{p{3cm}p{9cm}}
\texttt{x11\_singlewindow} & Displays plots on the screen (in X11 windows, using ghostview). Each time a new plot is generated, it replaces the old one, preventing the desktop from becoming flooded with old plots.\footnote{The author is aware of a bug, that this terminal can occasionally go blank when a new plot is generated. This is a known bug in ghostview, and can be worked around by selecting File $\to$ Reload within the ghostview window.} \textbf{[default when running interactively; see below]}\\
\texttt{x11\_multiwindow} & As above, but each new plot appears in a new window, and the old plots remain visible. As many plots as may be desired can be left on the desktop simultaneously.\\
\texttt{x11\_persist} & As above, but plot windows remain open after PyXPlot closes.\\
\texttt{postscript} & Sends output to a postscript file. The filename for this file should be set using \texttt{set output}. \textbf{[default when running non-interactively; see below]}\index{postscript output}\\
\texttt{eps} & As above, but produces encapsulated postscript.\index{encapsulated postscript}\index{postscript!encapsulated}\\
\texttt{colour} & Allows datasets to be plotted in colour. Automatically they will be displayed in a series of different colours, or alternatively colours may be specified using the \texttt{with colour} plot modifier (see below). \textbf{[default]}\index{colour output}\\
\texttt{color} & Equivalent to the above; provided for users of nationalities which can't spell. \smiley \\
\texttt{monochrome} & Opposite to the above; all datasets will be plotted in black.\index{monochrome output}\\
\texttt{portrait} & Sets plots to be displayed in upright (normal) orientation. \textbf{[default]}\index{portrait orientation}\\
\texttt{landscape} & Opposite of the above; produces side-ways plots. Not very useful when displayed on the screen, but you fit more on a sheet of paper that way around.\index{landscape orientation}\\
\texttt{gif} & Sends output to a gif image file; as above, the filename should be set using \texttt{set output}.\index{gif output}\\
\texttt{png} & As above, but produces a png image.\index{png output}\\
\texttt{jpg} & As above, but produces a jpeg image.\index{jpeg output}\\
\texttt{invert} & Modifier for the gif, png and jpg terminals; produces output with inverted colours.\footnote{This terminal setting is useful for producing plots to embed in talk slideshows, which often contain bright text on a dark background. It only works when producing bitmapped output, though a similar effect can be achieved in postscript using the \texttt{set textcolour} and \texttt{set axescolour} commands (see Section~\ref{set_colours}).}\index{colours!inverting}\\
\texttt{noinvert} & Modifier for the gif, png and jpg terminals; opposite to the above. \textbf{[default]}\\
\texttt{transparent} & Modifier for the gif and png terminals; produces output with a transparent background.\index{transparent terminal}\index{gif output!transparency}\index{png output!transparency}\\
\texttt{solid} & Modifier for the gif and png terminals; opposite to the above. \textbf{[default]}\\
\texttt{enlarge} & Enlarge or shrink contents to fit the current paper
size.\index{enlarging output}\\
\texttt{noenlarge} & Do not enlarge output; opposite to the above \textbf{[default]}\\
\end{longtable}
\label{terminals}

The default terminal is normally \texttt{x11\_singlewindow}, matching
approximately the behaviour of gnuplot. However, there is an exception to this.
When PyXPlot is used non-interactively -- i.e. one or more command scripts is
specified on the commandline, and PyXPlot exits as soon as it finishes
executing them -- the \texttt{x11\_singlewindow} is not a very sensible
terminal to use. Any plot window would close as soon as PyXPlot exited. The
default terminal in this case changes to \texttt{postscript}.

One exception to this is when the special `--' filename is specified in a list
of command scripts on the commandline, to produce an interactive terminal
between running a series of scripts. In this case, PyXPlot detects that the
session will be interactive, and defaults to the usual
\texttt{x11\_singlewindow} terminal.

An additional exception is on machines where the \texttt{DISPLAY} environment
variable\index{display environment variable@\texttt{DISPLAY} environment
variable} is not set. In this case, PyXPlot detects that it has access to no
X-terminal on which to display plots, and defaults to the \texttt{postscript}
terminal.

The \texttt{gif}, \texttt{png} and \texttt{jpg} terminals result in some loss
of quality, since the plot has to be sampled into a bitmapped graphic format.
By default, this sampling is performed at 300 dpi, though it may be changed
using the command \texttt{set dpi <value>}. Alternatively, it may be changed
using the \texttt{DPI} option in the \texttt{settings} section of a
configuration file (see Section~\ref{configuration}).\index{set dpi
command@\texttt{set dpi} command}\index{resolution of bitmap
output}\index{image resolution}

\subsection{Paper Sizes}

By default, when the postscript terminal produces printable, i.e. not
encapsulated, output, the papersize for this output is read from your system
locale settings. It may be changed, however, with the \texttt{set
papersize}\index{set papersize command@\texttt{set papersize} command} command,
which may be followed either by the name of a recognised papersize, or by the
dimensions of a user-defined size, specified as a
\texttt{height},\texttt{width} pair, both being measured in millimetres. For
example:

\begin{verbatim}
set papersize a4
set papersize 100,100
\end{verbatim}

A list of recognised paper size names is given in Figure~\ref{paper_sizes}.

\begin{figure}
\tiny \center
\begin{tabular}{|rrr|rrr|}
\hline
\textbf{Name} & \textbf{$h$/mm} & \textbf{$w$/mm} & \textbf{Name} & \textbf{$h$/mm} & \textbf{$w$/mm} \\
\hline
                       2a0 &   1681 &   1189 &           medium &    584 &    457 \\
                       4a0 &   2378 &   1681 &          monarch &    267 &    184 \\
                        a0 &   1189 &    840 &             post &    489 &    394 \\
                        a1 &    840 &    594 &        quad\_demy &   1143 &   889 \\
                       a10 &     37 &     26 &           quarto &    254 &    203 \\
                        a2 &    594 &    420 &            royal &    635 &    508 \\
                        a3 &    420 &    297 &        statement &    216 &    140 \\
                        a4 &    297 &    210 &       swedish\_d0 &   1542 &   1090 \\
                        a5 &    210 &    148 &       swedish\_d1 &   1090 &    771 \\
                        a6 &    148 &    105 &      swedish\_d10 &     48 &     34 \\
                        a7 &    105 &     74 &       swedish\_d2 &    771 &    545 \\
                        a8 &     74 &     52 &       swedish\_d3 &    545 &    385 \\
                        a9 &     52 &     37 &       swedish\_d4 &    385 &    272 \\
                        b0 &   1414 &    999 &       swedish\_d5 &    272 &    192 \\
                        b1 &    999 &    707 &       swedish\_d6 &    192 &    136 \\
                       b10 &     44 &     31 &       swedish\_d7 &    136 &     96 \\
                        b2 &    707 &    499 &       swedish\_d8 &     96 &     68 \\
                        b3 &    499 &    353 &       swedish\_d9 &     68 &     48 \\
                        b4 &    353 &    249 &       swedish\_e0 &   1241 &    878 \\
                        b5 &    249 &    176 &       swedish\_e1 &    878 &    620 \\
                        b6 &    176 &    124 &      swedish\_e10 &     38 &     27 \\
                        b7 &    124 &     88 &       swedish\_e2 &    620 &    439 \\
                        b8 &     88 &     62 &       swedish\_e3 &    439 &    310 \\
                        b9 &     62 &     44 &       swedish\_e4 &    310 &    219 \\
                        c0 &   1296 &    917 &       swedish\_e5 &    219 &    155 \\
                        c1 &    917 &    648 &       swedish\_e6 &    155 &    109 \\
                       c10 &     40 &     28 &       swedish\_e7 &    109 &     77 \\
                        c2 &    648 &    458 &       swedish\_e8 &     77 &     54 \\
                        c3 &    458 &    324 &       swedish\_e9 &     54 &     38 \\
                        c4 &    324 &    229 &       swedish\_f0 &   1476 &   1044 \\
                        c5 &    229 &    162 &       swedish\_f1 &   1044 &    738 \\
                        c6 &    162 &    114 &      swedish\_f10 &     46 &     32 \\
                        c7 &    114 &     81 &       swedish\_f2 &    738 &    522 \\
                        c8 &     81 &     57 &       swedish\_f3 &    522 &    369 \\
                        c9 &     57 &     40 &       swedish\_f4 &    369 &    261 \\
                     crown &    508 &    381 &       swedish\_f5 &    261 &    184 \\
                      demy &    572 &    445 &       swedish\_f6 &    184 &    130 \\
               double\_demy &    889 &    597 &       swedish\_f7 &    130 &     92 \\
                  elephant &    711 &    584 &       swedish\_f8 &     92 &     65 \\
               envelope\_dl &    110 &    220 &       swedish\_f9 &     65 &     46 \\
                 executive &    267 &    184 &       swedish\_g0 &   1354 &    957 \\
                  foolscap &    330 &    203 &       swedish\_g1 &    957 &    677 \\
         government\_letter &    267 &    203 &      swedish\_g10 &     42 &     29 \\
international\_businesscard &     85 &     53 &       swedish\_g2 &    677 &    478 \\
               japanese\_b0 &   1435 &   1015 &       swedish\_g3 &    478 &    338 \\
               japanese\_b1 &   1015 &    717 &       swedish\_g4 &    338 &    239 \\
              japanese\_b10 &     44 &     31 &       swedish\_g5 &    239 &    169 \\
               japanese\_b2 &    717 &    507 &       swedish\_g6 &    169 &    119 \\
               japanese\_b3 &    507 &    358 &       swedish\_g7 &    119 &     84 \\
               japanese\_b4 &    358 &    253 &       swedish\_g8 &     84 &     59 \\
               japanese\_b5 &    253 &    179 &       swedish\_g9 &     59 &     42 \\
               japanese\_b6 &    179 &    126 &       swedish\_h0 &   1610 &   1138 \\
               japanese\_b7 &    126 &     89 &       swedish\_h1 &   1138 &    805 \\
               japanese\_b8 &     89 &     63 &      swedish\_h10 &     50 &     35 \\
               japanese\_b9 &     63 &     44 &       swedish\_h2 &    805 &    569 \\
            japanese\_kiku4 &    306 &    227 &       swedish\_h3 &    569 &    402 \\
            japanese\_kiku5 &    227 &    151 &       swedish\_h4 &    402 &    284 \\
         japanese\_shiroku4 &    379 &    264 &       swedish\_h5 &    284 &    201 \\
         japanese\_shiroku5 &    262 &    189 &       swedish\_h6 &    201 &    142 \\
         japanese\_shiroku6 &    188 &    127 &       swedish\_h7 &    142 &    100 \\
                large\_post &    533 &    419 &       swedish\_h8 &    100 &     71 \\
                    ledger &    432 &    279 &       swedish\_h9 &     71 &     50 \\
                     legal &    356 &    216 &          tabloid &    432 &    279 \\
                    letter &    279 &    216 &  us\_businesscard &     89 &     51 \\
\hline
\end{tabular}
\caption{A list of all of the named paper sizes recognised by the \texttt{set papersize}\index{set
papersize command@\texttt{set papersize} command} command, with their heights, $h$, and widths, $w$, measured in millimetres.}
\label{paper_sizes}
\end{figure}

\section{Plotting}

In this section I outline some of the extensions of the \texttt{plot} command,
to give greater flexibility in the appearance of graphs.

\subsection{Configuring Axes}
\label{axis_extensions}\label{ranges_multiaxes}\label{multiple_axes}

By default, plots have only one $x$-axis and one $y$-axis. Further parallel
axes can be added and configured via statements such as:\index{axes
modifier@\texttt{axes} modifier}\index{set axis command@\texttt{set axis}
command}

\begin{verbatim}
set x3label 'foo'
plot sin(x) axes x3y1
set axis x3
\end{verbatim}

\noindent In the top statement, a further $x$ axis, called $x3$ is implicitly
created by giving it a label. In the next, the \texttt{axes} modifier is used
to tell the \texttt{plot} command to plot data against the $x3$-axis, which
also implicitly created such an axis if it doesn't already exist. In the third,
an $x3$-axis is explicitly created.

Unlike gnuplot, which allowed only a maximum of two parallel axes to be added
to plots, PyXPlot allows an unlimited number of axes to be used. Odd-numbered
$x$-axes appear below the plot, and even numbered $x$-axes above it; a similar
rule applies for $y$-axes, to the left and to the right.

As discussed in the previous chapter, the ranges of axes can be set either
using the \texttt{set xrange} command\index{set xrange command@\texttt{set
xrange} command}, or within the \texttt{plot} command. The following two
statements would set an equivalent range for the $x3$-axis:

\begin{verbatim}
set x3range [-2:2]
plot [:][:][:][:][-2:2] sin(x) axes x3y1
\end{verbatim}

\noindent As usual, the first two ranges specified in the \texttt{plot} command
apply to the $x$- and $y$-axes. The next pair apply to the $x2$- and $y2$-axes,
and so forth.

\index{axes!removal}\index{removing axes}\index{hidden
axes}\label{axis_removal} Having made axes with the above commands, they may
subsequently be removed using the \texttt{unset axis} command as follows:

\begin{verbatim}
unset axis x3
unset axis x3x5y3 y7
\end{verbatim}

\noindent The top statement, for example, would remove axis $x3$. The command
\texttt{unset axis} on its own, with no axes specified, returns all axes to
their default configuration.  The special case of \texttt{unset axis x1} does
not remove the first $x$-axis -- it cannot be removed -- but instead returns it
to its default configuration.

It should be noted, that if the following two commands are typed in succession,
the second may not entirely negate the first:

\begin{verbatim}
set x3label 'foo'
unset x3label 'foo'
\end{verbatim}

\noindent The first may have implicitly created an $x3$-axis, which would need
to be removed with the \texttt{unset axis x3} command.

A subtly different task is that of removing labels from axes, or setting axes
not to display. To achieve this, a number of special axis labels are used.
Labelling an axis ``\texttt{nolabels}''\index{nolabels
keyword@\texttt{nolabels} keyword} has the effect that no title or numerical
labels are placed upon it. Labelling it\label{nolabelstics}
``\texttt{nolabelstics}''\index{nolabelstics keyword@\texttt{nolabelstics}
keyword} is stronger still; this removes all tick marks from it as well
(similar in effect to \texttt{set noxtics} in gnuplot). Finally, labelling it
``\texttt{invisible}''\index{invisible keyword@\texttt{invisible} keyword}
makes an axis completely invisible.

Labels may be placed on such axes, by following the magic keywords above with a
colon and the desired title, for example:

\begin{verbatim}
set xlabel 'nolabels:Time'
\end{verbatim}

\noindent would produce an $x$-axis with no numeric labels, but a label of
`Time'.

Several examples of effects which can be achieved with these commands can be
found in Example~10 (see Section~\ref{ex10}).  In the unlikely event of wanting
to label a normal axis with one of these magic words\index{axes!reserved
labels}\index{magic axis labels}, this may be achieved by prefixing the magic
word with a space. There is one further magic axis label, \texttt{linkaxis},
which will be described in Section~\ref{linked_axes}.

The ticks of axes can be configured to point either inward, towards the plot,
as is the default, or outward towards the axis labels, or in both directions.
This is achieved using the \texttt{set xticdir} command, for example:

\begin{verbatim}
set xticdir inward
set y2ticdir outward
set x2ticdir both
\end{verbatim}

The position of ticks along each axis can be configured with the \texttt{set
xtics}\index{set xtics command@\texttt{set xtics} command} command. The
appearance of ticks along any axis can be turned off with the \texttt{set
noxtics}\index{set noxtics command@\texttt{set noxtics} command} command. The
syntax for these is given below:

\begin{verbatim}
set xtics { axis | border | inward | outward | both }
          {  autofreq
           | <increment>
           | <minimum>, <increment> { , <maximum> }
           | (     {"label"} <position>
               { , {"label"} <position> } .... )
          }
set noxtics
show xtics
\end{verbatim}

The keywords \texttt{inward}, \texttt{outward} and \texttt{both} alter the
directions of the ticks, and have the same effect as in the \texttt{set
xticdir} command. The keyword \texttt{axis} is an alias for \texttt{inward},
and \texttt{border} an alias for \texttt{outward}, both provided for gnuplot
compatibility. If the keyword \texttt{autofreq} is given, the automatic
placement of ticks on the axis is restored.

If \texttt{<minimum>, <increment>, <maximum>} are specified, then ticks are
placed at evenly spaced intervals between the specified limits. In the case of
logarithmic axes, \texttt{<increment>} is applied multiplicatively.

Alternatively, the final form allows ticks to be placed on an axis
individually, and each given its own textual label.

The following pair of examples would both place tick marks at $x=$2, 3, 4, 5.
In the second example, they would be labelled ``a'', ``b'', ``c'' and ``d'':

\begin{verbatim}
set xtics 2, 1, 5

set xtics ("a" 2, "b" 3, "c" 4, "d" 5)
\end{verbatim}

The following example would place tick marks at intervals of two units along
the $x$-axis:

\begin{verbatim}
set xtics 2
\end{verbatim}

The following example would restore the automatic placement of tics along the
$x3$-axis, placing those ticks facing outwards from the graph:

\begin{verbatim}
set xtics border autofreq
\end{verbatim}

Minor tick marks can be placed on axes with the \texttt{set mxtics} command,
which has the same syntax as above.

\subsection{Keys and Legends}

By default, plots are displayed with a legend in their top-right corners. The
textual description of each dataset is drawn by default from the command used
to plot it. Alternatively, the user may specify his own description for each
dataset by following the \texttt{plot} command with the \texttt{title}
modifier\index{title modifier@\texttt{title} modifier}, as follows:

\begin{verbatim}
plot sin(x) title 'A sine wave'
plot cos(x) title ''
\end{verbatim}

In the lower case, a blank title is specified, in which case, PyXPlot makes no
entry for this dataset in the legend. This is useful if it is desired to place
some but not all datasets into the legend of a plot.  Alternatively, the
production of the legend can be completely turned off for all datasets, by the
command \texttt{set nokey}. The opposite effect can be achieved by the
\texttt{set key}\index{set key@\texttt{set key} command} command.

This latter command can also be used to dictate where on the plot the legend
should be placed, using a syntax along the lines of:

\begin{verbatim}
set key top right
\end{verbatim}

The following recognised positioning keywords are self-explanatory:
\texttt{top}, \texttt{bottom}, \texttt{left}, \texttt{right}, \texttt{xcentre}
and \texttt{ycentre}. The word \texttt{outside} places the key outside the
plot, on its right side. The word \texttt{below} places the legend below the
plot.

In addition, two positional offset coordinates may be specified after such
keywords -- the first value is assumed to be an $x$-offset, and the second a
$y$-offset, in units approximately equal to the size of the plot. For example:

\begin{verbatim}
set key bottom left 0.0 -0.5
\end{verbatim}

\noindent would display a key below the bottom left corner of the graph.

By default, entries in the key are placed in a single vertical list. They can
instead be arranged into a number of columns by means of the \texttt{set
keycolumns} command.\index{set keycolumns command@\texttt{set keycolumns}
command}. This should be followed by the integer number of desired columns, for
example:

\begin{verbatim}
set keycolumns 2
\end{verbatim}

\subsection{The \texttt{linestyle} keyword}

At times, the string of style keywords following the \texttt{with} modifier in
\texttt{plot} commands can grow rather unwieldily long. For clarity, frequently
used plot styles can be stored as ``linestyles''; this is true of styles
involving points as well as lines. The syntax for setting a linestyle is:

\begin{verbatim}
set linestyle 2 points pointtype 3
\end{verbatim}

\noindent where the ``2'' is the identification number of the linestyle. In a
subsequent \texttt{plot} statement, this linestyle can be recalled as follows:

\begin{verbatim}
plot sin(x) with linestyle 2
\end{verbatim}

\subsection{Colour Plotting}

\index{colours!setting for datasets} In the \texttt{with}
clause of the plot command, the modifier \texttt{colour}, (abbrev.
`\texttt{c}'), allows the colour of each dataset to be manually selected. It
should be followed either by an integer, to set a colour from the present
palette, or by a colour name. A list of valid colour names is given in
Section~\ref{colour_names}. For example:

\begin{verbatim}
plot sin(x) with c 5
plot sin(x) with colour blue
\end{verbatim}

\noindent The \texttt{colour} modifier can also be used when defining linestyles.

\index{colours!setting the palette}\index{set palette command@\texttt{set
palette} command} PyXPlot has a palette of colours which it assigns
sequentially to datasets when colours are not manually assigned. This is also
the palette to which integers passed to \texttt{set colour} refer -- the `5'
above, for example. It may be set using the \texttt{set palette} command, which
differs in syntax from gnuplot. It should be followed by a comma-separated list
of colours, for example:

\begin{verbatim}
set palette red,green,blue
\end{verbatim}

Another way of setting the palette, in a configuration file, is described in
Section~\ref{config_files}; a list of valid colour names is given in
Section~\ref{colour_names}.

\subsection{General Extensions Beyond Gnuplot}

\begin{longtable}{p{3cm}lp{9cm}}

plot linewidths\index{linewidths!setting for datasets} & --- & For an unknown
reason, gnuplot doesn't allow \texttt{set linewidth 2} as valid syntax. This
setting is allowed to be made in PyXPlot. Furthermore, \texttt{set
pointlinewidth 2} will set the linewidth to be used when drawing data
\textit{points}. A similar effect can be achieved via:

\begin{verbatim}
plot sin(x) with points pointlinewidth 2
\end{verbatim}

\noindent  In both cases, the abbreviation \texttt{plw} is valid. \\

dots plot style\index{dots style@\texttt{dots} style} & --- & When using the
\texttt{dots} style, for example:

\begin{verbatim}
plot sin(x) with dots
\end{verbatim}

\noindent the size of the plotted dots can be varied with the
\texttt{pointsize} modifier, unlike in gnuplot, where the dots were of a fixed
size. For example, to display big dots, use:

\begin{verbatim}
plot sin(x) with dots pointsize 10
\end{verbatim}
\\

\texttt{select} keyword\index{select keyword@\texttt{select}
keyword}\label{select_modifier} & --- & As well as the \texttt{index},
\texttt{using} and \texttt{every} keywords which gnuplot used to allow users to
plot subsets of data from datafiles, PyXPlot also has a further modifier,
\texttt{select}. This can be used to plot only those datapoints in a datafile
which specify some given criterion. For example:

\begin{verbatim}
plot 'datafile' select ($8>5)
plot sin(x) select (($1>0) and ($2>0))
\end{verbatim}
\\ & &
In the second example, two select criteria are given, combined with the logical
and operator\footnote{See Table~\ref{operators_table} for a list of all
operators recognised by PyXPlot.}. The select modifier has many applications,
including plotting two-dimensional slices from three-dimensional datasets, and
selecting certain subsets of datapoints from a datafile for plotting.
\\ & &
Logical operators such as \texttt{and}, \texttt{or} and \texttt{not} can be
used, as seen in the second example above; indeed, any expression which is valid
Python can be used.
\\

arrows plot style\index{arrows plot style@\texttt{arrows} plot
style}\label{arrows_plot_style} & --- & The \texttt{arrows} plot style takes
four columns of data, $x_1$, $y_1$, $x_2$, $y_2$, and for each data point draws
an arrow from the point $(x_1,y_1)$ to $(x_2,y_2)$.  Three different kinds of
arrows can be drawn: ones with normal arrow heads, ones with no arrow heads,
which just appear as lines, and ones with arrow heads on both ends. The syntax
is:

\begin{verbatim}
plot 'datafile' with arrows_head
plot 'datafile' with arrows_nohead
plot 'datafile' with arrows_twohead
\end{verbatim}
\\ & &
The syntax `\texttt{with arrows}' is a shorthand for `\texttt{with arrows\_head}'.\\

lower and upper limit datapoints & --- & PyXPlot can plot datapoints using the
standard upper- and lower-limit symbols.\index{lower-limit
datapoints}\index{upper-limit datapoints} No special syntax is required for
this; these symbols are pointtypes\footnote{The \texttt{pointtype} modifier was
introduced in Section~\ref{pointtype_modifier}.} 12 and 13 respectively,
obtained as follows:

\begin{verbatim}
plot 'upperlimits' with points pointtype 12
plot 'lowerlimits' with points pointtype 13
\end{verbatim}
\\

plotting functions with errorbars and other plot styles & --- & In gnuplot,
when a function (as opposed to a datafile) is plotted, only those plot styles
which accept two columns of data can be used -- for example, \texttt{lines} or
\texttt{points}. It is not possible to plot a function with errorbars, for
example. In PyXPlot, by contrast, this is possible using the following syntax:

\begin{verbatim}
plot f(x):g(x) with yerrorbars
\end{verbatim}
\\ & &

Two functions are supplied, separated by a colon; plotting proceeds as if a
datafile had been supplied, containing values of $x$ in column 1, values of
$f(x)$ in column 2, and values of $g(x)$ in column 3. This may be useful, for
example, if $g(x)$ measures the intrinsic uncertainty in $f(x)$. The
\texttt{using} modifier may also be used:

\begin{verbatim}
plot f(x):g(x) using 2:3
\end{verbatim}
\\ & &
Here, $g(x)$ would be plotted on the $y$-axis, against $f(x)$ on the $x$-axis.
It should be noted, however, that the range of values of $x$ used would still
correspond to the range of the plot's horizontal axis. If the above were to be
attempted with an autoscaling horizontal axis, the result might be rather
unexpected -- PyXPlot would find itself autoscaling the $x$-axis range to the
spread of values of $f(x)$, but find that this itself changed depending upon
the range of the $x$-axis. \\

horizontally arranged datafiles\index{horizontal
datafiles}\index{datafiles!horizontal}\index{using rows modifier@\texttt{using
rows} modifier}\index{using columns modifier@\texttt{using columns}
modifier}\label{horizontal_datafiles} & --- & The command syntax for plotting
columns of datafiles against one another was previously described in
Section~\ref{plot_datafiles}.  In an extension of gnuplot's interface, it is
also possible to plot \textit{rows} of data against one another in
horizontally-arranged datafiles.  For this, the keyword `\texttt{rows}' is
placed after the \texttt{using} modifier:\index{rows keyword@\texttt{rows} keyword}

\begin{verbatim}
plot 'datafile' index 1 using rows 1:2
\end{verbatim}
\\ & &
The syntax `\texttt{using columns}' is also accepted, to specify the default
behaviour of plotting columns against one another:\index{columns keyword@\texttt{columns} keyword}

\begin{verbatim}
plot 'datafile' index 1 using columns 1:2
\end{verbatim}
\\ & &
When plotting horizontally-arranged datafiles, the meanings of the
\texttt{index} and \texttt{every} modifiers (see Section~\ref{plot_datafiles})
are altered slightly. The former continues to refer to vertical blocks of data
separated by two blank lines.  Blocks, as referenced in the \texttt{every}
modifier, continue to be vertical blocks of datapoints, separated by single
blank lines. The row numbers passed to the \texttt{using} modifier are counted
from the top of the current block.
\\ & &
However, the line-numbers specified in the \texttt{every} modifier -- i.e.
variables $a$, $c$ and $e$ in the system above -- now refer to horizontal
columns, rather than lines. For example:
\\ & &
\begin{verbatim}
plot 'datafile' using rows 1:2 every 2::3::9
\end{verbatim}
\\ & &
\noindent would plot the data in row 2 against that in row 1, using only the
values in every other column, between columns 3 and 9.
\\

errorbars\index{errorbars}\label{errorbars} & --- & In gnuplot, when one used
errorbars, one could either specify the size of the errorbar, or the min/max
range of the errorbar. Both of these usages shared a common syntax, and
gnuplot's behaviour depended upon the number of data columns provided:

\begin{verbatim}
plot 'datafile' with yerrorbars
\end{verbatim}
\\ & &
\noindent Given a datafile with three columns, this would take the third column
to indicate the size of the $y$-errorbar, and given a four-column datafile, it
would take the third and fourth columns to indicate the min/max range to be
marked out by the errorbar.
\\ & &
To avoid confusion, a different syntax is adopted in PyXPlot. The syntax:

\begin{verbatim}
plot 'datafile' with yerrorbars
\end{verbatim}
\\ & &
\noindent now always assumes the third column of the datafile to indicate the
size of the errorbar, regardless of whether a fourth is present. The syntax:

\begin{verbatim}
plot 'datafile' with yerrorrange
\end{verbatim}
\\ & &
\noindent always assumes the third and fourth columns to indicate the min/max
range of the errorbar.

\vspace{0.5cm}
For clarity, a complete list of errorbar styles is given below:
\\ & &
\begin{tabular}{p{2.5cm}p{5.5cm}}
\texttt{yerrorbars} & Vertical errorbars; size drawn from the third data-column. \\
\texttt{xerrorbars} & Horizontal errorbars; size drawn from the third data-column. \\
\texttt{xyerrorbars} & Horizontal and vertical errorbars; sizes drawn from the third and fourth data-columns respectively.\\
\texttt{errorbars} & Shorthand for \texttt{yerrorbars}. \\
\end{tabular}
\\ & &
\begin{tabular}{p{2.5cm}p{5.5cm}}
\texttt{yerrorrange} & Vertical errorbars; minimum drawn from the third data-column, maximum from the fourth.\\
\texttt{xerrorrange} & Horizontal errorbars; minimum drawn from the third data-column, maximum from the fourth.\\
\texttt{xyerrorrange} & Horizontal and vertical errorbars; horizontal minimum drawn from the third data-column, and maximum from the fourth; vertical minimum drawn from the fifth, and maximum from the sixth.\\
\texttt{errorrange} & Shorthand for \texttt{yerrorrange}. \\
\end{tabular}
\\

datafile wildcards\index{globbing}\index{wildcards}\index{datafiles!globbing} & --- & PyXPlot allows the wildcards `*' and `?' to be used both in the filenames of datafiles following the \texttt{plot} command, and also when specifying command files on the commandline and with the \texttt{load} command. For example, the following would plot all datafiles in the current directory with a `.dat' suffix, using the same plot options:
\\ & &
\begin{verbatim}
plot '*.dat' with linewidth 2
\end{verbatim}

In the legend, full filenames are displayed, allowing the datafiles to be distinguished.
\\ & &
As in gnuplot, a blank filename passed to the plot command causes the last used datafile to be used again.
\\

backing up overwritten files\index{set backup command@\texttt{set backup}
command}\index{overwriting files}\index{backup files}\label{filebackup} & ---
& By default, when plotting to a file, if the output filename matches that of
an existing file, that file is overwritten. This behaviour may be changed with
the \texttt{set backup} command, which has syntax:
\\ & &
\begin{verbatim}
set backup
set nobackup
\end{verbatim}
\\ & &
When this switch is turned on, pre-existing files will be renamed with a tilda at the end of their filenames, rather than being overwritten. \\

\end{longtable}

\section{Sundry Items (Arrows, Text Labels, and More)}

This section describes how to put arrows and text labels on plots; the syntax
is similar to that used by gnuplot, but slightly changed. It is now possible,
for example, to set the linestyles and colours with which arrows should be
drawn.  Also covered is how to put grids onto plots, and how to change the size
and colour of textual labels on plots.

\subsection{Arrows}

\label{set_arrow}\index{arrows}\index{set arrow command@\texttt{set arrow}
command} Arrows may be placed on plots using the \texttt{set arrow} command,
which has similar syntax to that used by gnuplot. A simple example would be:

\begin{verbatim}
set arrow 1 from 0,0 to 1,1
\end{verbatim}

\noindent The number `1' immediately following `set arrow' specifies an identification
number for the arrow, allowing it to be subsequently removed via:

\begin{verbatim}
unset arrow 1
\end{verbatim}

\noindent or equivalently, via:\index{set noarrow command@\texttt{set noarrow}
command}

\begin{verbatim}
set noarrow 1
\end{verbatim}

In PyXPlot, this syntax is extended; the \texttt{set arrow} command can be
followed by the keyword `\texttt{with}', to specify the style of the arrow. For
example, the specifiers `\texttt{nohead}', `\texttt{head}' and
`\texttt{twohead}', after the keyword `\texttt{with}', can be used to make
arrows with no arrow heads, normal arrow heads, or two arrow heads.
`\texttt{twoway}' is an alias for `\texttt{twohead}'.  For example:

\begin{verbatim}
set arrow 1 from 0,0 to 1,1 with nohead
\end{verbatim}

In addition, linestyles and colours can be specified after the keyword
`\texttt{with}':

\begin{verbatim}
set arrow 1 from 0,0 to 1,1 with nohead \
linetype 1 c blue
\end{verbatim}

As in gnuplot, the coordinates for the start and end points of the arrow can be
specified in a range of coordinate systems. `\texttt{first}', the default,
measures the graph using the $x$- and $y$-axes. `\texttt{second}' uses the $x2$-
and $y2$-axes. `\texttt{screen}' and `\texttt{graph}' both measure in centimetres
from the origin of the graph. In the following example, we use these
specifiers, and specify coordinates using variables rather than doing so
explicitly:

\begin{verbatim}
x0 = 0.0
y0 = 0.0
x1 = 1.0
y1 = 1.0
set arrow 1 from first  x0, first  x1 \
            to   screen x1, screen x1 \
            with nohead
\end{verbatim}

In addition to these four options, which are those available in gnuplot, the
syntax `\texttt{axis}\textit{n}' may also be used, to use the $n\,$th $x$- or
$y$-axis -- for example, `\texttt{axis3}'.\index{set arrow command@\texttt{set
arrow} command} This allows arrows to reference any arbitrary axis on plots
which make use of large numbers of parallel axes (see
Section~\ref{multiple_axes}).

\subsection{Text Labels}

Text labels may be placed on plots using the \texttt{set label}
command\index{set label command@\texttt{set label} command}. As with all
textual labels in PyXPlot, these are rendered in \LaTeX:

\begin{verbatim}
set label 1 'Hello World' at 0,0
\end{verbatim}

As in the previous section, the number `1' is a reference number, which allows
the label to be removed by either of the following two commands:

\begin{verbatim}
set nolabel 1
unset label 1
\end{verbatim}

\noindent The positional coordinates for the text label, placed after the
keyword `\texttt{at}', can be specified in any of the coordinate systems
described for arrows above. A rotation angle may optionally be specified after
the keyword `{\tt rotate}', to rotate text counter-clockwise by a given
angle, measured in degrees. For example, the following would produce
upward-running text:

\begin{verbatim}
set label 1 'Hello World' at axis3 3.0, axis4 2.7 rotate 90
\end{verbatim}

\index{fontsize}\index{text!size}\index{set fontsize command@\texttt{set
fontsize} command} The fontsize of these text labels can globally be set using
the \texttt{set fontsize x} command. This applies not only to the \texttt{set
label} command, but also to plot titles, axis labels, keys, etc. The value
given should be an integer in the range $-4 \leq x \leq 5$. The default is
zero, which corresponds to \LaTeX's \texttt{normalsize}; -4 corresponds to
\texttt{tiny} and 5 to \texttt{Huge}.

\index{text!colour}\index{colours!text}\index{set textcolour
command@\texttt{set textcolour} command} The \texttt{set textcolour} command
can be used to globally set the colour of all text output, and applies to all
of the text that the \texttt{set fontsize} command does. It is especially
useful when producing plots to be embedded in presentation slideshows, where
bright text on a dark background may be desired. It should be followed either
by an integer, to set a colour from the present palette, or by a colour name. A
list of the recognised colour names can be found in Section~\ref{colour_names}.
For example:

\begin{verbatim}
set textcolour 2
set textcolour blue
\end{verbatim}

\index{text!alignment}\index{alignment!text}\index{set texthalign command@\texttt{set texthalign} command}\index{set textvalign command@\texttt{set textvalign} command}By default, each label's specified position corresponds to its bottom left corner. This alignment may be changed with the \texttt{set texthalign} and \texttt{set textvalign} commands. The former takes the options \texttt{left}, \texttt{centre} or \texttt{right}, and the latter takes the options \texttt{bottom}, \texttt{centre} or \texttt{top}, for example:

\begin{verbatim}
set texthalign right
set textvalign top
\end{verbatim}

\subsection{Gridlines}

Gridlines may be placed on a plot and subsequently removed via the statements:

\begin{verbatim}
set grid
set nogrid
\end{verbatim}

\noindent respectively. The following commands are also valid:

\begin{verbatim}
unset grid
unset nogrid
\end{verbatim}

\noindent By default, gridlines are drawn from the major and minor ticks of the
$x$- and $y$-axes. However, the axes which should be used may be specified
after the \texttt{set grid} command\index{grid}\index{set grid
command@\texttt{set grid} command}:

\begin{verbatim}
set grid x2y2
set grid x x2y2
\end{verbatim}

\noindent The top example would connect the gridlines to the ticks of the $x2$-
and $y2$- axes, whilst the lower would draw gridlines from both the $x$- and
the $x2$-axes.

If one of the specified axes does not exist, then no gridlines will be drawn in
that direction.  Gridlines can subsequently be removed selectively from some
axes via:

\begin{verbatim}
unset grid x2x3
\end{verbatim}

The colours of gridlines can be controlled via the \texttt{set
gridmajcolour}\index{grid!colour}\index{colours!grid}\index{set gridmajcolour
command@\texttt{set gridmajcolour} command}\index{set gridmincolour
command@\texttt{set gridmincolour} command} and \texttt{set gridmincolour}
commands, which control the gridlines emanating from major and minor axis ticks
respectively. An example would be:

\begin{verbatim}
set gridmincolour blue
\end{verbatim}

\noindent Any of the colour names listed in Section~\ref{colour_names} can be
used.

A related command is \texttt{set
axescolour}\index{axes!colour}\index{colours!axes}\index{set
axescolourcommand@\texttt{set axescolour} command}, which has a syntax similar
to that above, and sets the colour of the graph's axes.
\label{set_colours}

\section{Multi-plotting}
\label{multiplot}
\index{multiplot}

Gnuplot has a plotting mode called ``multiplot'' which allows many graphs to be
plotted together, and display side-by-side. The basic syntax of this mode is
reproduced in PyXPlot, but is hugely extended.

The mode is entered by the command ``\texttt{set multiplot}''. This can be compared
to taking a blank sheet of paper on which to place plots. Plots are then placed
on that sheet of paper, as usual, with the \texttt{plot} command. The position
of each plot is set using the \texttt{set origin} command, which takes a
comma-separated $x,y$ coordinate pair, measured in centimetres. The following,
for example, would plot a graph of $\sin(x)$ to the left of a plot of
$\cos(x)$:\index{set origin command@\texttt{set origin} command}

\begin{verbatim} 
set multiplot
plot sin(x)
set origin 10,0
plot cos(x)
\end{verbatim}

The multiplot page may subsequently be cleared with the \texttt{clear} command,
and multiplot mode may be left using the ``\texttt{set nomultiplot}''
command.\index{clear command@\texttt{clear} command}

\subsection{Deleting, Moving and Changing Plots}

Each time a plot is placed on the multiplot page in PyXPlot, it is allocated a
reference number, which is output to the terminal. Reference numbers count up
from zero each time the multiplot page is cleared. A number of commands exist
for modifying plots after they have been placed on the page, selecting them by
making reference to their reference numbers.

Plots may be removed from the page with the \texttt{delete} command, and
restored with the \texttt{undelete} command:\index{delete
command@\texttt{delete} command}\index{undelete command@\texttt{undelete}
command}

\begin{verbatim} 
delete <number>
undelete <number>
\end{verbatim}

The reference numbers of deleted plots are not reused until the page is
cleared, as they may always be restored with the \texttt{undelete} command;
plots which have been deleted simply do not appear.

Plots may also be moved with the \texttt{move} command. For example, the
following would move plot 23 to position (8,8) measured in centimetres:

\begin{verbatim} 
move 23 to 8,8
\end{verbatim}

In multiplot mode, the \texttt{replot} command can be used to modify the last
plot added to the page. For example, the following would change the title of
the latest plot to ``foo'', and add a plot of $\cos(x)$ to it:

\begin{verbatim} 
set title 'foo'
replot cos(x)
\end{verbatim}

Additionally, it is possible to modify any plot on the page, by first selecting
it with the \texttt{edit} command. Subsequently, the \texttt{replot} will act
upon the selected plot. The following example would produce two plots, and then
change the colour of the text on the first:

\begin{verbatim} 
set multiplot
plot sin(x)
set origin 10,0
plot cos(x)
edit 0        # Select the first plot ...
set textcolour red
replot        # ... and replot it.
\end{verbatim}

The \texttt{edit} command can also be used to view the settings which are
applied to any plot on the multiplot page -- after executing ``edit 0'', the
\texttt{show} command will show the settings applied to plot zero.

When a new plot is added to the page, \texttt{replot} always switches to act
upon this most recent plot.

\index{refresh command@\texttt{refresh} command}\index{replotting}
\index{replot command@\texttt{replot} command} The \texttt{refresh} command is
rather similar to the \texttt{replot} command, but produces an exact copy of
the latest display. This can be useful, for example, after changing the
terminal type, to produce a second copy of a multiplot page in a different
format. But the crucial difference between this command and \texttt{replot} is
that it doesn't replot anything. Indeed, there could be only textual items and
arrows on the present multiplot page, and no graphs \textit{to} replot.

\subsection{Linked Axes}

The axes of plots can be linked together, in such a way that they always share
a common scale. This can be useful when placing plots next to one another,
firstly, of course, if it is of intrinsic interest to ensure that they are on a
common scale, but also because the two plots then do not both need their own
axis labels, and space can be saved by one sharing the labels from the other.
In PyXPlot, an axis which borrows its scale and labels from another is called a
``linked axis''.

Such axes are declared by setting the label of the linked axis to a magic
string such as ``\texttt{linkaxis 0}''\label{linked_axes}\index{axes!reserved
labels}\index{magic axis labels}. This magic label would set the axis to borrow
its scale from an axis from plot zero. The general syntax is
``\texttt{linkaxis} $n$ $m$'', where $n$ and $m$ are two integers, separated by
a comma or whitespace. The first, $n$, indicates the plot from which to borrow
an axis; the second, $m$, indicates whether to borrow the scale of axis $x1$,
$x2$, $x3$, etc. By default, $m=1$. The linking will fail, and a warning
result, if an attempt is made to link to an axis which doesn't exist.

The specimen plots in Section~\ref{gallery} show numerous examples of the use
of linked axes.


\subsection{Text Labels, Arrows and Images}

\label{text_command}\index{text command@\texttt{text} command} In addition to
placing plots on the multiplot page, text labels may also be inserted
independently of any plots, using the \texttt{text} command. This has the
following syntax:

\begin{verbatim} 
text 'This is some text' at x,y
\end{verbatim}

In this case, the string ``This is some text'' would be rendered at position
$(x,y)$ on the multiplot. As with the \texttt{set label} command, a rotation
angle may optionally be specified to rotate text labels through any given
angle, measured in degrees counter-clockwise, for example:

\begin{verbatim} 
text 'This is some text' at x,y rotate r
\end{verbatim}

The commands \texttt{set textcolour}, \texttt{set
texthalign} and \texttt{set textvalign}, which have already been described in
the context in the \texttt{set label} command, can also be used to set the
colour and alignment of text produced with the \texttt{text} command.\index{set
textcolour command@\texttt{set textcolour} command}\index{set texthalign
command@\texttt{set texthalign} command}\index{set textvalign
command@\texttt{set textvalign} command}. A useful application of this is to
produce centred headings at the top of multiplots.

As with plots, each text item has a unique identification number, and can be
moved around, deleted or undeleted with the \texttt{delete}, \texttt{undelete}
and \texttt{move} commands.
\index{delete command@\texttt{delete} command}
\index{undelete command@\texttt{undelete} command}

It should be noted that the \texttt{text} command can also be used outside of
the multiplot environment, to render a single piece of short text instead of a
graph. This has limited applications, but one is illustrated in
Section~\ref{powerpoint_example}.

\label{arrows} \index{arrow command@\texttt{arrow} command} Arrows may also be
placed on multiplot pages, independently of any plots, using the \texttt{arrow}
command, which has syntax:

\begin{verbatim} 
arrow from x,y to x,y
\end{verbatim}

As above, arrows receive unique identification numbers, and can be deleted and
undeleted.

The \texttt{arrow} command may be followed by the `\texttt{with}' keyword to
specify to style of the arrow. The style keywords which are accepted are
identical to those accepted by the \texttt{set arrow} command (see
Section~\ref{set_arrow}). For example:

\begin{verbatim} 
arrow from x1,y1 to x2,y2 \
with twohead colour red
\end{verbatim}

\index{jpeg command@\texttt{jpeg} command} Bitmap images in jpeg form may be
placed on the multiplot using the {\tt jpeg} command.  This has syntax:

\begin{verbatim}
jpeg 'filename' at x,y width w
\end{verbatim}

As an alternative to the {\tt width} modifier the height of the image can be
specified, using the analogous {\tt height} modifier.  An optional angle can
also be specified using the {\tt rotate} modifier; this causes the included
image to be rotated counter-clockwise by a specified angle (in degrees).

\index{eps command@\texttt{eps} command} Vector graphic images in eps format may
be placed on to a multiplot using the {\tt eps} command, which has a syntax
analogous to the {\tt jpeg} command.  However neither height nor width need be
specified; in this case the image will be included at its native size.  For
example:

\begin{verbatim}
eps 'badger.eps' at 3,2 rotate 5
\end{verbatim}

\noindent will place the eps file with its bottom-left corner at position
$(3,2)$ cm from the origin, rotated counter-clockwise through 5 degrees.

\subsection{Speed Issues}
\label{set_display}

By default, whenever an item is added to a multiplot, or an existing item moved
or replotted, the whole multiplot is replotted to show the change. This can be
a time consuming process on large and complex multiplots. For this reason, the
\texttt{set nodisplay}\index{set display command@\texttt{set display} command}
command is provided, which stops PyXPlot from producing any output. The
\texttt{set display} command can subsequently be issued to return to normal
behaviour.

This can be especially useful in scripts which produce large multiplots. There
is no point in producing output at each step in the construction of a large
multiplot, and so a great speed increase can be achieved by wrapping the script
with:

\begin{verbatim} 
set nodisplay
[...prepare large multiplot...]
set display
refresh
\end{verbatim}

The reader will observe that frequent use of this is made in the examples of
Chapter~\ref{examples}.

\section{Barcharts and Histograms}
\label{barcharts}\index{bar charts}
\index{steps plot style@\texttt{steps} plot style}
\index{fsteps plot style@\texttt{fsteps} plot style}
\index{histeps plot style@\texttt{histeps} plot style}
\index{impulses plot style@\texttt{impulses} plot style}
\index{set boxfrom command@\texttt{set boxfrom} command}

\subsection{Basic Operation}

As in gnuplot, bar charts and histograms can be produced using the
\texttt{boxes}\index{boxes plot style@\texttt{boxes} plot style} plot style:

\begin{verbatim} 
plot 'datafile' with boxes
\end{verbatim}

\noindent Horizontally, the interfaces between the bars are, by default, at the
midpoints along the $x$-axis between the specified datapoints (see, for
example, panel (a) of figure~\ref{fig_ex7}, and the script which produced it,
in Section~\ref{example7}).  Alternatively, the widths of the bars may be set
using the \texttt{set boxwidth} command. In this case, all of the bars will be
centred upon their specified $x$-coordinates, and have total widths equalling
that specified in the \texttt{set boxwidth}\index{set boxwidth
command@\texttt{set boxwidth} command} command. Consequently, there may be gaps
between them, or they may overlap, as seen in panel (c) of
figure~\ref{fig_ex7}.

Having set a fixed box width, the default automatic width mode may be restored
either with the \texttt{unset boxwidth} command, or by setting the boxwidth to
a negative width.

As a third alternative, it is also possible to specify different widths for
each bar manually, in a column of the input datafile. For this, the
\texttt{wboxes}\index{wboxes plot style@\texttt{wboxes} plot style} plot style
should be used:

\begin{verbatim} 
plot 'datafile' using 1:2:3 with wboxes
\end{verbatim}

\noindent This plot style expects three columns of data to be specified: the
$x$- and $y$-coordinates of each bar, and the width in the third column. Panel
(b) of figure~\ref{fig_ex7} shows an example of this plot style in use.

By default, the bars all originate from the line $y=0$, as is normally wanted
for a histogram. However, should it be desired for the bars to start from a
different vertical point, that may be achieved with the \texttt{set boxfrom}
command, for example:

\begin{verbatim} 
set boxfrom 5
\end{verbatim}

\noindent All of the bars would then originate from the line $y=5$. Panel (f)
of figure~\ref{fig_ex6} shows the kind of effect that is achieved; for
comparison, panel (b) of the same figure shows the same bar chart with the
boxes starting from their default position at $y=0$.

The bars may be filled using the \texttt{with fillcolour}\index{fillcolour
modifier@\texttt{fillcolour} modifier} modifier, followed by the name of a
colour:

\begin{verbatim} 
plot 'datafile' with boxes fillcolour blue
plot 'datafile' with boxes fc 4
\end{verbatim}

\noindent Panels (c) and (d) of figure~\ref{fig_ex7} demonstrate the use of
filled bars.

Finally, the \texttt{impulses} plot style, as in gnuplot, produces bars of zero
width; see panel (e) of figure~\ref{fig_ex6} for an example.

\subsection{Stacked Bar Charts}

If several datapoints are supplied at a common $x$-coordinate to the
\texttt{boxes} or \texttt{wboxes} plot styles, then the bars are stacked one
above another into a stacked barchart. Consider the following datafile:

\begin{verbatim} 
1 1
2 2
2 3
3 4
\end{verbatim}

The second bar at $x=2$ would be placed on top of the first, spanning the range
$2<y<5$, and having the same width as the first. If plot colours are being
automatically selected from the palette, then a different palette colour is
used to plot the upper bar.

\subsection{Steps}

As an alternative to solid boxes, a graph may also be plotted with ``steps'';
see panels (a), (c) and (d) of figure~\ref{fig_ex6} for examples. As is
illustrated by these panels, three flavours of steps are available (exactly as
in gnuplot):

\begin{verbatim}
plot 'datafile' with steps 
plot 'datafile' with fsteps 
plot 'datafile' with histeps
\end{verbatim}

When using the \texttt{steps} plot style, the datapoints specify the right-most
edges of each step. By contrast, they specify the left-most edges of the steps
when using the \texttt{fsteps} plot style. The \texttt{histeps} plot style
works rather like the \texttt{boxes} plot style; the interfaces between the
steps occur at the horizontal midpoints between the datapoints.

\section{Function Splicing}
\index{function splicing}
\index{splicing functions}

In PyXPlot, as in gnuplot, user-defined functions may be declared on the
commandline:

\begin{verbatim}
f(x) = x*sin(x)
\end{verbatim}

\noindent As an extension to what is possible in gnuplot, it is also possible
to declare functions which are only valid over a certain range of argument
space. For example, the following function would only be valid in the range
$-2<x<2$:\footnote{The syntax \texttt{[-2:2]} can also be written \texttt{[-2
to 2]}.}

\begin{verbatim}
f(x)[-2:2] = x*sin(x)
\end{verbatim}

\noindent The following function would only be valid when all of ${a,b,c}$ were
in the range $-1 \to 1$:

\begin{verbatim}
f(a,b,c)[-1:1][-1:1][-1:1] = a+b+c
\end{verbatim}

If an attempt is made to evaluate a function outside of its specified range,
then an error results. This may be useful, for example, for plotting a
function, but not continuing it outside some specified range. The following
would print the function $\sin(x)$, but only in the range $-2<x<7$:

\begin{verbatim}
f(x)[-2:7] = sin(x)
plot f(x)
\end{verbatim}

\label{splice} \noindent The output of this particular example can be seen in
panel (a) of figure~\ref{fig_ex9}. A similar effect could also have been
achieved with the \texttt{select} keyword; see Section~\ref{select_modifier}.

It is possible to make multiple declarations of the same function, over
different regions of argument space; if there is an overlap in the valid
argument space for multiple definitions, then later declarations take
precedence. This makes it possible to use different functional forms for a
function in different parts of parameter space, and is especially useful when
fitting a function to data, if different functional forms are to be spliced
together to fit different regimes in the data.

Another application of function splicing is to work with functions which do not
have analytic forms, or which are, by definition, discontinuous, such as
top-hat functions or Heaviside functions. The following example would define
$f(x)$ to be a Heaviside function:

\begin{verbatim}
f(x) = 0
f(x)[0:] = 1
\end{verbatim}

\noindent The declaration of a function similar to a top-hat function is
demonstrated in panel (b) of figure~\ref{fig_ex9}. The following example would
define $f(x)$ to follow the Fibonacci sequence, though it is not at all
computationally efficient, and it is inadvisable to evaluate it for $x>8$:

\begin{verbatim}
f(x) = 1
f(x)[2:] = f(x-1) + f(x-2)
plot [0:8] f(x)
\end{verbatim}

\section{Datafile Interpolation: Spline Fitting}
\label{spline_command}
\index{best fit lines}

Gnuplot allows data to be interpolated using its \texttt{csplines} plot style,
for example:\index{csplines plot style@\texttt{csplines} plot
style}\index{acsplines plot style@\texttt{acsplines} plot style}

\begin{verbatim}
plot 'datafile' with smooth csplines
plot 'datafile' with smooth acsplines
\end{verbatim}

\noindent where the upper statement fits a spline through all of the
datapoints, and the lower applies some smoothing to the data first. This syntax
is supported in PyXPlot but deprecated.  A similar effect can be achieved with
the new, more powerful, \texttt{spline} command\index{spline
command@\texttt{spline} command}. This has a syntax similar to that of the
\texttt{fit} command, for example:

\begin{verbatim}
spline f() 'datafile' index 1 using 2:3
\end{verbatim}

The function $f(x)$ now becomes a special function, representing a spline fit
to the given datafile. It can be plotted or otherwise used in exactly the same
way as any other function. This approach is more flexible than gnuplot's
syntax, as the spline $f(x)$ can subsequently be spliced together with other
functions (see the previous section), or used in any mathematical operation.
The following code snippet, for example, would fit splines through two
datasets, and then plot the interpolated differences between them, regardless,
for example, of whether the two datasets were sampled at exactly the same $x$
coordinates:

\begin{verbatim}
spline f() 'datafile1'
spline g() 'datafile2'
plot f(x)-g(x)
\end{verbatim}

Smoothed splines can also be produced:

\begin{verbatim}
spline f() 'datafile1' smooth 1.0
\end{verbatim}

\noindent where the value $1.0$ determines the degree of smoothing to apply;
the higher the value, the more smoothing is applied. The default behaviour is
not to smooth at all (equivalent to \texttt{smooth 0.0}); a value of $1.0$
corresponds to the default amount of smoothing applied in the
\texttt{acsplines} plot style.

\section{Numerical Integration and Differentiation}

Special functions are available for performing numerical integration and
differentiation of expressions: \texttt{int\_dx()} and
\texttt{diff\_dx()}\index{differentiation}\index{integration}\index{int\_dx()
function@\texttt{int\_dx()} function}\index{diff\_dx()
function@\texttt{diff\_dx()} function}. In each case, the ``\texttt{x}'' may be
replaced with any valid variable name, to integrate or differentiate with
respect to any given variable.

The function \texttt{int\_dx()} takes three parameters -- firstly the
expression to be integrated, followed by the minimum and maximum integration
limits. For example, the following would plot the integral of the function
$\sin(x)$:

\begin{verbatim}
plot int_dt(sin(t),0,x)
\end{verbatim} 

The function \texttt{diff\_dx()} takes two parameters and an optional third --
firstly the expression to be differentiated, then the point at which the
differential should be evaluated, and then an optional parameter, $\epsilon$.
The following example would evaluate the differential of the function $\cos(x)$
with respect to $x$ at $x=1.0$:

\begin{verbatim}
print diff_dx(cos(x), 1.0)
\end{verbatim}

Differentials are evaluated by a simple differencing algorithm, and the
parameter $\epsilon$ controls the spacing with which to perform the
differencing operation:

\begin{displaymath}
\left.\frac{\mathrm{d}f}{\mathrm{d}x}\right|_{x=x_0} \approx \frac{f(x_0+\epsilon/2) - f(x_0-\epsilon/2)}{\epsilon}
\end{displaymath}

\noindent By default, $\epsilon = 10^{-6}$.

Advanced users may be interested to know that integration is performed using
the \texttt{quad} function of the \texttt{integrate} package of the
\texttt{scipy} numerical toolkit for Python -- a general purpose integration
routine.

\section{Script Watching: pyxplot\_watch}

PyXPlot includes a simple tool for watching command script files, and executing
them whenever they are modified. This may be useful when developing a command
script, if one wants to make small modifications to it, and see the results in
a semi-live fashion. This tool is invoked by calling the
\texttt{pyxplot\_watch}\index{pyxplot\_watch}\index{watching scripts} command
from a shell prompt. The commandline syntax of \texttt{pyxplot\_watch} is
similar to that of PyXPlot itself, for example:

\begin{verbatim}
pyxplot_watch script
\end{verbatim}

\noindent would set \texttt{pyxplot\_watch} to watch the command script file
\texttt{script}. One difference, however, is that if multiple script files are
specified on the commandline, they are watched and executing independently,
\textit{not} sequentially, as PyXPlot itself would do. Wildcard characters can
also be used to set \texttt{pyxplot\_watch} to watch multiple
files.\footnote{Note that \texttt{pyxplot\_watch *.script} and
\texttt{pyxplot\_watch $\backslash$*.script} will behave differently in most
UNIX shells.  In the first case, the wildcard is expanded by your shell, and a
list of files passed to \texttt{pyxplot\_watch}. Any files matching the
wildcard, created after running \texttt{pyxplot\_watch}, will not be picked up.
In the latter case, the wildcard is expanded by \texttt{pyxplot\_watch} itself,
which \textit{will} pick up any newly created files.}

This is especially useful when combined with GhostView's watch facility. For
example, suppose that a script \texttt{foo} produces postscript output
\texttt{foo.ps}. The following two commands could be used to give a live view
of the result of executing this script:

\begin{verbatim}
gv --watch foo.ps &
pyxplot_watch foo
\end{verbatim}

\chapter{Configuring PyXPlot}

\section{Overview}

\label{configuration}

\index{set command@\texttt{set} command}
As is the case in gnuplot, PyXPlot can be configured using the \noindent
\texttt{set} command -- for example:

\begin{verbatim}set output 'foo.eps'\end{verbatim}

\noindent would set it to send its plotted output to the file
\texttt{foo.eps}.  Typing `\texttt{set}' on its own returns a list of all
recognised `\texttt{set}' configuration parameters. The \texttt{unset} command
may be used to return settings to their default values; it recognises a similar
set of parameter names, and once again, typing `\texttt{unset}' on its own
gives a list of them. The \texttt{show} command can be used to display the
values of settings.

\section{Configuration Files}
\label{config_files}

PyXPlot can also be configured by means of a configuration file, with filename
\texttt{.pyxplotrc}, which is scanned once upon startup. This file may be
placed either in the user's current working directory, or in his home
directory. In the event of both files existing, settings in the former override
those in the latter; in the event of neither file existing, PyXPlot uses its
own default settings.

The configuration file should take the form of a series of sections, each
headed by a section heading enclosed in square brackets, and followed by
variables declared using the format:

\begin{verbatim} 
OUTPUT=foo.eps
\end{verbatim}

The following sections are used, although they do not all need to be present in
any given file:

\begin{itemize}
\item \texttt{settings} -- contains parameters similar to those which can be set with the \texttt{set} command. A complete list is given in Section~\ref{configfile_settings} below.
\item \texttt{terminal} -- contains parameters for altering the behaviour and appearance of PyXPlot's interactive terminal. A complete list is given in Section~\ref{configfile_terminal}.
\item \texttt{variables} -- contains variable definitions. Any variables defined in this section will be predefined in the PyXPlot mathematical environment upon startup.
\item \texttt{functions} -- contains function definitions.
\item \texttt{colours} -- contains a variable `\texttt{palette}', which should be set to a comma-separated list of the sequence of colours in the palette used to plot datasets. The first will be called colour 1 in PyXPlot, the second colour 2, etc. A list of recognised colour names is given in Section~\ref{colour_names}.
\item \texttt{latex} -- contains a variable `\texttt{preamble}', which is
prefixed to the beginning of all \LaTeX\ text items, before the
\texttt{\textbackslash begin\{document\}} statement. It can be used to define
custom \LaTeX\ macros, or to include packages using the \texttt{\textbackslash
includepackage\{\}} command.  The preamble can be changed using the {\tt set
preabmle} command.
\end{itemize}

\section{An Example Configuration File}
\index{configuration files}
\noindent As an example, the following is a configuration file
which would represent PyXPlot's default configuration:

\begin{verbatim}
[settings]
ASPECT=1.0
AUTOASPECT=ON
AXESCOLOUR=Black
BACKUP=OFF
BAR=1.0
BOXFROM=0
BOXWIDTH=0
COLOUR=ON
DATASTYLE=points
DISPLAY=ON
DPI=300
ENLARGE=OFF
FONTSIZE=0
FUNCSTYLE=lines
GRID=OFF
GRIDAXISX=1
GRIDAXISY=1
GRIDMAJCOLOUR=Grey60
GRIDMINCOLOUR=Grey90
KEY=ON
KEYCOLUMNS=1
KEYPOS=TOP RIGHT
KEY_XOFF=0.0
KEY_YOFF=0.0
LANDSCAPE=OFF
LINEWIDTH=1.0
MULTIPLOT=OFF
ORIGINX=0.0
ORIGINY=0.0
OUTPUT=
POINTLINEWIDTH=1.0
POINTSIZE=1.0 
SAMPLES=250
TERMINVERT=OFF
TERMTRANSPARENT=OFF
TERMTYPE=X11_singlewindow
TEXTCOLOUR=Black
TEXTHALIGN=Left
TEXTVALIGN=Bottom
TITLE=
TIT_XOFF=0.0
TIT_YOFF=0.0
WIDTH=8.0

[terminal]
COLOUR=OFF
COLOUR_ERR=Red
COLOUR_REP=Green
COLOUR_WRN=Brown
SPLASH=ON

[variables]
pi = 3.14159265358979

[colours]
palette = Black, Red, Blue, Magenta, Cyan, Brown, Salmon, Gray,
Green, NavyBlue, Periwinkle, PineGreen, SeaGreen, GreenYellow,
Orange, CarnationPink, Plum

[latex]
PREAMBLE=
\end{verbatim}

\section{Configuration Options: \texttt{settings} section}
\label{configfile_settings}

The following table provides a brief description of the function of each of the
parameters in the \texttt{settings} section of the above configuration file,
with a list of possible values for each:

\begin{longtable}{p{3.4cm}p{9cm}}
\texttt{ASPECT} & \textbf{Possible values:} Any floating-point number.

                   \textbf{Analogous set command:} \texttt{set size ratio}\index{set size ratio command@\texttt{set size ratio} command}

                   Sets the aspect ratio of plots.
                   \\
\texttt{AUTOASPECT} & \textbf{Possible values:} ON / OFF

                   \textbf{Analogous set command:} \texttt{set size ratio}

                   Sets whether plots have the automatic aspect ratio, which is the golden ratio. If \texttt{ON}, then the above setting is ignored.
                   \\
\texttt{AXESCOLOUR} & \textbf{Possible values:} Any recognised colour.

                   \textbf{Analogous set command:} \texttt{set axescolour}\index{set axescolour command@\texttt{set axescolour} command}

                   Sets the colour of axis lines and ticks.
                   \\
\texttt{BACKUP} & \textbf{Possible values:} ON / OFF

                   \textbf{Analogous set command:} \texttt{set backup}\index{set backup command@\texttt{set backup} command}

                   When this switch is set to `ON', and plot output is being directed to file, attempts to write output over existing files cause a copy of the existing file to be preserved, with a tilda after its old filename (see Section~\ref{filebackup}).
                   \\
\texttt{BAR}     & \textbf{Possible values:}  Any floating-point number.

                   \textbf{Analogous set command:} \texttt{set bar}\index{set bar command@\texttt{set bar} command}

                   Sets the horizontal length of the lines drawn at the end of errorbars, in units of their default length.
                   \\
\texttt{BOXFROM} & \textbf{Possible values:} Any floating-point number.

                   \textbf{Analogous set command:} \texttt{set boxfrom}\index{set boxfrom command@\texttt{set boxfrom} command}

                   Sets the horizontal point from which bars on bar charts appear to emanate.
                   \\
\texttt{BOXWIDTH} & \textbf{Possible values:} Any floating-point number.

                   \textbf{Analogous set command:} \texttt{set boxwidth}\index{set boxwidth command@\texttt{set boxwidth} command}

                   Sets the default width of boxes on barcharts. If negative, then the boxes have automatically selected widths, so that the interfaces between bars occur at the horizontal midpoints between the specified datapoints.
                   \\
\texttt{COLOUR} & \textbf{Possible values:} ON / OFF

                   \textbf{Analogous set command:} \texttt{set terminal}\index{set terminal command@\texttt{set terminal} command}

                   Sets whether output should be colour (ON) or monochrome (OFF).
                   \\
\texttt{DATASTYLE} & \textbf{Possible values:} Any plot style. 

                   \textbf{Analogous set command:} \texttt{set data style}\index{set data style command@\texttt{set data style} command}
                   
                   Sets the plot style used by default when plotting datafiles.
                   \\
\texttt{DISPLAY} & \textbf{Possible values:} ON / OFF

                   \textbf{Analogous set command:} \texttt{set display}\index{set display command@\texttt{set display} command}

                   When set to `ON', no output is produced until the \texttt{set display} command is issued. This is useful for speeding up scripts which produce large multiplots; see Section~\ref{set_display} for more details.
                   \\
\texttt{DPI} & \textbf{Possible values:} Any floating-point number.

                   \textbf{Analogous set command:} \texttt{set dpi}\index{set dpi command@\texttt{set dpi} command}

                   Sets the sampling quality used, in dots per inch, when output is sent to a bitmapped terminal (the jpeg/gif/png terminals).
                   \\
\texttt{ENLARGE} & \textbf{Possible values:} ON / OFF

                   \textbf{Analogous set command:} \texttt{set terminal}\index{set terminal command@\texttt{set terminal} command}
                   
                   When set to `ON' output is enlarged or shrunk to fit the
                   current paper size.
                   \\

\texttt{FONTSIZE} & \textbf{Possible values:} Integers in the range $-4 \to 5$.

                   \textbf{Analogous set command:} \texttt{set fontsize}\index{set fontsize command@\texttt{set fontsize} command}

                   Sets the fontsize of text, varying between \LaTeX's \texttt{tiny} (-4) and \texttt{Huge} (5).
                   \\
\texttt{FUNCSTYLE} & \textbf{Possible values:} Any plot style.

                   \textbf{Analogous set command:} \texttt{set function style}\index{set function style command@\texttt{set function style} command}

                   Sets the plot style used by default when plotting functions.
                   \\
\texttt{GRID} & \textbf{Possible values:} ON / OFF

                   \textbf{Analogous set command:} \texttt{set grid}\index{set grid command@\texttt{set grid} command}

                   Sets whether a grid should be displayed on plots.
                   \\
\texttt{GRIDAXISX} & \textbf{Possible values:} Any integer.

                   \textbf{Analogous set command:} None

                   Sets the default $x$-axis to which gridlines should attach, if the \texttt{set grid} command is called without specifying which axes to use.
                   \\
\texttt{GRIDAXISY} & \textbf{Possible values:} Any integer.

                   \textbf{Analogous set command:} None

                   Sets the default $y$-axis to which gridlines should attach, if the \texttt{set grid} command is called without specifying which axes to use.
                   \\
\texttt{GRIDMAJCOLOUR} & \textbf{Possible values:} Any recognised colour.

                   \textbf{Analogous set command:} \texttt{set gridmajcolour}\index{set gridmajcolour command@\texttt{set gridmajcolour} command}

                   Sets the colour of major grid lines.
                   \\
\texttt{GRIDMINCOLOUR} & \textbf{Possible values:} Any recognised colour.

                   \textbf{Analogous set command:} \texttt{set gridmincolour}\index{set gridmincolour command@\texttt{set gridmincolour} command}

                   Sets the colour of minor grid lines.
                   \\
\texttt{KEY} & \textbf{Possible values:} ON / OFF

                   \textbf{Analogous set command:} \texttt{set key}\index{set key command@\texttt{set key} command}

                   Sets whether a legend is displayed on plots.
                   \\
\texttt{KEYCOLUMNS} & \textbf{Possible values:} Any integer $>0$.

                   \textbf{Analogous set command:} \texttt{set keycolumns}\index{set keycolumnscommand@\texttt{set keycolumns} command}

                   Sets the number of columns into which the legends of plots should be divided.
                   \\
\texttt{KEYPOS} & \textbf{Possible values:} ``TOP RIGHT'', ``TOP MIDDLE'', ``TOP LEFT'', ``MIDDLE RIGHT'', ``MIDDLE MIDDLE'', ``MIDDLE LEFT'', ``BOTTOM RIGHT'', ``BOTTOM MIDDLE'', ``BOTTOM LEFT'', ``BELOW'', ``OUTSIDE''.

                   \textbf{Analogous set command:} \texttt{set key}\index{set key command@\texttt{set key} command}

                   Sets where the legend should appear on plots.
                   \\
\texttt{KEY\_XOFF} & \textbf{Possible values:} Any floating-point number.

                   \textbf{Analogous set command:} \texttt{set key}\index{set key command@\texttt{set key} command}

                   Sets the horizontal offset, in approximate graph-widths, that should be applied to the legend, relative to its default position, as set by \texttt{KEYPOS}.
                   \\
\texttt{KEY\_YOFF} & \textbf{Possible values:} Any floating-point number.

                   \textbf{Analogous set command:} \texttt{set key}\index{set key command@\texttt{set key} command}

                   Sets the vertical offset, in approximate graph-heights, that should be applied to the legend, relative to its default position, as set by \texttt{KEYPOS}.
                   \\
\texttt{LANDSCAPE} & \textbf{Possible values:} ON / OFF

                   \textbf{Analogous set command:} \texttt{set terminal}\index{set terminal command@\texttt{set terminal} command}

                   Sets whether output is in portrait orientation (OFF), or landscape orientation (ON).
                   \\
\texttt{LINEWIDTH} & \textbf{Possible values:} Any floating-point number.

                   \textbf{Analogous set command:} \texttt{set linewidth}\index{set linewidth command@\texttt{set linewidth} command}

                   Sets the width of lines on plots, as a  multiple of the default.
                   \\
\texttt{MULTIPLOT} & \textbf{Possible values:} ON / OFF

                   \textbf{Analogous set command:} \texttt{set multiplot}\index{set multiplot command@\texttt{set multiplot} command}

                   Sets whether multiplot mode is on or off.
                   \\
\texttt{ORIGINX} & \textbf{Possible values:} Any floating point number.

                   \textbf{Analogous set command:} \texttt{set origin}\index{set origin command@\texttt{set origin} command}

                   Sets the horizontal position, in centimetres, of the default origin of plots on the page. Most useful when multiplotting many plots.
                   \\
\texttt{ORIGINY} & \textbf{Possible values:} Any floating point number.

                   \textbf{Analogous set command:} \texttt{set origin}\index{set origin command@\texttt{set origin} command}

                   Sets the vertical position, in centimetres, of the default origin of plots on the page. Most useful when multiplotting many plots.
                   \\
\texttt{OUTPUT} & \textbf{Possible values:} Any string.

                   \textbf{Analogous set command:} \texttt{set output}\index{set output command@\texttt{set output} command}

                   Sets the output filename for plots. If blank, the default filename of pyxplot.foo is used, where `foo' is an extension appropriate for the file format.
                   \\
\texttt{PAPER\_HEIGHT} & \textbf{Possible values:} Any floating-point number.

                   \textbf{Analogous set command:} \texttt{set papersize}\index{set papersize command@\texttt{set papersize} command}

                   Sets the height of the papersize for postscript output in millimetres.
                   \\
\texttt{PAPER\_WIDTH} & \textbf{Possible values:} Any floating-point number.

                   \textbf{Analogous set command:} \texttt{set papersize}\index{set papersize command@\texttt{set papersize} command}

                   Sets the width of the papersize for postscript output in millimetres.
                   \\
\texttt{POINTLINEWIDTH} & \textbf{Possible values:} Any floating-point number.

                   \textbf{Analogous set command:} \texttt{set pointlinewidth} / \texttt{plot with pointlinewidth}\index{set pointlinewidth command@\texttt{set pointlinewidth} command}

                   Sets the linewidth used to stroke points onto plots, as a multiple of the default.
                   \\
\texttt{POINTSIZE} & \textbf{Possible values:} Any floating-point number.

                   \textbf{Analogous set command:} \texttt{set pointsize} / \texttt{plot with pointsize}\index{set pointsize command@\texttt{set pointsize} command}

                   Sets the sizes of points on plots, as a multiple of their normal sizes.
                   \\
\texttt{SAMPLES} & \textbf{Possible values:} Any integer.

                   \textbf{Analogous set command:} \texttt{set samples}\index{set samples command@\texttt{set samples} command}

                   Sets the number of samples (datapoints) to be evaluated along the $x$-axis when plotting a function.
                   \\
\texttt{TERMINVERT} & \textbf{Possible values:} ON / OFF

                   \textbf{Analogous set command:} \texttt{set terminal}\index{set terminal command@\texttt{set terminal} command}

                   Sets whether jpeg/gif/png output has normal colours (OFF), or inverted colours (ON).
                   \\
\texttt{TERMTRANSPARENT} & \textbf{Possible values:} ON / OFF

                   \textbf{Analogous set command:} \texttt{set terminal}\index{set terminal command@\texttt{set terminal} command}

                   Sets whether jpeg/gif/png output has transparent background (ON), or solid background (OFF).
                   \\
\texttt{TERMTYPE} & \textbf{Possible values:} \texttt{X11\_singlewindow},

                   \texttt{X11\_multiwindow}, \texttt{X11\_persist}, \texttt{PS}, \texttt{EPS}, \texttt{PDF}, \texttt{PNG}, \texttt{JPG}, \texttt{GIF}

                   \textbf{Analogous set command:} \texttt{set terminal}\index{set terminal command@\texttt{set terminal} command}

                   Sets whether output is sent to the screen or to disk, and, in the latter case, the format of the output. The \texttt{ps} option should be used for both encapsulated and normal postscript output; these are distinguished using the \texttt{ENHANCED} option, above.
                   \\
\texttt{TEXTCOLOUR} & \textbf{Possible values:} Any recognised colour.

                   \textbf{Analogous set command:} \texttt{set textcolour}\index{set textcolour command@\texttt{set textcolour} command}

                   Sets the colour of all text output.
                   \\
\texttt{TEXTHALIGN} & \textbf{Possible values:} \texttt{Left}, \texttt{Centre}, \texttt{Right}

                   \textbf{Analogous set command:} \texttt{set texthalign}\index{set texthalign command@\texttt{set texthalign} command}

                   Sets the horizontal alignment of text labels to their given reference positions.
                   \\
\texttt{TEXTVALIGN} & \textbf{Possible values:} \texttt{Top}, \texttt{Centre}, \texttt{Bottom}

                   \textbf{Analogous set command:} \texttt{set textvalign}\index{set textvalign command@\texttt{set textvalign} command}

                   Sets the vertical alignment of text labels to their given reference positions.
                   \\
\texttt{TITLE} & \textbf{Possible values:} Any string.

                   \textbf{Analogous set command:} \texttt{set title}\index{set title command@\texttt{set title} command}

                   Sets the title to appear at the top of the plot.
                   \\
\texttt{TIT\_XOFF} & \textbf{Possible values:} Any floating point number.

                   \textbf{Analogous set command:} \texttt{set title}

                   Sets the horizontal offset of the title of the plot from its default central location.
                   \\
\texttt{TIT\_YOFF} & \textbf{Possible values:} Any floating point number.

                   \textbf{Analogous set command:} \texttt{set title}

                   Sets the vertical offset of the title of the plot from its default location at the top of the plot.
                   \\
\texttt{WIDTH} & \textbf{Possible values:} Any floating-point number.

                   \textbf{Analogous set command:} \texttt{set width} / \texttt{set size}\index{set width command@\texttt{set width} command}\index{set size command@\texttt{set size} command}

                   Sets the width of plots in centimetres.
                   \\
\end{longtable}

\section{Configuration Options: \texttt{terminal} section}
\label{configfile_terminal}

The following table provides a brief description of the function of each of the
parameters in the \texttt{terminal} section of the above configuration file,
with a list of possible values for each:

\begin{longtable}{p{3.4cm}p{9cm}}
\texttt{COLOUR} & \textbf{Possible values:} ON / OFF

                  \textbf{Analogous commandline switches:} \texttt{-c}, \texttt{--colour}, \texttt{-m}, \texttt{--monochrome}

                  Sets whether colour highlighting should be used in the interactive terminal. If turned on, output is displayed in green, warning messages in amber, and error messages in red; these colours are configurable, as described below. Note that not all UNIX terminals support the use of colour.
                   \\
\texttt{COLOUR\_ERR} & \textbf{Possible values:} Any recognised terminal colour.

                  \textbf{Analogous commandline switches:} None.

                  Sets the colour in which error messages are displayed when colour highlighting is used. Note that the list of recognised colour names differs from that used in PyXPlot; a list is given at the end of this section.
                   \\
\texttt{COLOUR\_REP} & \textbf{Possible values:} Any recognised terminal colour.

                  \textbf{Analogous commandline switches:} None.

                  As above, but sets the colour in which PyXPlot displays its non-error-related output.
                   \\
\texttt{COLOUR\_WRN} & \textbf{Possible values:} Any recognised terminal colour.

                  \textbf{Analogous commandline switches:} None.

                  As above, but sets the colour in which PyXPlot displays its warning messages.
                   \\
\texttt{SPLASH} & \textbf{Possible values:} ON / OFF

                  \textbf{Analogous commandline switches:} \texttt{-q}, \texttt{--quiet}, \texttt{-V}, \texttt{--verbose}

                  Sets whether the standard welcome message is displayed upon startup.
                   \\
\end{longtable}

The colours recognised by the \texttt{COLOUR\_XXX} configuration options above are: \texttt{Red}, \texttt{Green}, \texttt{Brown}, \texttt{Blue}, \texttt{Purple}, \texttt{Magenta}, \texttt{Cyan}, \texttt{White}, \texttt{Normal}. The final option produces the default foreground colour of your terminal.

\section{Recognised Colour Names}
\label{colour_names}

The following is a complete list of the colour names which PyXPlot recognises in the \texttt{set textcolour}, \texttt{set axescolour} commands, and in the \texttt{colours} section of its configuration file. It should be noted that they are case-insensitive:

\index{configuration file!colours}\index{colours!configuration file}
GreenYellow, Yellow, Goldenrod, Dandelion, Apricot, Peach, Melon, YellowOrange, Orange, BurntOrange, Bittersweet, RedOrange, Mahogany, Maroon, BrickRed, Red, OrangeRed, RubineRed, WildStrawberry, Salmon, CarnationPink, Magenta, VioletRed, Rhodamine, Mulberry, RedViolet, Fuchsia, Lavender, Thistle, Orchid, DarkOrchid, Purple, Plum, Violet, RoyalPurple, BlueViolet, Periwinkle, CadetBlue, CornflowerBlue, MidnightBlue, NavyBlue, RoyalBlue, Blue, Cerulean, Cyan, ProcessBlue, SkyBlue, Turquoise, TealBlue, Aquamarine, BlueGreen, Emerald, JungleGreen, SeaGreen, Green, ForestGreen, PineGreen, LimeGreen, YellowGreen, SpringGreen, OliveGreen, RawSienna, Sepia, Brown, Tan, Gray, Grey, Black, White, white, black.

The following further colours provide a scale of shades of grey from dark to light, also case-insensitive:

\index{colours!shades of grey}
grey05, grey10, grey15, grey20, grey25, grey30, grey35, grey40, grey45, grey50, grey55, grey60, grey65, grey70, grey75, grey80, grey85, grey90, grey95.

The US mis-spelling of grey (``gray'') is also accepted.

For a colour chart of these colours, the reader is referred to Appendix B of the \textit{PyX Reference Manual}.\footnote{\url{http://pyx.sourceforge.net/manual/colorname.html}}

\chapter{Command Reference}

This chapter contains a list of all the commands that PyXPlot understands,
listed by alphabetical order.

\section{arrow}\index{arrow command@\texttt{arrow} command}

\begin{verbatim}
arrow [from] <x>, <y> [to] <x>, <y> [with <option> ... ]
\end{verbatim}

Arrows may be placed on multiplot pages independently of any plots using the
{\tt arrow} command, which has syntax:

\begin{verbatim}
arrow from x,y to x,y
\end{verbatim}

The arrow command may be followed by the `{\tt with}' keyword to specify to
style of the arrow. The style keywords which are accepted are `{\tt nohead}',
`{\tt head}' (default) or `{\tt twohead}', in addition to keywords such as `{\tt
colour}', `{\tt linewidth}' or `{\tt linetype}', which have the same syntax and
meaning as they do in the plot command. An example would be:

\begin{verbatim}
arrow from x,y to x,y with twohead linetype 2 colour blue
\end{verbatim}

Arrows receive unique identification numbers which count sequentially from one,
and which are output to the terminal after the arrow command is called. By
reference to these numbers, they can later be deleted and undeleted with the
{\tt delete} and {\tt undelete} commands respectively. For example:

\begin{verbatim}
delete 2
\end{verbatim}

\section{cd}\index{cd command@\texttt{cd} command}

\begin{verbatim}
cd <directory>
\end{verbatim}

PyXPlot's {\tt cd} command is very similar to the shell cd command; it can be used to
change the current working directory. For example:

\begin{verbatim}
cd foo
\end{verbatim}


\section{clear}\index{clear command@\texttt{clear} command}

\begin{verbatim}
clear
\end{verbatim}

In multiplot mode the {\tt clear} command removes all current plots, arrows and
text objects from the working page. In single plot mode it is not especially
useful; it removes the current plot to leave a blank page.

The {\tt clear} command should not be followed by any parameters.


\section{delete}\index{delete command@\texttt{delete} command}

\begin{verbatim}
delete <plot number>, ...
\end{verbatim}

The {\tt delete} command is part of the multiplot environment; it removes
plots, arrows or text items from a multiplot page. The desired items should be
identified using a comma-separated list of their reference numbers, which count
sequentially from zero for the first item created on a multiplot page, and are
displayed on the terminal when items are created.  For example:

\begin{verbatim}
delete 1,2,3
\end{verbatim}

\noindent removes item numbers 1, 2 and 3.

Having been deleted, multiplot items can be restored using the {\tt undelete}
command.

\section{edit}\index{edit command@\texttt{edit} command}

\begin{verbatim}
edit <plot number>
\end{verbatim}

The {\tt edit} command is part of the multiplot environment; it allows one to
modify the properties of any plot on a multiplot. The desired plot should be
identified using the reference number which it was given when it was created
using the {\tt plot} command; it would have been displayed on the terminal at
that time. For example, consider the following command sequence:

\begin{verbatim}
edit 1
set textcolour red
replot
\end{verbatim}

Here, the {\tt edit} command sets the following {\tt set textcolour} command to
affect the plot with reference number 1 -- the first plot which would have been
placed on the multiplot. The {\tt set textcolour red} command then affects this
plot, although does not take effect until the {\tt replot} command is called.

The {\tt edit} command also has the effect of resetting all of PyXPlot's plot
settings to those used to produce the chosen plot, and so in conjunction with
the {\tt show} command, can be used to inspect as well as modify the settings of
any plot on a multiplot page. For example:

\begin{verbatim}
edit 1
show textcolour
\end{verbatim}

\noindent would show the text colour used in plot 1.

Having issued the {\tt edit} command, no further command needs to be issued to
return to a state of adding plots to a multiplot rather than editing the
existing plots; simply call the {\tt plot} command rather than the {\tt replot}
command to do this.


\section{eps}\index{eps command@\texttt{eps} command}

\begin{verbatim}
eps '<filename>' [at <x>, <y>] [rotate <angle>] [width <width>] [height <height>]
\end{verbatim}

The {\tt eps } command places an eps format file into the current plot.  The
{\tt at} modifier can be used to specify the position of the bottom-left corner
of the file, otherwise it is placed at the origin.  If the {\tt rotate} modifier
is used the image is rotated by the specified angle counter-clockwise.  Either
the {\tt width} or {\tt height} can be specified, followed by the width or
height respectively in cm that the resulting image should be; otherwise the
native file width will be used.  The {\tt eps} command is perhaps most useful in
multiplot mode, where included files can be combined with plots, text labels,
etc.

\section{exit}\index{exit command@\texttt{exit} command}

\begin{verbatim}
exit
\end{verbatim}

The {\tt exit} command can be used to quit PyXPlot. If multiple command files,
or a mixture of command files and interactive sessions, are specified on the
commandline, then PyXPlot moves onto the next commandline item after receiving
the {\tt exit} command.

PyXPlot may also be quit be pressing CTRL-D or via the {\tt quit} command. In
interactive mode, CTRL-C terminates the current command, if one is running.
When running a script, CTRL-C terminates execution of it.


\section{fit}\index{fit command@\texttt{fit} command}

\begin{verbatim}
fit [<range specifier> ...] <function> '<datafile>'
    [index <index specifier>] [using <using specifier>]
    via <variable>[, <variable>, ...]
\end{verbatim}

The {\tt fit} command may be used to fit functional forms to data points in
datafiles. A simple example might be:

\begin{verbatim}
f(x) = a*x+b
fit f(x) 'datafile' index 1 using 2:3 via a,b
\end{verbatim}

The coefficients to be varied are listed after the keyword `{\tt via}'; the
keywords `{\tt index}', `{\tt every}' and `{\tt using}' have the same meanings as in
the {\tt plot} command.

This is useful for producing best-fit lines and also has applications for
estimating the gradients of datasets.  The syntax is essentially identical to
the used by gnuplot, though a few points are worth noting, which are outlined
in Section~\ref{fit_command}.

\section{help}\index{help command@\texttt{help} command}

\begin{verbatim}
help [<topic> [<sub-topic> ... ] ]
\end{verbatim}

The {\tt help} command provides an easily-navigable source of information which
is supplementary to that in this manual.  To obtain information on any
particular topic, type {\tt help} followed by the name of the topic. For
example:

\begin{verbatim}
help commands
\end{verbatim}

\noindent provides information on PyXPlot's commands. Some topics have subtopics; these
are listed at the end of each help page. To view them, add further words to the
end of your help request -- an example might be:

\begin{verbatim}
help commands help
\end{verbatim}

Information is arranged with general information about PyXPlot under the heading
{\tt about}, and information about PyXPlot's commands under {\tt commands}.
Information about the format that input datafiles should take can be found
under {\tt datafile}.  Other categories are self-explanatory.

To exit any help page, press the `{\tt q}' key.

\section{jpeg}\index{jpeg command@\texttt{jpeg} command}

\begin{verbatim}
jpeg '<filename>' [at <x>, <y>] [rotate <angle>] [width <width>] [height <height>]
\end{verbatim}

The {\tt jpeg} command places a jpeg format bitmap image into the current plot.
The {\tt at} modifier can be used to specify the position of the bottom-left
corner of the image, otherwise it is placed at the origin.  If the {\tt rotate}
modifier is used the image is rotated by the specified angle counter-clockwise.
Either the {\tt width} or {\tt height} modifier should be specified, followed by
the width or height respectively in cm that the resulting image should be.  The
{\tt jpeg} command is perhaps most useful in multiplot mode, where images can be
combined with plots, text labels, etc.

\section{load}\index{load command@\texttt{load} command}

\begin{verbatim}
load '<filename>'
\end{verbatim}

The {\tt load} command executes a PyXPlot command script file, just as if its
contents had been typed into the current terminal. For example:

\begin{verbatim}
load 'foo'
\end{verbatim}

would have the same effect as typing the contents of the file foo into the
present session.

Wildcards can be used in the load command, in which case \textit{all}
commandfiles matching the given wildcard are executed, for example:

\begin{verbatim}
load '*.script'
\end{verbatim}


\section{move}\index{move command@\texttt{move} command}

\begin{verbatim}
move <plot number> to <x>, <y>
\end{verbatim}

The {\tt move} command is part of the multiplot environment; it can be used to
move items around on a multiplot page. The desired item to be moved should be
identified using the reference number which it was given when it was created;
it would have been displayed on the terminal at that time. For example:

\begin{verbatim}
move 23 to 8,8
\end{verbatim}
  
If 23 were a plot, this would move it to position 8,8 (measured in
centimetres). The end result would be the same as if the command {\tt set origin
8,8} had been called before plotting plot number 23.

\section{!}\index{! command@! command}

\begin{verbatim}
! <shell command>
<command> `<shell command>` ...
\end{verbatim}

Shell commands can be executed from within PyXPlot by pre-fixing them with
pling (!) characters, for example:

\begin{verbatim}
!mkdir foo
\end{verbatim}

As an alternative, back-quotes (`) can be used to substitute the output of a
shell command into a PyXPlot command, for example:

\begin{verbatim}
set xlabel `echo "'" ; ls ; echo "'"`
\end{verbatim}

Note that back-quotes cannot be used inside quote characters, and so the
following would \textit{not} work:

\begin{verbatim}
set xlabel '`ls`'
\end{verbatim}


\section{plot}\index{plot command@\texttt{plot} command}

\begin{verbatim}
plot [<range specifier> ...] ('<filename>'|<function>)
     [using <using specifier>] [axes <axis specifier>]
     [select <select specifier>]
     [index <index specifier>]
     [every <every specifier>]
     [with <style> [<style modifier> ... ] ]
\end{verbatim}

The {\tt plot} command is the main workhorse command of PyXPlot, which is used
to produce all plots. For example to plot the sine function:

\begin{verbatim}
plot sin(x)
\end{verbatim}

Ranges for the axes of a graph can be specified by placing them in
square-brackets before the name of the function to be plotted. Leaving a set of
brackets empty specifies that an axis will be automatically scaled, as happens
by default. An example of this syntax would be:

\begin{verbatim}
plot [-pi:pi] sin(x)
\end{verbatim}

\noindent which would plot the function $\sin(x)$ across some default range of
values on the $x$-axis.

Datafiles may also be plotted as well as functions, in which case the filename
of the datafile to be plotted should be enclosing in apostrophes. An example of
this syntax would be:

\begin{verbatim}
plot 'datafile' with points
\end{verbatim}

\noindent which would plot the file called `{\tt datafile}'.  Section
\ref{plot_datafiles} should be studied for further details of the format that is
expected of input datafiles, and how PyXPlot may be directed to plot only
certain portions of datafiles.

In plots which have multiple parallel axes -- for example, an $x$-axis along its
lower edge and an $x2$-axis along its upper edge -- the pair of axes against
which data should be plotted should be specified using the modifier {\tt axes}
following the name of the function or datafile to be plotted, for example:

\begin{verbatim}
plot sin(x) axes x2y1
\end{verbatim}

The style in which data should be plotted may be specified following the
modifier {\tt with}, with the following syntax:

\begin{verbatim}
plot sin(x) with points
\end{verbatim}

The following plot styles are recognised: {\tt lines}, {\tt points}, {\tt
linespoints}, {\tt dots}, {\tt boxes}, {\tt wboxes}, {\tt impulses}, {\tt
steps}, {\tt histeps}, {\tt fsteps}, {\tt xerrorbars}, {\tt yerrorbars}, {\tt
xyerrorbars}, {\tt xerrorrange}, {\tt yerrorrange}, {\tt xyerrorrange}, {\tt
arrows\_head}, {\tt arrows\_nohead}, {\tt arrows\_twohead}, {\tt csplines}, {\tt
acsplines}.

In addition, {\tt lp} and {\tt pl} are recognised as abbreviations
for {\tt linespoints}; {\tt errorbars} is recognised as an abbreviation for {\tt
yerrorbars}; {\tt errorrange} is recognised as an abbreviation for {\tt
yerrorrange}; and {\tt arrows\_twoway} is recognised as an alternative for {\tt
arrows\_twohead}.

As well as plot styles, the {\tt with} modifier can also be followed by the
following keywords:

\begin{description}
\item[{\tt linetype}] -- specifies the linetype (e.g. dotted) used by the lines plot style. 
\item[{\tt linewidth}] -- specifies the width of line, in pt, used by the lines plot style.
\item[{\tt pointsize}] -- specifies the size of datapoints, relative to the
default size, used by the points plot style. 
\item[{\tt pointlinewidth}] -- as above, but specifies the linewidth, in pt,
used to render the crosses, circles, etc, used to mark datapoints. 
\item[{\tt linestyle}] -- this can be used in conjunction with the {\tt set linestyle} command to save default plot styles. 
\item[{\tt colour}] -- specifies the colour used to plot the dataset, either by
one of the recognised colour names or by an integer,
to use one from the current palette.  \item[{\tt fillcolour}] -- relavant to the
{\tt boxes} and {\tt wboxes} plot
styles, specifies a colour in which bar charts should be filled.
\end{description}

An example using several of these keywords would be:

\begin{verbatim}
plot sin(x) axes x2y1 with colour blue linetype 2 \
                           linewidth 5
\end{verbatim}

Multiple datasets can be plotted on a single graph by listing them with commas
separating them:

\begin{verbatim}
plot sin(x) with colour blue, cos(x) with linetype 2
\end{verbatim}


\section{print}\index{print command@\texttt{print} command}

\begin{verbatim}
print <expression>
\end{verbatim}

The {\tt print} command outputs the value of a mathematical expression to the
terminal.  It is most often used to find the value of a variable, though it can
also be used to produce formatted output from a PyXPlot script. For example:

\begin{verbatim}
print a
\end{verbatim}

\noindent would print the value of the variable $a$.


\section{pwd}\index{pwd command@\texttt{pwd} command}

\begin{verbatim}
pwd
\end{verbatim}

The {\tt pwd} command prints the location of the current working directory.


\section{?}\index{? command@\texttt{?} command}

\begin{verbatim}
? [<help option> ... ]
\end{verbatim}

The {\tt ?} symbol is a shortcut to the {\tt help} command.


\section{quit}\index{quit command@\texttt{quit} command}

\begin{verbatim}
quit
\end{verbatim}

The {\tt quit} command can be used to exit PyXPlot. If multiple command files,
or a mixture of command files and interactive sessions, are specified on the
commandline, then PyXPlot moves onto the next commandline item after receiving
the {\tt exit} command.

PyXPlot may also be quit be pressing CTRL-D or via the {\tt exit} command. In
interactive mode, CTRL-C terminates the current command, if one is running.
When running a script, CTRL-C terminates execution of it.


\section{refresh}\index{refresh command@\texttt{refresh} command}

\begin{verbatim}
refresh
\end{verbatim}

The {\tt refresh} command produces an exact copy of the latest display. This can
be useful, for example, after changing the terminal type, to produce a second
copy of a plot in a different graphic format. It differs from the {\tt replot}
command in that it doesn't replot anything; subsequent usages of the {\tt set}
command since the previous {\tt plot} command have no affect on the output. The
{\tt refresh} command is also especially useful in the multiplot environment; it
can be used to produce second copies of multiplot pages where there need not
necessarily even be any plots; there might perhaps only be textual items and
arrows.


\section{replot}\index{replot command@\texttt{replot} command}

\begin{verbatim}
replot [<plot number>]
\end{verbatim}

In single plot mode, the {\tt replot} command causes the most recent plot
command to be re-run.  This can be useful to replot a datafile which has changed
in the meantime, but also to change some aspect of a plot within PyXPlot itself.
Usages of the {\tt set} command between the original {\tt plot} command and the
calling of the {\tt replot} command are applied to the new plot. For example:

\begin{verbatim}
plot sin(x)
set textcolour red
replot
\end{verbatim}

In multiplot mode, the {\tt replot} command acts by default upon the last plot
item which was added to the multiplot page, and causes that to be replotted. It
is possible to change this behaviour by first calling the {\tt edit} command, in
which case any given plot within a multiplot can be modified and replotted.

Specifying a function or datafile after the {\tt replot} command causes that
function or data file to be added to the plot. The syntax here is the same as
for the {\tt plot} command.  For example:

\begin{verbatim}
replot sin(x) axes x2y1 with linespoints
\end{verbatim}

will add a plot of the function $\sin(x)$ to the current plot.


\section{reset}\index{reset command@\texttt{reset} command}

\begin{verbatim}
reset
\end{verbatim}

The {\tt reset} command returns the values of all settings that have been
changed with the {\tt set} command back to their default values.


\section{save}\index{save command@\texttt{save} command}

\begin{verbatim}
save '<filename>'
\end{verbatim}

The {\tt save} command saves a list of all of the commands which have been
executed in the current interactive PyXPlot session into a given file. The
filename of the desired location for this file should be placed in quotes, for
example:

\begin{verbatim}
save 'foo'
\end{verbatim}

\noindent would save a command history into the file named `{\tt foo}'.


\section{set}\index{set command@\texttt{set} command}

\begin{verbatim}
set <option> <value>
\end{verbatim}

The {\tt set} command sets the value of various operational parameters within
PyXPlot.  For example:

\begin{verbatim}
set pointsize 2
\end{verbatim}

would sets the default point size to 2. The basic syntax always follows that
above: the {\tt set} command should be followed by some keyword specifying which
setting it is which should be set. If a further parameter is needed to specify
what value to set this setting to, it should follow this keyword. Settings
which work in an on/off fashion tend to take a syntax along the lines of:

\begin{tabular}{ll}
{\tt set key} & Set option ON \\
{\tt set nokey} & Set option OFF
\end{tabular}

More details of the functions of each individual setting can be found in the
subsections below, which represents a complete list of the recognised setting
keywords.

The reader should also see the {\tt show} command, which can be used to display
the current values of settings, and the {\tt unset} command, which returns
settings to their default values. Section~\ref{config_files} describes how
commonly used settings can be saved into a configuration file.

\subsection{arrow}\index{set arrow command@\texttt{set arrow} command}

\begin{verbatim}
set arrow <arrow number> from [<co-ordinate>] <x>,
          [<co-ordinate>] <y> to [<co-ordinate>] <x>,
          [<co-ordinate>] <y> [with <modifier> ]
\end{verbatim}

\begin{verbatim}
<co-ordinate> = ( first | second | screen | graph |
                  axis<axisnumber>                  )
\end{verbatim}

The {\tt set arrow} command causes an arrow to be added to a plot. An example of
its syntax would be:

\begin{verbatim}
set arrow 1 from 0,0 to 1,1
\end{verbatim}

\noindent which would cause an arrow to be drawn between the points 0,0 and 1,1, as
measured on the $x$ and $y$ axes.  The tag `1' immediately following the {\tt
arrow} keyword is an identification number, and allows the arrow to be removed
later with the {\tt unset arrow} command.  By default the co-ordinates are
measured relative to the first $x$- and $y$-axes, but can be specified in a range
of coordinate systems. These are specified as follows:

\begin{verbatim}
set arrow 1 from first 0, second 0 to axis3 1, axis4 1
\end{verbatim}

As can be seen, the name of the desired coordinate system precedes the position
value in that coordinate system. The coordinate system {\tt first}, the default,
measures the graph using the $x$- and $y$-axes. {\tt second} uses the $x2$- and
$y2$-axes.  {\tt screen} and {\tt graph} both measure in centimetres from the
origin of the graph.  The syntax {\tt axisn} may also be
used, to use the $n$ th $x$- or $y$-axis; for example, {\tt axis3} above.

The {\tt set arrow} command can be followed by the keyword `{\tt with}', to
specify the style of the arrow. For example, the specifiers `{\tt nohead}', `{\tt
head}' and `{\tt twohead}', after the keyword `{\tt with}', can be used to make
arrows with no arrow heads, normal arrow heads, or two arrow heads. `{\tt twoway}'
is an alias for `{\tt twohead}'.  Normal line type modifiers can also be used
here.  For example:

\begin{verbatim}
set arrow 2 from first 0, second 2.5 to axis3 0,
             axis4 2.5 with colour blue nohead
\end{verbatim}



\subsection{autoscale}\index{set autoscale command@\texttt{set autoscale} command}

\begin{verbatim}
set autoscale <axis>[<axis>... ] 
\end{verbatim}

The {\tt autoscale} setting causes PyXPlot to choose the scaling for an axis
automatically based on the data and/or functions to be plotted against it. As
an example of the syntax:

\begin{verbatim}
set autoscale x1
\end{verbatim}

\noindent would cause the size of the first $x$-axis to be scaled to fit the
data.  Multiple axes can be specified, viz.:

\begin{verbatim}
set autoscale x1y3
\end{verbatim}


\subsection{axescolour}\index{set axescolour command@\texttt{set axescolour} command}

\begin{verbatim}
set axescolour <colour>
\end{verbatim}

The {\tt axescolour} setting changes the colour of the plot's axes.  For example:

\begin{verbatim}
set axescolour blue
\end{verbatim}

\noindent changes the axes to be blue. Any of the recognised colour names listed in
Section~\ref{colour_names} can be used.
 

\subsection{axis}\index{set axis command@\texttt{set axis} command}

\begin{verbatim}
set axis <axis>, ...
\end{verbatim}

The command:

\begin{verbatim}
set axis x2
\end{verbatim}

\noindent may be used to add a second $x$-axis to a plot, with default settings. In
general, there is no practical reason to use this command, as a second $x$-axis
would implicitly be created anyway by any of the following statements:

\begin{verbatim}
set x2label 'foo' \\
set x2ticdir outwards \\
plot sin(x) axes x2y1
\end{verbatim}

Of more practical use is the `{\tt unset x2}' command, which is used to remove an
axis once it has been added to a plot. After executing:

\begin{verbatim}
set x2label 'foo'
\end{verbatim}

\noindent for example, the only way to tell PyXPlot to subsequently produce a plot
without a second $x$-axis would be to delete this axis with the following
command:

\begin{verbatim}
unset axis x2
\end{verbatim}

Note that in this case, the {\tt unset x2label} command would be sufficent to
remove the label `foo' placed on the new axis, but not sufficient to delete the
new axis that the {\tt set x2label} command implicitly created. Multiple axes
can be deleted in a single {\tt unset axis} statement, for example:

\begin{verbatim}
unset axis x2x4x5
\end{verbatim}

In the special cases of {\tt unset axis x1} or {\tt unset axis y1}, these axes
cannot be deleted; a plot must have at least one $x$- and one $y$-axis. Instead,
the {\tt unset axis} command restores these axes to their default
configurations, removing any set titles or ranges that they might have been
given.

\subsection{backup}\index{set backup command@\texttt{set backup} command}

\begin{verbatim}
set backup
\end{verbatim}

The setting {\tt backup} changes PyXPlot's behaviour when it detects that a file
which it is about to write is going to overwrite an existing file. Whereas by
default the existing file would be overwritten by the new one, when the
{\tt backup} setting is turned on, it is renamed, placing a tilde at the end of
its filename. For example, suppose that a plot were to be written with filename
`{\tt out.ps}', but such a file already existed.  With the backup setting turned on
the existing file would be renamed `{\tt out.ps~}' to save it from being overwritten.

The setting may be turned off via {\tt set nobackup}.


\subsection{bar}\index{set bar command@\texttt{set bar} command}

\begin{verbatim}
set bar ( large | small | <barsize> )
\end{verbatim}

The {\tt bar} setting changes the size of the bar on the end of the errorbars,
relative to the current pointsize.  For example:

\begin{verbatim}
set bar 2
\end{verbatim}

\noindent sets the bars to be twice the size of the points.  The options `{\tt large}' and
`{\tt small}' are equivalent to 1 (the default) and 0 (no bar) respectively.


\subsection{boxfrom}\index{set boxfrom command@\texttt{set boxfrom} command}

\begin{verbatim}
set boxfrom <value>
\end{verbatim}

The `{\tt boxfrom}' setting alters PyXPlot's behaviour when plotting bar charts.
It changes the horizontal line (vertical point; $y$-axis value) from which the
boxes of bar charts appear to emanate.  The default value is zero (i.e. boxes
extend from the line of the $y$-axis). An example of its syntax would be:

\begin{verbatim}
set boxfrom 2
\end{verbatim}

\noindent which would make the boxes of a barchart emanate vertically from the line $y=2$.


\subsection{boxwidth}\index{set boxwidth command@\texttt{set boxwidth} command}

\begin{verbatim}
set boxwidth <width>
\end{verbatim}

The `{\tt boxwidth}' setting alters PyXPlot's behaviour when plotting bar charts.
It sets the default width of the boxes used, in graph $x$-axis units.  If the
specified width is negative then, as happens by default, the boxes have
automatically selected widths, such that the interfaces between them occur at
the horizontal midpoints between their specified $x$-positions.  For example:

\begin{verbatim}
set boxwidth 2
\end{verbatim}

\noindent would set all boxes to be two units wide.

\begin{verbatim}
set boxwidth -2
\end{verbatim}

\noindent would set all of the bars to have differing widths, centred upon their
specified $x$-positions, such that their interfaces occur at the horizontal
midpoints between them.


\subsection{data style}\index{set data style command@\texttt{set data style} command}

See `{\tt set style data}'.

\subsection{display}\index{set display command@\texttt{set display} command}

\begin{verbatim}
set [no]display
\end{verbatim}

By default, whenever an item is added to a multiplot, or an existing item moved
or replotted, the whole multiplot is replotted to show the change. This can be
a time consuming process on large and complex multiplots. For this reason, the
`{\tt set nodisplay}' command is provided, which stops PyXPlot from producing any
output. The `{\tt set display}' command can subsequently be issued to return to
normal behaviour.

This can be especially useful in scripts which produce large multiplots. There
is no point in producing output at each step in the construction of a large
multiplot, and so a great speed increase can be achieved by wrapping the script
with:

\begin{verbatim}
set nodisplay 
[...prepare large multiplot...] 
set display 
refresh
\end{verbatim}


\subsection{dpi}\index{set dpi command@\texttt{set dpi} command}

\begin{verbatim}
set dpi <value>
\end{verbatim}

When PyXPlot is set to produce bitmapped graphics output, using the {\tt gif},
{\tt jpg} or {\tt png} terminals (see the `{\tt set terminal}' command), the `{\tt
dpi}' setting changes how many dots per inch these graphics files are produced
with. That is to say, it changes the image resolution of these file formats:

\begin{verbatim}
set dpi 100
\end{verbatim}

\noindent sets the output to a resolution of 100 dots per inch. Higher dpi
values yield better quality images, but larger file sizes.

\subsection{fontsize}\index{set fontsize command@\texttt{set fontsize} command}

\begin{verbatim}
set fontsize <value>
\end{verbatim}

The {\tt fontsize} setting changes the size of the fount\footnote{This is not a
spelling mistake. `font', by contrast, \textit{would} be a spelling mistake. See the
Oxford English Dictionary.} used to render all text labels which appear on a
plot, including keys, axis labels, etc. The value specified should be an integer
in the range -4 to 5, corresponding to \LaTeX's tiny (-4) and Huge (5) sizes,
for example:

\begin{verbatim}
set fontsize 2
\end{verbatim}

The default value is zero, \LaTeX's normal fount size. As an alternative, fount
sizes can be specified directly in the \LaTeX text of labels, for example:

\begin{verbatim}
set xlabel '\Large This is a BIG label'
\end{verbatim}

\subsection{function style}\index{set function style command@\texttt{set function style} command}

See `{\tt set style function}'.

\subsection{grid}\index{set grid command@\texttt{set grid} command}

\begin{verbatim}
set [no]grid <axis> ...
\end{verbatim}

The {\tt grid} setting controls whether a grid is placed behind a plot or not.
Issuing the command:

\begin{verbatim}
set grid
\end{verbatim}

\noindent would cause a grid to be drawn with its gridlines connecting to the ticks of
the default axes (usually the first $x$- and $y$-axes). Conversely, issuing:

\begin{verbatim}
unset grid
\end{verbatim}

\noindent would remove from the plot all gridlines associated with the ticks of any axes.
One or more axes can be specified for the {\tt set grid} command; a grid will
then be drawn to connect with the ticks of these axes. An example of this syntax
would be:

\begin{verbatim}
set grid x1 y3
\end{verbatim}

\noindent which would cause gridlines to be drawn from ticks of the first $x$- and third
$y$-axes.

It is possible, though not always aesthetically very pleasing, to draw
gridlines from multiple parallel axes, for example:

\begin{verbatim}
set grid x1x2x3
\end{verbatim}


\subsection{gridmajcolour}\index{set gridmajcolour command@\texttt{set gridmajcolour} command}

\begin{verbatim}
set gridmajcolour <colour>
\end{verbatim}

The `{\tt gridmajcolour}' setting changes the colour that is used to plot the
gridlines (see the {\tt set grid} command) which are associated with the major
ticks of axes (i.e. major gridlines). For example:

\begin{verbatim}
set gridmajcolour purple
\end{verbatim}

\noindent would cause the major grid lines to be drawn in purple. Any of the recognised
colour names listed in Section~\ref{colour_names} can be used.

See also the {\tt set gridmincolour} command.


\subsection{gridmincolour}\index{set gridmincolour command@\texttt{set gridmincolour} command}

\begin{verbatim}
set gridmincolour <colour>
\end{verbatim}

The {\tt gridmincolour} setting changes the colour that is used to plot the
gridlines (see the {\tt set grid} command) which are associated with the minor
ticks of axes (i.e. minor gridlines). For example:

\begin{verbatim}
set gridmincolour purple
\end{verbatim}

\noindent would cause the minor grid lines to be drawn in purple. Any of the recognised
colour names listed in Section~\ref{colour_names} can be used.

\begin{verbatim}
See also the set gridmajcolour command.
\end{verbatim}


\subsection{key}\index{set key command@\texttt{set key} command}

\begin{verbatim}
set key [ <position> ... ] [<xoffset>, <yoffset>]
\end{verbatim}

The setting `{\tt key}' determines whether a legend is placed on a plot, and if
so, where it should be located on the plot. Issuing the command:

\begin{verbatim}
set key
\end{verbatim}

\noindent simply causes a legend to be added to the plot in its default position, usually
the plot's upper-right corner. The converse action is achieved by:

\begin{verbatim}
set nokey
\end{verbatim}

\noindent or:

\begin{verbatim}
unset key
\end{verbatim}

\noindent both of which cause a plot to have no legend. A position for the key may also
be specified after the {\tt set key} command, for example:

\begin{verbatim}
set key bottom left
\end{verbatim}

Recognised positions are `{\tt top}', `{\tt bottom}', `{\tt left}', `{\tt right}', `{\tt
below}', `{\tt outside}', `{\tt xcentre}' and `{\tt ycentre}'. In addition, if none of
these quite achieved the desired result, a positional offset may be specified
after one of the position keywords above.  The first value is assumed to be an
$x$-offset, and the second a $y$-offset, in units approximately equal to the
size of the plot. For example:

\begin{verbatim}
set key bottom left 0.0 -0.5
\end{verbatim}

\noindent would display a key below the bottom left corner of the graph.


\subsection{keycolumns}\index{set keycolumns command@\texttt{set keycolumns} command}

\begin{verbatim}
set keycolumns <value>
\end{verbatim}

The `{\tt keycolumns}' settings sets how many columns the legend of a plot should
be arranged into. By default, all of the entries in the legends of plots are
arranged in a single vertical list. However, for plots with very large number
of datasets, it may be preferably to split this list into several columns. The
{\tt set keycolumns} command can be followed by any positive integer, for
example:

\begin{verbatim}
set keycolumns 3
\end{verbatim}


\subsection{label}\index{set label command@\texttt{set label} command}

\begin{verbatim}
set label <label number> '<text>' [<co-ordinate>] <x>,
                                  [<co-ordinate>] <y>
                                  [rotate <angle>]
\end{verbatim}

\begin{verbatim}
<co-ordinate> = ( first | second | screen | graph |
                  axis<axisnumber>                  )
\end{verbatim}

The {\tt set label} command can be used to place text labels onto a plot.  For
example:

\begin{verbatim}
set label 1 'Hello' 0, 0
\end{verbatim}

\noindent would place the word `Hello' at plot co-ordinates (0,0), as measured on the $x$-
and $y$-axes.  The tag `{\tt 1}' immediately following the `{\tt label}' keyword is an
identification number, and allows the label to be removed later with the {\tt
unset label} command.  By default the position coordinates of the label are
measured relative to the first $x$- and $y$-axes, but can be specified in a
range of coordinate systems. These are specified as follows:

\begin{verbatim}
set label 1 'Hello' first 0, second 0
\end{verbatim}

As can be seen, the name of the desired coordinate system precedes the position
value in that coordinate system. Following gnuplot's nomenclature, the
coordinate system {\tt first} the default, measures the graph using the $x$- and
$y$-axes. {\tt second} uses the $x2$- and $y2$-axes.  {\tt screen} and {\tt
graph} both measure in centimetres from the origin of the graph.  The syntax
{\tt axisn} may also be used, to use the $n$ th $x$- or $y$-axis; for example,
{\tt axis3}:

\begin{verbatim}
set label 1 'Hello' axis3 1, axis4 1
\end{verbatim}

A rotation angle may optionally be specified after the keyword `{\tt rotate}'
to produce text rotated to any arbitrary angle, measured in degrees
counter-clockwise. The following example would produce upward-running text:

\begin{verbatim}
set label 1 'Hello' 1.2, 2.5 rotate 90
\end{verbatim}


\subsection{linestyle}\index{set linestyle command@\texttt{set linestyle} command}

\begin{verbatim}
set linestyle <style number> <style specifier> ...
\end{verbatim}

At times, the string of style keywords following the `{\tt with}' modifier in plot
commands can grow rather unwieldily long. For clarity, frequently used plot
styles can be stored as {\tt linestyles}; this is true of styles involving
points as well as lines. The syntax for setting a linestyle is:

\begin{verbatim}
set linestyle 2 points pointtype 3
\end{verbatim}

\noindent where the `{\tt 2}' is the identification number of the linestyle. In a subsequent
plot statement, this linestyle can be recalled as follows:

\begin{verbatim}
plot sin(x) with linestyle 2
\end{verbatim}


\subsection{linewidth}\index{set linewidth command@\texttt{set linewidth} command}

\begin{verbatim}
set linewidth <value>
\end{verbatim}

Sets the default linewidth, in units of pt, of the lines used to plot datasets
onto graphs with the `{\tt lines}' plot style (see the {\tt plot} command for
details of plot styles), for example in the following statement:

\begin{verbatim}
plot sin(x) with lines
\end{verbatim}

The linewidths of individual datasets can be set as follows; the {\tt set
linewidth} setting only affects plot statements where no linewidth is manually
specified:

\begin{verbatim}
plot sin(x) with lines linewidth 5.0
\end{verbatim}


\subsection{logscale}\index{set logscale command@\texttt{set logscale} command}

\begin{verbatim}
set logscale [<axis> ... ]
\end{verbatim}

The `{\tt logscale}' setting causes an axis to be laid out with logarithmically,
rather than linearly, spaced intervals.  For example, issuing the command:

\begin{verbatim}
set log
\end{verbatim}

\noindent would cause all of the axes of a plot to be scaled logarithmically. Alternatively
only one, or a selection of axes, can be set to scale logarithmically as
follows:

\begin{verbatim}
set log x1 y2
\end{verbatim}

This would cause the first $x$- and second $y$-axes to be scaled logarithmically.
Linear scaling can be restored to all axes via:

\begin{verbatim}
set nolog
\end{verbatim}

\noindent or:

\begin{verbatim}
unset log
\end{verbatim}

\noindent and to only one, or a selection of axes, via:

\begin{verbatim}
set nolog x1 y2
\end{verbatim}

\noindent or:

\begin{verbatim}
unset log x1y2
\end{verbatim}
   

\subsection{multiplot}\index{set multiplot command@\texttt{set multiplot} command}

\begin{verbatim}
set multiplot
\end{verbatim}

Issuing the command:

\begin{verbatim}
set multiplot
\end{verbatim}

\noindent causes PyXPlot to enter multiplot mode, which allows many graphs to
be plotted together and displayed side-by-side. See Section~\ref{multiplot} for
a full discussion of multiplot mode.


\subsection{mxtics}\index{set mxtics command@\texttt{set mxtics} command}

See {\tt set xtics}.
   

\subsection{mytics}\index{set mytics command@\texttt{set mytics} command}

See {\tt set xtics}.
   

\subsection{noarrow}\index{set noarrow command@\texttt{set noarrow} command}

\begin{verbatim}
set noarrow [<arrow number>]
\end{verbatim}

Issuing the command:

\begin{verbatim}
set noarrow
\end{verbatim}

\noindent removes all arrows, as set using the {\tt set arrow} command, from the current
plot. Alternatively, individual arrows can be removed using the syntax:

\begin{verbatim}
set noarrow 2
\end{verbatim}

where the tag `{\tt 2}' here is the identification number given to the arrow to be
removed when it was initially set using the {\tt set arrow} command.


\subsection{noaxis}\index{set noaxis command@\texttt{set noaxis} command}

\begin{verbatim}
set noaxis <axis specification>, ...
\end{verbatim}

The {\tt set noaxis} command is equivalent to the {\tt unset axis} command. It
should be followed by a comma-separated lists of axes, which are to be removed
from the current axis configuration.


\subsection{nobackup}\index{set nobackup command@\texttt{set nobackup} command}

See {\tt backup}.


\subsection{nodisplay}\index{set nodisplay command@\texttt{set nodisplay} command}

See {\tt display}.


\subsection{nogrid}\index{set nogrid command@\texttt{set nogrid} command}

\begin{verbatim}
set norgrid [<axis> ... ]
\end{verbatim}

Issuing the command {\tt set nogrid} removes gridlines from the current plot. On
its own, the command removes all gridlines from the plot, but alternatively,
those gridlines connected to the ticks of certain axes can selectively be
removed.  The syntax for doing this is as follows:

\begin{verbatim}
set nogrid x1 y2
\end{verbatim}


\subsection{nokey}\index{set nokey command@\texttt{set nokey} command}

\begin{verbatim}
set nokey
\end{verbatim}

Issuing the command {\tt set nokey} causes plots to be generated with no legend.
See the command {\tt set key} for more details.


\subsection{nolabel}\index{set nolabel command@\texttt{set nolabel} command}

\begin{verbatim}
set nolabel [<label number> ... ]
\end{verbatim}

Issuing the command:

\begin{verbatim}
set nolabel
\end{verbatim}

removes all text labels, as set using the {\tt set label} command, from the
current plot. Alternatively, individual labels can be removed using the syntax:

\begin{verbatim}
set nolabel 2
\end{verbatim}

where the tag `{\tt 2}' here is the identification number given to the label to be 
removed when it was initially set using the {\tt set label} command.


\subsection{nolinestyle}\index{set nolinestyle command@\texttt{set nolinestyle} command}

\begin{verbatim}
set nolinestyle <style number>
\end{verbatim}

The {\tt nolinestyle} setting deletes a line style. For example, the command:

\begin{verbatim}
set nolinestyle 3
\end{verbatim}

\noindent would delete the third linestyle, if defined. See the command {\tt set
linestyle} for more details.


\subsection{nologscale}\index{set nologscale command@\texttt{set nologscale} command}

\begin{verbatim}
set nologscale [<axis> ... ]
\end{verbatim}

The {\tt logscale} setting causes an axis to be laid out with logarithmically,
rather than linearly, spaced intervals. Conversely, the {\tt nologscale} setting
is used to restore linear scaling. For example, issuing the command:

\begin{verbatim}
set nolog 
\end{verbatim}

\noindent would cause all of the axes of a plot to be scaled linearly. Alternatively only one,
or a selection of axes, can be set to scale linearly as follows:

\begin{verbatim}
set nologscale x1 y2
\end{verbatim}

This would cause the first $x$- and second $y$-axes to be scaled linearly.


\subsection{nomultiplot}\index{set nomultiplot command@\texttt{set nomultiplot} command}

\begin{verbatim}
set nomultiplot
\end{verbatim}

Issuing the command {\tt set nomultiplot} places PyXPlot into single plotting
mode.  See above for a detailed discussion of PyXPlot's multiplot and
single plot modes. Broadly speaking, single plot mode is used to produce single
graphs on their own; multiplot mode is used to produce galleries of many plots
side-by-side.


\subsection{notitle}\index{set notitle command@\texttt{set notitle} command}

\begin{verbatim}
set notitle
\end{verbatim}

Issuing the command {\tt set notitle} will cause graphs to be produced with no
title at the top.


\subsection{noxtics}\index{set noxtics command@\texttt{set noxtics} command}

\begin{verbatim}
set no<axis sepcification>tics
\end{verbatim}

This command causes graphs to be produced with no tick marks along their $x$-axes.

\subsection{noytics}\index{set noytics command@\texttt{set noytics} command}

See {\tt set noxtics}.


\subsection{origin}\index{set origin command@\texttt{set origin} command}

\begin{verbatim}
set origin <x>, <y>
\end{verbatim}

The `{\tt origin}' setting controls the default location of graphs on a multiplot.
For example, the command:

\begin{verbatim}
set origin 3,5
\end{verbatim}

\noindent would cause the next graph to be plotted at position $(3,5)$ centimetres on the
multiplot page. The {\tt set origin} command is of little use outside multiplot
mode.


\subsection{output}\index{set output command@\texttt{set output} command}

\begin{verbatim}
set output '<filename>'
\end{verbatim}

The {\tt output} setting controls the name of the file that is produced for
non-interactive terminals ({\tt postscript}, {\tt eps}, {\tt jpeg}, {\tt gif}
and {\tt png}).  For example:

\begin{verbatim}
set output 'badger.eps'
\end{verbatim}

\noindent causes the output to be written to the file `{\tt badger.eps}'.


\subsection{palette}\index{set palette command@\texttt{set palette} command}

\begin{verbatim}
set palette <colour>, [<colour> ... ]
\end{verbatim}

PyXPlot has a palette of colours which it assigns sequentially to datasets when
colours are not manually assigned. This is also the palette to which is
referred if the user issues a command such as:

\begin{verbatim}
plot sin(x) with colour 5
\end{verbatim}

\noindent requesting the fifth colour from the palette. By default, this palette contains
a range of distinctive colours, however the user can choose to substitute his
own list of colours for these using the {\tt set palette} command. It should be
followed by a comma-separated list of colour names, for example:

\begin{verbatim}
set palette red,green,blue
\end{verbatim}

If, after issuing this command, the following plot statement were to be
executed:

\begin{verbatim}
plot sin(x), cos(x), tan(x), exp(x)
\end{verbatim}

\noindent the first function would be plotted in red, the second in green, and the third
in blue. Upon reaching the fourth, the palette would cycle back to red.

Any of the recognised colour names listed in Section~\ref{colour_names} can be used.

\subsection{papersize}\index{set papersize command@\texttt{set papersize} command}

\begin{verbatim}
set papersize (size|<x>, <y>)
\end{verbatim}

The {\tt papersize} option sets the size of output produced by the postscript
terminal. This can take the form of either a recognised papersize name -- a
list of these is given below -- or a height, width pair of values, both measured
in millimetres. For example:

\begin{verbatim}
set papersize a4
set papersize letter
set papersize 200,100
\end{verbatim}

A list of recognised papersizes can be found in Figure~\ref{paper_sizes}.

\subsection{pointlinewidth}\index{set pointlinewidth command@\texttt{set pointlinewidth} command}

\begin{verbatim}
set pointlinewidth <value>
\end{verbatim}

The `{\tt pointlinewidth}' setting changes the width of the lines that are used to
plot datapoints.  For instance:

\begin{verbatim}
set pointlinewidth 20
\end{verbatim}

\noindent would cause points to be plotted with lines 20 times the default thickness.

Note that `{\tt pointlinewidth}' can be abbreviated as `{\tt plw}'.

\subsection{pointsize}\index{set pointsize command@\texttt{set pointsize} command}

\begin{verbatim}
set pointsize <value>
\end{verbatim}

The `{\tt pointsize}' setting changes the size at which points are plotted
relative to their default size. It should be followed by a single value, the
relative size, which can be any positive number. For example:

\begin{verbatim}
set pointsize 1.5
\end{verbatim}

\noindent would cause points to be plotted 1.5 times the default size.

\subsection{preamble}\index{set preamble command@\texttt{set preamble} command}

The {\tt premble} setting changes the preamble that is prepended to each item of
text rendered using \LaTeX{}.  This allows, for example, different packages to
be loaded by default and user-defined macros to be set up.

\subsection{samples}\index{set samples command@\texttt{set samples} command}

The {\tt samples} setting determines the number of values along the $x$-axis at
which functions are evaluated when they are plotted. For example:

\begin{verbatim}
set samples 100
\end{verbatim}

\noindent causes 100 points to be evaluated.  Increasing this value will cause functions
to be plotted more smoothly, but also more slowly, and the resulting postscript
files generated will be correspondingly larger.

When functions are plotted with the {\tt points} plot style, this also affects
the number of points plotted.


\subsection{size}\index{set size command@\texttt{set size} command}

\begin{verbatim}
set size (<width>|ratio <ratio>|noratio|square)
\end{verbatim}

The setting {\tt size} is deprecated; use {\tt set width} instead.  It sets the
width of the plot in centimetres. However, the command {\tt set size}, when
followed by the keyword {\tt ratio}, is still used to set the aspect ratio of
plots. See the `{\tt ratio}' setting below for details.

\subsubsection{noratio}\index{set size command!noratio modifier@\texttt{noratio} modifier}

\begin{verbatim}
set size noratio
\end{verbatim}

Running:

\begin{verbatim}
set size noratio
\end{verbatim}

\noindent resets PyXPlot to produce plots with its default aspect ratio, which is the
golden section. Other aspect ratios can be set with the {\tt set size ratio}
command.


\subsubsection{ratio}\index{set size command!ratio modifier@\texttt{ratio} modifier}

\begin{verbatim}
set size ratio <ratio>
\end{verbatim}

The command:

\begin{verbatim}
set size ratio x
\end{verbatim}

\noindent sets the aspect ratio of plots produced by PyXPlot.  The height of resulting
plots will equal the plot width, as set by the {\tt set width} command,
muliplied by this aspect ratio. The value $x$ in the above statement can be
substituted with any positive value, for example:

\begin{verbatim}
set size ratio 2.0
\end{verbatim}

\noindent would cause PyXPlot to produce plots that are twice as high as they are wide.

The default aspect ratio which PyXPlot uses is a golden ratio of
$2/(1+\sqrt{5})$, which matches that of a sheet of A4 paper.


\subsubsection{square}\index{set size command!square modifier@\texttt{square} modifier}

\begin{verbatim}
set size square
\end{verbatim}

The command:

\begin{verbatim}
set size square
\end{verbatim}

\noindent sets PyXPlot to produce square plots, i.e. with unit aspect ratio. Other aspect
ratios can be set with the {\tt set size ratio} command.

\subsection{style}\index{set style command@\texttt{set style} command}

\begin{verbatim}
set style { data | function } <style modifier> ...
\end{verbatim}

The {\tt set style data} command affects the default style that data from a file
is plotted with.  Likewise the {\tt set style function} command changes the
default style that functions are plotted with.  Any valid style modifier can be
used.  For example:

\begin{verbatim}
set style data points
set style function lines linestyle 1
\end{verbatim}

would cause datafiles to be plotted by default using points and functions using
lines with the first defined linestyle.
 
\subsection{terminal}\index{set terminal command@\texttt{set terminal} command}

\begin{verbatim}
set terminal <terminal type> [<option> ... ]
\end{verbatim}

Syntax:

\begin{verbatim}
set terminal { X11_singlewindow | X11_multiwindow | X11_persist | 
               postscript | eps | pdf | gif | png | jpg } 
             { colour | color | monochrome } 
             { portrait | landscape } 
             { invert | noinvert } 
             { transparent | solid }
             { enlarge | noenlarge }
\end{verbatim}

The {\tt set terminal} command controls the graphic format in which PyXPlot
should output plots, for example setting whether it should output plots to files
or display them in a window on the screen. Various options can also be set
within many of the graphic formats which PyXPlot supports using this command.

The following graphic formats are supported:  {\tt X11\_singlewindow}, {\tt
X11\_multiwindow}, {\tt X11\_persist}, {\tt postscript}, {\tt eps}, {\tt pdf},
{\tt gif}, {\tt jpeg}, {\tt png}. To select one of these formats, simply type
the name of the desired format after the {\tt set terminal} command. To obtain
more details on each, see the subtopics below.

The following settings, which can also be typed following the {\tt set terminal}
command, are used to change the options within some of these graphic formats:
{\tt colour}, {\tt monochrome}, {\tt enhanced}, {\tt noenhanced}, {\tt
portrait}, {\tt landscape}, {\tt invert}, {\tt noinvert}, {\tt transparent},
{\tt solid}, {\tt enlarge}, {\tt noenlarge}. Details of each of these can be
found below.

\subsubsection{colour}\index{set terminal command!colour modifier@\texttt{colour} modifier}

The {\tt colour} terminal option causes plots to be produced in colour.

\subsubsection{color}\index{set terminal command!color modifier@\texttt{color} modifier}

The {\tt color} terminal option is provided for the convenience of users unable
to spell {\tt colour}.

\subsubsection{enlarge}\index{set terminal command!enlarge modifier@\texttt{enlarge} modifier}

The {\tt enlarge} terminal option causes the complete plot to be enlarged or
shrunk to fit the current paper size.

\subsubsection{eps}\index{set terminal command!eps modifier@\texttt{eps} modifier}

\begin{verbatim}
set terminal eps [<option> ... ]
\end{verbatim}

Sends output to eps files.  The filename to which output is to be sent should be
set using the {\tt set output} command; the default is `\texttt{pyxplot.eps}'.  This
terminal produces encapsulated postscript suitable for including in, for
example, \LaTeX documents.


\subsubsection{gif}\index{set terminal command!gif modifier@\texttt{gif} modifier}

\begin{verbatim}
set terminal gif [<option> ... ]
\end{verbatim}

The {\tt gif} terminal renders output as gif files. The filename to which output
is to be sent should be set using the {\tt set output} command; the default is
{\tt pyxplot.gif}. The number of dots per inch used can be changed using the dpi
option; the filename using {\tt set output}. Transparent gifs can be produced
with the {\tt transparent} option. Also of relevance is the {\tt invert} option
for producing gifs with inverted colours.


\subsubsection{invert}\index{set terminal command!invert modifier@\texttt{invert} modifier}

The {\tt invert} terminal option causes the bitmap terminals ({\tt gif}, {\tt
jpeg}, {\tt png}) to produce output with inverted colours. Useful for producing
plots for slideshows, where bright colours on a dark background may be desired.


\subsubsection{jpeg}\index{set terminal command!jpeg modifier@\texttt{jpeg} modifier}

\begin{verbatim}
set terminal jpeg [<option> ... ]
\end{verbatim}

The {\tt jpeg} terminal renders output as jpeg files. The filename to which
output is to be sent should be set using the {\tt set output} command; the
default is {\tt pyxplot.jpg}.  The number of dots per inch used can be changed
using the dpi option. Of relevance is the {\tt invert} option for producing
jpegs with inverted colours.

\subsubsection{landscape}\index{set terminal command!landscape modifier@\texttt{landscape} modifier}

The {\tt landscape} terminal option causes PyXPlot's output to be displayed in
rotated orientation.  Useful for printing as you get more on your sheet of
paper that way around; probably less useful for plotting things on screen.

\subsubsection{monochrome}\index{set terminal command!monochrome modifier@\texttt{monochrome} modifier}

The {\tt monochrome} terminal option causes plots to be rendered in black and
white; by default, different dash styles are used to differentiate between
lines on plots with several datasets.

\subsubsection{noenlarge}\index{set terminal command!noenlarge modifier@\texttt{noenlarge} modifier}

The {\tt noenlarge} terminal option causes the output not to be scaled (the
opposite of {\tt enlarge} above).

\subsubsection{noinvert}\index{set terminal command!noinvert modifier@\texttt{noinvert} modifier}

The {\tt noinvert} terminal option causes the bitmap terminals ({\tt gif}, {\tt
jpeg}, {\tt png}) to produce normal output without inverted colours. The
converse of {\tt inverse}.


\subsubsection{pdf}\index{set terminal command!pdf modifier@\texttt{pdf}
modifier}

\begin{verbatim}
set terminal pdf [<option> ... ]
\end{verbatim}

The {\tt pdf} terminal options causes pdf format output files to be produced.

\subsubsection{png}\index{set terminal command!png modifier@\texttt{png} modifier}

\begin{verbatim}
set terminal png [<option> ... ]
\end{verbatim}

The {\tt png} terminal renders output as png files. The filename to which output
is to be sent should be set using the {\tt set output} command; the default is
{\tt pyxplot.png}. The number of dots per inch used can be changed using the dpi
option; the filename using {\tt set output}. Transparent pngs can be produced
with the {\tt transparent} option. Also of relevance is the {\tt invert} option
for producing pngs with inverted colours.


\subsubsection{portrait}\index{set terminal command!portrait modifier@\texttt{portrait} modifier}

The {\tt portrait} terminal option causes PyXPlot's output to be displayed in
upright (normal) orientation.
 

\subsubsection{postscript}\index{set terminal command!postscript modifier@\texttt{postscript} modifier}

\begin{verbatim}
set terminal postscript [<option> ... ]
\end{verbatim}

Sends output to postscript files. The filename to which output is to be sent
should be set using the {\tt set output} command; the default is {\tt
pyxplot.ps}.  This terminal produces non-encapsulated postscript suitable for
sending directly to a printer.

\subsubsection{solid}\index{set terminal command!solid modifier@\texttt{solid} modifier}

The solid option causes the {\tt gif} and {\tt png} terminals to produce output
with a non-transparent background. The converse of {\tt transparent}.


\subsubsection{transparent}\index{set terminal command!transparent modifier@\texttt{transparent} modifier}

The {\tt transparent} terminal option causes the {\tt gif} and {\tt png}
terminals to produce output with a transparent background.


\subsubsection{X11\_multiwindow}\index{set terminal command!X11\_multiwindow modifier@\texttt{X11\_multiwindow} modifier}

Displays plots on the screen (in X11 windows, using ghostview). Each time a new
plot is generated it appears in a new window, and the old plots remain visible.
As many plots as may be desired can be left on the desktop simultaneously.

\subsubsection{X11\_persist}\index{set terminal command!X11\_persist
modifier@\texttt{X11\_persist} modifier}

Displays plots on the screen in X11 windows, using ghostview.  Each time a new
plot is generated it appears in a new window, and the old plots remain visible.
When PyXPlot is exited the windows remain in place until they are closed
manually.

\subsubsection{X11\_singlewindow}\index{set terminal command!X11\_singlewindow modifier@\texttt{X11\_singlewindow} modifier}

Displays plots on the screen (in X11 windows, using ghostview). Each time a new
plot is generated it replaces the old one, preventing the desktop from becoming
flooded with old plots. This terminal is the default when running
interactively.

\subsection{textcolour}\index{set textcolour command@\texttt{set textcolour} command}

\begin{verbatim}
set textcolour <colour>
\end{verbatim}

The `{\tt textcolour}' setting changes the colour of all text displayed on a plot.
For example:

\begin{verbatim}
set textcolour red
\end{verbatim}

\noindent causes all text labels, including the labels on graph axes and
legends, etc. to be rendered in red. Any of the recognised colour names listed
in Section~\ref{colour_names} can be used.

\subsection{texthalign}\index{set texthalign command@\texttt{set texthalign} command}

\begin{verbatim}
set texthalign ( left | centre | right )
\end{verbatim}

The `{\tt texthalign}' setting controls how text labels, placed on plots using the
{\tt set label} command, and upon multiplots using the {\tt text} command, are
justified horizontally with respect to their specified positions. Three options
are available:

\begin{verbatim}
set texthalign left
set texthalign centre
set texthalign right
\end{verbatim}

\subsection{textvalign}\index{set textvalign command@\texttt{set textvalign} command}

\begin{verbatim}
set textvalign ( bottom | center | top )
\end{verbatim}

The `{\tt textvalign}' setting controls how text labels, placed on plots using the
{\tt set label} command, and upon multiplots using the {\tt text} command, are
justified vertically with respect to their specified positions. Three options
are available:

\begin{verbatim}
set textvalign bottom 
set textvalign centre
set textvalign top
\end{verbatim}

\subsection{title}\index{set title command@\texttt{set title} command}

\begin{verbatim}
set title '<title>'
\end{verbatim}

The `{\tt title}' setting can be used to set a title for a plot, to be displayed
above it.  For example, the command:

\begin{verbatim}
set title 'foo'
\end{verbatim}

would cause a title `foo' to be displayed above a graph. The easiest way to
remove a title, having set one, is via:

\begin{verbatim}
set title ''
\end{verbatim}
   
\subsection{width}\index{set width command@\texttt{set width} command}

\begin{verbatim}
set width <value>
\end{verbatim}

The {\tt width} setting controls the size of a graph.  For example:

\begin{verbatim}
set width 10
\end{verbatim}

\noindent sets output to be 10 centimetres in width.  For the bitmap terminals ({\tt gif},
{\tt jpg} and {\tt png}) this setting, in conjunction with the {\tt dpi}
setting, controls the number of pixels across the final image.

\subsection{xlabel}\index{set xlabel command@\texttt{set xlabel} command}

\begin{verbatim}
set xlabel '<text>'
\end{verbatim}

The {\tt xlabel} setting controls the label placed on its $x$-axis (abscissa).
For example:

\begin{verbatim}
set xlabel '$x$'
\end{verbatim}

\noindent sets the label on the $x$-axis to `$x$'.  Labels can be placed on higher axes by
inserting their number after the `\texttt{x}', for example:

\begin{verbatim}
set x10label 'foo'
\end{verbatim}

\noindent would label the tenth $x$ axis.

Similarly, labels can be placed on $y$-axes as follows:

\begin{verbatim}
set ylabel '$y$' 
set y2label 'foo'
\end{verbatim}


\subsection{xrange}\index{set xrange command@\texttt{set xrange} command}

\begin{verbatim}
set x[<axisnumber>]range '<text>'
\end{verbatim}

The {\tt xrange} setting controls the range of values along the $x$-axes of
plots.  For function plots, this is also the domain across which the function
will be evaluated.  For example:

\begin{verbatim}
set xrange [0:10]
\end{verbatim}

\noindent sets the first $x$ axis to be between 0 and 10.  Higher numbered axes may be
referred to be inserting their number after the $x$; $y$-axes similarly be
replacing the $x$ with a $y$.  Hence:

\begin{verbatim}
set y23range [-5:5]
\end{verbatim}

sets the range of the 23rd $y$-axis to be between -5 and 5.

The following command:

\begin{verbatim}
set xrange [:10]
\end{verbatim}

would set the $x$-axis to have an upper limit of 10, but an
automatically-scaling lower-limit.


\subsection{xticdir}\index{set xticdir command@\texttt{set xticdir} command}

\begin{verbatim}
set (x|y)[<axisnumber>]ticdir (inward|outward|both)
\end{verbatim}

The `{\tt xticdir}' setting can be used to set whether the ticks along the
$x$-axis of a plot point inwards, towards the graph, as by default, or outwards,
towards the numeric labels along the axis. They can also be set to point in both
directions simultaneously. The syntax for this is as follows:

\begin{verbatim}
set xticdir inward 
set xticdir outward 
set xticdir both
\end{verbatim}

The same setting can also be made on higher numbered axes, by inserting their
numbers after the `{\tt x}', for example:

\begin{verbatim}
set x10ticdir outward
\end{verbatim}

Similarly, the `{\tt x}' can be substituted with a `{\tt y}' to set the directions of ticks
on vertical axes:

\begin{verbatim}
set yticdir inward
set y10ticdir both
\end{verbatim}

\subsection{xtics}\index{set xtics command@\texttt{set xtics} command}

\begin{verbatim}
set [m]x[<axisnumber>]tics 
         [axis|border|inward|outward|both] 
         [auto 
          | [<minimum>,] <increment[, <maximum>] 
          | ( '<label>' <position> ... ) 
         ] 
\end{verbatim}

The {\tt xtics} option specifies the positions of tick marks on the $x$-axis
(similarly, {\tt ytics} acts on the $y$-axis).  One can specify:

\begin{itemize}
\item The axis to modify; if none is specified, then the command acts upon all axes.

\item {\tt mxtics} to alter the placement of minor tic marks.

\item The keywords {\tt inward}, {\tt outward} and {\tt both}, which alter the
directions of the tics.  {\tt axis} is an alias for {\tt inward}, {\tt border}
for {\tt outward}.

\item The {\tt autofreq} keyword restores automatic placement of the tics

\item If {\tt minimum}, {\tt increment}, {\tt maximum} are specified, then ticks
are placed at evenly spaced intervals between the specified limits. In the case
of logarithmic axes, increment is applied multiplicatively. 

\item The final form allows ticks to be placed on an axis manually with
individual labels.
\end{itemize}
   
Two examples:

\begin{verbatim}
set xtics 2 1 5
\end{verbatim}

\noindent will set tick marks on the $x$-axis at positions 2, 3, 4 and 5.

\begin{verbatim}
set x2tics ("a" 2, "b" 3)
\end{verbatim}

\noindent will set tick marks on the second $x$-axis at positions 2 and 3 reading `a' and
`b' respectively.


\subsection{ylabel}\index{set ylabel command@\texttt{set ylabel} command}

See {\tt xlabel}.


\subsection{yrange}\index{set yrange command@\texttt{set yrange} command}

See {\tt xrange}.
   

\subsection{yticdir}\index{set yticdir command@\texttt{set yticdir} command}

See {\tt xticdir}.


\subsection{ytics}\index{set ytics command@\texttt{set ytics} command}

See {\tt xtics}.

\section{show}\index{show command@\texttt{show} command}

\begin{verbatim}
show ( all | settings | axes | variables | functions |
       <parameter> ...                                 )
\end{verbatim}

The {\tt show} command displays the values of PyXPlot's internal parameters. For
example:

\begin{verbatim}
show pointsize
\end{verbatim}

\noindent will display the current default point size.

Details of the various settings that can be shown can be found under the {\tt
set} command; any keyword which can follow the {\tt set} command can also follow
the {\tt show} command.

In addition, {\tt show all} shows the configuration state of all aspects of
PyXPlot. The command {\tt show settings} shows all of PyXPlot's settings, as
distinct from variables, functions and axes. {\tt show axes} shows the
configuration of all of PyXPlot's axes. {\tt show variables} lists all of the
currently defined variables. And finally, {\tt show functions} lists all of the
current user-defined functions.


\section{spline}\index{spline command@\texttt{spline} command}

\begin{verbatim}
spline [<range specification>] <function name> '<filename>' 
       [index <index specification>] [every <every specification>]
       [using <using specification>]
\end{verbatim}

The {\tt spline} command fits a spline to a datafile. A special function is
created that represents the spline fit and can be used in the same way as any
other user-defined function. For example:

\begin{verbatim}
spline f() 'data.1'
\end{verbatim}

\noindent would create a function $f(x)$ that is a fit to the data in the file {\tt
data.1}. By default, the {\tt spline} command uses the first two columns of a
data file in a manner analogous to the plot command. The {\tt index}, {\tt
every} and {\tt using} modifiers can be used in the same way as in the {\tt
plot} command to select which parts of the datafile should be used; see the
{\tt datafile} section for more details.

Note that trying to generate splines of multi-valued functions will not, in
general, produce useful results.

\section{text}\index{text command@\texttt{text} command}

\begin{verbatim}
text '<text string>' [at <x>, <y>] [rotate <angle>]
\end{verbatim}

The {\tt text} command is primarily part of the multiplot environment; it can be
used to add blocks of text to a multiplot. It can, however, also be used in
single plot mode, in a way that is described below. As always in PyXPlot, the
text is rendered using \LaTeX. An example would be:

\begin{verbatim}
text 'Hello world' at 0,2
\end{verbatim}

\noindent which would render the text `Hello world' at position $(0,2)$,
measured in centimetres. The alignment of the text item with respect to this
position can be set using the {\tt set texthalign} and {\tt set textvalign}
commands.

A rotation angle may optionally be specified after the keyword `{\tt rotate}'
to produce text rotated to any arbitrary angle, measured in degrees
counter-clockwise. The following example would produce upward-running text:

\begin{verbatim}
text 'Hello' at 1.5, 3.6 rotate 90
\end{verbatim}

Outside of multiplot mode, the text command can be used to produce images
consisting simply of one single text item. This can be useful for importing
\LaTeX ed equations as gif images into slideshow programs such as Microsoft
Powerpoint which are incapable of producing such neat mathematical notation
by themselves.

\section{undelete}\index{undelete command@\texttt{undelete} command}

\begin{verbatim}
undelete <item number>, ...
\end{verbatim}

The {\tt undelete} command is part of the multiplot environment; it can be used
to reverse the effect of deleting a multiplot item with the {\tt delete}
command. The desired item to be undeleted should be identified using the
reference number which it was given when it was created; it would have been
displayed on the terminal at that time. For example:

\begin{verbatim}
undelete 1
\end{verbatim}

\noindent will cause the previously item numbered {\tt 1} to reappear.
  
\section{unset}\index{unset command@\texttt{unset} command}

\begin{verbatim}
unset <setting>
\end{verbatim}

The {\tt unset} command causes a setting that has been changed using the set
command to be returned to its default value.  For example:

\begin{verbatim}
unset linewidth
\end{verbatim}

\noindent returns the linewidth to its default value.

The list of keywords which can follow the {\tt unset} command are essentially
the same as those which can follow the {\tt set} command.

\chapter{Examples}
\label{examples}

This chapter contains a few example PyXPlot plot scripts to illustrate its
features. For each example, the plotting script is given, and an illustration
of the resulting output.

\section{Example 1: Plotting Functions -- A Simple First Plot}

As a simple first example, we plot two trigonometric functions. The syntax here
is exactly as would have been used in the original gnuplot. The output is shown
in figure~\ref{fig_ex1}.

\vspace{1cm}
\noindent \textbf{PyXPlot Script:}

\VerbatimInput{examples/example1}

\newpage
\section{Example 2: Stacking Many Plots Together -- Multiplot}

\index{multiplot}
\index{set origin command@\texttt{set origin} command}
In this example, we use the multiplot environment to produce a gallery of
several plots. The \texttt{set origin} command is used to position each one. We
also make use of multiple $y$-axes in the top-left plot: the functions
$\sin(x)$ and $\sin^2(x)$ are plotted together, but on different $y$ scales.
The output is shown in figure~\ref{fig_ex2}.

\vspace{1cm}
\noindent \textbf{PyXPlot Script:}

\VerbatimInput{examples/example2} 

\newpage
\section{Example 3: Plotting A Datafile -- Using Multiple Axes}

\index{multiple axes}\index{axes!multiple}
This is a more complicated example. First of all, we plot two datafiles, one
using a line, and another using points. We label our lines using arrows and
text labels, using the same syntax that gnuplot uses. We also have multiple
axes, this time having three $x$-axes on the same plot. The output is shown in
figure~\ref{fig_ex3}.

\vspace{1cm}
\noindent \textbf{PyXPlot Script:}

\VerbatimInput{examples/example3}

\newpage
\section{Example 4: Something Completely Different}

\label{powerpoint_example}

\index{Microsoft Powerpoint!importing figures into}
\index{presentations!importing figures into}

In this example, we demonstrate something rather different that PyXPlot can
do. There is a common problem of trying to incorporate \LaTeX ed equations into
various multimedia/graphics packages: the postscript format which \LaTeX\
produces is not supported by programs such as Microsoft Powerpoint. PyXPlot
offers a very quick and simple solution to this problem.

First of all, we set our terminal to produce png output. To overlay our output
onto a Powerpoint slide, we will want it to have a transparent background, and
so we also use the ``transparent'' terminal option (see Section~\ref{terminals}
for a discussion of PyXPlot terminal options). Finally, if we're producing a
Powerpoint presentation with light-coloured text on a dark background, we will
want to invert the colours to have white text, and so use the ``invert''
terminal option.

\index{text command@\texttt{text} command}
We can now produce plots which can readily be imported into Powerpoint. To
produce \LaTeX ed equations, we use the multiplot environment's \texttt{text}
command (see Section~\ref{text_command}).

\index{refresh command@\texttt{refresh} command}
Finally, as such a figure would not be very easy to incorporate into this User
Manual, we produce a normal eps version of our equation, illustrating how to
use the \texttt{refresh} command to produce multiple copies of the same figure
in different graphic formats.

The output is shown in figure~\ref{fig_ex4}.

\vspace{1cm}
\noindent \textbf{PyXPlot Script:}

\VerbatimInput{examples/example4}

\newpage
\section{Example 5: Multiplot -- Linked Axes}

\index{multiplot!linked axes} In this example, we illustrate how to link the
axes of plots on a multiplot, so that they share a common scale, and also
demonstrate how to set the colours of datasets using the \texttt{with
colour}\index{colours!setting for datasets} plot modifier. In the top-right
panel, we also make use of the multiplot environment to add a plot
inset.\index{multiplot!inset plots} Finally, we render this plot using the
\texttt{landscape}\index{landscape orientation} terminal setting, showing how
to fit more plot onto our sheet of paper. The output is shown in
figure~\ref{fig_ex5}.

Notice how the linked axes autoscale intelligently. The right two plots both
require larger vertical ranges than those plots to their lefts, to whose
vertical axes they are linked. But once they are linked, the plots autoscale
together, to ensure that they all have sufficient range for their data.

\vspace{1cm}
\noindent \textbf{PyXPlot Script:}

\VerbatimInput{examples/example5}

\newpage
\section{Example 6: Bar Charts and Steps}

\label{example6}
\index{bar charts}
\index{boxes plot style@\texttt{boxes} plot style}
\index{steps plot style@\texttt{steps} plot style}
\index{fsteps plot style@\texttt{fsteps} plot style}
\index{histeps plot style@\texttt{histeps} plot style}
\index{impulses plot style@\texttt{impulses} plot style}
\index{set boxfrom command@\texttt{set boxfrom} command}

In this example, we illustrate the \texttt{boxes}, \texttt{impulses} and
\texttt{steps} plot styles, described in Section~\ref{barcharts}, which operate
similarly to how they operate in gnuplot. Panels (a) and (b) illustrates the
\texttt{impulses} plot style for a sine wave, using the \texttt{set boxfrom}
command to define the point from which the lines originate. Panel (c)
illustrates the \texttt{fsteps} plot style, (d) \texttt{steps}, (e)
\texttt{histeps} and (f) \texttt{boxes}. The output is shown in
figure~\ref{fig_ex6}.

\vspace{1cm}
\noindent \textbf{PyXPlot Script:}

\VerbatimInput{examples/example6}

\newpage
\section{Example 7: Bar Charts -- Box Widths}

\label{example7}
\index{bar charts}
\index{boxes plot style@\texttt{boxes} plot style}
\index{wboxes plot style@\texttt{wboxes} plot style}
\index{fillcolour modifier@\texttt{fillcolour} modifier}
\index{colours!fillcolour}

In this example, we demonstrate different ways of specifying the widths of bars
on a bar chart. In panel (a), the widths are automatically determined from the
data, changing bar midway between datapoints. In panel (b), the \texttt{wboxes}
plot style is used, which reads the widths of the bars from a third column in
the datafile. In panel (c), we demonstrate how the \texttt{set boxfrom} command
can be applied to bar charts, as well as to impulses. And in panel (d) we
illustrate how the \texttt{fillcolour} modifier can be used to produce coloured
bars. The output is shown in figure~\ref{fig_ex7}.

\vspace{1cm}
\noindent \textbf{PyXPlot Script:}

\VerbatimInput{examples/example7}

\newpage
\section{Example 8: Fitting Functions to Data}

\index{function fitting}
\index{fit command@\texttt{fit} command}
\index{spline command@\texttt{spline} command}

The \texttt{fit} command works in PyXPlot in essentially the same way as in
gnuplot (see Section~\ref{fit_command}). In this example, we take a series of
data points, and first fit parabolas through them. For the first fit, $f(x)$,
we do not take the errorbars into account; in the second, $g(x)$, we do. Then,
we use the \texttt{spline} command to fit a spline, $h(x)$, through the same
data (see Section~\ref{spline_command}). Strong oscillation is seen in this
example because of the angular nature of the data; it is not well-fit by a
spline. The output is shown in figure~\ref{fig_ex8}.

\vspace{1cm}
\noindent \textbf{PyXPlot Script:}

\VerbatimInput{examples/example8}

\newpage
\section{Example 9: Simple Examples of Function Splicing}

Here, we demonstrate simple use of function splicing (see
Section~\ref{splice}). In panel (a), we plot the function $\sin(x)$, but
specify that we only want it to be drawn in the range $-2<x<7$. In panel (b),
we show how to define a discontinuous function similar to a top-hat function,
also demonstrating how to set movable boundaries between the spliced components
of functions, in this case using the variable $a$ for this purpose.

Panels (c) and (d) demonstrate a more complex example, involving the splicing
of a two-dimensional function.

\vspace{1cm}
\noindent \textbf{PyXPlot Script:}

\VerbatimInput{examples/example9}

\newpage
\section{Example 10: Removal of Unwanted Axes}
\label{ex10}

In this example, we use the magic axis labels \texttt{nolabels},
\texttt{nolabelsticks} and \texttt{invisible}, which were described in
Section~\ref{axis_removal}. In the lower-left plot, we show how to create a
graph without mirrored $x$- and $y$-axes on the top and right sides of the
plot. In the lower-right panel, we produce a plot with only $x$-axes visible,
using them to produce a gallery showing the appearance resulting from the use
of each of these magic labels. The top-left plot shows a simple sketch-graph
with completely unlabelled axes. We also draw arrows over the top of the axes
in this example, to give them arrowheads. Finally, in the top-right panel, we
show one artistic application of plotting functions with no axes visible at
all, creating a simple logo. The output is shown in figure~\ref{fig_ex10}.

\vspace{1cm}
\noindent \textbf{PyXPlot Script:}

\VerbatimInput{examples/example10}

\newpage
\section{Example 11: The Arrows Plot Style}
\label{ex11}

Here, we show two possible applications of the arrows plot style (see
Section~\ref{arrows_plot_style}).\index{arrows plot style@\texttt{arrows} plot
style} In the left panel, we plot a map of fluid flow around a vortex core, the
dotted circle showing the outline of the vortex core. The source for this is a
datafile mapping fluid velocity as a function of position. In the right panel,
we show a series of datapoints before and after some correction factor is
applied to them, showing how the data are moved in the process. The output of
this example is shown in figure~\ref{fig_ex11}.

\vspace{1cm}
\noindent \textbf{PyXPlot Script:}

\VerbatimInput{examples/example11}

\newpage
\section{Output Produced by Examples}
\label{gallery}

\begin{figure}[!h]
\centerline{\includegraphics[width=\textwidth]{examples/eps/example1.eps}}
\caption{The output produced by example script~1, \textit{Plotting Functions -- A Simple First Plot}.}
\label{fig_ex1}
\end{figure}

\begin{figure}[!h]
\centerline{\includegraphics[width=\textwidth]{examples/eps/example2.eps}}
\caption{The output produced by example script~2, \textit{Stacking Many Plots Together -- Multiplot}.}
\label{fig_ex2}
\end{figure}

\begin{figure}[!h]
\centerline{\includegraphics[width=\textwidth]{examples/eps/example3.eps}}
\caption{The output produced by example script~3, \textit{Plotting A Datafile -- Using Multiple Axes}.}
\label{fig_ex3}
\end{figure}

\begin{figure}
\centerline{\includegraphics[width=5cm]{examples/eps/example4.eps}}
\caption{The output produced by example script~4, \textit{Something Completely Different}.}
\label{fig_ex4}
\end{figure}

\begin{figure}
\centerline{\includegraphics[width=\textwidth]{examples/eps/example5.eps}}
\caption{The output produced by example script~5, \textit{Multiplot -- Linked Axes}.}
\label{fig_ex5}
\end{figure}

\begin{figure}
\centerline{\includegraphics[width=\textwidth]{examples/eps/example6.eps}}
\caption{The output produced by example script~6, \textit{Bar Charts and Steps}.}
\label{fig_ex6}
\end{figure}

\begin{figure}
\centerline{\includegraphics[width=\textwidth]{examples/eps/example7.eps}}
\caption{The output produced by example script~7, \textit{Bar Charts -- Box Widths}.}
\label{fig_ex7}
\end{figure}

\begin{figure}
\centerline{\includegraphics[width=\textwidth]{examples/eps/example8.eps}}
\caption{The output produced by example script~8, \textit{Fitting Functions to Data}.}
\label{fig_ex8}
\end{figure}

\begin{figure}
\centerline{\includegraphics[width=\textwidth]{examples/eps/example9.eps}}
\caption{The output produced by example script~9, \textit{Simple Examples of Function Splicing}.}
\label{fig_ex9}
\end{figure}

\begin{figure}
\centerline{\includegraphics[width=\textwidth]{examples/eps/example10.eps}}
\caption{The output produced by example script~10, \textit{Removal of Unwanted Axes}.}
\label{fig_ex10}
\end{figure}

\begin{figure}
\centerline{\includegraphics[width=\textwidth]{examples/eps/example11.eps}}
\caption{The output produced by example script~11, \textit{The Arrows Plot Style}.}
\label{fig_ex11}
\end{figure}

\chapter{The \texttt{fit} Command: Mathematical Details}
\label{fit_math}
\index{fit command@\texttt{fit} command}

In this section, the mathematical details of the workings of the \texttt{fit}
command are described. This may be of interest in diagnosing its limitations,
and also in understanding the various quantities that it outputs after a fit is
found. This discussion must necessarily be a rather brief treatment of a large
subject; for a fuller account, the reader is referred to D.S. Sivia's
\textit{Data Analysis: A Bayesian Tutorial}.

\section{Notation}
\label{bayes_notation}

I shall assume that we have some function $f()$, which takes $n_\mathrm{x}$
parameters, $x_0$...$x_{n_\mathrm{x}-1}$, the set of which may collectively be
written as the vector $\mathbf{x}$. We are supplied a datafile, containing a
number $n_\mathrm{d}$ of datapoints, each consisting of a set of values for
each of the $n_\mathrm{x}$ parameters, and one for the value which we are
seeking to make $f(\mathbf{x})$ match. I shall call of parameter values for the
$i$th datapoint $\mathbf{x}_i$, and the corresponding value which we are trying
to match $f_i$. The datafile may contain error estimates for the values $f_i$,
which I shall denote $\sigma_i$. If these are not supplied, then I shall
consider these quantities to be unknown, and equal to some constant
$\sigma_\mathrm{data}$.

Finally, I assume that there are $n_\mathrm{u}$ coefficients within the
function $f()$ that we are able to vary, corresponding to those variable names
listed after the \texttt{via} statement in the \texttt{fit} command. I shall
call these coefficients $u_0$...$u_{n_\mathrm{u}-1}$, and refer to them
collectively as $\mathbf{u}$.

I model the values $f_i$ in the supplied datafile as being noisy
Gaussian-distributed observations of the true function $f()$, and within this
framework, seek to find that vector of values $\mathbf{u}$ which is most
probable, given these observations. The probability of any given $\mathbf{u}$
is written
$\mathrm{P}\left( \mathbf{u} | \left\{ \mathbf{x}_i, f_i, \sigma_i \right\} \right)$.

\section{The Probability Density Function}
\label{bayes_pdf}

Bayes' Theorem states that:

\begin{equation}
\mathrm{P}\left( \mathbf{u} | \left\{ \mathbf{x}_i, f_i, \sigma_i \right\} \right) =
\frac{
\mathrm{P}\left( \left\{f_i \right\} | \mathbf{u}, \left\{ \mathbf{x}_i, \sigma_i \right\} \right)
\mathrm{P}\left( \mathbf{u} | \left\{ \mathbf{x}_i, \sigma_i \right\} \right)
}{
\mathrm{P}\left( \left\{f_i \right\} | \left\{ \mathbf{x}_i, \sigma_i \right\} \right)
}
\end{equation}

Since we are only seeking to maximise the quantity on the left, and the
denominator, termed the Bayesian \textit{evidence}, is independent of
$\mathbf{u}$, we can neglect it and replace the equality sign with a
proportionality sign.  Furthermore, if we assume a uniform prior, that is, we
assume that we have no prior knowledge to bias us towards certain more favoured
values of $\mathbf{u}$, then $\mathrm{P}\left( \mathbf{u} \right)$ is also a
constant which can be neglected. We conclude that maximising $\mathrm{P}\left(
\mathbf{u} | \left\{ \mathbf{x}_i, f_i, \sigma_i \right\} \right)$ is
equivalent to maximising $\mathrm{P}\left( \left\{f_i \right\} | \mathbf{u},
\left\{ \mathbf{x}_i, \sigma_i \right\} \right)$.

Since we are assuming $f_i$ to be Gaussian-distributed observations of the true
function $f()$, this latter probability can be written as a product of
$n_\mathrm{d}$ Gaussian distributions:

\begin{equation}
\mathrm{P}\left( \left\{f_i \right\} | \mathbf{u}, \left\{ \mathbf{x}_i, \sigma_i \right\} \right)
=
\prod_{i=0}^{n_\mathrm{d}-1} \frac{1}{\sigma_i\sqrt{2\pi}} \exp \left(
\frac{
-\left[f_i - f_\mathbf{u}(\mathbf{x}_i)\right]^2
}{
2 \sigma_i^2
} \right)
\end{equation}

The product in this equation can be converted into a more computationally
workable sum by taking the logarithm of both sides. Since logarithms are
monotonically increasing functions, maximising a probability is equivalent to
maximising its logarithm. We may write the logarithm $L$ of $\mathrm{P}\left(
\mathbf{u} | \left\{ \mathbf{x}_i, f_i, \sigma_i \right\} \right)$ as:

\begin{equation}
L = \sum_{i=0}^{n_\mathrm{d}-1}
\left( \frac{
-\left[f_i - f_\mathbf{u}(\mathbf{x}_i)\right]^2
}{
2 \sigma_i^2
} \right) + k
\end{equation}

\noindent where $k$ is some constant which does not affect the maximisation
process. It is this quantity, the familiar sum-of-square-residuals, that we
numerically maximise to find our best-fitting set of parameters, which I shall
refer to from here on as $\mathbf{u}^0$.

\section{Estimating the Error in $\mathbf{u}^0$}

To estimate the error in the best-fitting parameter values that we find, we
assume $\mathrm{P}\left( \mathbf{u} | \left\{ \mathbf{x}_i, f_i, \sigma_i
\right\} \right)$ to be approximated by an $n_\mathrm{u}$-dimensional Gaussian
distribution around $\mathbf{u}^0$. Taking a Taylor expansion of
$L(\mathbf{u})$ about $\mathbf{u}^0$, we can write:

\begin{eqnarray}
L(\mathbf{u}) & = & L(\mathbf{u}^0) +
    \underbrace{
      \sum_{i=0}^{n_\mathrm{u}-1} \left( u_i - u^0_i \right)
      \left.\frac{\partial L}{\partial u_i}\right|_{\mathbf{u}^0}
    }_{\textrm{Zero at $\mathbf{u}^0$ by definition}} + \label{L_taylor_expand}\\
& & \sum_{i=0}^{n_\mathrm{u}-1} \sum_{j=0}^{n_\mathrm{u}-1} \frac{\left( u_i - u^0_i \right) \left( u_j - u^0_j \right)}{2}
    \left.\frac{\partial^2 L}{\partial u_i \partial u_j}\right|_{\mathbf{u}^0} +
    \mathcal{O}\left( \mathbf{u} - \mathbf{u}^0\right)^3 \nonumber
\end{eqnarray}

Since the logarithm of a Gaussian distribution is a parabola, the quadratic
terms in the above expansion encode the Gaussian component of the probability
distribution $\mathrm{P}\left( \mathbf{u} | \left\{ \mathbf{x}_i, f_i, \sigma_i
\right\} \right)$ about $\mathbf{u}^0$.\footnote{The use of this is called
\textit{Gauss' Method}. Higher order terms in the expansion represent any
non-Gaussianity in the probability distribution, which we neglect. See MacKay,
D.J.C., \textit{Information Theory, Inference and Learning Algorithms}, CUP
(2003).} We may write the sum of these terms, which we denote $Q$, in matrix
form:

\begin{equation}
Q = \frac{1}{2} \left(\mathbf{u} - \mathbf{u}^0\right)^\mathbf{T} \mathbf{A} \left(\mathbf{u} - \mathbf{u}^0\right)
\label{Q_vector}
\end{equation}

\noindent where the superscript $^\mathbf{T}$ represents the transpose of the
vector displacement from $\mathbf{u}^0$, and $\mathbf{A}$ is the Hessian matrix
of $L$, given by:

\begin{equation}
A_{ij} = \nabla\nabla L = \left.\frac{\partial^2 L}{\partial u_i \partial u_j}\right|_{\mathbf{u}^0}
\end{equation}
\index{Hessian matrix}

This is the Hessian matrix which is output by the \texttt{fit} command. In
general, an $n_\mathrm{u}$-dimensional Gaussian distribution such as that given
by equation~(\ref{L_taylor_expand}) yields elliptical contours of
equiprobability in parameter space, whose principal axes need not be aligned
with our chosen coordinate axes -- the variables $u_0 ... u_{n_u-1}$. The
eigenvectors $\mathbf{e}_i$ of $\mathbf{A}$ are the principal axes of these
ellipses, and the corresponding eigenvalues $\lambda_i$ equal $1/\sigma_i^2$,
where $\sigma_i$ is the standard deviation of the probability density function
along the direction of these axes.

This can be visualised by imagining that we diagonalise $\mathbf{A}$, and
expand equation~(\ref{Q_vector}) in our diagonal basis. The resulting
expression for $L$ is a sum of square terms; the cross terms vanish in this
basis by definition. The equations of the equiprobability contours become the
equations of ellipses:

\begin{equation}
Q = \frac{1}{2} \sum_{i=0}^{n_\mathrm{u}-1} A_{ii} \left(u_i - u^0_i\right)^2 = k
\end{equation}

\noindent where $k$ is some constant. By comparison with the equation for the
logarithm of a Gaussian distribution, we can associate $A_{ii}$ with
$-1/\sigma_i^2$ in our eigenvector basis.

The problem of evaluating the standard deviations of our variables $u_i$ is
more complicated, however, as we are attempting to evaluate the width of these
elliptical equiprobability contours in directions which are, in general, not
aligned with their principal axes. To achieve this, we first convert our
Hessian matrix into a covariance matrix.

\section{The Covariance Matrix}
\index{covariance matrix}

The terms of the covariance matrix $V_{ij}$ are defined by:

\begin{equation}
\label{def_covar}
V_{ij} = \left< \left(u_i - u^0_i\right) \left(u_j - u^0_j\right) \right>
\end{equation}

\noindent Its leading diagonal terms may be recognised as equalling the
variances of each of our $n_\mathrm{u}$ variables; its cross terms measure the
correlation between the variables. If a component $V_{ij} > 0$, it implies that
higher estimates of the coefficient $u_i$ make higher estimates of $u_j$ more
favourable also; if $V_{ij} < 0$, the converse is true.

It is a standard statistical result that $\mathbf{V} = (-\mathbf{A})^{-1}$. In
the remainder of this section we prove this; readers who are willing to accept
this may skip onto Section~\ref{correlation_matrix}.

Using $\Delta u_i$ to denote $\left(u_i - u^0_i\right)$, we may proceed by
rewriting equation~(\ref{def_covar}) as:

\begin{eqnarray}
V_{ij} & = & \idotsint_{u_i=-\infty}^{\infty}
\Delta u_i \Delta u_j
\mathrm{P}\left(
\mathbf{u} | \left\{ \mathbf{x}_i, f_i, \sigma_i \right\} \right)
\,\mathrm{d}^{n_\mathrm{u}}\mathbf{u} \\
 & = & \frac{
\idotsint_{u_i=-\infty}^{\infty} \Delta u_i \Delta u_j \exp(-Q) \,\mathrm{d}^{n_\mathrm{u}}\mathbf{u}
}{
\idotsint_{u_i=-\infty}^{\infty} \exp(-Q) \,\mathrm{d}^{n_\mathrm{u}}\mathbf{u}
}
\nonumber
\end{eqnarray}

The normalisation factor in the denominator of this expression, which we denote
as $Z$, the \textit{partition function}, may be evaluated by
$n_\mathrm{u}$-dimensional Gaussian integration, and is a standard result:

\begin{eqnarray}
Z & = & \idotsint_{u_i=-\infty}^{\infty} \exp\left(\frac{1}{2} \Delta \mathbf{u}^\mathbf{T} \mathbf{A} \Delta \mathbf{u} \right) \,\mathrm{d}^{n_\mathrm{u}}\mathbf{u} \\
& = & \frac{(2\pi)^{n_\mathrm{u}/2}}{\mathrm{Det}(\mathbf{-A})} \nonumber
\end{eqnarray}

Differentiating $\log_e(Z)$ with respect of any given component of the Hessian
matrix $A_{ij}$ yields:

\begin{equation}
-2 \frac{\partial}{\partial A_{ij}} \left[ \log_e(Z) \right] = \frac{1}{Z}
\idotsint_{u_i=-\infty}^{\infty} \Delta u_i \Delta u_j \exp(-Q) \,\mathrm{d}^{n_\mathrm{u}}\mathbf{u}
\end{equation}

\noindent which we may identify as equalling $V_{ij}$:

\begin{eqnarray}
\label{v_zrelate}
V_{ij} & = & -2 \frac{\partial}{\partial A_{ij}} \left[ \log_e(Z) \right] \\
& = & -2 \frac{\partial}{\partial A_{ij}} \left[ \log_e((2\pi)^{n_\mathrm{u}/2}) - \log_e(\mathrm{Det}(\mathbf{-A})) \right] \nonumber \\
& = & 2 \frac{\partial}{\partial A_{ij}} \left[ \log_e(\mathrm{Det}(\mathbf{-A})) \right] \nonumber
\end{eqnarray}

\noindent This expression may be simplified by recalling that the determinant
of a matrix is equal to the scalar product of any of its rows with its
cofactors, yielding the result:

\begin{equation}
\frac{\partial}{\partial A_{ij}} \left[\mathrm{Det}(\mathbf{-A})\right] = -a_{ij}
\end{equation}

\noindent where $a_{ij}$ is the cofactor of $A_{ij}$. Substituting this into
equation~(\ref{v_zrelate}) yields:

\begin{equation}
V_{ij} = \frac{-a_{ij}}{\mathrm{Det}(\mathbf{-A})}
\end{equation}

Recalling that the adjoint $\mathbf{A}^\dagger$ of the Hessian matrix is the
matrix of cofactors of its transpose, and that $\mathbf{A}$ is symmetric, we
may write:

\begin{equation}
V_{ij} = \frac{-\mathbf{A}^\dagger}{\mathrm{Det}(\mathbf{-A})} \equiv (-\mathbf{A})^{-1}
\end{equation}

\noindent which proves the result stated earlier.

\section{The Correlation Matrix}
\label{correlation_matrix}
\index{correlation matrix}

Having evaluated the covariance matrix, we may straightforwardly find the
standard deviations in each of our variables, by taking the square roots of the
terms along its leading diagonal. For datafiles where the user does not specify
the standard deviations $\sigma_i$ in each value $f_i$, the task is not quite
complete, as the Hessian matrix depends critically upon these uncertainties,
even if they are assumed the same for all of our $f_i$. This point is returned
to in Section~\ref{finding_sigmai}.

The correlation matrix $\mathbf{C}$, whose terms are given by:

\begin{equation}
C_{ij} = \frac{V_{ij}}{\sigma_i\sigma_j}
\end{equation}

\noindent may be considered a more user-friendly version of the covariance
matrix for inspecting the correlation between parameters. The leading diagonal
terms are all clearly equal unity by construction. The cross terms lie in the
range $-1 \leq C_{ij} \leq 1$, the upper limit of this range representing
perfect correlation between parameters, and the lower limit perfect
anti-correlation.

\section{Finding $\sigma_i$}
\label{finding_sigmai}

Throughout the preceding sections, the uncertainties in the supplied target
values $f_i$ have been denoted $\sigma_i$ (see Section~\ref{bayes_notation}).
The user has the option of supplying these in the source datafile, in which
case the provisions of the previous sections are now complete; both
best-estimate parameter values and their uncertainties can be calculated. The
user may also, however, leave the uncertainties in $f_i$ unstated, in which
case, as described in Section~\ref{bayes_notation}, we assume all of the data
values to have a common uncertainty $\sigma_\mathrm{data}$, which is an
unknown.

In this case, where $\sigma_i = \sigma_\mathrm{data} \,\forall\, i$, the best
fitting parameter values are independent of $\sigma_\mathrm{data}$, but the
same is not true of the uncertainties in these values, as the terms of the
Hessian matrix do depend upon $\sigma_\mathrm{data}$. We must therefore
undertake a further calculation to find the most probable value of
$\sigma_\mathrm{data}$, given the data. This is achieved by maximising
$\mathrm{P}\left( \sigma_\mathrm{data} | \left\{ \mathbf{x}_i, f_i \right\}
\right)$. Returning once again to Bayes' Theorem, we can write:

\begin{equation}
\mathrm{P}\left( \sigma_\mathrm{data} | \left\{ \mathbf{x}_i, f_i \right\} \right)
= \frac{
\mathrm{P}\left( \left\{ f_i \right\} | \sigma_\mathrm{data}, \left\{ \mathbf{x}_i \right\} \right)
\mathrm{P}\left( \sigma_\mathrm{data} | \left\{ \mathbf{x}_i \right\} \right)
}{
\mathrm{P}\left( \left\{ f_i \right\} | \left\{ \mathbf{x}_i \right\} \right)
}
\end{equation}

As before, we neglect the denominator, which has no effect upon the
maximisation problem, and assume a uniform prior $\mathrm{P}\left(
\sigma_\mathrm{data} | \left\{ \mathbf{x}_i \right\} \right)$. This reduces the
problem to the maximisation of $\mathrm{P}\left( \left\{ f_i \right\} |
\sigma_\mathrm{data}, \left\{ \mathbf{x}_i \right\} \right)$, which we may
write as a marginalised probability distribution over $\mathbf{u}$:

\begin{eqnarray}
\label{p_f_given_sigma}
\mathrm{P}\left( \left\{ f_i \right\} | \sigma_\mathrm{data}, \left\{ \mathbf{x}_i \right\} \right) =
\idotsint_{-\infty}^{\infty}
&
\mathrm{P}\left( \left\{ f_i \right\} | \sigma_\mathrm{data}, \left\{ \mathbf{x}_i \right\}, \mathbf{u} \right)
\times & \\ &
\mathrm{P}\left( \mathbf{u} | \sigma_\mathrm{data}, \left\{ \mathbf{x}_i \right\} \right)
\,\mathrm{d}^{n_\mathrm{u}}\mathbf{u}
& \nonumber
\end{eqnarray}

Assuming a uniform prior for $\mathbf{u}$, we may neglect the latter term in
the integral, but even with this assumption, the integral is not generally
tractable, as $\mathrm{P}\left( \left\{ f_i \right\} | \sigma_\mathrm{data},
\left\{ \mathbf{x}_i \right\}, \left\{ \mathbf{u}_i \right\} \right)$ may well
be multimodal in form. However, if we neglect such possibilities, and assume
this probability distribution to be approximate a Gaussian \textit{globally},
we can make use of the standard result for an $n_\mathrm{u}$-dimensional Gaussian integral:

\begin{equation}
\idotsint_{-\infty}^{\infty}
\exp \left(
\frac{1}{2}\mathbf{u}^\mathbf{T} \mathbf{A} \mathbf{u}
\right) \,\mathrm{d}^{n_\mathrm{u}}\mathbf{u}
=
\frac{
(2\pi)^{n_\mathrm{u}/2}
}{
\sqrt{\mathrm{Det}\left(-\mathbf{A}\right)}
}
\end{equation}

\noindent We may thus approximate equation~(\ref{p_f_given_sigma}) as:

\begin{eqnarray}
\mathrm{P}\left( \left\{ f_i \right\} | \sigma_\mathrm{data}, \left\{ \mathbf{x}_i \right\} \right)
& \approx &
\mathrm{P}\left( \left\{ f_i \right\} | \sigma_\mathrm{data}, \left\{ \mathbf{x}_i \right\}, \mathbf{u}^0 \right)
\times \\
& &
\mathrm{P}\left( \mathbf{u}^0 | \sigma_\mathrm{data}, \left\{ \mathbf{x}_i, f_i \right\} \right)
\frac{
(2\pi)^{n_\mathrm{u}/2}
}{
\sqrt{\mathrm{Det}\left(-\mathbf{A}\right)}
}
\nonumber
\end{eqnarray}

As in Section~\ref{bayes_pdf}, it is numerically easier to maximise this
quantity via its logarithm, which we denote $L_2$, and can write as:

\begin{eqnarray}
L_2 & = &
\sum_{i=0}^{n_\mathrm{d}-1}
\left(
\frac{
-\left[f_i - f_{\mathbf{u}^0}(\mathbf{x}_i)\right]^2
}{
2\sigma_\mathrm{data}^2
}
- \log_e \left(2\pi\sqrt{\sigma_\mathrm{data}} \right)
\right) +
\\ & & \nonumber
\log_e \left(
\frac{
(2\pi)^{n_\mathrm{u}/2}
}{
\sqrt{\mathrm{Det}\left(-\mathbf{A}\right)}
}
\right)
\end{eqnarray}

This quantity is maximised numerically, a process simplified by the fact that
$\mathbf{u}^0$ is independent of $\sigma_\mathrm{data}$.

\chapter{ChangeLog}
\index{ChangeLog}

\noindent \textbf{2007 Jan 19: PyXPlot 0.6.2}
\begin{itemize}
\item `enlarge' terminal option implemented.
\item pdf terminal implemented.
\item set preamble command implemented.
\item LaTeX preambles bugfixed, so that the `$\backslash$usepackage' command can now be used.
\item `X11\_persist' terminal bugfixed to work correctly from non-interactive sessions.
\item Large number of minor bugfixes.
\end{itemize}

\noindent \textbf{2006 Dec 27: PyXPlot 0.6.1}
\begin{itemize} 
\item Major bug-fixes to the set and unset commands.
\item Command reference chapter added to User Manual.
\item Command syntax references added to the help command system.
\item `text' and `set label' commands extended to allow rotation of text through arbitrary angles.
\item Semi-functional `jpeg' and `eps' commands added. Left undocumented because they're unstable and need a bit of work.
\end{itemize}

\noindent \textbf{2006 Nov 12: PyXPlot 0.6.0}
\begin{itemize}
\item New more-reliable command parser implemented, with improved syntax errors.
\item delete\_arrow, delete\_text, undelete\_arrow, undelete\_text and move\_text commands removed from API. The move, delete and undelete commands now act on all kinds of multiplot object.
\item set terminal command no longer recognised enhanced and noenhanced keywords. The postscript and eps terminal keywords should now be used in their place.
\item Automatic ticking of axes overhauled, and the set xtics and set mxtics commands implemented for those who do not like the default ticking schemes.
\item set log and set nolog command now allow axes to work with log bases other than 10.
\item The select modifier after the plot, fit, replot and spline commands can now only be used once; to specify multiple select criteria, use the and logical operator.
\item X11\_persist terminal implemented.
\item Requirement on python 2.4 eased to python 2.3.
\item Requirements on scipy and readline eased; PyXPlot will now work in reduced form without them.
\item Requirements on dvips and gs are dropped; postscript handling now done by PyXPlot itself.
\end{itemize}

\noindent \textbf{2006 Sep 09: PyXPlot 0.5.8}
\begin{itemize}
\item Many bugfixes to error trapping and reporting.
\end{itemize}

\noindent \textbf{2006 Aug 26: PyXPlot 0.5.7}
\begin{itemize}
\item set display command implemented.
\item set keycolumns command implemented.
\item CTRL-C behaviour changed; no longer quits PyXPlot.
\item plot `*.dat' now arranges files alphabetically.
\item Escaping of LaTeX $<$ and $>$ symbols fixed.
\item Major bugfix to fit command's error estimation.
\item Major bugfix to the positioning of legends in the ``outside'' and ``below'' positions to avoid overlapping with axes.
\item help command text substantially revised.
\end{itemize}

\noindent \textbf{2006 Aug 18: PyXPlot 0.5.6}
\begin{itemize}
\item Ability to unset variables via ``a='' implemented.
\item Handling on scipy error messages in the int\_dx and spline commands improved.
\item Colour-highlighted terminal added.
\item The inline help system made much more complete.
\item select modifier implemented.
\item set texthalign and set textvalign implemented.
\item set xticdir command implemented.
\item Support for CSV input datafiles implemented.
\item pyxplot\_watch quiet mode added. Also, behaviour changed to allow the watching of files, even when they do not initially exist.
\item Labels can now be placed on ``nolabels'', ``nolabelstics'' and ``invisible'' axes. Example 10 changed to demonstrate this.
\item set log, when issued on its own, now applies to all axes, rather than throwing an error.
\end{itemize}

\noindent \textbf{2006 Jul 25: PyXPlot 0.5.5}
\begin{itemize}
\item pyxplot\_watch implemented.
\item fit command now gives error estimates, as well as correlation matrices.
\item Many new pointtypes added, including upper and lower limit symbols.
\item Handling of SIGINT improved; now exits current command in interactive mode, and exits PyXPlot when running a script.
\item Quote characters can now be escaped in LaTeX strings, to allow strings with both ' and " characters to be rendered.
\item Installer no longer creates any files belonging to root in the user's homespace.
\item show xlabel and show xrange implemented.
\item Bug fix: cd command no longer crashes if target directory doesn't exist.
\item Bug fix: some commands, e.g. plot, which previously didn't work when capitalised, now do.
\item Major bug fix to int\_dx and diff\_dx functions.
\end{itemize}

\noindent \textbf{2006 Jul 3: PyXPlot 0.5.4}
\begin{itemize}
\item edit command implemented.
\item Numerical integration and differentiation functions implemented.
\item New makefile installer added.
\item man page added.
\item Brief tour of gnuplot syntax added to documentation.
\item Many minor bug fixes.
\end{itemize}

\noindent \textbf{2006 Jun 27: PyXPlot 0.5.3}
\begin{itemize}
\item set bar and set palette implemented.
\item Stacked barcharts implemented.
\item Command history files and the save command implemented.
\item Plotting of functions with errorbars implemented.
\item Ability to define a LaTeX preamble implemented.
\item Bug fix to smoothed splines, to ensure that smoothing is always applied to a sensible degree by default.
\item Bug fix to the autoscaling of bar charts, histograms and errorbars, to ensure that their full extent is contained within the plot area.
\item Bug fix to arrow plotting, to prevent PyX from crashing if arrows of zero lengths are plotting (they have no direction...)
\end{itemize}

\noindent \textbf{2006 Jun 14: PyXPlot 0.5.2}
\begin{itemize}
\item spline command, and csplines/acsplines plot styles implemented.
\item Syntax plot[0:1], with no space, now allowed.
\item Automatic names of datasets in legends no longer have full paths, but only the path in the form that the user specified it.
\item Bug fix to the handling of LaTeX special characters in the automatic names of datasets, e.g. file paths containing underscores.
\item Error messages now sent to stderr, rather than stdout.
\item multiplot mode now plots items in the order that they are plotted; previously all arrows and text labels had been plotted in front of plots.
\item set backup command implemented, for keeping backups of overwritten files.
\item Bug fix, enabling the use of axis x5 without axis x3, and likewise for y.
\item unset axis command implemented, for removing axes from plots.
\item `invisible', `nolabels', and `nolabelsticks' axis title implemented, for producing axes without text labels.
\item plot 'every' modifier re-implemented, to use the same syntax as gnuplot.
\item fit command re-implemented to work with functions of $>$ 1 variable.
\item plot with pointlines defined as alias for `linespoints'.
\item plot using rows syntax implemented, for plotting horizontally-arranged datafiles.
\item Bug fix to replot command in multiplot mode, to take account of any move commands applied to the last plot.
\item Bug fix to errorbar pointsizes. pointsize modifier now produces sensible output with all errorbar plot styles.
\item show command re-implemented to accept any word that the set command will.
\end{itemize}

\noindent \textbf{2006 Jun 2: PyXPlot 0.5.1}
\begin{itemize}
\item Pling and cd commands implemented; ` ` shell command substitution implemented.
\item Arrows (both from set arrow and the arrow command) can now have linetypes and colours set.
\item Colours can now be specified as either palette indices or PyX colour names in all contexts -- e.g. `plot with colour red'.
\item Function plotting fixed to allow plotting of functions which are not valid across the whole range of the x-axis.
\item Transparent terminals now have anti-aliasing disabled.
\item Warnings now issued when too many columns are specified in plot command; duplicate errors filtered out in two-pass plotting.
\item Function splicing implemented.
\item Documentation: sections on barcharts, function splicing, and datafile globbing added.
\end{itemize}

\noindent \textbf{2006 May 27: PyXPlot 0.5.0}
\begin{itemize}
\item Name changed to PyXPlot.
\item Change to distribution format: PyX Version 0.9 now ships with package.
\item Safety installer added; checks for required packages.
\item `errorrange' plot styles added; allow errorbars to be given as min/max values, rather than as a standard deviation.
\item `boxes', `wboxes', `steps', `fsteps', `histeps' and `impulses' plot styles implemented -- allow the production of histograms and bar charts.
\item plot with fillcolour implemented, to allow coloured bar charts.
\item Handling of broken datafiles sanitised: now warns for each broken line.
\item gridlines on multiple axes, e.g. `set grid x1x2x3' now allowed.
\item Major bugfix to the way autoscaling works: linked axes share information and scale intelligently between plots.
\item --help and --version commandline options implemented.
\item `using' specifiers for datafiles can now include expressions, such as \$(2+x).
\item eps terminal fixed to produce encapsulated postscript.
\item datafile names now glob, so that plot `*' will plot many datafiles.
\item Documentation: examples 6,7 and 8 added.
\end{itemize}

\noindent \textbf{2006 May 18: GnuPlot+ 0.4.3}
\begin{itemize}
\item text and arrow commands now accept expressions rather than just floats for positional coordinates.
\item clear command major bug-fixed.
\item `plot with' clause bugfixed; state variable was not resetting.
\item Automatical key titles for datafile datasets made more informative.
\item Autoscaling of multiple axes bugfixed.
\item Autoscaling of inverted axes fixed.
\item set grid command fixed to only produce x/y gridlines when requested.
\item X11\_singlewindow changed to use \texttt{gv --watch}.
\item landscape terminal postscript header detection bugfixed.
\item noenhanced terminal changed to produce proper postscript.
\item Plotting of single column datafiles without using specifier fixed.
\end{itemize}

\noindent \textbf{2006 May 4: GnuPlot+ 0.4.2}
\begin{itemize}
\item Autoscaling redesigned, no longer uses PyX for this.
\item Numerical expression handling fixed in set title, set origin and set label.
\item Handling of children fixed, to prevent zombies from lingering around.
\item arrow command implemented.
\item set textcolour, set axescolour, set gridmajcolour, set gridmincolour and set fontsize implemented.
\item Colour palette can now be set in configuration file.
\item Ranges for axes other then x1/y1 can now be set in the plot command.
\item Postscript noenhanced can now produce plots almost as big as an A4 sheet.
\item Plotting of one column datafiles, against datapoint number, implemented.
\item Negative errorbars error trapped.
\item Comment lines now allowed in command files.
\end{itemize}

\noindent \textbf{2006 May 1: GnuPlot+ 0.4.1}
\begin{itemize}
\item Documentation converted from ASCII to LaTeX.
\item ChangeLog added.
\item Configuration files now supported.
\item Prevention of temporary files in /tmp overwriting pre-existing files.
\item set term enhanced / noenhanced / landscape / portrait / png / gif / jpeg / transparent / solid / invert / noinvert implemented.
\item set dpi implemented, to allow user to choose quality of gif/jpg/png output.
\item 'set grid' command now allows user to specify which axes grid attaches to (extended API).
\item Support introduced for plotting gzipped datafiles. Filenames ending in `.gz' are assumed to be gzipped.
\item load command implemented.
\item move command implemented.
\item Long lines can now be split using `\' linesplit character at the end of a line. Any whitespace at the beginning of the next is omitted.
\item text / delete\_text / undelete\_text / move\_text commands implemented.
\item refresh command implemented. (extended API)
\item point types, line styles, and colours now start at 1, for gnuplot compatibility.
\item default terminal changed to postscript for non-interactive sessions.
\end{itemize}

\noindent \textbf{2006 Apr 27: GnuPlot+ 0.4.0}
\begin{itemize}
\item Bug fix: now looks for input scripts in the user's cwd, not in /tmp.
\item `set logscale' is now valid syntax (as in gnuplot), as well as `set log'.
\item multiplot implemented, including linked axes, though with some brokenness
if linked axes are allowed to autoscale.
\item `dots' plotting style implemented.
\item Bug fix: can now include a plot `with' clause after an `axes' clause;
could not previously without an error message arising.
\item Pointstyles now increment between plotted datasets, even in a colour
terminal where the colours also increment.
\item garbage collection of .eps files from the X11 terminal added. Previously
they were left to fester in /tmp.
\item pointlinewidth added as a plot style, specifying the linewidth to be used
in plotting points. `set plw' and `set lw' both added (extended API).
\item delete, clear and undelete commands added to the multiplot environment.
\item unset command implemented.
\item set notitle implemented.
\end{itemize}

\noindent \textbf{2006 Apr 14: GnuPlot+ 0.3.2}
\begin{itemize}
\item The autoscaling of logarithmic axes made more trust-worthy: error checks
to ensure that they do not try to cover negative ordinates.
\item Error checks put in place to prevent empty keys being plotted, which made
PyX crash previously. Now can plot empty graphs happily.
\item Datasets with blank titles removed from the key, to allow users to plot
some datasets to be omitted from the key. This is not possible in gnuplot.
\item Bug fix to prevent PyX's texrunner from crashing irreparably upon
receiving bad LaTeX. Now uses a spanner to attempt to return it to working
order for the next plot.
\item Bug fix to the autoscaling of axes with no range of data -- previous did
not work for negative ordinates. Now displays an axes with a range of +/-
1.0 around the data.
\end{itemize}

\noindent \textbf{2006 Apr 12: GnuPlot+ 0.3.1}
\begin{itemize}
\item Plotting of functions fixed: plot command will now plot any algebraic
expression, not just functions of the form f(x).
\item Space added after command prompt.
\end{itemize}

\noindent \textbf{2006 Apr 12: GnuPlot+ 0.3.0}
\begin{itemize}
\item X11\_singlewindow and X11\_multiwindow terminals implemented, as distinct
from just standard X11.
\item Key positioning allowed to be xcentre, ycentre, below and outside, as
well as in the corners of the plot. Key allowed to be offseted in position.
\item Datasets colours can be set via `plot with colour $<$n$>$'
\item Datasets are split when there is a blank line in the datafile; lines are
not joined up between the two segments.
\item set size implemented; can now change aspect ratio of plots.
\item working directory of GnuPlot+ changed to /tmp, so that LaTeX's temporary
files are stored there rather than in the user's cwd.
\end{itemize}

\noindent \textbf{2006 Mar 30: GnuPlot+ 0.2.0}
\begin{itemize}
\item Standard GnuPlot dual axes improved upon, allowing users to add x3, x4
axes, etc, up to any number of axes that may be desired.
\item Autocomplete mechanism for commandline substantially cleaned up and
debugged.
\item Bug fixes to the plotting of arrows/labels. Now appear \textit{above} gridlines,
not below.
\end{itemize}

\noindent \textbf{2006 Feb 26: GnuPlot+ 0.1.0}

\printindex
\end{document}
