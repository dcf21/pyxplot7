% VECTOR_GRAPHICS.TEX
%
% The documentation in this file is part of PyXPlot
% <http://www.pyxplot.org.uk>
%
% Copyright (C) 2006-8 Dominic Ford <coders@pyxplot.org.uk>
%               2008   Ross Church
%
% $Id$
%
% PyXPlot is free software; you can redistribute it and/or modify it under the
% terms of the GNU General Public License as published by the Free Software
% Foundation; either version 2 of the License, or (at your option) any later
% version.
%
% You should have received a copy of the GNU General Public License along with
% PyXPlot; if not, write to the Free Software Foundation, Inc., 51 Franklin
% Street, Fifth Floor, Boston, MA  02110-1301, USA

% ----------------------------------------------------------------------------

% LaTeX source for the PyXPlot Users' Guide

\chapter{Labelling Plots and Producing Galleries}
\chaptermark{Labelled Plots and Galleries}

So far, we have talked exclusively about how to plot graphs with PyXPlot. In
this chapter, we discuss how to label graphs and place simple vector graphics
around them.

\section{Adding Arrows and Text Labels to Plots}

This section describes how to put arrows and text labels on plots; the syntax
is similar, though not identical, to that used by \gnuplot.  PyXPlot extends
\gnuplot's syntax to make it possible to to set the colours and styles of
arrows and text labels.

\subsection{Arrows}

\label{set_arrow}\index{arrows} Arrows may be placed on plots using the
\indcmdt{set arrow}. A simple example would be:

\begin{verbatim}
set arrow 1 from 0,0 to 1,1
\end{verbatim}

\noindent The number {\tt 1} immediately following \indcmdts{set arrow}
specifies an identification number for the arrow, allowing it to be
subsequently removed via the command:

\begin{verbatim}
unset arrow 1
\end{verbatim}

\noindent or equivalently, via:\indcmd{set noarrow}

\begin{verbatim}
set noarrow 1
\end{verbatim}

The {\tt set arrow} command can be followed by the keyword {\tt with} to
specify the style of the arrow. For example, the keywords \indkeyt{nohead},
\indkeyt{head} and \indkeyt{twohead}, placed after the keyword {\tt with}, can
be used to generate arrows with no arrow heads, normal arrow heads, or two
arrow heads.  \indkeyt{twoway} is an alias for \indkeyt{twohead}.  For example:

\begin{verbatim}
set arrow 1 from 0,0 to 1,1 with nohead
\end{verbatim}

\noindent Line types and colours can also be specified after the keyword {\tt
with}:

\begin{verbatim}
set arrow 1 from 0,0 to 1,1 with nohead \
linetype 1 c blue
\end{verbatim}

As in \gnuplot, the coordinates for the start and end points of the arrow can
be specified in a range of coordinate systems. The co-ordinate system to be
used should be specified immediately before the co-ordinate value. The default
system, \indcot{first} measures the graph using the $x$- and $y$-axes. The
\indcot{second} system uses the $x2$- and $y2$-axes. The \indcot{screen} and
\indcot{graph} systems both measure in centimetres from the origin of the
graph. In the following example, we use these specifiers, and specify
coordinates using variables rather than doing so explicitly:

\begin{verbatim}
x0 = 0.0
y0 = 0.0
x1 = 1.0
y1 = 1.0
set arrow 1 from first  x0, first  y0 \
            to   screen x1, screen y1 \
            with nohead
\end{verbatim}

In addition to these four options, which are those available in \gnuplot, the
syntax `\indcot{axis{\it n}}' may also be used, to use the $n\,$th $x$- or
$y$-axis -- for example, `{\tt axis3}'.\indcmd{set arrow} This allows arrows to
reference any arbitrary axis on plots which make use of large numbers of
parallel axes (see Section~\ref{multiple_axes}).

\subsection{Text Labels}

Text labels may be placed on plots using the \indcmdt{set label}. As with all
textual labels in PyXPlot, these are rendered in \LaTeX:

\begin{verbatim}
set label 1 'Hello World' at 0,0
\end{verbatim}

As in the previous section, the number {\tt 1} is a reference number, which
allows the label to be removed by either of the following two commands:

\begin{verbatim}
set nolabel 1
unset label 1
\end{verbatim}

\noindent The positional coordinates for the text label, placed after the {\tt
at} keyword, can be specified in any of the coordinate systems described for
arrows above. A rotation angle may optionally be specified after the keyword
\indkeyt{rotate}, to rotate text counter-clockwise by a given angle, measured
in degrees. For example, the following would produce upward-running text:

\begin{verbatim}
set label 1 'Hello World' at axis3 3.0, axis4 2.7 rotate 90
\end{verbatim}

A colour can also be specified, if desired, using the {\tt with colour}
modifier.  For example, the following would produce a green label at the origin:

\begin{verbatim}
set label 2 'This label is green' at 0, 0 with colour green
\end{verbatim}

\index{fontsize}\index{text!size} The fontsize of these text labels can be set
globally using the \indcmdt{set fontsize}. This applies not only to the {\tt
set label} command, but also to plot titles, axis labels, keys, etc. The value
given should be an integer in the range $-4 \leq x \leq 5$. The default is
zero, which corresponds to \LaTeX's {\tt normalsize}; $-4$ corresponds to {\tt
tiny} and 5 to {\tt Huge}.

\index{text!colour}\index{colours!text} The \indcmdt{set textcolour} can be
used to globally set the colour of all text output, and applies to all of the
text that the {\tt set fontsize} command does. It is especially useful when
producing plots to be embedded in presentation slideshows, where bright text on
a dark background may be desired. It should be followed either by an integer,
to set a colour from the present palette, or by a colour name. A list of the
recognised colour names can be found in Section~\ref{colour_names}.  For
example:

\begin{verbatim}
set textcolour 2
set textcolour blue
\end{verbatim}

\index{text!alignment}\index{alignment!text}By default, each label's specified
position corresponds to its bottom left corner. This alignment may be changed
with the \indcmdts{set texthalign} and \indcmdts{set textvalign} commands. The
former takes the options \indkeyt{left}, \indkeyt{centre} or \indkeyt{right},
and the latter takes the options \indkeyt{bottom}, \indkeyt{centre} or
\indkeyt{top}, for example:

\begin{verbatim}
set texthalign right
set textvalign top
\end{verbatim}

An example of a somewhat unconventional plot containing many labels and lines can be found in Figure~\ref{fig:ex_map}.

\begin{figure}
\begin{center}
\includegraphics[width=\textwidth]{examples/eps/ex_map.eps}
\end{center}
\caption{A map of Australia, plotted using PyXPlot.  The data were obtained
from \protect\url{http://www.maproom.psu.edu/dcw/} (for the coastal outlines
and state boundaries) and \protect\url{http://en.wikipedia.org} (for the city
locations).  The data files and script used to produce this map can be
downloaded from the PyXPlot website at
\protect\url{http://www.pyxplot.org.uk/examples/Manual/08map/}.}
\label{fig:ex_map}
\end{figure}

\section{Gridlines}

Gridlines may be placed on a plot and subsequently removed via the statements:

\begin{verbatim}
set grid
set nogrid
\end{verbatim}

\noindent respectively. The following commands are also valid:

\begin{verbatim}
unset grid
unset nogrid
\end{verbatim}

\noindent By default, gridlines are drawn from the major and minor ticks of the
default $x$- and $y$-axes (which are the first $x$- and $y$-axes unless set
otherwise in the configuration file; see Chapter~\ref{configuration}). However,
the axes which should be used may be specified after the \indcmdt{set
grid}\index{grid}:

\begin{verbatim}
set grid x2y2
set grid x x2y2
\end{verbatim}

\noindent The top example would connect the gridlines to the ticks of the $x2$-
and $y2$-axes, whilst the lower would draw gridlines from both the $x$- and the
$x2$-axes.

If one of the specified axes does not exist, then no gridlines will be drawn in
that direction.  Gridlines can subsequently be removed selectively from some
axes via:

\begin{verbatim}
unset grid x2x3
\end{verbatim}

The colours of gridlines\index{grid!colour}\index{colours!grid} can be
controlled via the \indcmdts{set gridmajcolour} and \indcmdts{set
gridmincolour} commands, which control the gridlines emanating from major and
minor axis ticks respectively. An example would be:

\begin{verbatim}
set gridmincolour blue
\end{verbatim}

\noindent Any of the colour names listed in Section~\ref{colour_names} can be
used.

A related command\index{axes!colour}\index{colours!axes} is \indcmdts{set
axescolour}, which has a syntax similar to that above, and sets the colour of
the graph's axes.\label{set_colours}

\section{Multi-plotting}
\label{multiplot}
\index{multiplot}

Gnuplot has a plotting mode called {\it multiplot} which allows many graphs to
be plotted together and displayed side-by-side. The basic syntax of this mode
is reproduced in PyXPlot, but it is hugely extended.

The mode is entered by the command \indcmdts{set multiplot}.  This can be
compared to taking a blank sheet of paper on which to place plots.  Plots are
then placed on that sheet of paper, as usual, with the {\tt plot} command. The
position of each plot is set using the \indcmdt{set origin}, which takes a
comma-separated $(x,y)$ coordinate pair, measured in centimetres. The
following, for example, would plot a graph of $\sin(x)$ to the left of a plot
of $\cos(x)$:

\begin{verbatim} 
set multiplot
plot sin(x)
set origin 10,0
plot cos(x)
\end{verbatim}

The multiplot page may subsequently be cleared with the \indcmdt{clear}, and
multiplot mode may be left using the \indcmdt{set nomultiplot}.

\subsection{Deleting, Moving and Changing Plots}

Each time a plot is placed on the multiplot page in PyXPlot, it is allocated a
reference number, which is output to the terminal. Reference numbers count up
from zero each time the multiplot page is cleared. A number of commands exist
for modifying plots after they have been placed on the page, selecting them by
making reference to their reference numbers.

Plots may be removed from the page with the \indcmdt{delete}, and restored with
the \indcmdt{undelete}:

\begin{verbatim} 
delete <number>
undelete <number>
\end{verbatim}

The reference numbers of deleted plots are not reused until the page is
cleared, as they may always be restored with the \indcmdt{undelete}; plots
which have been deleted simply do not appear.

Plots may also be moved with the \indcmdt{move}. For example, the following
would move plot 23 to position $(8,8)$ measured in centimetres:

\begin{verbatim} 
move 23 to 8,8
\end{verbatim}

In multiplot mode, the \indcmdt{replot} can be used to modify the last plot
added to the page. For example, the following would change the title of the
latest plot to `foo', and add a plot of $\cos(x)$ to it:

\begin{verbatim} 
set title 'foo'
replot cos(x)
\end{verbatim}

Additionally, it is possible to modify any plot on the page, by first selecting
it with the \indcmdt{edit}. Subsequently, the \indcmdt{replot} will act upon
the selected plot. The following example would produce two plots, and then
change the colour of the text on the first:

\begin{verbatim} 
set multiplot
plot sin(x)
set origin 10,0
plot cos(x)
edit 0        # Select the first plot ...
set textcolour red
replot        # ... and replot it.
\end{verbatim}

The \indcmdt{edit} can also be used to view the settings which are applied to
any plot on the multiplot page -- after executing {\tt edit~0}, the
\indcmdt{show} will show the settings applied to plot zero.

When a new plot is added to the page, the \indcmdt{replot} always switches to
act upon this most recent plot.

\subsection{Listing Items on a Multiplot}

A listing of all of the items on a multiplot, giving their reference numbers
and the commands used to produce them, can be obtained using the
\indcmdt{list}. For example:

\begin{verbatim}
pyxplot> list
#  ID | Command 
    0   plot f(x) 
d   1   text 'Figure 1: A plot of f(x)' 
    2   text 'Figure 1: A plot of $f(x)$' 

# Items marked 'd' are deleted 
\end{verbatim}

In this example, the user has plotted a graph of $f(x)$, and added a caption to
it. The {\tt ID} column lists the reference numbers of each multiplot item.
Item {\tt 1} has been deleted, and this is indicated by the {\tt d} to the left
of its reference number.

\subsection{Linked Axes}

The axes of plots can be linked together, in such a way that they always share
a common scale. This can be useful when placing plots next to one another,
firstly, of course, if it is of intrinsic interest to ensure that they are on a
common scale, but also because the two plots then do not both need their own
axis labels, and space can be saved by one sharing the labels from the other.
In PyXPlot, an axis which borrows its scale and labels from another is called a
{\it linked axis}.

Such axes are declared by setting the label of the linked axis to a magic
string such as {\tt linkaxis 0}\label{linked_axes}\index{axes!reserved
labels}\index{magic axis labels}. This magic label would set the axis to borrow
its scale from an axis from plot zero. The general syntax is `{\tt linkaxis}
$n$ $m$', where $n$ and $m$ are two integers, separated by a comma or
whitespace. The first, $n$, indicates the plot from which to borrow an axis;
the second, $m$, indicates whether to borrow the scale of axis $x1$, $x2$,
$x3$, etc. By default, $m=1$. The linking will fail, and a warning result, if
an attempt is made to link to an axis which doesn't exist.

\subsection{Text Labels, Arrows and Images}

\label{text_command} In addition to placing plots on the multiplot page, text
labels may also be inserted independently of any plots, using the
\indcmdt{text}. This has the following syntax:

\begin{verbatim} 
text 'This is some text' at x,y
\end{verbatim}

In this case, the string `This is some text' would be rendered at position
$(x,y)$ on the multiplot. As with the \indcmdt{set label}, a colour may
optionally be specified with the {\tt with colour} modifier, as well as a
rotation angle to rotate text labels through any given angle, measured in
degrees counter-clockwise. For example:\indkey{rotate}

\begin{verbatim} 
text 'This is some text' at x,y rotate r with colour red
\end{verbatim}

The commands \indcmdts{set textcolour}, \indcmdts{set texthalign} and
\indcmdts{set textvalign}, which have already been described in the context in
the {\tt set label} command, can also be used to set the colour and alignment
of text produced with the \indcmdt{text}.  A useful application of this is to
produce centred headings at the top of multiplots.

As with plots, each text item has a unique identification number, and can be
moved around, deleted or undeleted with the \indcmdts{move},
\indcmdts{delete} and \indcmdts{undelete} commands.

It should be noted that the \indcmdt{text} can also be used outside of the
multiplot environment, to render a single piece of short text instead of a
graph. One obvious application is to produce equations rendered as graphical
files for inclusion in talks.\index{presentations}

\label{arrows} Arrows may also be placed on multiplot pages, independently of
any plots, using the \indcmdt{arrow}, which has syntax:

\begin{verbatim} 
arrow from x,y to x,y
\end{verbatim}

As above, arrows receive unique identification numbers, and can be deleted and
undeleted.

The \indcmdt{arrow} may be followed by the \indmodt{with} keyword to specify to
style of the arrow. The style keywords which are accepted are identical to
those accepted by the {\tt set arrow} command (see Section~\ref{set_arrow}).
For example:

\begin{verbatim} 
arrow from x1,y1 to x2,y2 \
with twohead colour red
\end{verbatim}

Bitmap images in jpeg format may be placed on the multiplot using the
\indcmdt{jpeg}.  This has syntax:

\begin{verbatim}
jpeg 'filename' at x,y width w
\end{verbatim}

As an alternative to the \indkeyt{width} keyword the height of the image can be
specified, using the analogous \indkeyt{height} keyword.  An optional angle can
also be specified using the \indkeyt{rotate} keyword; this causes the included
image to be rotated counter-clockwise by a specified angle, measured in
degrees.

Vector graphic images in eps format may be placed on to a multiplot using the
\indcmdt{eps}, which has a syntax analogous to the {\tt jpeg} command.  However
neither height nor width need be specified; in this case the image will be
included at its native size.  For example:

\begin{verbatim}
eps 'filename' at 3,2 rotate 5
\end{verbatim}

\noindent will place the eps file with its bottom-left corner at position
$(3,2)$\,cm from the origin, rotated counter-clockwise through 5 degrees.

\subsection{Speed Issues}
\label{set_display}

By default, whenever an item is added to a multiplot, or an existing item moved
or replotted, the whole multiplot is replotted to show the change. This can be
a time consuming process on large and complex multiplots. For this reason, the
\indcmdt{set nodisplay} is provided, which stops PyXPlot from producing any
output. The \indcmdt{set display} can subsequently be issued to return to
normal behaviour.

This can be especially useful in scripts which produce large multiplots. There
is no point in producing output at each step in the construction of a large
multiplot, and a great speed increase can be achieved by wrapping the script
with:

\begin{verbatim} 
set nodisplay
[...prepare large multiplot...]
set display
refresh
\end{verbatim}

\subsection{The \indcmdt{refresh}}

\index{replotting} The \indcmdt{refresh} is rather similar to the
\indcmdt{replot}, but produces an exact copy of the latest display. This can be
useful, for example, after changing the terminal type, to produce a second copy
of a multiplot page in a different format. But the crucial difference between
this command and {\tt replot} is that it doesn't replot anything. Indeed, there
could be only textual items and arrows on the present multiplot page, and no
graphs {\it to} replot.

\section{LaTeX and PyXPlot}

The \indcmdt{text} can straightforwardly be used to render simple one-line
\LaTeX\index{latex} strings, but sometimes the need arises to place more
substantial blocks of text onto a plot. For this purpose, it can be useful to
use the \LaTeX\ {\tt parbox} or {\tt minipage} environments\footnote{Remember,
any valid \LaTeX\ string can be passed to the \indcmdt{text} and \indcmdt{set
label}.} For example:

\begin{verbatim} 
text '\parbox[t]{6cm}{\setlength{\parindent}{1cm} \
\noindent There once was a lady from Hyde, \\ \
Who ate a green apple and died, \\ \
\indent While her lover lamented, \\ \
\indent The apple fermented, \\ \
and made cider inside her inside.}'
\end{verbatim}

If unusual mathematical symbols are required, for example those in the {\tt
amsmath} package\index{amsmath package@{\tt amsmath} package}, such a package
can be loaded using the \indcmdt{set preamble}. For example:

\begin{verbatim} 
set preamble \usepackage{marvosym}
text "{\Huge\Dontwash\ \NoIroning\ \NoTumbler}$\;$ Do not \
wash, iron or tumble-dry this plot."
\end{verbatim}

