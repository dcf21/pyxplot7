\chapter{Plotting}
\section{Chosing which data to plot}
\label{select_modifier} 
As well as the \texttt{index}, \texttt{using} and \texttt{every} keywords which
allow users to plot subsets of data from datafiles, PyXPlot also has a further
modifier, \index{select keyword@\texttt{select} keyword}. This can be used to
plot only those datapoints in a datafile which specify some given criterion. For
example:

\begin{verbatim}
plot 'datafile' select ($8>5)
plot sin(x) select (($1>0) and ($2>0))
\end{verbatim}

In the second example, two select criteria are given, combined with the logical
and operator\footnote{See Table~\ref{operators_table} for a list of all
operators recognised by PyXPlot.}. The select modifier has many applications,
including plotting two-dimensional slices from three-dimensional datasets, and
selecting certain subsets of datapoints from a datafile for plotting.

Logical operators such as \texttt{and}, \texttt{or} and \texttt{not} can be
used, as seen in the second example above; indeed, any expression which is valid
Python can be used.

\section{Plot styles}

The following basic plotting styles are supported by PyXPlot:
\begin{itemize}
\item dots -- places a small dot at each datum
\item points -- places a marker symbol at each datum
\item lines -- connects the data with lines
\item linespoints -- a combination of both lines and points
\item impulses -- draws a line from the $x$-axis up to each datum
\end{itemize}

When using the \texttt{points}, \texttt{linespoints} and \texttt{dots} plotting
styles the size of the plotted points or dots can be varied with the
\texttt{pointsize} modifier. For example, to display big dots, use:

\begin{verbatim}
plot sin(x) with dots pointsize 10
\end{verbatim}

The width of lines can similarly be controlled with the
\texttt{linewidth}\index{linewidth modifier@\texttt{linewidth} modifier}.
Similarly, the width of lines used to draw point symbols can be controlled with
the \texttt{pointlinewidth}\index{pointlinewidth
modifier@\texttt{pointlinewidth} modifier}.  For example:

\begin{verbatim}
plot sin(x) with points pointlinewidth 2
\end{verbatim}

\noindent
Furthermore, \texttt{set pointlinewidth 2} will set the default linewidth to be
used when drawing data points.  In both cases, the abbreviation \texttt{plw} is
valid. 

PyXPlot can plot datapoints using the
standard upper- and lower-limit symbols.\index{lower-limit
datapoints}\index{upper-limit datapoints} No special syntax is required for
this; these symbols are pointtypes\footnote{The \texttt{pointtype} modifier was
introduced in Section~\ref{pointtype_modifier}.} 12 and 13 respectively,
obtained as follows:

\begin{verbatim}
plot 'upperlimits' with points pointtype 12
plot 'lowerlimits' with points pointtype 13
\end{verbatim}

\subsection{\texttt{arrows} plot style}
\label{arrows_plot_style} 

The \index{arrows plot style@\texttt{arrows} plot style}
takes four columns of data, $x_1$, $y_1$, $x_2$, $y_2$, and for each data point
draws an arrow from the point $(x_1,y_1)$ to $(x_2,y_2)$.  Three different kinds
of arrows can be drawn: ones with normal arrow heads, ones with no arrow heads,
which just appear as lines, and ones with arrow heads on both ends. The syntax
is:

\begin{verbatim}
plot 'datafile' with arrows_head
plot 'datafile' with arrows_nohead
plot 'datafile' with arrows_twohead
\end{verbatim}

The syntax `\texttt{with arrows}' is a shorthand for `\texttt{with arrows\_head}'.


\subsection{Plotting with error bars}

\index{errorbars}\label{errorbars}
In gnuplot, when one used
errorbars, one could either specify the size of the errorbar, or the min/max
range of the errorbar. Both of these usages shared a common syntax, and
gnuplot's behaviour depended upon the number of data columns provided:

\begin{verbatim}
plot 'datafile' with yerrorbars
\end{verbatim}

\noindent Given a datafile with three columns, this would take the third column
to indicate the size of the $y$-errorbar, and given a four-column datafile, it
would take the third and fourth columns to indicate the min/max range to be
marked out by the errorbar.

To avoid confusion, a different syntax is adopted in PyXPlot. The syntax:

\begin{verbatim}
plot 'datafile' with yerrorbars
\end{verbatim}

\noindent now always assumes the third column of the datafile to indicate the
size of the errorbar, regardless of whether a fourth is present. The syntax:

\begin{verbatim}
plot 'datafile' with yerrorrange
\end{verbatim}

\noindent always assumes the third and fourth columns to indicate the min/max
range of the errorbar.

\vspace{0.5cm}
For clarity, a complete list of errorbar styles is given below:

\begin{tabular}{p{2.5cm}p{5.5cm}}
\texttt{yerrorbars} & Vertical errorbars; size drawn from the third data-column. \\
\texttt{xerrorbars} & Horizontal errorbars; size drawn from the third data-column. \\
\texttt{xyerrorbars} & Horizontal and vertical errorbars; sizes drawn from the third and fourth data-columns respectively.\\
\texttt{errorbars} & Shorthand for \texttt{yerrorbars}. \\
\end{tabular}

\begin{tabular}{p{2.5cm}p{5.5cm}}
\texttt{yerrorrange} & Vertical errorbars; minimum drawn from the third data-column, maximum from the fourth.\\
\texttt{xerrorrange} & Horizontal errorbars; minimum drawn from the third data-column, maximum from the fourth.\\
\texttt{xyerrorrange} & Horizontal and vertical errorbars; horizontal minimum drawn from the third data-column, and maximum from the fourth; vertical minimum drawn from the fifth, and maximum from the sixth.\\
\texttt{errorrange} & Shorthand for \texttt{yerrorrange}. \\
\end{tabular}

In gnuplot, when a function (as opposed to a datafile) is plotted, only those
plot styles which accept two columns of data can be used -- for example,
\texttt{lines} or \texttt{points}. It is not possible to plot a function with
errorbars, for example. In PyXPlot, by contrast, this is possible using the
following syntax:

\begin{verbatim}
plot f(x):g(x) with yerrorbars
\end{verbatim}

Two functions are supplied, separated by a colon; plotting proceeds as if a
datafile had been supplied, containing values of $x$ in column 1, values of
$f(x)$ in column 2, and values of $g(x)$ in column 3. This may be useful, for
example, if $g(x)$ measures the intrinsic uncertainty in $f(x)$. The
\texttt{using} modifier may also be used:

\begin{verbatim}
plot f(x):g(x) using 2:3
\end{verbatim}

Here, $g(x)$ would be plotted on the $y$-axis, against $f(x)$ on the $x$-axis.
It should be noted, however, that the range of values of $x$ used would still
correspond to the range of the plot's horizontal axis. If the above were to be
attempted with an autoscaling horizontal axis, the result might be rather
unexpected -- PyXPlot would find itself autoscaling the $x$-axis range to the
spread of values of $f(x)$, but find that this itself changed depending upon
the range of the $x$-axis.

\section{Horizontally arranged datafiles}

\index{horizontal
datafiles}\index{datafiles!horizontal}\index{using rows modifier@\texttt{using
rows} modifier}\index{using columns modifier@\texttt{using columns}
modifier}\label{horizontal_datafiles} 
The command syntax for plotting
columns of datafiles against one another was previously described in
Section~\ref{plot_datafiles}.  In an extension of gnuplot's interface, it is
also possible to plot \textit{rows} of data against one another in
horizontally-arranged datafiles.  For this, the keyword `\texttt{rows}' is
placed after the \texttt{using} modifier:\index{rows keyword@\texttt{rows} keyword}

\begin{verbatim}
plot 'datafile' index 1 using rows 1:2
\end{verbatim}

The syntax `\texttt{using columns}' is also accepted, to specify the default
behaviour of plotting columns against one another:\index{columns keyword@\texttt{columns} keyword}

\begin{verbatim}
plot 'datafile' index 1 using columns 1:2
\end{verbatim}

When plotting horizontally-arranged datafiles, the meanings of the
\texttt{index} and \texttt{every} modifiers (see Section~\ref{plot_datafiles})
are altered slightly. The former continues to refer to vertical blocks of data
separated by two blank lines.  Blocks, as referenced in the \texttt{every}
modifier, continue to be vertical blocks of datapoints, separated by single
blank lines. The row numbers passed to the \texttt{using} modifier are counted
from the top of the current block.

However, the line-numbers specified in the \texttt{every} modifier -- i.e.
variables $a$, $c$ and $e$ in the system above -- now refer to horizontal
columns, rather than lines. For example:

\begin{verbatim}
plot 'datafile' using rows 1:2 every 2::3::9
\end{verbatim}

\noindent would plot the data in row 2 against that in row 1, using only the
values in every other column, between columns 3 and 9.


\section{Barcharts and Histograms}
\label{barcharts}\index{bar charts}
\index{steps plot style@\texttt{steps} plot style}
\index{fsteps plot style@\texttt{fsteps} plot style}
\index{histeps plot style@\texttt{histeps} plot style}
\index{impulses plot style@\texttt{impulses} plot style}
\index{set boxfrom command@\texttt{set boxfrom} command}

\subsection{Basic Operation}

As in gnuplot, bar charts and histograms can be produced using the
\texttt{boxes}\index{boxes plot style@\texttt{boxes} plot style} plot style:

\begin{verbatim} 
plot 'datafile' with boxes
\end{verbatim}

\noindent Horizontally, the interfaces between the bars are, by default, at the
midpoints along the $x$-axis between the specified datapoints (see, for
example, Panel~(a) of Figure~\ref{fig:ex_barchart2}).  Alternatively, the
widths of the bars may be set using the \texttt{set boxwidth} command. In this
case, all of the bars will be centred upon their specified $x$-coordinates, and
have total widths equalling that specified in the \texttt{set
boxwidth}\index{set boxwidth command@\texttt{set boxwidth} command} command.
Consequently, there may be gaps between them, or they may overlap, as seen in
Panel~(c) of Figure~\ref{fig:ex_barchart2}.

\begin{figure}
\begin{center}
\includegraphics{examples/eps/ex_barchart2.eps}
\end{center}
\caption{A gallery of different bar chart styles which PyXPlot can produce. See the text for more details.}
\label{fig:ex_barchart2}
\end{figure}

Having set a fixed box width, the default automatic width mode may be restored
either with the \texttt{unset boxwidth} command, or by setting the boxwidth to
a negative width.

As a third alternative, it is also possible to specify different widths for
each bar manually, in a column of the input datafile. For this, the
\texttt{wboxes}\index{wboxes plot style@\texttt{wboxes} plot style} plot style
should be used:

\begin{verbatim} 
plot 'datafile' using 1:2:3 with wboxes
\end{verbatim}

\noindent This plot style expects three columns of data to be specified: the
$x$- and $y$-coordinates of each bar, and the width in the third column.
Panel~(b) of Figure~\ref{fig:ex_barchart2} shows an example of this plot style
in use.

By default, the bars all originate from the line $y=0$, as is normal for a
histogram. However, should it be desired for the bars to start from a different
vertical point, that may be achieved with the \texttt{set boxfrom} command, for
example:

\begin{verbatim} 
set boxfrom 5
\end{verbatim}

\noindent All of the bars would then originate from the line $y=5$. Panel~(f)
of Figure~\ref{fig:ex_barchart1} shows the kind of effect that is achieved; for
comparison, panel~(b) of the same figure shows the same bar chart with the
boxes starting from their default position at $y=0$.

\begin{figure}
\begin{center}
\includegraphics{examples/eps/ex_barchart1.eps}
\end{center}
\caption{A second gallery of different bar chart styles which PyXPlot can produce. See the text for more details.}
\label{fig:ex_barchart1}
\end{figure}

The bars may be filled using the \texttt{with fillcolour}\index{fillcolour
modifier@\texttt{fillcolour} modifier} modifier, followed by the name of a
colour:

\begin{verbatim} 
plot 'datafile' with boxes fillcolour blue
plot 'datafile' with boxes fc 4
\end{verbatim}

\noindent Panels~(c) and (d) of Figure~\ref{fig:ex_barchart2} demonstrate the
use of filled bars.

Finally, the \texttt{impulses} plot style, as in gnuplot, produces bars of zero
width; see Panel~(e) of Figure~\ref{fig:ex_barchart1} for an example.

\subsection{Stacked Bar Charts}

If several datapoints are supplied at a common $x$-coordinate to the
\texttt{boxes} or \texttt{wboxes} plot styles, then the bars are stacked one
above another into a stacked barchart. Consider the following datafile:

\begin{verbatim} 
1 1
2 2
2 3
3 4
\end{verbatim}

The second bar at $x=2$ would be placed on top of the first, spanning the range
$2<y<5$, and having the same width as the first. If plot colours are being
automatically selected from the palette, then a different palette colour is
used to plot the upper bar.

\subsection{Steps}

As an alternative to solid boxes, a graph may also be plotted with ``steps'';
see Panels~(a), (c) and (d) of Figure~\ref{fig:ex_barchart1} for examples. As
is illustrated by these panels, three flavours of steps are available (exactly
as in gnuplot):

\begin{verbatim}
plot 'datafile' with steps 
plot 'datafile' with fsteps 
plot 'datafile' with histeps
\end{verbatim}

When using the \texttt{steps} plot style, the datapoints specify the right-most
edges of each step. By contrast, they specify the left-most edges of the steps
when using the \texttt{fsteps} plot style. The \texttt{histeps} plot style
works rather like the \texttt{boxes} plot style; the interfaces between the
steps occur at the horizontal midpoints between the datapoints.




\section{Multi-plotting}
\label{multiplot}
\index{multiplot}

Gnuplot has a plotting mode called ``multiplot'' which allows many graphs to be
plotted together, and display side-by-side. The basic syntax of this mode is
reproduced in PyXPlot, but is hugely extended.

The mode is entered by the command ``\texttt{set multiplot}''. This can be compared
to taking a blank sheet of paper on which to place plots. Plots are then placed
on that sheet of paper, as usual, with the \texttt{plot} command. The position
of each plot is set using the \texttt{set origin} command, which takes a
comma-separated $x,y$ coordinate pair, measured in centimetres. The following,
for example, would plot a graph of $\sin(x)$ to the left of a plot of
$\cos(x)$:\index{set origin command@\texttt{set origin} command}

\begin{verbatim} 
set multiplot
plot sin(x)
set origin 10,0
plot cos(x)
\end{verbatim}

The multiplot page may subsequently be cleared with the \texttt{clear} command,
and multiplot mode may be left using the ``\texttt{set nomultiplot}''
command.\index{clear command@\texttt{clear} command}

\subsection{Deleting, Moving and Changing Plots}

Each time a plot is placed on the multiplot page in PyXPlot, it is allocated a
reference number, which is output to the terminal. Reference numbers count up
from zero each time the multiplot page is cleared. A number of commands exist
for modifying plots after they have been placed on the page, selecting them by
making reference to their reference numbers.

Plots may be removed from the page with the \texttt{delete} command, and
restored with the \texttt{undelete} command:\index{delete
command@\texttt{delete} command}\index{undelete command@\texttt{undelete}
command}

\begin{verbatim} 
delete <number>
undelete <number>
\end{verbatim}

The reference numbers of deleted plots are not reused until the page is
cleared, as they may always be restored with the \texttt{undelete} command;
plots which have been deleted simply do not appear.

Plots may also be moved with the \texttt{move} command. For example, the
following would move plot 23 to position (8,8) measured in centimetres:

\begin{verbatim} 
move 23 to 8,8
\end{verbatim}

In multiplot mode, the \texttt{replot} command can be used to modify the last
plot added to the page. For example, the following would change the title of
the latest plot to ``foo'', and add a plot of $\cos(x)$ to it:

\begin{verbatim} 
set title 'foo'
replot cos(x)
\end{verbatim}

Additionally, it is possible to modify any plot on the page, by first selecting
it with the \texttt{edit} command. Subsequently, the \texttt{replot} will act
upon the selected plot. The following example would produce two plots, and then
change the colour of the text on the first:

\begin{verbatim} 
set multiplot
plot sin(x)
set origin 10,0
plot cos(x)
edit 0        # Select the first plot ...
set textcolour red
replot        # ... and replot it.
\end{verbatim}

The \texttt{edit} command can also be used to view the settings which are
applied to any plot on the multiplot page -- after executing ``edit 0'', the
\texttt{show} command will show the settings applied to plot zero.

When a new plot is added to the page, \texttt{replot} always switches to act
upon this most recent plot.

\index{refresh command@\texttt{refresh} command}\index{replotting}
\index{replot command@\texttt{replot} command} The \texttt{refresh} command is
rather similar to the \texttt{replot} command, but produces an exact copy of
the latest display. This can be useful, for example, after changing the
terminal type, to produce a second copy of a multiplot page in a different
format. But the crucial difference between this command and \texttt{replot} is
that it doesn't replot anything. Indeed, there could be only textual items and
arrows on the present multiplot page, and no graphs \textit{to} replot.

\subsection{Linked Axes}

The axes of plots can be linked together, in such a way that they always share
a common scale. This can be useful when placing plots next to one another,
firstly, of course, if it is of intrinsic interest to ensure that they are on a
common scale, but also because the two plots then do not both need their own
axis labels, and space can be saved by one sharing the labels from the other.
In PyXPlot, an axis which borrows its scale and labels from another is called a
``linked axis''.

Such axes are declared by setting the label of the linked axis to a magic
string such as ``\texttt{linkaxis 0}''\label{linked_axes}\index{axes!reserved
labels}\index{magic axis labels}. This magic label would set the axis to borrow
its scale from an axis from plot zero. The general syntax is
``\texttt{linkaxis} $n$ $m$'', where $n$ and $m$ are two integers, separated by
a comma or whitespace. The first, $n$, indicates the plot from which to borrow
an axis; the second, $m$, indicates whether to borrow the scale of axis $x1$,
$x2$, $x3$, etc. By default, $m=1$. The linking will fail, and a warning
result, if an attempt is made to link to an axis which doesn't exist.

\subsection{Text Labels, Arrows and Images}

\label{text_command}\index{text command@\texttt{text} command} In addition to
placing plots on the multiplot page, text labels may also be inserted
independently of any plots, using the \texttt{text} command. This has the
following syntax:

\begin{verbatim} 
text 'This is some text' at x,y
\end{verbatim}

In this case, the string ``This is some text'' would be rendered at position
$(x,y)$ on the multiplot. As with the \texttt{set label} command, a rotation
angle may optionally be specified to rotate text labels through any given
angle, measured in degrees counter-clockwise, for example:

\begin{verbatim} 
text 'This is some text' at x,y rotate r
\end{verbatim}

The commands \texttt{set textcolour}, \texttt{set
texthalign} and \texttt{set textvalign}, which have already been described in
the context in the \texttt{set label} command, can also be used to set the
colour and alignment of text produced with the \texttt{text} command.\index{set
textcolour command@\texttt{set textcolour} command}\index{set texthalign
command@\texttt{set texthalign} command}\index{set textvalign
command@\texttt{set textvalign} command}. A useful application of this is to
produce centred headings at the top of multiplots.

As with plots, each text item has a unique identification number, and can be
moved around, deleted or undeleted with the \texttt{delete}, \texttt{undelete}
and \texttt{move} commands.
\index{delete command@\texttt{delete} command}
\index{undelete command@\texttt{undelete} command}

It should be noted that the \texttt{text} command can also be used outside of
the multiplot environment, to render a single piece of short text instead of a
graph. One obvious application is to produce equations rendered as graphical
files for inclusion in talks.

\label{arrows} \index{arrow command@\texttt{arrow} command} Arrows may also be
placed on multiplot pages, independently of any plots, using the \texttt{arrow}
command, which has syntax:

\begin{verbatim} 
arrow from x,y to x,y
\end{verbatim}

As above, arrows receive unique identification numbers, and can be deleted and
undeleted.

The \texttt{arrow} command may be followed by the `\texttt{with}' keyword to
specify to style of the arrow. The style keywords which are accepted are
identical to those accepted by the \texttt{set arrow} command (see
Section~\ref{set_arrow}). For example:

\begin{verbatim} 
arrow from x1,y1 to x2,y2 \
with twohead colour red
\end{verbatim}

\index{jpeg command@\texttt{jpeg} command} Bitmap images in jpeg form may be
placed on the multiplot using the {\tt jpeg} command.  This has syntax:

\begin{verbatim}
jpeg 'filename' at x,y width w
\end{verbatim}

As an alternative to the {\tt width} modifier the height of the image can be
specified, using the analogous {\tt height} modifier.  An optional angle can
also be specified using the {\tt rotate} modifier; this causes the included
image to be rotated counter-clockwise by a specified angle (in degrees).

\index{eps command@\texttt{eps} command} Vector graphic images in eps format may
be placed on to a multiplot using the {\tt eps} command, which has a syntax
analogous to the {\tt jpeg} command.  However neither height nor width need be
specified; in this case the image will be included at its native size.  For
example:

\begin{verbatim}
eps 'filename' at 3,2 rotate 5
\end{verbatim}

\noindent will place the eps file with its bottom-left corner at position
$(3,2)$ cm from the origin, rotated counter-clockwise through 5 degrees.

\subsection{Speed Issues}
\label{set_display}

By default, whenever an item is added to a multiplot, or an existing item moved
or replotted, the whole multiplot is replotted to show the change. This can be
a time consuming process on large and complex multiplots. For this reason, the
\texttt{set nodisplay}\index{set display command@\texttt{set display} command}
command is provided, which stops PyXPlot from producing any output. The
\texttt{set display} command can subsequently be issued to return to normal
behaviour.

This can be especially useful in scripts which produce large multiplots. There
is no point in producing output at each step in the construction of a large
multiplot, and so a great speed increase can be achieved by wrapping the script
with:

\begin{verbatim} 
set nodisplay
[...prepare large multiplot...]
set display
refresh
\end{verbatim}
