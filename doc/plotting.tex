% PLOTTING.TEX
%
% The documentation in this file is part of PyXPlot
% <http://www.pyxplot.org.uk>
%
% Copyright (C) 2006-7 Dominic Ford <coders@pyxplot.org.uk>
%               2008   Ross Church
%
% $Id$
%
% PyXPlot is free software; you can redistribute it and/or modify it under the
% terms of the GNU General Public License as published by the Free Software
% Foundation; either version 2 of the License, or (at your option) any later
% version.
%
% You should have received a copy of the GNU General Public License along with
% PyXPlot; if not, write to the Free Software Foundation, Inc., 51 Franklin
% Street, Fifth Floor, Boston, MA  02110-1301, USA

% ----------------------------------------------------------------------------

% LaTeX source for the PyXPlot Users' Guide

\chapter{Advanced Plotting}

In this chapter, we continue to explore the various options of the {\tt plot}
command. Specifically, we turn to those aspects which differ from \gnuplot's
{\tt plot} command.

\section{A Tour of PyXPlot's Plot Styles}

We begin by reviewing the various plot styles which are available in PyXPlot.
Two of these we have already met: {\tt lines}, which draws straight lines
between \datapoint s, and {\tt points}, which does not connect \datapoint s.

\subsection{Lines and Points}

The following are PyXPlot's most basic plot styles\footnote{This is not an
exhaustive list; see Section~\ref{list_of_plotstyles} for such a list.}:
\begin{itemize}
\item \indpst{dots} -- places a small dot at each datum.
\item \indpst{points} -- places a marker symbol at each datum.
\item \indpst{lines} -- connects adjacent \datapoint s with straight lines.
\item \indpst{linespoints} -- a combination of both lines and points.
\end{itemize}

When using the \indpst{points}, \indpst{linespoints} and \indpst{dots} plot
styles, the size of the plotted points or dots can be varied by using the
\indmodt{pointsize} modifier, for example:

\begin{verbatim}
set samples 25
plot sin(x) with dots pointsize 10
\end{verbatim}

which would represent data with large dots. The default value of this setting
is $1.0$.

\noindent The width of lines can similarly be controlled with the
\indmodt{linewidth} modifier, and the width of the lines used to draw point
symbols can be controlled with the \indmodt{pointlinewidth} modifier. For
example:

\begin{verbatim}
set samples 25
plot sin(x) with points pointlinewidth 2
\end{verbatim}

\noindent In addition to setting these parameters on a per-plot basis, their
default values can also be changed. The command:

\begin{verbatim}
set pointlinewidth 2
\end{verbatim}

\noindent would set the default line width used when drawing \datapoint s. Both
here, and in the {\tt plot} command, the abbreviation {\tt plw} is valid. 

\subsection{Upper and Lower Limit Data Points}

PyXPlot can plot \datapoint s using the standard upper- and lower-limit
symbols.\index{lower-limit datapoints}\index{upper-limit datapoints} No special
syntax is required for this; these symbols are pointtypes\footnote{The
{\tt pointtype} modifier was introduced in
Section~\ref{pointtype_modifier}.} 12 and 13 respectively, obtained as follows:

\begin{verbatim}
plot 'upperlimits.dat' with points pointtype 12
plot 'lowerlimits.dat' with points pointtype 13
\end{verbatim}

\subsection{Drawing Arrows}
\label{arrows_plot_style} 

Data may be represented as arrows connecting two points on a plot by using the
\indpst{arrows} plot style.  This takes four columns of data -- $x_1$, $y_1$,
$x_2$ and $y_2$ -- and for each \datapoint\ draws an arrow from the point
$(x_1,y_1)$ to the point $(x_2,y_2)$.  Three different kinds of arrows can be
drawn: ones with normal arrow heads, ones with no arrow heads, which just
appear as lines, and ones with arrow heads on both ends. The syntax to obtain
these varieties is:

\begin{verbatim}
plot 'data.dat' with arrows_head
plot 'data.dat' with arrows_nohead
plot 'data.dat' with arrows_twohead
\end{verbatim}

The syntax {\tt with arrows} is a shorthand for {\tt with arrows\_head}.  This
plot style is analogous to the \indpst{vectors} plot style in Version~4 of
\gnuplot.

\subsection{Error Bars}

\index{errorbars}\label{errorbars}
In \gnuplot, when one uses errorbars, one can specify either the size of the
errorbar, or the minimum to maximum range of the errorbar. Both of these usages
share a common syntax, and \gnuplot's behaviour depends upon the number of
columns of data provided:

\begin{verbatim}
plot 'data.dat' with yerrorbars
\end{verbatim}

\noindent Given a \datafile\ with three columns, this takes the third column to
indicate the size of the $y$-errorbar. Given a four-column \datafile, it takes
the third and fourth columns to indicate the minimum to maximum range to be
marked out by the errorbar.

To avoid confusion, a different syntax is adopted in PyXPlot. The syntax:

\begin{verbatim}
plot 'data.dat' with yerrorbars
\end{verbatim}

\noindent always assumes that the third column of the \datafile\ indicates the
size of the errorbar, regardless of whether a fourth is present. The syntax:

\begin{verbatim}
plot 'data.dat' with yerrorrange
\end{verbatim}

\noindent always assumes that the third and fourth columns indicate the minimum
to maximum range of the errorbar.

\vspace{0.5cm}
For clarity, a complete list of the errorbar plot styles available in PyXPlot
is given below:

\begin{tabular}{p{2.5cm}p{7.5cm}}
\indpst{yerrorbars} & Vertical errorbars; size drawn from the third data column. \\
\indpst{xerrorbars} & Horizontal errorbars; size drawn from the third data column. \\
\indpst{xyerrorbars} & Horizontal and vertical errorbars; sizes drawn from the third and fourth data columns respectively.\\
\indpst{errorbars} & Shorthand for {\tt yerrorbars}. \\
\end{tabular}

\begin{tabular}{p{2.5cm}p{7.5cm}}
\indpst{yerrorrange} & Vertical errorbars; minimum drawn from the third data column, maximum from the fourth.\\
\indpst{xerrorrange} & Horizontal errorbars; minimum drawn from the third data column, maximum from the fourth.\\
\indpst{xyerrorrange} & Horizontal and vertical errorbars; horizontal minimum drawn from the third data column and maximum from the fourth; vertical minimum drawn from the fifth and maximum from the sixth.\\
\indpst{errorrange} & Shorthand for {\tt yerrorrange}. \\
\end{tabular}

\subsection{Plotting Functions with Errorbars, Arrows, or More}

In \gnuplot, when a function (as opposed to a \datafile) is plotted, only those
plot styles which accept two columns of data can be used -- for example,
{\tt lines} or {\tt points}. This means that it is not possible to plot a
function with errorbars. In PyXPlot, this is possible using the following
syntax:

\begin{verbatim}
plot f(x):g(x) with yerrorbars
\end{verbatim}

\noindent Two functions are supplied, separated by a colon; plotting proceeds
as if a \datafile\ had been supplied, containing values of $x$ in column 1,
values of $f(x)$ in column 2, and values of $g(x)$ in column 3. This may be
useful, for example, if $g(x)$ measures the intrinsic uncertainty in $f(x)$.
The {\tt using} modifier may also be used:

\begin{verbatim}
plot f(x):g(x) using 2:3
\end{verbatim}

Here, $g(x)$ would be plotted on the $y$-axis, against $f(x)$ on the $x$-axis.
It should be noted, however, that the range of values of $x$ used would still
correspond to the range of the plot's horizontal axis. If the above were to be
attempted with an autoscaling horizontal axis, the result might be rather
unexpected -- PyXPlot would find itself autoscaling the $x$-axis range to the
spread of values of $f(x)$, but find that this itself changed depending upon
the range of the $x$-axis.\footnote{We're aware that this is not good. Expect
it to change in a future release.}

\section{Barcharts and Histograms}
\label{barcharts}\index{bar charts}
\index{steps plot style@{\tt steps} plot style}
\index{fsteps plot style@{\tt fsteps} plot style}
\index{histeps plot style@{\tt histeps} plot style}
\index{impulses plot style@{\tt impulses} plot style}

\subsection{Basic Operation}

As in \gnuplot, bar charts and histograms can be produced using the
\indpst{boxes} plot style:

\begin{verbatim} 
plot 'data.dat' with boxes
\end{verbatim}

\noindent Horizontally, the interfaces between the bars are, by default, at the
midpoints along the $x$-axis between the specified \datapoint s (see, for
example, Figure~\ref{fig:ex_barchart2}a).  Alternatively, the widths of the
bars may be set using the {\tt set boxwidth} command. In this case, all of
the bars will be centred upon their specified $x$-coordinates, and have total
widths equalling that specified in the \indcmdt{set boxwidth}. Consequently, there may be
gaps between them, or they may overlap, as seen in
Figure~\ref{fig:ex_barchart2}(b).

\begin{figure}
\begin{center}
\includegraphics{examples/eps/ex_barchart2.eps}
\end{center}
\caption{A gallery of the various bar chart styles which PyXPlot can produce. See the text for more details.  The script and data file used to produce this image are available on the PyXPlot website at XXX.}
\label{fig:ex_barchart2}
\end{figure}

Having set a fixed box width, the default behaviour of scaling box widths
automatically may be restored either with the {\tt unset boxwidth} command,
or by setting the boxwidth to a negative width.

As a third alternative, it is also possible to specify different widths for
each bar manually, in an additional column of the input \datafile. To achieve
this behaviour, the \indpst{wboxes} plot style should be used:

\begin{verbatim} 
plot 'data.dat' using 1:2:3 with wboxes
\end{verbatim}

\noindent This plot style expects three columns of data to be provided: the
$x$- and $y$-coordinates of each bar in the first two, and the width of the
bars in the third.  Figure~\ref{fig:ex_barchart2}(c) shows an example of this
plot style in use.

By default, the bars originate from the line $y=0$, as is normal for a
histogram. However, should it be desired for the bars to start from a different
vertical point, this may be achieved by using the \indcmdt{set boxfrom},
for example:

\begin{verbatim} 
set boxfrom 5
\end{verbatim}

\noindent In this case, all of the bars would now originate from the line
$y=5$. Figure~\ref{fig:ex_barchart1}(1) shows the kind of effect that is
achieved; for comparison, Figure~\ref{fig:ex_barchart1}(b) shows the same bar
chart with the boxes starting from their default position of $y=0$.

\begin{figure}
\begin{center}
\includegraphics{examples/eps/ex_barchart1.eps}
\end{center}
\caption{A second gallery of the various bar chart styles which PyXPlot can produce. See the text for more details.  The script and data file used to produce this image are available on the PyXPlot website at XXX.}
\label{fig:ex_barchart1}
\end{figure}

The bars may be filled using the {\tt with} \indmodt{fillcolour} modifier,
followed by the name of a colour:

\begin{verbatim} 
plot 'data.dat' with boxes fillcolour blue
plot 'data.dat' with boxes fc 4
\end{verbatim}

\noindent Figures~\ref{fig:ex_barchart2}(b) and (d) demonstrate the use of
filled bars.

Finally, the \indpst{impulses} plot style, as in \gnuplot, produces bars of zero
width; see Figure~\ref{fig:ex_barchart1}(c) for an example.

\subsection{Stacked Bar Charts}

If several \datapoint s are supplied to the \indpst{boxes} or \indpst{wboxes}
plot styles at a common $x$-coordinate, then the bars are stacked one above
another into a stacked barchart. Consider the following \datafile:

\begin{verbatim} 
1 1
2 2
2 3
3 4
\end{verbatim}

\noindent The second bar at $x=2$ would be placed on top of the first, spanning
the range $2<y<5$, and having the same width as the first. If plot colours are
being automatically selected from the palette, then a different palette colour
is used to plot the upper bar.

\subsection{Steps}

The plot styles met so far plot data as solid bars, with left, right and top
sides all drawn. Data may also be plotted with {\it steps}, with the left and
right sides of each bar omitted. Some examples are shown in
Figures~\ref{fig:ex_barchart1}(d), (e) and (f).  As is illustrated in these
panels, three flavours of steps are available, exactly as in \gnuplot:

\begin{verbatim}
plot 'data.dat' with steps 
plot 'data.dat' with fsteps 
plot 'data.dat' with histeps
\end{verbatim}

\noindent When using the \indpst{steps} plot style, the \datapoint s specify the
right-most edges of each step. When using the \indpst{fsteps} plot style, they
specify the left-most edges of the steps. The \indpst{histeps} plot style works
rather like the {\tt boxes} plot style; the interfaces between the steps occur
at the horizontal midpoints between the \datapoint s.

\section{Choosing which Data to Plot}
\label{select_modifier} 
As well as the {\tt plot} command's {\tt index}, {\tt using} and {\tt every}
modifiers, which allow users to plot subsets of data from \datafile s, it also
has a further modifier, \indmodt{select}. This can be used to plot only those
\datapoint s in a \datafile\ which specify some given criterion. For example:

\begin{verbatim}
plot 'data.dat' select ($8>5)
plot sin(x) select (($1>0) and ($2>0))
\end{verbatim}

\noindent In the second example, two selection criteria are given, combined
with the logical {\tt and} operator. A full list of all of the operators
recognised by PyXPlot, including logical operators, was given in Chapter~2; see
Table~\ref{operators_table}.  The select modifier has many applications, for
example, plotting two-dimensional slices from three-dimensional datasets and
plotting subsets of \datapoint s from \datafile s.

When plotting using the \indpst{lines}, the default behaviour is for the lines
plotted not to be broken if a set of datapoints are removed by the select
modifier.  However, this functionality is sometimes useful.  To cause the
plotted line to break when points are removed the \indmodt{discontinuous}\ is
supplied.  For example:

\begin{verbatim}
plot sin(x) select ($1>0) discontinuous
\end{verbatim}

plots a disconnected set of peaks from the sine function.

\section{Horizontally arranged \Datafile s}

\index{horizontal datafiles}\index{datafiles!horizontal}\index{using rows
modifier@{\tt using rows} modifier}\index{using columns modifier@{\tt using
columns} modifier}\label{horizontal_datafiles} The command syntax for plotting
columns of \datafile s against one another was previously described in
Section~\ref{plot_datafiles}.  In an extension of what is possible in \gnuplot,
PyXPlot also allows one to plot {\it rows} of data against one another in
horizontally-arranged \datafile s.  For this, the keyword \indkeyt{rows} is
placed after the {\tt using} modifier:

\begin{verbatim}
plot 'data.dat' index 1 using rows 1:2
\end{verbatim}

\noindent For completeness, the syntax {\tt using} \indkeyt{columns} is also
accepted, to specify the default behaviour of plotting columns against one
another:

\begin{verbatim}
plot 'data.dat' index 1 using columns 1:2
\end{verbatim}

When plotting horizontally-arranged \datafile s, the meanings of the {\tt
index} and {\tt every} modifiers (see Section~\ref{plot_datafiles}) are altered
slightly. The former continues to refer to vertically-displaced blocks of data
separated by two blank lines.  Blocks, as referenced in the {\tt every}
modifier, likewise continue to refer to vertically-displaced blocks of
\datapoint s, separated by single blank lines. The row numbers passed to the
{\tt using} modifier are counted from the top of the current block.

However, the line-numbers specified in the \indmodt{every} modifier -- i.e.\
variables $a$, $c$ and $e$ in the system introduced in
Section~\ref{introduce_every} -- now refer to horizontal columns, rather than
lines. For example:

\begin{verbatim}
plot 'data.dat' using rows 1:2 every 2::3::9
\end{verbatim}

\noindent would plot the data in row 2 against that in row 1, using only the
values in every other column, between columns 3 and 9.

\section{Configuring Axes}
\label{axis_extensions}\label{ranges_multiaxes}\label{multiple_axes}

By default, plots have only one $x$-axis and one $y$-axis. Further parallel
axes can be added and configured via statements such as:\index{axes
modifier@{\tt axes} modifier}\indcmd{set axis}

\begin{verbatim}
set x3label 'foo'
plot sin(x) axes x3y1
set axis x3
\end{verbatim}

\noindent In the top statement, a further horizontal axis, called the
$x3$-axis, is implicitly created by giving it a label. In the next, the {\tt
axes} modifier is used to tell the {\tt plot} command to plot data using the
horizontal $x3$-axis and the vertical $y$-axis. Here again, the axis would be
implicitly created if it didn't already exist.  In the third statement, an
$x3$-axis is explicitly created.

Unlike \gnuplot, which allows only a maximum of two parallel axes to be
attached to any given plot, PyXPlot allows an unlimited number of axes to be
used. Odd-numbered $x$-axes appear below the plot, and even numbered $x$-axes
above it; a similar rule applies for $y$-axes, to the left and to the right.
This is illustrated in Figure~\ref{fig:ex_multiaxes}.

\begin{figure}
\begin{center}
\includegraphics{examples/eps/ex_multiaxes.eps}
\end{center}
\caption{A plot demonstating the use of large numbers of axes. Odd-numbered
$x$-axes appear below the plot, and even numbered $x$-axes above it; a similar
rule applies for $y$-axes, to the left and to the right.}
\label{fig:ex_multiaxes}
\end{figure}

As discussed in the previous chapter, the ranges of axes can be set either
using the \indcmdt{set xrange}, or within the {\tt plot} command. The following
two statements would set equivalent ranges for the $x3$-axis:

\begin{verbatim}
set x3range [-2:2]
plot [:][:][:][:][-2:2] sin(x) axes x3y1
\end{verbatim}

\noindent As usual, the first two ranges specified in the {\tt plot} command
apply to the $x$- and $y$-axes. The next pair apply to the $x2$- and $y2$-axes,
and so forth.

\index{axes!removal}\index{removing axes}\index{hidden
axes}\label{axis_removal} Having made axes with the above commands, they may
subsequently be removed using the \indcmdt{unset axis} as follows:

\begin{verbatim}
unset axis x3
unset axis x3x5y3 y7
\end{verbatim}

\noindent The top statement, for example, would remove axis $x3$. The command
{\tt unset axis} on its own, with no axes specified, returns all axes to
their default configuration.  The special case of {\tt unset axis x1} does
not remove the first $x$-axis -- it cannot be removed -- but instead returns it
to its default configuration.

It should be noted that if the following two commands are typed in succession,
the second may not entirely negate the first:

\begin{verbatim}
set x3label 'foo'
unset x3label 'foo'
\end{verbatim}

\noindent If an $x3$-axis did not previously exist, then the first will have
implicitly created one. This would need to be removed with the {\tt unset axis
x3} command if it was not desired.

A subtly different task is that of removing labels from axes, or setting axes
not to display. To achieve this, a number of special axis labels are used.
Labelling an axis \indkeyt{nolabels} has the effect that no title or numerical
labels are placed upon it. Labelling it\label{nolabelstics}
\indkeyt{nolabelstics} is stronger still; this removes all tick marks from it
as well (similar in effect to the {\tt set noxtics} command; see below).
Finally, labelling it \indkeyt{invisible} makes an axis completely invisible.

Labels may be placed on such axes, by suffixing the magic keywords above with a
colon and the desired title. For example:

\begin{verbatim}
set xlabel 'nolabels:Time'
\end{verbatim}

\noindent would produce an $x$-axis with no numeric labels, but a label of
`Time'.

In the unlikely event of wanting
to label a normal axis with one of these magic words\index{axes!reserved
labels}\index{magic axis labels}, this may be achieved by prefixing the magic
word with a space. There is one further magic axis label, {\tt linkaxis},
which will be described in Section~\ref{linked_axes}.

The ticks of axes can be configured to point either inward, towards the plot,
as is the default, or outward towards the axis labels, or in both directions.
This is achieved using the {\tt set xticdir} command, for example:

\begin{verbatim}
set xticdir inward
set y2ticdir outward
set x2ticdir both
\end{verbatim}

The position of ticks along each axis can be configured with the \indcmdt{set
xtics}. The appearance of ticks along any axis can be turned off with the
\indcmdt{set noxtics}. The syntax for these is given below:

\begin{verbatim}
set xtics { axis | border | inward | outward | both }
          {  autofreq
           | <increment>
           | <minimum>, <increment> { , <maximum> }
           | (     {"label"} <position>
               { , {"label"} <position> } .... )
          }
set noxtics
show xtics
\end{verbatim}

The keywords \indkeyt{inward}, \indkeyt{outward} and \indkeyt{both} alter the
directions of the ticks, and have the same effect as in the \indcmdt{set
xticdir}. The keyword \indkeyt{axis} is an alias for \indkeyt{inward}, and
\indkeyt{border} an alias for \indkeyt{outward}; both are provided for
compatibility with \gnuplot. If the keyword \indkeyt{autofreq} is given, the
automatic placement of ticks along the axis is restored.

If {\tt <minimum>, <increment>, <maximum>} are specified, then ticks are
placed at evenly spaced intervals between the specified limits. In the case of
logarithmic axes, {\tt <increment>} is applied multiplicatively.

Alternatively, the final form allows ticks to be placed on an axis
individually, and each given its own textual label.

The following example sets the $x1$-axis to have tick marks at
$x=0.05, 0.1, 0.2$ and $0.4$.  The $x2$-axis has symbolically labelled tics at
$x=1/\pi, 2/\pi$, etc., pointing outwards from the plot.  The left-hand
$y$-axis has tick marks placed automatically whereas the $y2$-axis has no tics
at all.  The overall effect is shown in Figure~\ref{fig:ex_axistics}.

\begin{verbatim}
set log x1x2
set grid x2
set xtics 0.05, 2, 0.4
set x2tics border \
     ("$\frac{1}{\pi}$" 1/pi,      "$\frac{1}{2\pi}$" 1/(2*pi), \
      "$\frac{1}{3\pi}$" 1/(3*pi), "$\frac{1}{4\pi}$" 1/(4*pi), \
      "$\frac{1}{5\pi}$" 1/(5*pi), "$\frac{1}{6\pi}$" 1/(6*pi))
set ytics autofreq
set noy2tics
\end{verbatim}

\begin{figure}
\begin{center}
\includegraphics{examples/eps/ex_axistics.eps}
\end{center}
\caption{A plot illustrating some of the crossing points of the function
$\exp(x)\sin(1/x)$.  The commands used to set up ticking on the axes in this
plot are as given in the text.}
\label{fig:ex_axistics}
\end{figure}

Minor tick marks can be placed on axes with the \indcmdt{set mxtics}, which has
the same syntax as above.

\section{Keys and Legends}\index{keys}\index{legends}

By default, plots are displayed with legends in their top-right corners. The
textual description of each dataset is drawn by default from the command used
to plot it. Alternatively, the user may specify his own description for each
dataset by following the {\tt plot} command with the \indmodt{title} modifier,
as follows:

\begin{verbatim}
plot sin(x) title 'A sine wave'
plot cos(x) title ''
\end{verbatim}

In the lower case, a blank title is specified, in which case PyXPlot makes no
entry for the dataset in the legend. This is useful if it is desired to place
some but not all datasets into the legend of a plot.  Alternatively, the
production of the legend can be completely turned off for all datasets using
the command \indcmdts{set nokey}. The opposite effect can be achieved by the
\indcmdt{set key}.

This latter command can also be used to dictate where on the plot the legend
should be placed, using a syntax along the lines of:

\begin{verbatim}
set key top right
\end{verbatim}

The following recognised positioning keywords are self-explanatory:
\indkeyt{top}, \indkeyt{bottom}, \indkeyt{left}, \indkeyt{right},
\indkeyt{xcentre} and \indkeyt{ycentre}. The word \indkeyt{outside} places the
key outside the plot, on its right side. The words \indkeyt{below} and
\indkeyt{above} place legends below and above the plot respectively.

In addition, two positional offset coordinates may be specified after such
keywords -- the first value is assumed to be an $x$-offset, and the second a
$y$-offset, both in units of centimetres. For example:

\begin{verbatim}
set key bottom left 0.0 -2
\end{verbatim}

\noindent would display a key below the bottom left corner of the graph.

By default, entries in the key are placed in a single vertical list. They can
instead be arranged into a number of columns by means of the \indcmdt{set
keycolumns}. This should be followed by the integer number of desired columns,
for example:

\begin{verbatim}
set keycolumns 2
\end{verbatim}

\noindent An example of a plot with a two-column legend is given in
Figure~\ref{fig:ex_legends}.

\begin{figure}
\begin{center}
\includegraphics{examples/eps/ex_legends.eps}
\end{center}
\caption{This plot shows how rapidly three functions, often approximated as
$x$, deviate from that approximation.  Furthermore it is an example of a plot
with a two-column legend, positioned below the plot using {\tt set key below}.
The complete script used to produce the plot can be found on the PyXPlot
website at XXX.}
\label{fig:ex_legends}
\end{figure}

\section{The {\tt linestyle} Keyword}

At times, the string of style keywords placed after the {\tt with} modifier
in {\tt plot} commands can grow rather unwieldy in its length. For clarity,
frequently used plot styles can be stored as {\it linestyles}; despite the name,
this is true of styles involving points as well as lines. The syntax for
setting a linestyle is:

\begin{verbatim}
set linestyle 2 points pointtype 3
\end{verbatim}

\noindent where the {\tt 2} is the identification number of the linestyle. In a
subsequent {\tt plot} statement, this linestyle can be recalled as follows:

\begin{verbatim}
plot sin(x) with linestyle 2
\end{verbatim}

\section{Colour Plotting}

\index{colours!setting for datasets} In the {\tt with} clause of the plot
command, the modifier {\tt colour}, which can be abbreviated to
`{\tt c}', can be used to manually select the colour in which each dataset
is to be plotted. It should be followed either by an integer, to set a colour
from the present palette, or by a colour name. A list of valid colour names is
given in Section~\ref{colour_names}. For example:

\begin{verbatim}
plot sin(x) with c 5
plot sin(x) with colour blue
\end{verbatim}

\noindent The {\tt colour} modifier can also be used when defining linestyles.

\index{palette}\index{colours!setting the palette}
PyXPlot has a palette of colours which it assigns sequentially to datasets when
colours are not manually assigned. This is also the palette to which integers
passed to {\tt set colour} refer -- the {\tt 5} above, for example. It may be
set using the \indcmdt{set palette}, which differs in syntax from \gnuplot. It
should be followed by a comma-separated list of colours, for example:

\begin{verbatim}
set palette BrickRed, LimeGreen, CadetBlue
\end{verbatim}

\noindent Another way of setting the palette, in a configuration file, is
described in Section~\ref{config_files}; a list of valid colour names is given
in Section~\ref{colour_names}.

\section{Plotting Many Files at Once}

\index{globbing}\index{wildcards}\index{datafiles!globbing}

PyXPlot allows the wildcards {\tt *} and {\tt ?} to be used in the filenames of
\datafile s supplied to the \indcmdt{plot}.  For example, the following would
plot all \datafile s in the current directory with a {\tt .dat} suffix, using
the same plot options:

\begin{verbatim}
plot '*.dat' with linewidth 2
\end{verbatim}

\noindent In the legend, full filenames are displayed, allowing the \datafile s
to be distinguished.  As in \gnuplot, a blank filename passed to the plot
command causes the last used \datafile\ to be used again, for example:

\begin{verbatim}
plot 'data.dat' using 1:2, '' using 2:3
\end{verbatim}

\noindent or even:

\begin{verbatim}
plot '*.dat' using 1:2, '' using 2:3
\end{verbatim}

The {\tt *} and {\tt ?} wildcards can be used in a similar fashion in the
\indcmdt{load}.

\section{Backing Up Over-Written Files}

\index{overwriting files}\index{backup files}\label{filebackup}

By default, when graphical output is sent to a file -- i.e.\ a postscript file or
a bitmap image -- pre-existing files are overwritten if their filenames match
that of the file which PyXPlot generates. This behaviour may be changed with
the \indcmdt{set backup}, which has the syntax:

\begin{verbatim}
set backup
set nobackup
\end{verbatim}

When this switch is turned on, pre-existing files will be renamed with a tilde
at the end of their filenames, rather than being overwritten.

