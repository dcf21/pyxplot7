% FIRST_STEPS.TEX
%
% The documentation in this file is part of PyXPlot
% <http://www.pyxplot.org.uk>
%
% Copyright (C) 2006-7 Dominic Ford <coders@pyxplot.org.uk>
%
% $Id$
%
% PyXPlot is free software; you can redistribute it and/or modify it under the
% terms of the GNU General Public License as published by the Free Software
% Foundation; either version 2 of the License, or (at your option) any later
% version.
%
% You should have received a copy of the GNU General Public License along with
% PyXPlot; if not, write to the Free Software Foundation, Inc., 51 Franklin
% Street, Fifth Floor, Boston, MA  02110-1301, USA

% ----------------------------------------------------------------------------

% LaTeX source for the PyXPlot Users' Guide

\chapter{First Steps With PyXPlot}
\label{gnuplot_intro}

In this chapter, I shall provide a brief overview of the basic operation of
PyXPlot, essentially covering those areas of syntax which are borrowed directly
from gnuplot. Users who are already familiar with gnuplot may wish to skim or
skip this chapter, though Section~\ref{sec:latex_incompatibility}, describing
the use of \LaTeX\ to render text, and Section~\ref{missing_features},
detailing which parts of gnuplot's interface are and are not supported in
PyXPlot, may be of interest. In the following chapter, I shall go on to
describe PyXPlot's extensions of gnuplot's interface.

Describing gnuplot's interface in its entirety is a substantial task, and what
follows is only an overview; novice users can find many excellent tutorials on
the web which will greatly supplement what is provided below.

\section{Getting Started}

The simplest way to start PyXPlot is simply to type ``\texttt{pyxplot}'' at a
shell prompt to start an interactive session. A PyXPlot command-line prompt will
appear, into which commands can be typed. PyXPlot can be exited either by
typing ``\texttt{exit}'', ``\texttt{quit}'', or by pressing CTRL-D.\index{exit
command@\texttt{exit} command}\index{quit command@\texttt{quit} command}

Alternatively, a list of commands to be executed may be stored in a command
script, and executed by passing the filename of the command script to PyXPlot
on the shell command line, for example:\index{command line syntax}

\begin{verbatim}
pyxplot foo
\end{verbatim}

\noindent In this case, PyXPlot would exit immediately after finishing
executing the commands from the file \texttt{foo}. Several filenames may be
passed on the command line, to be executed in sequence:

\begin{verbatim}
pyxplot foo1 foo2 foo3
\end{verbatim}

\noindent Wildcards can also be used; the following would execute all command
scripts in the presenting working directory whose filenames end with a
\texttt{.plot} suffix:

\begin{verbatim}
pyxplot *.plot
\end{verbatim}

It is possible to use PyXPlot both interactively, and from command scripts, in
the same session. One way to do this is to pass the magic filename `--' on the
command line:

\begin{verbatim}
pyxplot foo1 - foo2
\end{verbatim}

\noindent This magic filename represents an interactive session, which
commences after the execution of \texttt{foo1}, and should be terminated in the
usual way after use, with the ``\texttt{exit}'' or ``\texttt{quit}'' commands.
Afterwards, the command script \texttt{foo2} would execute.

From within an interactive session, it is possible to run a command script
using the \texttt{load}\index{load command@\texttt{load} command} command:

\begin{verbatim}
pyxplot> load 'foo'
\end{verbatim}

\noindent This example would have the same effect as typing the contents of the
file \texttt{foo} into the present session.

A related command is ``\texttt{save}''\index{save command@\texttt{save}
command}, which stores a history of the commands executed in the present
interactive session to file.

All command files can include comment lines, which should begin with a hash
character, for example:\index{comment lines}\index{command scripts!comment
lines}

\begin{verbatim}
# This is a comment
\end{verbatim}

Long commands may be split over multiple lines in the script by terminating
each line of it with a backslash character, whereupon the following line will
be appended to the end of it.

\section{First Plots}
\label{first_plots}

The basic workhorse command of PyXPlot is the \texttt{plot} command, which is
used to produce all plots. The following simple example would plot the function
$\sin(x)$:\index{plot command@\texttt{plot} command}

\begin{verbatim}
plot sin(x)
\end{verbatim}

\noindent It is also possible to plot data from files. The following would plot
data from a file `\texttt{datafile}', taking the $x$-coordinate of each point
from the first column of the datafile, and the $y$-coordinate from the second.
The datafile is assumed to be in plain text format, with columns separated by
whitespace and/or commas\footnote{If the filename of a datafile ends with a
\texttt{.gz} suffix, it is assuming to be gzipped plaintext, and is decoded
accordingly.}:

\begin{verbatim}
plot 'datafile'
\end{verbatim}

Several items can be plotted on the same graph by separating them by commas:

\begin{verbatim}
plot 'datafile', sin(x), cos(x)
\end{verbatim}

\noindent It is possible to define one's own variables and functions, and then
plot them:

\begin{verbatim}
a = 2
b = 1
c = 1.5
f(x) = a*(x**2) + b*x + c
plot f(x)
\end{verbatim}

\noindent To unset a variable or function once it has been set, the following
syntax should be
used:\index{variables!unsetting}\index{functions!unsetting}\index{unsetting
variables}

\begin{verbatim}
a =
f(x) =
\end{verbatim}

\section{Axis Labels and Titles}
\label{sec:latex_incompatibility}

Labels can be applied to the two axes of the plot, and a title put at the top:

\begin{verbatim}
set xlabel 'This is the X axis'
set ylabel 'This is the Y axis'
set title 'A plot of sin(x)'
plot sin(x)
\end{verbatim}

\begin{figure}
\begin{center}
\includegraphics{examples/eps/ex_axislab.eps}
\end{center}
\caption{A simple first plot with axis labels. See the text for more details.}
\label{fig:ex_axislab}
\end{figure}

\noindent The output from this simple example is shown in
Figure~\ref{fig:ex_axislab}. All such text labels should be placed between
either single (') or double (") quotes. They are displayed using \LaTeX, and so
any \LaTeX\ commands can be used, for example to put equations on axes:

\begin{verbatim}
set xlabel '$\frac{x^2}{c^2}$'
\end{verbatim}

\noindent As a caveat, however, this does mean that care needs to be taken to
escape any of \LaTeX's reserved characters -- i.e.:
$\backslash$~\&~\%~\#~\{~\}~\$~\_~\^{} or $\sim$.

Because of the use of quotes to delimit text labels, special care needs to be
taken when apostrophe and quote characters are used. The following command
would raise an error, because the apostrophe would be interpretted as marking
the end of the text label:

\begin{verbatim}
set xlabel 'My plot's X axis'
\end{verbatim}

\noindent The following would achieve the desired effect:

\begin{verbatim}
set xlabel "My plot's X axis"
\end{verbatim}

\noindent To allow one to render \LaTeX\ strings containing both single and
double quote characters -- for example, to put a German umlaut on the name
``J\"org'' in the \LaTeX\ string ``\texttt{J$\backslash$"org's Data}'' --
PyXPlot recognises the backslash character to be an escape character when
followed by either ' or " in a \LaTeX\ string. This is the \textit{only} case
in which PyXPlot considers $\backslash$ an escape character. To render the
example string above, one would type:

\begin{verbatim}
set xlabel "J\\"org's Data"
\end{verbatim}

Two backslashes are used.  The first backslash is the \LaTeX\ escape character
used to produce the umlaut; the second is a PyXPlot escape character, used so
that the " character is not interpretted as delimiting the string. \index{escape
characters}\index{quote characters}\index{special characters}

Having set labels and titles, they may be removed thus:

\begin{verbatim}
set xlabel ''
set ylabel ''
set title ''
\end{verbatim}

\noindent These are two other ways of removing the title from a plot:

\begin{verbatim}
set notitle
unset title
\end{verbatim}

The \texttt{unset} command\index{unset command@\texttt{unset} command} may be
followed by essentially any word that can follow the \texttt{set} command, such
as \texttt{xlabel} or \texttt{title}, to return that setting to its default
configuration. The \texttt{reset} command\index{reset command@\texttt{reset}
command} restores all configurable parameters to their default states.

\section{Operators and Functions}

As has already been seen above, some mathematical functions such as $\sin(x)$ are
pre-defined within PyXPlot. A list of frequently-used functions which are
predefined in PyXPlot is given in Table~\ref{functions_table}\footnote{Users
with some experience in Python may be interested to know that all of the
functions in the core and math modules are recognised.}. A list of
operators recognised by PyXPlot is given in Table~\ref{operators_table}.

\begin{table}
\begin{longtable}{|lp{8cm}|}
\hline
acos($x$)&
Return the arc cosine (measured in radians) of $x$.\\
asin($x$)&
Return the arc sine (measured in radians) of $x$.\\
atan($x$)&
Return the arc tangent (measured in radians) of $x$.\\
atan2($y,x$)&
Return the arc tangent (measured in radians) of $y/x$. Unlike $\mathrm{atan}(y/x)$, the signs of both $x$ and $y$ are considered.\\
ceil($x$)&
Return the ceiling of $x$ as a float. This is the smallest integral value $\geq x$.\\
cos($x$)&
Return the cosine of $x$ (measured in radians).\\
cosh($x$)&
Return the hyperbolic cosine of $x$.\\
degrees($x$)&
Convert angle $x$ from radians to degrees.\\
exp($x$)&
Return $e$ raised to the power of $x$.\\
fabs($x$)&
Return the absolute value of the float $x$.\\
floor($x$)&
Return the floor of $x$ as a float. This is the largest integral value $\leq x$.\\
fmod($x,y$)&
Return fmod(x, y), according to platform C.  x \% y may differ.\\
frexp($x$)&
Return the mantissa and exponent of $x$, as pair $(m,e)$. $m$ is a float and $e$ is an int, such that $x = m \times 2^e$. If $x$ is 0, $m$ and $e$ are both 0.  Else $0.5 \leq \mathrm{abs}(m) < 1.0$.\\
hypot($x,y$)&
Return the Euclidean distance, $\sqrt{x^2 + y^2}$.\\
ldexp($x, i$)&
Return $x \times 2^i$. \\
log($x[,base]$)&
Return the logarithm of $x$ to the given base. If the base not specified, returns the natural logarithm (base $e$) of $x$.\\
log10($x$)&
Return the base 10 logarithm of $x$.\\
max($x$,$y$,...)&
Return the greatest of the numerical values supplied.\\
min($x$,$y$,...)&
Return the least of the numerical values supplied.\\
modf($x$)&
Return the fractional and integer parts of $x$.  Both results carry the sign of $x$.  The integer part is returned as a real.\\
pow($x,y$)&
Return $x^y$.\\
radians($x$)&
Converts angle $x$ from degrees to radians.\\
sin($x$)&
Return the sine of $x$ (measured in radians).\\
sinh($x$)&
Return the hyperbolic sine of $x$.\\
sqrt($x$)&
Return the square root of $x$.\\
tan($x$)&
Return the tangent of $x$ (measured in radians).\\
tanh($x$)&
Return the hyperbolic tangent of $x$.\\
\hline
\end{longtable}
\caption{A list of mathematical functions which are pre-defined within PyXPlot.}
\label{functions_table}
\end{table}

\begin{table}
\begin{longtable}{|lp{8cm}|}
\hline
\texttt{+} & Algebraic sum \\
\texttt{-} & Algebraic subtraction \\
\texttt{*} & Algebraic multiplication \\
\texttt{**} & Algebraic exponentiation \\
\texttt{/} & Algebraic division \\
\texttt{\%} & Modulo operator \\
\texttt{<<} & Left binary shift \\
\texttt{>>} & Right binary shift \\
\texttt{\&} & Binary and \\
\texttt{|} & Binary or \\
\texttt{\^{}} & Logical exclusive or \\
\texttt{<} & Magnitude comparison \\
\texttt{>} & Magnitude comparison \\
\texttt{<=} & Magnitude comparison \\
\texttt{>=} & Magnitude comparison \\
\texttt{==} & Equality comparison \\
\texttt{!=} & Equality comparison \\
\texttt{<>} & Alias for \texttt{!=} \\
\texttt{and} & Logical and \\
\texttt{or} & Logical or \\
\hline
\end{longtable}
\caption{A list of mathematical operators which PyXPlot recognises.}
\label{operators_table}
\end{table}

\section{Plotting Datafiles}
\label{plot_datafiles}

In the simple example of the previous section, we plotted the first column of a
datafile against the second. It is also possible to plot any arbitrary column
of a datafile against any other; the syntax for doing this is:

\begin{verbatim}
plot 'datafile' using 3:5
\end{verbatim}

\noindent This example would plot the fifth column of the file
\texttt{datafile} against the third. As mentioned above, columns in datafiles
can be separated using whitespace and/or commas, which means that PyXPlot is
compatible both with the format used by gnuplot, and also with
comma-separated-value (CSV)\index{csv files} files which many
spreadsheets\index{spreadsheets, importing data from} produce. Algebraic
expressions may also be used in place of column numbers, for example:

\begin{verbatim}
plot 'datafile' using (3+$1+$2):(2+$3)
\end{verbatim}

\noindent In algebraic expressions, column numbers should be prefixed by dollar
signs, to distinguish them from numerical constants. The example above would
plot the sum of the values in the first two columns of the datafile, plus
three, on the horizontal axis, against two plus the value in the third column
on the vertical axis. A more advanced example might be:

\newpage % WHY IS THIS NECESSARY??

\begin{verbatim}
plot 'datafile' using 3.0:$($2)
\end{verbatim}

\noindent This would place all of the datapoints on the line $x=3$, drawing
their vertical positions from the value of some column $n$ in the datafile,
where the value of $n$ is itself read from the second column of the datafile.

Later, in Section~\ref{horizontal_datafiles}, I shall discuss how to plot rows
of datafiles against one another, in horizontally arranged datafiles.

It is also possible to plot data from only a range of lines within a datafile.
When PyXPlot reads a datafile, it looks for any blank lines in the file. It
divides the datafile up into ``data blocks'', each being separated by single
blank lines. The first datablock is numbered 0, the next 1, and so on.
\index{datafile format}

When two or more blank lines are found together, the datafile is divided up
into ``index blocks''. Each index block may be made up of a series of data
blocks. To clarify this, a labelled example datafile is shown in
Figure~\ref{sample_datafile}.

\begin{figure}
\begin{tabular}{p{2.2cm}l}
\hline
\texttt{0.0 \ 0.0} & Start of index 0, data block 0. \\
\texttt{1.0 \ 1.0} & \\
\texttt{2.0 \ 2.0} & \\
\texttt{3.0 \ 3.0} & \\
                   & A single blank line marks the start of a new data block. \\
\texttt{0.0 \ 5.0} & Start of index 0, data block 1. \\
\texttt{1.0 \ 4.0} & \\
\texttt{2.0 \ 2.0} & \\
                   & A double blank line marks the start of a new index. \\
                   & ... \\
\texttt{0.0 \ 1.0} & Start of index 1, data block 0. \\
\texttt{1.0 \ 1.0} & \\
                   & A single blank line marks the start of a new data block. \\
\texttt{0.0 \ 5.0} & Start of index 1, data block 1. \\
                   & $<$etc$>$ \\
\hline
\end{tabular}
\caption{An example PyXPlot datafile -- the datafile is shown in the two left-hand columns, and commands are shown to the right.}
\label{sample_datafile}
\end{figure}

By default, when a datafile is plotted, all data blocks in all index blocks are
plotted. To plot only the data from one index block, the following syntax may
be used:

\begin{verbatim}
plot 'datafile' index 1
\end{verbatim}

\noindent To achieve the default behaviour of plotting all index blocks, the
\texttt{index} modifier should be followed by a negative number.\index{index
modifier@\texttt{index} modifier}

It is also possible to specify which lines and/or data blocks to plot from
within each index. For this purpose the \texttt{every} modifier is used, which
takes six values, separated by colons:\index{every modifier@\texttt{every}
modifier}

\begin{verbatim}
plot 'datafile' every a:b:c:d:e:f
\end{verbatim}

The values have the following meanings:

\begin{longtable}{p{1.0cm}p{10.5cm}}
$a$ & Plot data only from every $a\,$th line in datafile. \\
$b$ & Plot only data from every $b\,$th block within each index block. \\
$c$ & Plot only from line $c$ onwards within each block. \\
$d$ & Plot only data from block $d$ onwards within each index block. \\
$e$ & Plot only up to the $e\,$th line within each block. \\
$f$ & Plot only up to the $f\,$th block within each index block. \\
\end{longtable}

\noindent Any or all of these values can be omitted, and so the following would
both be valid statements:

\begin{verbatim}
plot 'datafile' index 1 every 2:3
plot 'datafile' index 1 every :::3
\end{verbatim}

\noindent The first would plot only every other data point from every third
data block; the second from the third line onwards within each data block.

A final modifier for selecting which parts of a datafile are plotted is
\texttt{select}, which plots only those data points which satisfy some given
criterion. This is 
%an extension of gnuplot's original syntax, and is 
described in Section~\ref{select_modifier}.

\section{Directing Where Output Goes}
\label{directing_output}

By default, when PyXPlot is used interactively, all plots are displayed on the
screen. It is also possible to produce postscript output, to be read into other
programs or embedded into \LaTeX\ documents, as well as a variety of other
graphical formats. The \texttt{set terminal} command\index{set terminal
command@\texttt{set terminal} command}\footnote{gnuplot users should note that
the syntax of the \texttt{set terminal} command in PyXPlot is rather different;
see Section~\ref{set_terminal2}.} is used to specify the output format that is
required, and the \texttt{set output} command\index{set output
command@\texttt{set output} command} the file to which output should be
directed. For example,

\begin{verbatim}
set terminal postscript
set output 'myplot.eps'
plot sin(x)
\end{verbatim}

\noindent would produce a postscript plot of $\sin(x)$ to the file
\texttt{myplot.eps}.

The \texttt{set terminal} command can also be used to configure further aspects
of the output file format. For example, the following would produce
black-and-white and colour output respectively:

\begin{verbatim}
set terminal monochrome
set terminal colour
\end{verbatim}

\noindent The former is useful for preparing plots for black-and-white
publications, the latter for preparing plots for colourful presentations.

Both encapsulated and non-encapsulated postscript can be produced. The former
is recommended for producing figures to embed into documents, the latter for
plots which are to be printed without further processing. The
\texttt{postscript} terminal produces the latter; the \texttt{eps} terminal
should be used to produce the former.  Similarly the \texttt{pdf} terminal
produces pdf files:

\begin{verbatim}
set terminal postscript
set terminal eps
set terminal pdf
\end{verbatim}

It is also possible to produce plots in the gif, png and jpeg graphic formats,
as follows:

\begin{verbatim}
set terminal gif
set terminal png
set terminal jpg
\end{verbatim}

More than one of the above keywords can be combined on a single line, for
example:

\begin{verbatim}
set terminal postscript colour
set terminal gif monochrome
\end{verbatim}

To return to the default state of displaying plots on screen, the \texttt{x11}
terminal should be selected:

\begin{verbatim}
set terminal x11
\end{verbatim}

For more details of the \texttt{set terminal} command, including how to produce
transparent gifs and pngs, see Section~\ref{set_terminal2}.

We finally note that, after changing terminals, the \texttt{replot} command is
especially useful; it repeats the last \texttt{plot} command.\index{replot
command@\texttt{replot} command}. If any plot items are placed after it, they
are added to the last plot.

\section{Data Styles}

By default, data from files are plotted with points, and functions are plotted
with lines. However, either kinds of data can be plotted in a variety of ways.
To plot a function with points, for example, the following syntax is
used\footnote{Note that when a plot command contains both
\texttt{using}/\texttt{every} modifiers, and the \texttt{with} modifier, the
latter must come last.}\index{with modifier@\texttt{with} modifier}:

\begin{verbatim}
plot sin(x) with points
\end{verbatim}

\noindent The number of points displayed (i.e. the number of samples of the
function) can be set as follows\index{set samples command@\texttt{set samples}
command}:

\begin{verbatim}
set samples 100
\end{verbatim}

Likewise, datafiles can be plotted with lines:

\begin{verbatim}
plot 'datafile' with lines
\end{verbatim}

A variety of other styles are available. \texttt{linespoints} combines both the
\texttt{points} and \texttt{lines} styles, drawing lines through points.
Errorbars can also be drawn, as follows:

\begin{verbatim}
plot 'datafile' with yerrorbars
\end{verbatim}

\noindent In this case, three columns of data need to be specified: the $x$-
and $y$-coordinates of each datapoint, plus the size of the vertical errorbar
on that datapoint. By default, the first three columns of the datafile are
used, but once again (see Section~\ref{plot_datafiles}), the \texttt{using}
modifier can be used:

\begin{verbatim}
plot 'datafile' using 2:3:7 with yerrorbars
\end{verbatim}

More details of the errorbars plot style can be found in
Section~\ref{errorbars}. Other plots styles supported by PyXPlot are listed in
Section~\ref{missing_features}, and their details can be found in many gnuplot
tutorials. Bar charts will be discussed further in Section~\ref{barcharts}.

The modifiers ``\texttt{pointtype}''\index{pointtype
modifier@\texttt{pointtype} modifier} and ``\texttt{linetype}''\index{linetype
modifier@\texttt{linetype} modifier}\label{pointtype_modifier}, which can be
abbreviated to ``\texttt{pt}'' and ``\texttt{lt}'' respectively, can also be
placed after the \texttt{with} modifier. Each should be followed by an integer.
The former specifies what shape of points should be used to plot the dataset,
and the latter whether a line should be continuous, dotted, dash-dotted, etc.
Different integers correspond to different styles.

The default plotting style referred to above can also be changed.  For example:

\begin{verbatim}
set style data lines
\end{verbatim}

\noindent The default style for plotting data from files is then changed to
lines.  Similarly the ``{\tt set style function}''\index{set style
command@\texttt{set style} command} command changes the default style for
plotting functions.

\section{Setting Axis Ranges}

In Section~\ref{first_plots}, the \texttt{set xlabel} configuration command was
previously introduced for placing text labels on axes. In this section, the
configuration of axes is extended to setting their ranges.

By default, PyXPlot automatically scales axes to some sensible range which
contains all of the plotted data. However, it is also possible for the user to
override this and set his own range.\index{axes!setting ranges} This can be
done directly from the plot command, for example:

\begin{verbatim}
plot [-1:1][-2:2] sin(x)
\end{verbatim}
\label{plot_ranges}

\noindent The ranges are specified immediately after the \texttt{plot}
statement, with the syntax \texttt{[minimum:maximum]}.\footnote{An alternative
valid syntax is to replace the colon with the word `to': \texttt{[minimum to
maximum]}.} The first specified range applies to the $x$-axis, and the second
to the $y$-axis.\footnote{As will be discussed in
Section~\ref{ranges_multiaxes}, if further ranges are specified, they apply to
the $x2$-axis, then the $y2$-axis, and so forth.} Any of the values can be
omitted, for example:

\begin{verbatim}
plot [:][-2:2] sin(x)
\end{verbatim}

\noindent would only set a range on the $y$-axis.

Alternatively, ranges can be set before the \texttt{plot} statement, using the
\texttt{set xrange}\index{set xrange command@\texttt{set xrange} command}
statement, for example:

\begin{verbatim}
set xrange [-2:2]
set y2range [a:b]
\end{verbatim}

Having done so, a range may subsequently be turned off, and an axis returned to
its default autoscaling behaviour, using the \texttt{set autoscale}\index{set
autoscale command@\texttt{set autoscale} command} command, which takes a list
of axes to which it is to apply. If no list is supplied, then the command is
applied to all axes.

\begin{verbatim}
set autoscale x y
set autoscale
\end{verbatim}

Axes can be set to have logarithmic scales using the \texttt{set
logscale}\index{set logscale command@\texttt{set logscale} command} command,
which also takes a list of axes to which it should apply. Its converse is
\texttt{set nologscale}\index{set nologscale command@\texttt{set nologscale}
command}:

\begin{verbatim}
set logscale
set nologscale y x x2
\end{verbatim}

Further discussion of the configuration of axes can be found in
Section~\ref{axis_extensions}.

\section{Function Fitting}
\index{fit command@\texttt{fit} command}
\label{fit_command}

It is possible to fit functional forms to data points in datafiles using the
\texttt{fit} command. A simple example might be:\footnote{In gnuplot, this example would have been written \texttt{fit f(x) ...}, rather than \texttt{fit f() ...}. This syntax is supported in PyXPlot, but deprecated.}

\begin{verbatim}
f(x) = a*x+b
fit f() 'datafile' index 1 using 2:3 via a,b
\end{verbatim}

The coefficients to be varied are listed after the keyword ``\texttt{via}'';
the keywords \texttt{index}, \texttt{every} and \texttt{using} have the same
meanings as in the plot command.\footnote{The \texttt{select} keyword, to be
introduced in Section~\ref{select_modifier} can also be used.}

This is useful for producing best-fit lines\index{best fit
lines}\footnote{Another way of producing best-fit lines is a to use a cubic
spline; more details in given in Section~\ref{spline_command}}, and also has
applications for estimating the gradients of datasets.  The syntax is
essentially identical to that used by gnuplot, though a few points are worth
noting:

\begin{itemize}
\item When fitting a function of $n$ variables, at least $n+1$ columns (or
rows -- see Section~\ref{horizontal_datafiles}) must be specified after the
\texttt{using} modifier. By default, the first $n+1$ columns are used. These
correspond to the values of each of the $n$ inputs to the function, plus
finally the value which the output from the function is aiming to match.
\item If an additional column is specified, then this is taken to contain the
standard error in the value that the output from the function is aiming to
match, and can be used to weight the datapoints which are input into the
\texttt{fit} command. 
\item By default, the starting values for each of the fitting parameters is
$1.0$. However, if the variables to be used in the fitting process are already
set before the \texttt{fit} command is called, these initial values are used
instead. For example, the following would use the initial values
$\{a=100,b=50\}$:
\begin{verbatim}
f(x) = a*x+b
a = 100
b = 50
fit f() 'datafile' index 1 using 2:3 via a,b
\end{verbatim}

\item As with all numerical fitting procedures, the \texttt{fit} command comes
with caveats. It uses a generic fitting algorithm, and may not work well with
poorly behaved or ill-constrained problems. It works best when all of the
values it is attempting to fit are of order unity. For example, in a problem
where $a$ was of order $10^{10}$, the following might fail:
\begin{verbatim}
f(x) = a*x
fit f() 'datafile' via a
\end{verbatim}
However, better results might be achieved if $a$ were artificially made of
order unity, as in the following script:
\begin{verbatim}
f(x) = 1e10*a*x
fit f() 'datafile' via a
\end{verbatim}

\item A series of ranges may be specified after the \texttt{fit} command, using
the same syntax as in the \texttt{plot} command, as described in
Section~\ref{plot_ranges}. If ranges are specified then only datapoints falling
within these ranges are used in the fitting process; the ranges refer to each
of the $n$ variables of the fitted function in order.

\item For those interested in the mathematical details, the workings of the
\texttt{fit} command is discussed in more detail in Chapter~\ref{fit_math}.

\end{itemize}

At the end of the fitting process, the best-fitting values of each parameter
are output to the terminal, along with an estimate of the uncertainty in each.
Additionally, the Hessian, covariance and correlation matrices are output in
both human-readable and machine-readable formats, allowing a more complete
assessment of the probability distribution of the parameters.

\section{Interactive Help}

In addition to this Users' Guide, PyXPlot also has a \texttt{help} command,
which provides a hierarchical source of information. Typing `help' alone gives a
brief introduction to the help system, as well as a list of topics on which
help is available. To display help on any given topic, type `help' followed by
the name of the topic. For example:

\begin{verbatim}
help commands
\end{verbatim}

\noindent provides information on PyXPlot's commands. Some topics have
subtopics, which are listed at the end of each page. To view them, add further
words to the end of your help request -- an example might be:

\begin{verbatim}
help commands help
\end{verbatim}

\noindent which would display help on the \texttt{help} command itself.

\section{Differences Between PyXPlot and Gnuplot}
\label{missing_features}

The commands supported by PyXPlot are only a subset of those available in
gnuplot, although most of its functionality is present. Features which are
supported by this version include:

\begin{itemize}
\item Allocation of user-defined variables and functions.
\item The \texttt{print}, \texttt{help}, \texttt{exit} and \texttt{quit} commands.
\item The \texttt{reset} and \texttt{clear} commands.
\item The \texttt{!} command, to execute the remainder of the line as a shell command, e.g. \texttt{!ls}.
\item The \texttt{cd} and \texttt{pwd} commands, to change and display the current working directory.
\item The use of ` ` back-quotes to substitute the output of a shell command.\footnote{It should be noted that back-quotes can only be used outside quotes. For example, \texttt{set xlabel '`ls`'} will not work. The best way to do this would be: \texttt{set xlabel `echo "'" ; ls ; echo "'"`}.}
\item Set plot titles, axis labels, axis ranges, pointsize, linestyles, etc.
\item Fitting of functions to data via the \texttt{fit} command.
\item Basic 2d plotting and replotting of functions and datafiles, with the following styles: \texttt{lines}, \texttt{points}, \texttt{linespoints}, \texttt{dots}, \texttt{boxes}, \texttt{steps}, \texttt{fsteps}, \texttt{histeps}, \texttt{impulses}, \texttt{csplines}, \texttt{acsplines} and errorbars of all flavours (see Section \ref{errorbars} for details of changes to errorbars).
\item Automatic and manual selection of linestyles, linetypes, linewidths, pointtypes and pointsizes.
\item Use of dual axes. Note: Operation here differs slightly from original gnuplot; dual axes are displayed whenever they are defined, there is no need to \texttt{set xtics nomirror}. See the details in the following chapter.
\item Placing arrows and textual labels on plots.
\item Putting grids on plots (colour can be set, but not linestyle).
\item Setting plot aspect ratios with \texttt{set size ratio} or \texttt{set size square}.
\item Multiplot (which is very significantly improved over gnuplot; see Section~\ref{multiplot}).
\item Setting major/minor tics with the \texttt{set xtics} and \texttt{set mxtics} commands.
\end{itemize}

Gnuplot features which PyXPlot does not presently support include:

\begin{itemize}
\item Parametric function plotting.
\item Three-dimensional plotting (i.e. gnuplot's \texttt{splot} command).
\end{itemize}
