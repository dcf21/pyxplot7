% FIRST_STEPS.TEX
%
% The documentation in this file is part of PyXPlot
% <http://www.pyxplot.org.uk>
%
% Copyright (C) 2006-7 Dominic Ford <coders@pyxplot.org.uk>
%               2008   Ross Church
%
% $Id$
%
% PyXPlot is free software; you can redistribute it and/or modify it under the
% terms of the GNU General Public License as published by the Free Software
% Foundation; either version 2 of the License, or (at your option) any later
% version.
%
% You should have received a copy of the GNU General Public License along with
% PyXPlot; if not, write to the Free Software Foundation, Inc., 51 Franklin
% Street, Fifth Floor, Boston, MA  02110-1301, USA

% ----------------------------------------------------------------------------

% LaTeX source for the PyXPlot Users' Guide

\chapter{First Steps With PyXPlot}
\label{gnuplot_intro}

In this chapter, we provide a brief overview of the basic operation of PyXPlot,
principally covering those areas of syntax which are borrowed directly from
\gnuplot. Users who are already familiar with \gnuplot\ may wish to skim or
skip this chapter, though Section~\ref{sec:latex_incompatibility}, which
describes the use of \LaTeX\ to render text, and
Section~\ref{missing_features}, which details those parts of \gnuplot's
interface that are not supported by PyXPlot, may be of interest. In the
following chapters, we shall go on to describe the ways in which PyXPlot
extends \gnuplot's interface.

Describing \gnuplot's interface in its entirety is a substantial task, and what
follows is only an overview; novice users can find many excellent tutorials on
the web which will greatly supplement what is provided below.

\section{Getting Started}

The simplest way to start PyXPlot is to type `{\tt pyxplot}' at a shell prompt
to start an interactive session. A PyXPlot command-line prompt will appear,
into which commands can be typed. PyXPlot can be exited either by typing
\indcmdts{exit}, \indcmdts{quit}, or by pressing CTRL-D.

As you begin to plot increasingly complicated graphs, the number of commands
require to set them up and plot them will grow.  It will soon become
preferable, instead of typing these commands into an interactive session, to
store lists of commands as scripts, which are simply text files containing
PyXPlot commands. These may be executed by passing the filename of the command
script to PyXPlot on the shell command line, for example:\index{command-line
syntax}

\begin{verbatim}
pyxplot foo.ppl
\end{verbatim}

\noindent In this case, PyXPlot would execute all of the commands in the file
{\tt foo.ppl} and then exit immediately afterwards.  By convention, we suffix
the filenames of PyXPlot command scripts with `{\tt .ppl}', though this is not
strictly necessary. Several filenames may be passed on a single command line,
indicating a series of scripts to be executed in sequence:

\begin{verbatim}
pyxplot foo1.ppl foo2.ppl foo3.ppl
\end{verbatim}

\noindent Wildcards can also be used; the following would execute all command
scripts in the present working directory whose filenames end with a
{\tt .ppl} suffix:

\begin{verbatim}
pyxplot *.ppl
\end{verbatim}

It is possible to use a single PyXPlot session both interactively and from
command scripts. One way to do this is to pass the magic filename `--' on the
command line:

\begin{verbatim}
pyxplot foo1.ppl - foo2.ppl
\end{verbatim}

\noindent This magic filename represents an interactive session, which
commences after the execution of {\tt foo1.ppl}, and should be terminated in
the usual way after use, with the \indcmdts{exit} or \indcmdts{quit} commands.
Afterwards, the command script {\tt foo2.ppl} would execute.

From within an interactive session, it is possible to run a command script
using the \indcmdt{load}:

\begin{verbatim}
pyxplot> load 'foo.ppl'
\end{verbatim}

\noindent This example would have the same effect as typing the contents of the
file {\tt foo.ppl} into the present session.

Usually a text editor is used to produce PyXPlot command scripts, but the
\indcmdt{save} may also assist. This stores a history of the commands executed
in the present interactive session to file.

Command files can include comment lines, which should begin with a hash
character, for example:\index{comment lines}\index{command scripts!comment
lines}

\begin{verbatim}
# This is a comment
\end{verbatim}

\noindent Comments may also be placed on the same line as commands, for
example:

\begin{verbatim}
set nokey # I'll have no key on _my_ plot
\end{verbatim}

Long commands may be split over multiple lines in the script by terminating
each line of it with a backslash character, whereupon the following line will
be appended to the end of it.

\section{First Plots}
\label{first_plots}

The basic workhorse command of PyXPlot is the \indcmdt{plot}, which is used to
produce all plots. The following simple example would plot the function
$\sin(x)$:

\begin{verbatim}
plot sin(x)
\end{verbatim}

\noindent It is also possible to plot data stored in files on disk. The
following would plot data from a file {\tt data.dat}, taking the
$x$-coordinate of each point from the first column of the \datafile, and the
$y$-coordinate from the second.  The \datafile\ is assumed to be in plain text
format\footnote{If the filename of a \datafile\ ends with a {\tt .gz} suffix,
it is assuming to be gzipped plaintext, and is decoded accordingly.}, with
columns separated by whitespace and/or commas\footnote{This format is
compatible with the Comma Separated Values (CSV) format produced by many
applications, including Microsoft Excel.}\index{csv files}\index{spreadsheets,
importing data from}\index{Microsoft Excel}\index{gzip}:

\begin{verbatim}
plot 'data.dat'
\end{verbatim}

Several items can be plotted on the same graph by separating them by commas:

\begin{verbatim}
plot 'data.dat', sin(x), cos(x)
\end{verbatim}

\noindent It is possible to define one's own variables and functions, and then
plot them:

\begin{verbatim}
a = 2
b = 1
c = 1.5
f(x) = a*(x**2) + b*x + c
plot f(x)
\end{verbatim}

\noindent To unset a variable or function once it has been set, the following
syntax should be
used:\index{variables!unsetting}\index{functions!unsetting}\index{unsetting
variables}

\begin{verbatim}
a =
f(x) =
\end{verbatim}

\section{Printing Text}
\label{string_subs_op}

PyXPlot has a \indcmdt{print} for displaying strings and the results of
calculations to the terminal, for example:

\begin{verbatim}
a=2
print "Hello World!"
print a
\end{verbatim}

\begin{verbatim}
f(x) = x**2
a=3
print "The value of",a,"squared is",f(a)
\end{verbatim}

\noindent Values may also be substituted into strings using the {\tt \%}
operator\index{\% operator@{\tt \%} operator}, which works in a similar fashion
to Python string substitution operator\index{string
operators!substitution}\footnote{For a description of this, see Guido van
Rossum's\index{van Rossum, Guido} {\it Python Library Reference}\index{Python
Library Reference}: \url{http://docs.python.org/lib/typesseq-strings.html}}.
The list of values to be substituted into the string should be a ()-bracketed
list\footnote{Unlike in Python, the brackets are obligatory; {\tt '\%d'\%2} is
{\it not} valid in PyXPlot.}:

\begin{verbatim}
print "The value of %d squared is %d."%(a,f(a))
print "The %s of f(%f) is %d."%("value",sqrt(2),f(sqrt(2)))
\end{verbatim}

\section{Axis Labels and Titles}
\label{sec:latex_incompatibility}

Labels can be added to the two axes of a plot, and a title put at the top:

\begin{verbatim}
set xlabel 'This is the X axis'
set ylabel 'This is the Y axis'
set title 'A plot of sin(x)'
plot sin(x)
\end{verbatim}

\begin{figure}
\begin{center}
\includegraphics{examples/eps/ex_axislab.eps}
\end{center}
\caption{A simple first plot with axis labels. See the text for more details.}
\label{fig:ex_axislab}
\end{figure}

\noindent The output from a simple example is shown in
Figure~\ref{fig:ex_axislab}. All text labels should be placed between either
single (') or double (") quotes. They are rendered using \LaTeX, and so any
\LaTeX\ commands can be used, for example to put equations on axes:

\begin{verbatim}
set xlabel '$\frac{x^2}{c^2}$'
\end{verbatim}

\noindent As a caveat, however, this does mean that care needs to be taken to
escape any of \LaTeX's reserved characters -- i.e.:
$\backslash$~\&~\%~\#~\{~\}~\$~\_~\^{} or $\sim$.

Because of the use of quotes to delimit text labels, special care needs to be
taken when apostrophe and quote characters are used. The following command
would raise an error, because the apostrophe would be interpreted as marking
the end of the text label:

\begin{dontdo}
set xlabel 'My plot's X axis'
\end{dontdo}

\noindent The following would achieve the desired effect:

\begin{dodo}
set xlabel "My plot's X axis"
\end{dodo}

To make it possible to render \LaTeX\ strings containing both single and double
quote characters -- for example, to put a German umlaut on the name `J\"org' in
the \LaTeX\ string `{\tt J$\backslash$"org's Data}' -- PyXPlot recognises
the backslash character to be an escape character when followed by either ' or
" in a \LaTeX\ string. This is the \textit{only} case in which PyXPlot
considers $\backslash$ an escape character. To render the example string above,
one would type:\index{escape characters}\index{backslash
character}\index{accented characters}

\begin{verbatim}
set xlabel "J\\"org's Data"
\end{verbatim}

\noindent In this example, two backslashes are required.  The first is the
\LaTeX\ escape character used to produce the umlaut; the second is a PyXPlot
escape character, used so that the " character is not interpreted as
delimiting the string. \index{escape characters}\index{quote
characters}\index{special characters}

Having set labels and titles, they may be removed thus:

\begin{verbatim}
set xlabel ''
set ylabel ''
set title ''
\end{verbatim}

\noindent These are two other ways of removing the title from a plot:

\begin{verbatim}
set notitle
unset title
\end{verbatim}

The \indcmdt{unset} may be followed by almost any word that can follow the {\tt
set} command, such as {\tt xlabel} or {\tt title}, to return that setting to
its default configuration. The \indcmdt{reset} restores all configurable
parameters to their default states.

\section{Operators and Functions}

As has already been seen above, some mathematical functions such as $\sin(x)$
are pre-defined within PyXPlot. A list of all of PyXPlot's pre-defined
functions is given in Table~\ref{functions_table}. A list of operators
recognised by PyXPlot is given in
Table~\ref{operators_table}.\index{functions!pre-defined}\index{operators}

\begin{table}
\begin{longtable}{|lp{8cm}|}
\hline
acos($x$)&
Return the arc cosine (measured in radians) of $x$.\\
asin($x$)&
Return the arc sine (measured in radians) of $x$.\\
atan($x$)&
Return the arc tangent (measured in radians) of $x$.\\
atan2($y,x$)&
Return the arc tangent (measured in radians) of $y/x$. Unlike $\mathrm{atan}(y/x)$, the signs of both $x$ and $y$ are considered.\\
ceil($x$)&
Return the ceiling of $x$ as a float. This is the smallest integral value $\geq x$.\\
cos($x$)&
Return the cosine of $x$ (measured in radians).\\
cosh($x$)&
Return the hyperbolic cosine of $x$.\\
degrees($x$)&
Convert angle $x$ from radians to degrees.\\
erf($x$)&
Return the error function, i.e.\ the Gaussian (normal) distribution function.\\
exp($x$)&
Return $e$ raised to the power of $x$.\\
fabs($x$)&
Return the absolute value of the float $x$.\\
floor($x$)&
Return the floor of $x$ as a float. This is the largest integral value $\leq x$.\\
fmod($x,y$)&
Return fmod(x, y), according to platform C.  x \% y may differ.\\
frexp($x$)&
Return the mantissa and exponent of $x$, as pair $(m,e)$. $m$ is a float and $e$ is an int, such that $x = m \times 2^e$. If $x$ is 0, $m$ and $e$ are both 0.  Else $0.5 \leq \mathrm{abs}(m) < 1.0$.\\
hypot($x,y$)&
Return the Euclidean distance, $\sqrt{x^2 + y^2}$.\\
ldexp($x, i$)&
Return $x \times 2^i$. \\
log($x[,base]$)&
Return the logarithm of $x$ to the given base. If the base not specified, returns the natural logarithm (base $e$) of $x$.\\
log10($x$)&
Return the base 10 logarithm of $x$.\\
max($x$,$y$,...)&
Return the greatest of the numerical values supplied.\\
min($x$,$y$,...)&
Return the least of the numerical values supplied.\\
modf($x$)&
Return the fractional and integer parts of $x$.  Both results carry the sign of $x$.  The integer part is returned as a real.\\
\hline
\end{longtable}
\caption{A list of mathematical functions which are pre-defined within PyXPlot (contd.\ in Table~\ref{functions_table2}).}
\label{functions_table}
\end{table}

\begin{table}
\begin{longtable}{|lp{8cm}|}
\hline
pow($x,y$)&
Return $x^y$.\\
radians($x$)&
Converts angle $x$ from degrees to radians.\\
random()&
Return a pseudo-random number in the range $0\to1$.\\
sin($x$)&
Return the sine of $x$ (measured in radians).\\
sinh($x$)&
Return the hyperbolic sine of $x$.\\
sqrt($x$)&
Return the square root of $x$.\\
tan($x$)&
Return the tangent of $x$ (measured in radians).\\
tanh($x$)&
Return the hyperbolic tangent of $x$.\\
\hline
\end{longtable}
\caption{A list of mathematical functions which are pre-defined within PyXPlot (contd.\ from Table~\ref{functions_table}).}
\label{functions_table2}
\end{table}

\begin{table}
\begin{longtable}{|lp{8cm}|}
\hline
{\tt +} & Algebraic sum \\
{\tt -} & Algebraic subtraction \\
{\tt *} & Algebraic multiplication \\
{\tt **} & Algebraic exponentiation \\
{\tt /} & Algebraic division \\
{\tt \%} & Modulo operator \\
{\tt <<} & Left binary shift \\
{\tt >>} & Right binary shift \\
{\tt \&} & Binary and \\
{\tt |} & Binary or \\
{\tt \^{}} & Logical exclusive or \\
{\tt <} & Magnitude comparison \\
{\tt >} & Magnitude comparison \\
{\tt <=} & Magnitude comparison \\
{\tt >=} & Magnitude comparison \\
{\tt ==} & Equality comparison \\
{\tt !=} & Equality comparison \\
{\tt <>} & Alias for {\tt !=} \\
{\tt and} & Logical and \\
{\tt or} & Logical or \\
\hline
\end{longtable}
\caption{A list of mathematical operators which PyXPlot recognises.}
\label{operators_table}
\end{table}

\section{Plotting \Datafile s}
\label{plot_datafiles}

In the simple example of the previous section, we plotted the first column of a
\datafile\ against the second. It is also possible to plot any arbitrary column
of a \datafile\ against any other; the syntax for doing this is:\indmod{using}

\begin{verbatim}
plot 'data.dat' using 3:5
\end{verbatim}

\noindent This example would plot the contents of the fifth column of the file
{\tt data.dat} on the vertical axis, against the the contents of the third
column on the horizontal axis. As mentioned above, columns in \datafile s can be
separated using whitespace and/or commas.  Algebraic expressions may also be
used in place of column numbers, for example:

\begin{verbatim}
plot 'data.dat' using (3+$1+$2):(2+$3)
\end{verbatim}

\noindent In such expressions, column numbers are prefixed by dollar signs, to
distinguish them from numerical constants. The example above would plot the sum
of the values in the first two columns of the \datafile, plus three, on the
horizontal axis, against two plus the value in the third column on the vertical
axis. A more advanced example might be:

\begin{verbatim}
plot 'data.dat' using 3.0:$($2)
\end{verbatim}

\noindent This would place all of the \datapoint s on the line $x=3$, meanwhile
drawing their vertical positions from the value of some column $n$ in the
\datafile, where the value of $n$ is itself read from the second column of the
\datafile.

Later, in Section~\ref{horizontal_datafiles}, I shall discuss how to plot rows
of \datafile s against one another, in horizontally arranged \datafile s.

It is also possible to plot data from only selected lines within a \datafile.
When PyXPlot reads a \datafile, it looks for any blank lines in the file. It
divides the \datafile\ up into {\it data blocks}, each being separated from the
next by a single blank line. The first datablock is numbered~0, the next~1, and
so on.  \index{datafile format}

When two or more blank lines are found together, the \datafile\ is divided up
into {\it index blocks}. The first index block is numbered~0, the next~1, and
so on. Each index block may be made up of a series of data blocks. To clarify
this, a labelled example \datafile\ is shown in Figure~\ref{sample_datafile}.

\begin{figure}
\begin{tabular}{p{2.2cm}l}
\hline
{\tt 0.0 \ 0.0} & Start of index 0, data block 0. \\
{\tt 1.0 \ 1.0} & \\
{\tt 2.0 \ 2.0} & \\
{\tt 3.0 \ 3.0} & \\
                   & A single blank line marks the start of a new data block. \\
{\tt 0.0 \ 5.0} & Start of index 0, data block 1. \\
{\tt 1.0 \ 4.0} & \\
{\tt 2.0 \ 2.0} & \\
                   & A double blank line marks the start of a new index. \\
                   & ... \\
{\tt 0.0 \ 1.0} & Start of index 1, data block 0. \\
{\tt 1.0 \ 1.0} & \\
                   & A single blank line marks the start of a new data block. \\
{\tt 0.0 \ 5.0} & Start of index 1, data block 1. \\
                   & $<$etc$>$ \\
\hline
\end{tabular}
\caption{An example PyXPlot \datafile\ -- the \datafile\ is shown in the two left-hand columns, and commands are shown to the right.}
\label{sample_datafile}
\end{figure}

By default, when a \datafile\ is plotted, all data blocks in all index blocks are
plotted. To plot only the data from one index block, the following syntax may
be used:

\begin{verbatim}
plot 'data.dat' index 1
\end{verbatim}

\noindent To achieve the default behaviour of plotting all index blocks, the
{\tt index} modifier should be followed by a negative number.\indmod{index}

It is also possible to specify which lines and/or data blocks to plot from
within each index. To do so, the \indmodt{every} modifier is used, which takes
up to six values, separated by colons:\label{introduce_every}

\begin{verbatim}
plot 'data.dat' every a:b:c:d:e:f
\end{verbatim}

\noindent The values have the following meanings:

\begin{longtable}{p{1.0cm}p{10.5cm}}
$a$ & Plot data only from every $a\,$th line in \datafile. \\
$b$ & Plot only data from every $b\,$th block within each index block. \\
$c$ & Plot only from line $c$ onwards within each block. \\
$d$ & Plot only data from block $d$ onwards within each index block. \\
$e$ & Plot only up to the $e\,$th line within each block. \\
$f$ & Plot only up to the $f\,$th block within each index block. \\
\end{longtable}

\noindent Any or all of these values can be omitted, and so the following would
both be valid statements:

\begin{verbatim}
plot 'data.dat' index 1 every 2:3
plot 'data.dat' index 1 every :::3
\end{verbatim}

\noindent The first would plot only every other \datapoint\ from every third
data block; the second from the third line onwards within each data block.

\newpage % One day when I understand LaTeX better I might understand why I need this line...

A final modifier for selecting which parts of a \datafile\ are plotted is
{\tt select}, which plots only those \datapoint s which satisfy some given
criterion. This is described in Section~\ref{select_modifier}.

\section{Directing Where Output Goes}
\label{directing_output}

By default, when PyXPlot is used interactively, all plots are displayed on the
screen. It is also possible to produce postscript output, to be read into other
programs or embedded into \LaTeX\ documents, as well as a variety of other
graphical formats. The \indcmdt{set terminal}\footnote{Gnuplot users should
note that the syntax of the {\tt set terminal} command in PyXPlot is
somewhat different from that which they are used to; see
Section~\ref{set_terminal2}.} is used to specify the output format that is
required, and the \indcmdt{set output} is used to specify the file to which
output should be directed. For example,

\begin{verbatim}
set terminal postscript
set output 'myplot.eps'
plot sin(x)
\end{verbatim}

\noindent would output a postscript plot of $\sin(x)$ to the file
{\tt myplot.eps}.

The \indcmdt{set terminal} can also be used to configure various output options
within each supported file format.  For example, the following commands would
produce black-and-white or colour output respectively:

\begin{verbatim}
set terminal monochrome
set terminal colour
\end{verbatim}

\noindent The former is useful for preparing plots for black-and-white
publications, the latter for preparing plots for colourful presentations.

Both encapsulated and non-encapsulated postscript can be produced. The former
is recommended for producing figures to embed into documents, the latter for
plots which are to be printed without further processing. The
{\tt postscript} terminal produces the latter; the {\tt eps} terminal
should be used to produce the former.  Similarly the {\tt pdf} terminal
produces files in the portable document format (pdf)\index{pdf format} read by
Adobe Acrobat\index{Adobe Acrobat}:

\begin{verbatim}
set terminal postscript
set terminal eps
set terminal pdf
\end{verbatim}

It is also possible to produce plots in the gif, png and jpeg graphic formats,
as follows:

\begin{verbatim}
set terminal gif
set terminal png
set terminal jpg
\end{verbatim}

More than one of the above keywords can be combined on a single line, for
example:

\begin{verbatim}
set terminal postscript colour
set terminal gif monochrome
\end{verbatim}

To return to the default state of displaying plots on screen, the {\tt x11}
terminal should be selected:

\begin{verbatim}
set terminal x11
\end{verbatim}

For more details of the \indcmdt{set terminal}, including how to produce gif
and png images with transparent backgrounds, see Section~\ref{set_terminal2}.

We finally note that, after changing terminals, the \indcmdt{replot} is
especially useful; it repeats the last {\tt plot} command. If any plot items
are placed after it, they are added to the pre-existing plot.

\section{Setting the Size of Output}

The widths of plots may be set be means of two commands -- {\tt set
size}\indcmd{set size} and {\tt set width}\indcmd{set width}. Both are
equivalent, and should be followed by the desired width measured in
centimetres, for example:

\begin{verbatim}
set width 20
\end{verbatim}

The {\tt set size} command can also be used to set the aspect ratio of plots by
following it with the keyword {\tt ratio}\indcmd{set size ratio}. The number
which follows should be the desired ratio of height to width. The following,
for example would produce plots three times as high as they are wide:

\begin{verbatim}
set size ratio 3.0
\end{verbatim}

The command {\tt set size noratio} returns to PyXPlot's default aspect ratio of
the golden ratio\footnote{Artists have used this aspect ratio since ancient
times. The Pythagoreans observed its frequent occurance in geometry, and
Phidias (490\,-–\,430~{\scriptsize BC}) used it repeatedly in the architecture
of the Parthenon. Renaissance artists such as Dal\'i, in keeping with their
discipleship to classical aesthetics in many areas, often used the ratio.
Leonardi Da Vinci observed that many bodily proportions closely approximate the
golden ratio. Some even went so far as to suggest that the ratio had a divine
origin (e.g.\ Pacioli~1509). As for the authors of this present work, we do
assert that plots with golden aspect ratios are pleasing to the eye, but we
leave the ponderance of the theological significance as an exercise for the
reader.}, i.e.\ $\left((1+\sqrt{5})/2\right)^{-1}$. The special command {\tt
set size square}\indcmd{set size square} sets the aspect ratio to unity.

\section{Data Styles}

By default, data from files are plotted with points and functions are plotted
with lines. However, either kinds of data can be plotted in a variety of ways.
To plot a function with points, for example, the following syntax is
used\footnote{Note that when a plot command contains {\tt using}, {\tt every}
and {\tt with} modifiers, the {\tt with} modifier must come
last.}\indmod{with}:

\begin{verbatim}
plot sin(x) with points
\end{verbatim}

\noindent The number of points displayed (i.e.\ the number of samples of the
function) can be set as follows\indcmd{set samples}:

\begin{verbatim}
set samples 100
\end{verbatim}

\noindent Likewise, \datafile s can be plotted with a line connecting the data
points:

\begin{verbatim}
plot 'data.dat' with lines
\end{verbatim}

A variety of other styles are available. The \indpst{linespoints} plot style
combines both the \indpst{points} and \indpst{lines} styles, drawing lines
through points. Errorbars can also be drawn as follows:\indps{yerrorbars}

\begin{verbatim}
plot 'data.dat' with yerrorbars
\end{verbatim}

\noindent In this case, three columns of data need to be specified: the $x$-
and $y$-coordinates of each \datapoint, plus the size of the vertical errorbar
on that \datapoint. By default, the first three columns of the \datafile\ are
used, but once again (see Section~\ref{plot_datafiles}), the {\tt using}
modifier can be used:

\begin{verbatim}
plot 'data.dat' using 2:3:7 with yerrorbars
\end{verbatim}

More details of the {\tt errorbars} plot style can be found in
Section~\ref{errorbars}. Other plot styles supported by PyXPlot are listed in
Section~\ref{list_of_plotstyles}, and their details can be found in many
\gnuplot\ tutorials. Bar charts will be discussed further in
Section~\ref{barcharts}.

\label{pointtype_modifier}
The modifiers \indpst{pointtype} and \indpst{linetype}, which can be
abbreviated to {\tt pt} and {\tt lt} respectively, can also be placed after the
{\tt with} modifier. Each should be followed by an integer.  The former
specifies what shape of points should be used to plot the dataset, and the
latter whether a line should be continuous, dotted, dash-dotted, etc.
Different integers correspond to different styles.

The default plotting style referred to above can also be changed.  For example:

\begin{verbatim}
set style data lines
\end{verbatim}

\noindent would change the default style used for plotting data from files to
lines. Similarly, the \indcmdt{set style function} changes the default style
used when functions are plotted.

\section{Setting Axis Ranges}

In Section~\ref{first_plots}, the {\tt set xlabel} configuration command was
previously introduced for placing text labels on axes. In this section, the
configuration of axes is extended to setting their ranges.

By default, PyXPlot automatically scales axes to some sensible range which
contains all of the plotted data. However, it is possible for the user to
override this and set his own range.\index{axes!setting ranges} This can be
done directly from the plot command, for example:

\begin{verbatim}
plot [-1:1][-2:2] sin(x)
\end{verbatim}
\label{plot_ranges}

\noindent The ranges are specified immediately after the \indcmdt{plot}, with
the syntax {\tt [minimum:maximum]}.\footnote{An alternative valid syntax is to
replace the colon with the word {\tt to}: {\tt [minimum to maximum]}.} The
first specified range applies to the $x$-axis, and the second to the
$y$-axis.\footnote{As will be discussed in Section~\ref{ranges_multiaxes}, if
further ranges are specified, they apply to the $x2$-axis, then the $y2$-axis,
and so forth.} Any of the values can be omitted, for example:

\begin{verbatim}
plot [:][-2:2] sin(x)
\end{verbatim}

\noindent would only set a range on the $y$-axis.

Alternatively, ranges can be set before the {\tt plot} statement, using the
\indcmdt{set xrange}, for example:

\begin{verbatim}
set xrange [-2:2]
set y2range [a:b]
\end{verbatim}

If an asterisk is supplied in place of either of the limits in this command, then
any limit which had previously been set is switched off, and the axis returns to
its default autoscaling behaviour:

\begin{verbatim}
set xrange [-2:*]
\end{verbatim}

\noindent A similar effect may be obtained using the \indcmdt{set autoscale},
which takes a list of the axes to which it is to apply. Both the upper and
lower limits of these axes are set to scale automatically. If no list is
supplied, then the command is applied to all axes.

\begin{verbatim}
set autoscale x y
set autoscale
\end{verbatim}

Axes can be set to have logarithmic scales by using the \indcmdt{set logscale},
which also takes a list of axes to which it should apply. Its converse is
\indcmdts{set nologscale}:

\begin{verbatim}
set logscale
set nologscale y x x2
\end{verbatim}

Further discussion of the configuration of axes can be found in
Section~\ref{axis_extensions}.

\section{Function Fitting}
\label{fit_command}

It is possible to fit functional forms to \datapoint s read from files by using
the \indcmdt{fit}. A simple example might be:\footnote{In \gnuplot, this
example would have been written {\tt fit f(x) ...}, rather than {\tt fit f()
...}. This syntax is supported in PyXPlot, but is deprecated.}

\begin{verbatim}
f(x) = a*x+b
fit f() 'data.dat' index 1 using 2:3 via a,b
\end{verbatim}

The first line specifies the functional form which is to be used.  The
coefficients within this function which are to be varied during the fitting
process are listed after the keyword \indkeyt{via} in the {\tt fit} command.
The modifiers \indmodt{index}, \indmodt{every} and
\indmodt{using}\indmod{select} have the same meanings here as in the plot
command.\footnote{The {\tt select} modifier, to be introduced in
Section~\ref{select_modifier} can also be used.}

This is useful for producing best-fit lines\index{best fit
lines}\footnote{Another way of producing best-fit lines is to use a cubic
spline; more details are given in Section~\ref{spline_command}}, and also has
applications for estimating the gradients of datasets.  The syntax is
essentially identical to that used by \gnuplot, though a few points are worth
noting:

\begin{itemize}
\item When fitting a function of $n$ variables, at least $n+1$ columns (or
rows -- see Section~\ref{horizontal_datafiles}) must be specified after the
{\tt using} modifier. By default, the first $n+1$ columns are used. These
correspond to the values of each of the $n$ inputs to the function, plus
finally the value which the output from the function is aiming to match.
\item If an additional column is specified, then this is taken to contain the
standard error in the value that the output from the function is aiming to
match, and can be used to weight the \datapoint s which are input into the
{\tt fit} command. 
\item By default, the starting values for each of the fitting parameters is
$1.0$. However, if the variables to be used in the fitting process are already
set before the {\tt fit} command is called, these initial values are used
instead. For example, the following would use the initial values
$\{a=100,b=50\}$:
\begin{verbatim}
f(x) = a*x+b
a = 100
b = 50
fit f() 'data.dat' index 1 using 2:3 via a,b
\end{verbatim}

\item As with all numerical fitting procedures, the {\tt fit} command comes
with caveats. It uses a generic fitting algorithm, and may not work well with
poorly behaved or ill-constrained problems. It works best when all of the
values it is attempting to fit are of order unity. For example, in a problem
where $a$ was of order $10^{10}$, the following might fail:
\begin{verbatim}
f(x) = a*x
fit f() 'data.dat' via a
\end{verbatim}
However, better results might be achieved if $a$ were artificially made of
order unity, as in the following script:
\begin{verbatim}
f(x) = 1e10*a*x
fit f() 'data.dat' via a
\end{verbatim}

\item A series of ranges may be specified after the {\tt fit} command, using
the same syntax as in the {\tt plot} command, as described in
Section~\ref{plot_ranges}. If ranges are specified then only \datapoint s falling
within these ranges are used in the fitting process; the ranges refer to each
of the $n$ variables of the fitted function in order.

\item For those interested in the mathematical details, the workings of the
{\tt fit} command is discussed in more detail in Chapter~\ref{fit_math}.

\end{itemize}

At the end of the fitting process, the best-fitting values of each parameter
are output to the terminal, along with an estimate of the uncertainty in each.
Additionally, the Hessian, covariance and correlation matrices are output in
both human-readable and machine-readable formats, allowing a more complete
assessment of the probability distribution of the parameters.

\section{Interactive Help}

In addition to this {\it Users' Guide}, PyXPlot also has a \indcmdt{help},
which provides a hierarchical source of information. Typing {\tt help} alone
gives a brief introduction to the help system, as well as a list of topics on
which help is available. To display help on any given topic, type {\tt help}
followed by the name of the topic. For example:

\begin{verbatim}
help commands
\end{verbatim}

\noindent provides information on PyXPlot's commands. Some topics have
sub-topics, which are listed at the end of each page. To view them, add further
words to the end of your help request -- an example might be:

\begin{verbatim}
help commands help
\end{verbatim}

\noindent which would display help on the {\tt help} command itself.

\section{Shell Commands}

Shell commands\index{shell commands!executing} may be executed directly from
within PyXPlot by prefixing them with an \indcmdts{!} character. The
remainder of the line is sent directly to the shell, for example:

\begin{verbatim}
!ls -l
\end{verbatim}

\noindent Semi-colons cannot be used to place further PyXPlot commands after a
shell command on the same line.

\begin{dontdo}
!ls -l ; set key top left
\end{dontdo}

It is also possible to substitute the output of a shell command into a PyXPlot
command. To do this, the shell command should be enclosed in back-quotes (`).
For example:\index{backquote character}\index{shell commands!substituting}

\begin{verbatim}
a=`ls -l *.ppl | wc -l`
print "The current directory contains %d PyXPlot scripts."%(a)
\end{verbatim}

It should be noted that back-quotes can only be used outside quotes. For
example:

\begin{dontdo}
set xlabel '`ls`'
\end{dontdo}

\noindent will not work. The best way to do this would be:

\begin{dodo}
set xlabel `echo "'" ; ls ; echo "'"`
\end{dodo}

Note that it is not possible to change the current working directory by sending
the {\tt cd} command to a shell, as this command would only change the working
directory of the shell in which the single command is executed:

\begin{dontdo}
!cd ..
\end{dontdo}

PyXPlot has its own \indcmdt{cd} for this purpose, as well as its own
\indcmdt{pwd}:

\begin{dodo}
cd ..
\end{dodo}

\section{Differences Between PyXPlot and \gnuplot}
\label{missing_features}

Because PyXPlot is still work in progress, it does not implement all of the
features of \gnuplot. It currently does not implement any three-dimensional or
surface plotting -- i.e.\ the \indcmdt{splot} of \gnuplot. It also does not
support the plotting of parametric functions.

Some of \gnuplot's features have been significantly re-worked to improve upon
their operation. The prime example is \gnuplot's multiplot mode, which allows
multiple graphs to be placed side-by-side. While we retain a similar syntax, we
have made it significantly more flexible. The use of dual axes is another
example: PyXPlot now places no limit on the number of parallel horizontal and
vertical axes which may be drawn on a graph.

These extensions to \gnuplot's interface are described in detail in the
following chapters.
