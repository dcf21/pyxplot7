% NUMERICS.TEX
%
% The documentation in this file is part of PyXPlot
% <http://www.pyxplot.org.uk>
%
% Copyright (C) 2006-7 Dominic Ford <coders@pyxplot.org.uk>
%               2008   Ross Church
%
% $Id$
%
% PyXPlot is free software; you can redistribute it and/or modify it under the
% terms of the GNU General Public License as published by the Free Software
% Foundation; either version 2 of the License, or (at your option) any later
% version.
%
% You should have received a copy of the GNU General Public License along with
% PyXPlot; if not, write to the Free Software Foundation, Inc., 51 Franklin
% Street, Fifth Floor, Boston, MA  02110-1301, USA

% ----------------------------------------------------------------------------

% LaTeX source for the PyXPlot Users' Guide

\chapter{Numerical Analysis}
\label{gnuplot_ext_last}

In this chapter, we outline the facilities provided for simple numerical
analysis and data processing within PyXPlot.

\section{Function Splicing}
\index{function splicing}
\index{splicing functions}

In PyXPlot, as in \gnuplot, user-defined functions may be declared on the
command line:

\begin{verbatim}
f(x) = x*sin(x)
\end{verbatim}

\noindent It is also possible to declare functions which are valid only over
certain ranges of argument space. For example, the following function would
only be valid within the range $-2<x<2$:\footnote{The syntax {\tt [-2:2]} can
also be written {\tt [-2 to 2]}.}

\begin{verbatim}
f(x)[-2:2] = x*sin(x)
\end{verbatim}

\noindent The following function would only be valid when all of ${a,b,c}$ were
in the range $-1 \to 1$:

\begin{verbatim}
f(a,b,c)[-1:1][-1:1][-1:1] = a+b+c
\end{verbatim}

If an attempt is made to evaluate a function outside of its specified range,
then an error results. This may be useful, for example, for plotting a function
only within some specified range. The following would plot the function
$\sinc(x)$, but only in the range $-2<x<7$:

\begin{verbatim}
f(x)[-2:7] = sin(x)/x
plot f(x)
\end{verbatim}

\begin{figure}
\begin{center}
\includegraphics{examples/eps/ex_funcsplice1.eps}
\end{center}
\caption{A simple example of the use of function splicing to truncate the function $\sinc(x)$ at $x=-2$ and $x=7$. See details in the text.}
\label{fig:ex_funcsplice1}
\end{figure}

\label{splice} \noindent The output of this particular example can be seen in
Figure~\ref{fig:ex_funcsplice1}. A similar effect could also have been achieved
with the {\tt select} keyword; see Section~\ref{select_modifier}.

It is possible to make multiple declarations of the same function, over
different regions of argument space; if there is an overlap in the valid
argument space for multiple definitions, then later declarations take
precedence. This makes it possible to use different functional forms for
functions in different parts of parameter space, and is especially useful when
fitting functions to data, if different functional forms are to be spliced
together to fit different regimes in the data.

Another application of function splicing is to work with functions which do not
have analytic forms, or which are, by definition, discontinuous, such as
top-hat functions or Heaviside functions. The following example would define
$f(x)$ to be a Heaviside function:

\begin{verbatim}
f(x) = 0
f(x)[0:] = 1
\end{verbatim}

\noindent The following example would define $f(x)$ to follow the Fibonacci
sequence, though it is not at all computationally efficient, and it is
inadvisable to evaluate it for $x\gtrsim8$:

\begin{verbatim}
f(x) = 1
f(x)[2:] = f(x-1) + f(x-2)
plot [0:8] f(x)
\end{verbatim}

\begin{figure}
\begin{center}
\includegraphics{examples/eps/ex_funcsplice2.eps}
\end{center}
\caption{An example of the use of function splicing to define a function which does not have an analytic form -- in this case, the Fibonacci sequence. See the text for details.}
\label{fig:ex_funcsplice2}
\end{figure}

\noindent The output of this example can be seen in Figure~\ref{fig:ex_funcsplice2}

\section{Datafile Interpolation: Spline Fitting}
\label{spline_command}
\index{best fit lines}

Gnuplot allows data to be interpolated using its \indpst{csplines} plot style,
for example:\indps{acsplines}

\begin{verbatim}
plot 'data.dat' with csplines
plot 'data.dat' with acsplines
\end{verbatim}

\noindent where the upper statement fits a spline through all of the
datapoints, and the lower applies some smoothing to the data first. This syntax
also is supported in PyXPlot, though splines may also be fit through data using
a new, more powerful, \indcmdt{spline}. This has a syntax similar to that of
the \indcmdt{fit}, for example:

\begin{verbatim}
spline f() 'data.dat' index 1 using 2:3
\end{verbatim}

\noindent In this example, the function $f(x)$ now becomes a special function,
representing a spline fit to the given datafile. It can be plotted or otherwise
used in exactly the same way as any other function. This approach is more
flexible than \gnuplot's syntax, as the spline $f(x)$ can subsequently be
spliced together with other functions (see the previous section), or used in
any mathematical operation.  The following code snippet, for example, would fit
splines through two datasets, and then plot the interpolated differences
between them, regardless, for example, of whether the two datasets were sampled
at exactly the same $x$ co-ordinates:

\begin{verbatim}
spline f() 'data1.dat'
spline g() 'data2.dat'
plot f(x)-g(x)
\end{verbatim}

Smoothed splines can also be produced:

\begin{verbatim}
spline f() 'data1.dat' smooth 1.0
\end{verbatim}

\noindent where the value $1.0$ determines the degree of smoothing to apply;
the higher the value, the more smoothing is applied. The default behaviour is
not to smooth at all -- equivalent to {\tt smooth 0.0} -- and a value of $1.0$
corresponds to the default amount of smoothing applied in \gnuplot's {\tt
acsplines} plot style.

\section{Tabulating Functions and Slicing Data Files}

PyXPlot's \indcmdt{tabulate} can be used to produce a text file containing the
values of a function at a set of points.  For example, the following would
produce a data file called {\tt sine.dat} containing a list of values of the
sine function:

\begin{verbatim}
set output 'sine.dat'
tabulate [-pi:pi] sin(x)
\end{verbatim}

\noindent Multiple functions may be tabulated into the same file, either by
using the \indmodt{using} modifier:

\begin{verbatim}
tabulate [0:2*pi] sin(x):cos(x):tan(x) u 1:2:3:4
\end{verbatim}

\noindent or by placing them in a comma-separated list, as in the {\tt plot}
command:

\begin{verbatim}
tabulate [0:2*pi] sin(x), cos(x), tan(x)
\end{verbatim}

The {\tt samples} setting can be used to control the number of points that are
inserted into the data file:\indcmd{set samples}

\begin{verbatim}
set samples 200
\end{verbatim}

\noindent If the $x$-axis is set to be logarithmic then the points at which the
functions are evaluated are spaced logarithmically.

The {\tt tabulate} command can also be used to select portions of data files
for output into a new file.  For example, the following would output the third,
sixth and ninth columns of the datafile {\tt input.dat}, but only when the
arcsine of the value in the fourth column is positive:

\begin{verbatim}
set output 'filtered.dat'
tabulate 'input.dat' u 3:6:9 select (asin($4)>0)
\end{verbatim}

\noindent The \indmodt{select}, \indmodt{using} and \indmodt{every} modifiers
operate in the same manner as with the {\tt plot} command.

The format used in each column of the output file is chosen automatically with
integers and small numbers treated intelligently to produce output which
preserves accuracy, but is also easily human-readable. If desired, however, a
format statement may be specified using the {\tt with format} specifier. The
syntax for this is similar to that expected by the Python string substitution
operator ({\tt \%})\index{\% operator@{\tt \%} operator}\index{string
operators!substitution}\footnote{Note that this operator can also be used
within PyXPlot; see Section~\ref{string_subs_op} for details.}.  For example,
to tabulate the values of $x^2$ to very many significant figures one could use:

\begin{verbatim}
tabulate x**2 with format "%27.20e"
\end{verbatim}

If there are not enough columns present in the supplied format statement it
will be repeated in a cyclic fashion; e.g. in the example above the single
supplied format is used for both columns.

\section{Numerical Integration and Differentiation}

\index{differentiation}\index{integration} Special functions are available for
performing numerical integration and differentiation of expressions:
\indfunt{int\_dx()} and \indfunt{diff\_dx()}. In each case, the `{\tt x}' may
be replaced with any valid one-letter variable name, to integrate or
differentiate with respect to that dummy variable.

The function {\tt int\_dx()} takes three parameters -- firstly the
expression to be integrated, which should be placed in quotes as a string,
followed by the minimum and maximum integration limits. For example, the
following would plot the integral of the function $\sin(x)$:

\begin{verbatim}
plot int_dt('sin(t)',0,x)
\end{verbatim} 

The function {\tt diff\_dx()} takes two obligatory parameters plus two further
optional parameters. The first is the expression to be differentiated, which,
as above, should be placed in quotes as a string, followed by the point at
which the differential should be evaluated, followed by optional parameters
$\epsilon_1$ and $\epsilon_2$ which are described below.  The following example
would evaluate the differential of the function $\cos(x)$ with respect to $x$
at $x=1.0$:

\begin{verbatim}
print diff_dx('cos(x)', 1.0)
\end{verbatim}

Differentials are evaluated by a simple differencing algorithm, and a parameter
$\epsilon$ controls the spacing with which to perform the differencing
operation:

\begin{displaymath}
\left.\frac{\mathrm{d}f}{\mathrm{d}x}\right|_{x=x_0} \approx \frac{f(x_0+\epsilon/2) - f(x_0-\epsilon/2)}{\epsilon}
\end{displaymath}

\noindent where $\epsilon = \epsilon_1 + x \epsilon_2$. By default, $\epsilon_1
= \epsilon_2 = 10^{-6}$, which is appropriate for the differentiation of most
well-behaved functions.

Advanced users may be interested to know that integration is performed using
the {\tt quad} function of the {\tt integrate} package of the
{\tt scipy} numerical toolkit for Python -- a general purpose integration
routine.\index{scipy}

\section{Histograms}

The \indcmdt{histogram} takes data from a datafile and bins it, producing a
function that represents the frequency distribution of the supplied data.  A
histogram is defined as a function consisting of discrete intervals, the area
under each of which is equal to the number of points binned in that interval.
For example:

\begin{verbatim}
histogram f() 'input.dat'
\end{verbatim}

\noindent would bin the points in the first column of the file {\tt input.dat}
into bins of unit width and produce a function $f()$, the value of which at any
given point was equal to the number of items in the bin at that point.

Three modifiers can be supplied to the {\tt histogram} command. The {\tt
binwidth} modifier sets the width of the bins used and the {\tt binorigin}
modifier their origin. Alternatively the {\tt bins} modifier allows an
arbitrary set of bins to be specified. For example the command:

\begin{verbatim}
histogram g() 'input.dat' bins (1, 2, 4)
\end{verbatim}

\noindent would bin the points in the first column of the file {\tt input.dat}
into two bins, $x=1\to 2$ and $x=2\to 4$.

A range can be supplied immediately following the command, using the same
syntax as in the {\tt plot} and {\tt fit} commands; only points that fall in
that range will then be binned.  In the same way as for the {\tt plot} command,
the \indmodt{index}, \indmodt{every}, \indmodt{using} and \indmodt{select}
modifiers can also be used to bin different portions of a datafile.

Note that, although a histogram is similar to a bar chart, they are subltly
different.  A bar chart has the {\it height} of the bar equal to the number of
points that it represents; for a histogram the {\it area} of the bar is equal to
the number of points.  To draw a bar chart of a set of data use the histogram
command and then multiply by the bin width when plotting.

If the function produced by the histogram command is plotted using the
\indmod{with boxes} modifier, box boundaries will be drawn to coincide with the
bins that the histogram was made with.
