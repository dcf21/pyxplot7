% REFERENCE.TEX
%
% The documentation in this file is part of PyXPlot
% <http://www.pyxplot.org.uk>
%
% Copyright (C) 2006-7 Dominic Ford <coders@pyxplot.org.uk>
%               2008   Ross Church
%
% $Id$
%
% PyXPlot is free software; you can redistribute it and/or modify it under the
% terms of the GNU General Public License as published by the Free Software
% Foundation; either version 2 of the License, or (at your option) any later
% version.
%
% You should have received a copy of the GNU General Public License along with
% PyXPlot; if not, write to the Free Software Foundation, Inc., 51 Franklin
% Street, Fifth Floor, Boston, MA  02110-1301, USA

% ----------------------------------------------------------------------------

% LaTeX source for the PyXPlot Users' Guide

\chapter{Command Reference}

This chapter contains an alphabetically ordered list of all the commands that
PyXPlot understands.

\section{!}\indcmd{!}

\begin{verbatim}
! <shell command>
<command> `<shell command>` ...
\end{verbatim}

Shell commands can be executed from within PyXPlot by pre-fixing them with
pling (!) characters, for example:

\begin{verbatim}
!mkdir foo
\end{verbatim}

\noindent As an alternative, back-quotes (`) can be used to substitute the
output of a shell command into a PyXPlot command, for example:

\begin{verbatim}
set xlabel `echo "'" ; ls ; echo "'"`
\end{verbatim}

\noindent Note that back-quotes cannot be used inside quote characters, and so
the following would \textit{not} work:

\begin{verbatim}
set xlabel '`ls`'
\end{verbatim}

\section{arrow}\indcmd{arrow}

\begin{verbatim}
arrow [from] <x>, <y> [to] <x>, <y> [with <option> ... ]
\end{verbatim}

Arrows may be placed on multiplot pages independently of any plots using the
{\tt arrow} command, which has the syntax:

\begin{verbatim}
arrow from x,y to x,y
\end{verbatim}

The {\tt arrow} command may be followed by the \indmodt{with} keyword to
specify to style of the arrow. The style keywords which are accepted are
\indkeyt{nohead}, \indkeyt{head} (default) or \indkeyt{twohead}, in addition to
keywords such as \indkeyt{colour}, \indkeyt{linewidth} or \indkeyt{linetype},
which have the same syntax and meaning as in the {\tt plot} command. An example
would be:

\begin{verbatim}
arrow from x,y to x,y with twohead linetype 2 colour blue
\end{verbatim}

Arrows receive unique multiplot identification numbers which count sequentially
from one, and which are output to the terminal after the {\tt arrow} command is
called. By reference to these numbers, they can later be deleted and undeleted
with the {\tt delete} and {\tt undelete} commands respectively.

\section{cd}\indcmd{cd}

\begin{verbatim}
cd <directory>
\end{verbatim}

PyXPlot's {\tt cd} command is very similar to the shell cd command; it can be used to
change the current working directory. For example:

\begin{verbatim}
cd foo
\end{verbatim}

\section{clear}\indcmd{clear}

\begin{verbatim}
clear
\end{verbatim}

In multiplot mode, the {\tt clear} command removes all current plots, arrows
and text objects from the working page. In single plot mode it is not
especially useful; it removes the current plot to leave a blank page.

The {\tt clear} command should not be followed by any parameters.

\section{delete}\indcmd{delete}

\begin{verbatim}
delete <plot number>, ...
\end{verbatim}

The {\tt delete} command is part of the multiplot environment; it removes
plots, arrows or text items from a multiplot page. The desired items should be
identified using a comma-separated list of their reference numbers, which count
sequentially from zero for the first item created on a multiplot page, and are
displayed on the terminal when items are created.  For example:

\begin{verbatim}
delete 1,2,3
\end{verbatim}

\noindent would remove item numbers 1, 2 and 3.

Having been deleted, multiplot items can be restored using the {\tt undelete}
command.

\section{edit}\indcmd{edit}

\begin{verbatim}
edit <plot number>
\end{verbatim}

The {\tt edit} command is part of the multiplot environment; it allows one to
modify the properties of any plot on a multiplot. The desired plot should be
identified using the reference number which it was given when it was created
using the {\tt plot} command; it would have been displayed on the terminal at
that time. For example, consider the following command sequence:

\begin{verbatim}
edit 1
set textcolour red
replot
\end{verbatim}

\noindent Here, the {\tt edit} command is used to select the plot with
reference number~1. The {\tt set textcolour red} command which follows then
changes the settings of this plot, taking effect when the {\tt replot} command
is called.

The {\tt edit} command also has the effect of resetting all of PyXPlot's plot
settings to those used to produce the chosen plot, and so in conjunction with
the {\tt show} command, can be used to inspect as well as modify the settings of
any plot on a multiplot page. For example:

\begin{verbatim}
edit 1
show textcolour
\end{verbatim}

\noindent would show the text colour used in plot 1.

Having issued the {\tt edit} command, no further command needs to be issued to
return to a state of adding plots to a multiplot rather than editing the
existing plots; simply call the {\tt plot} command rather than the {\tt replot}
command to do this.

\section{eps}\indcmd{eps}

\begin{verbatim}
eps '<filename>' [at <x>, <y>] [rotate <angle>] [width <width>]
                 [height <height>]
\end{verbatim}

The {\tt eps} command inserts an image into the current multiplot from an
encapsulated postscript (eps) file.  The {\tt at} modifier can be used to
specify where the bottom-left corner of the image should be placed; if it is
not, then the image is placed at the origin. The {\tt rotate} modifier can be
used to rotate the image by any angle, measured in degrees counter-clockwise.
The {\tt width} or {\tt height} modifiers can be used to specify the width or
height with which the image should be rendered; both should be specified in
centimetres. If neither is specified then the image will be rendered with the
native dimensions specified within the postscript.  The {\tt eps} command is
often useful in multiplot mode, allowing postscript images to be combined with
plots, text labels, etc.

\section{exec}\indcmd{exec}

The {\tt exec} command can be used to execute PyXPlot commands contained within
string variables. For example:

\begin{verbatim}
terminal="eps"
exec "set terminal %s"%(terminal)
\end{verbatim}

It can also be used to write obfuscated PyXPlot scripts.

\section{exit}\indcmd{exit}

\begin{verbatim}
exit
\end{verbatim}

The {\tt exit} command can be used to quit PyXPlot. If multiple command files,
or a mixture of command files and interactive sessions, were specified on
PyXPlot's command line, then PyXPlot moves onto the next command-line item
after receiving the {\tt exit} command.

PyXPlot may also be quit be pressing CTRL-D or via the {\tt quit} command. In
interactive mode, CTRL-C terminates the current command, if one is running.
When running a script, CTRL-C terminates execution of it.


\section{fit}\indcmd{fit}

\begin{verbatim}
fit [<range specifier> ...] <function> '<datafile>'
    [index <index specifier>] [using <using specifier>]
    via <variable>[, <variable>, ...]
\end{verbatim}

The {\tt fit} command may be used to fit functional forms to \datapoint s in
\datafile s. A simple example might be:

\begin{verbatim}
f(x) = a*x+b
fit f(x) 'data.dat' index 1 using 2:3 via a,b
\end{verbatim}

\noindent The coefficients to be varied are listed after the keyword
\indkeyt{via} keyword; the modifiers \indmodt{index}, \indmodt{every} and
\indmodt{using} have the same meanings as in the {\tt plot} command.

This is useful for producing best-fit lines and also has applications for
estimating the gradients of datasets.  The syntax is essentially identical to
the used by \gnuplot, though a few points, outlined in
Section~\ref{fit_command}, are worth noting.

\section{help}\indcmd{help}

\begin{verbatim}
help [<topic> [<sub-topic> ... ] ]
\end{verbatim}

The {\tt help} command provides an hierachical source of information which
is supplementary to that in this manual.  To obtain information on any
particular topic, type {\tt help} followed by the name of the topic. For
example:

\begin{verbatim}
help commands
\end{verbatim}

\noindent provides information on PyXPlot's commands. Some topics have
sub-topics; these are listed at the end of each help page. To view them, add
further words to the end of your help request -- an example might be:

\begin{verbatim}
help commands help
\end{verbatim}

Information is arranged with general information about PyXPlot under the heading
{\tt about} and information about PyXPlot's commands under {\tt commands}.
Information about the format that input \datafile s should take can be found
under {\tt datafile}.  Other categories are self-explanatory.

To exit any help page, press the {\tt q} key.

\section{histogram}\indcmd{histogram}

\begin{verbatim}
histogram [range specification] <function name> '<datafile>' 
     [using <using specifier>] [select <select specifier>]
     [index <index specifier>] [every <every specifier>]
     [binwidth <bin width>] [binorigin <bin origin>]
     [bins (x1, x2, ...)]
\end{verbatim}

The {\tt histogram} command takes a \datafile\ and counts the number of points
in various bins, producing a function the area under which is equal to the
number of points for each bin.  The width and starting position of the bins can
be specified using the {\tt binwidth} and {\tt binorigin} modifiers, or a
user-supplied set of bins can be used with the {\tt bins} modifier.  For
example:

\begin{verbatim}
histogram f() 'output.dat' u 2 binwidth 2
\end{verbatim}

\noindent produces a function $f()$, which contains the the data in the second
column of the {\tt output.dat} file binned into bins of width 2.  A range
specifier can be used to restrict the set of data in the \datafile\ that is to be
binned; for example:

\begin{verbatim}
histogram [0:10] f() 'data.dat' bins (0,1,3,6,10)
\end{verbatim}

\noindent would only bin data between~0 and~10, and would do so into the
user-specified bins.

\section{history}\indcmd{history}

\begin{verbatim}
history [N]
\end{verbatim}

The {\tt history} command outputs the current command-line history to the
terminal.  The optional parameter, {\tt N}, if supplied, causes only the first
$N$ lines to be printed.

\section{jpeg}\indcmd{jpeg}

\begin{verbatim}
jpeg '<filename>' [at <x>, <y>] [rotate <angle>] [width <width>]
                  [height <height>]
\end{verbatim}

The {\tt jpeg} command inserts an image into the current multiplot from a jpeg
file in disk.  The {\tt at} modifier can be used to specify where the
bottom-left corner of the image should be placed; if it is not, then the image
is placed at the origin. The {\tt rotate} modifier can be used to rotate the
image by any angle, measured in degrees counter-clockwise.  The {\tt width} or
{\tt height} modifiers can be used to specify the width or height with which
the image should be rendered; both should be specified in centimetres. If
neither is specified then the image will be rendered with the native dimensions
specified within the jpeg file (if any). The {\tt jpeg} command is often useful
in multiplot mode, allowing postscript images to be combined with plots, text
labels, etc.

\section{list}\indcmd{list}

The {\tt list} command outputs a listing of all of the items on a multiplot,
giving their reference numbers and the commands used to produce them. For
example:

\begin{verbatim}
pyxplot> list
#  ID | Command 
    0   plot f(x) 
d   1   text 'Figure 1: A plot of f(x)' 
    2   text 'Figure 1: A plot of $f(x)$' 

# Items marked 'd' are deleted 
\end{verbatim}

In this example, the user has plotted a graph of $f(x)$, and added a caption to
it. The {\tt ID} column lists the reference numbers of each multiplot item.
Item {\tt 1} has been deleted, and this is indicated by the {\tt d} to the left
of its reference number.

\section{load}\indcmd{load}

\begin{verbatim}
load '<filename>'
\end{verbatim}

The {\tt load} command executes a PyXPlot command script file, just as if its
contents had been typed into the current terminal. For example:

\begin{verbatim}
load 'foo'
\end{verbatim}

\noindent would have the same effect as typing the contents of the file {\tt
foo} into the present session.

Wildcards can be used in the load command, in which case {\it all} command
files matching the given wildcard are executed, for example:

\begin{verbatim}
load '*.script'
\end{verbatim}

\section{move}\indcmd{move}

\begin{verbatim}
move <plot number> to <x>, <y>
\end{verbatim}

The {\tt move} command is part of the multiplot environment; it can be used to
move items around on a multiplot page. The item to be moved should be specified
using the reference number which it was given when it was created; it would
have been displayed on the terminal at that time. For example:

\begin{verbatim}
move 23 to 8,8
\end{verbatim}
  
\noindent This would move multiplot item~23 to position $(8,8)$ centimetres. If
this item were a plot, the end result would be the same as if the command {\tt
set origin 8,8} had been called before it had originally been plotted.

\section{plot}\indcmd{plot}

\begin{verbatim}
plot [<range specifier> ...] ('<filename>'|<function>)
     [using <using specifier>] [axes <axis specifier>]
     [select <select specifier>]
     [index <index specifier>]
     [every <every specifier>]
     [with <style> [<style modifier> ... ] ]
\end{verbatim}

The {\tt plot} command is the main workhorse command of PyXPlot, which is used
to produce all plots. For example to plot the sine function:

\begin{verbatim}
plot sin(x)
\end{verbatim}

Ranges for the axes of a graph can be specified by placing them in
square-brackets before the name of the function to be plotted. Leaving a set of
brackets empty specifies that an axis will be automatically scaled, as happens
by default. An example of this syntax would be:

\begin{verbatim}
plot [-pi:pi] sin(x)
\end{verbatim}

\noindent which would plot the function $\sin(x)$ across some default range of
values on the $x$-axis.

\Datafile s may also be plotted as well as functions, in which case the filename
of the \datafile\ to be plotted should be enclosing in apostrophes. An example of
this syntax would be:

\begin{verbatim}
plot 'data.dat' with points
\end{verbatim}

\noindent which would plot the file called `{\tt data.dat}'.  Section
\ref{plot_datafiles} should be studied for further details of the format that is
expected of input \datafile s, and how PyXPlot may be directed to plot only
certain portions of \datafile s.

In plots which have multiple parallel axes -- for example, an $x$-axis along its
lower edge and an $x2$-axis along its upper edge -- the pair of axes against
which data should be plotted should be specified using the modifier {\tt axes}
following the name of the function or \datafile\ to be plotted, for example:

\begin{verbatim}
plot sin(x) axes x2y1
\end{verbatim}

The style in which data should be plotted may be specified following the
modifier {\tt with}, with the following syntax:

\begin{verbatim}
plot sin(x) with points
\end{verbatim}

\label{list_of_plotstyles}
The following plot styles are recognised: {\tt lines}, {\tt points}, {\tt
linespoints}, {\tt dots}, {\tt boxes}, {\tt wboxes}, {\tt impulses}, {\tt
steps}, {\tt histeps}, {\tt fsteps}, {\tt xerrorbars}, {\tt yerrorbars}, {\tt
xyerrorbars}, {\tt xerrorrange}, {\tt yerrorrange}, {\tt xyerrorrange},
\newline\noindent % WHY IS THIS NECESSARY?
{\tt arrows\_head}, {\tt arrows\_nohead}, {\tt arrows\_twohead}, {\tt
csplines}, {\tt acsplines}.

In addition, {\tt lp} and {\tt pl} are recognised as abbreviations
for {\tt linespoints}; {\tt errorbars} is recognised as an abbreviation for {\tt
yerrorbars}; {\tt errorrange} is recognised as an abbreviation for {\tt
yerrorrange}; and {\tt arrows\_twoway} is recognised as an alternative for {\tt
arrows\_twohead}.

As well as plot styles, the {\tt with} modifier can also be followed by the
following keywords:

\begin{description}
\item[{\tt linetype}] -- specifies the linetype (e.g.\ dotted) used by the lines plot style. 
\item[{\tt linewidth}] -- specifies the width of line, in pt, used by the lines plot style.
\item[{\tt pointsize}] -- specifies the size of \datapoint s, relative to the
default size, used by the points plot style. 
\item[{\tt pointlinewidth}] -- as above, but specifies the linewidth, in pt,
used to render the crosses, circles, etc, used to mark \datapoint s. 
\item[{\tt linestyle}] -- this can be used in conjunction with the {\tt set linestyle} command to save default plot styles. 
\item[{\tt colour}] -- specifies the colour used to plot the dataset, either by
one of the recognised colour names or by an integer,
to use one from the current palette.  \item[{\tt fillcolour}] -- relavant to the
{\tt boxes} and {\tt wboxes} plot
styles, specifies a colour in which bar charts should be filled.
\end{description}

An example using several of these keywords would be:

\begin{verbatim}
plot sin(x) axes x2y1 with colour blue linetype 2 \
                           linewidth 5
\end{verbatim}

Multiple datasets can be plotted on a single graph by listing them with commas
separating them:

\begin{verbatim}
plot sin(x) with colour blue, cos(x) with linetype 2
\end{verbatim}


\section{print}\indcmd{print}

\begin{verbatim}
print <expression>
\end{verbatim}

The {\tt print} command outputs the value of a mathematical expression to the
terminal.  It is most often used to find the value of a variable, though it can
also be used to produce formatted output from a PyXPlot script. For example:

\begin{verbatim}
print a
\end{verbatim}

\noindent would print the value of the variable $a$.


\section{pwd}\indcmd{pwd}

\begin{verbatim}
pwd
\end{verbatim}

The {\tt pwd} command prints the location of the current working directory.


\section{?}\indcmd{?}

\begin{verbatim}
? [<help option> ... ]
\end{verbatim}

The {\tt ?} symbol is a shortcut to the {\tt help} command.


\section{quit}\indcmd{quit}

\begin{verbatim}
quit
\end{verbatim}

The {\tt quit} command can be used to exit PyXPlot. If multiple command files,
or a mixture of command files and interactive sessions, are specified on the
command line, then PyXPlot moves onto the next command-line item after receiving
the {\tt exit} command.

PyXPlot may also be quit be pressing CTRL-D or via the {\tt exit} command. In
interactive mode, CTRL-C terminates the current command, if one is running.
When running a script, CTRL-C terminates execution of it.


\section{refresh}\indcmd{refresh}

\begin{verbatim}
refresh
\end{verbatim}

The {\tt refresh} command produces an exact copy of the latest display. This can
be useful, for example, after changing the terminal type, to produce a second
copy of a plot in a different graphic format. It differs from the {\tt replot}
command in that it doesn't replot anything; subsequent usages of the {\tt set}
command since the previous {\tt plot} command have no affect on the output. The
{\tt refresh} command is also especially useful in the multiplot environment; it
can be used to produce second copies of multiplot pages where there need not
necessarily even be any plots; there might perhaps only be textual items and
arrows.


\section{replot}\indcmd{replot}

\begin{verbatim}
replot [<plot number>]
\end{verbatim}

In single plot mode, the {\tt replot} command causes the most recent plot
command to be re-run.  This can be useful to replot a \datafile\ which has changed
in the meantime, but also to change some aspect of a plot within PyXPlot itself.
Usages of the {\tt set} command between the original {\tt plot} command and the
calling of the {\tt replot} command are applied to the new plot. For example:

\begin{verbatim}
plot sin(x)
set textcolour red
replot
\end{verbatim}

In multiplot mode, the {\tt replot} command acts by default upon the last plot
item which was added to the multiplot page, and causes that to be replotted. It
is possible to change this behaviour by first calling the {\tt edit} command, in
which case any given plot within a multiplot can be modified and replotted.

Specifying a function or \datafile\ after the {\tt replot} command causes that
function or \datafile\ to be added to the plot. The syntax here is the same as
for the {\tt plot} command.  For example:

\begin{verbatim}
replot sin(x) axes x2y1 with linespoints
\end{verbatim}

\noindent will add a plot of the function $\sin(x)$ to the current plot.

\section{reset}\indcmd{reset}

\begin{verbatim}
reset
\end{verbatim}

The {\tt reset} command returns the values of all settings that have been
changed with the {\tt set} command back to their default values.


\section{save}\indcmd{save}

\begin{verbatim}
save '<filename>'
\end{verbatim}

The {\tt save} command saves a list of all of the commands which have been
executed in the current interactive PyXPlot session into a given file. The
filename of the desired location for this file should be placed in quotes, for
example:

\begin{verbatim}
save 'foo'
\end{verbatim}

\noindent would save a command history into the file named `{\tt foo}'.


\section{set}\indcmd{set}

\begin{verbatim}
set <option> <value>
\end{verbatim}

The {\tt set} command sets the value of various operational parameters within
PyXPlot.  For example:

\begin{verbatim}
set pointsize 2
\end{verbatim}

\noindent would sets the default point size to 2. The basic syntax always
follows that above: the {\tt set} command should be followed by some keyword
specifying which setting it is which should be set. If a further parameter is
needed to specify what value to set this setting to, it should follow this
keyword. Settings which work in an on/off fashion tend to take a syntax along
the lines of:

\begin{tabular}{ll}
{\tt set key} & Set option ON \\
{\tt set nokey} & Set option OFF
\end{tabular}

More details of the functions of each individual setting can be found in the
subsections below, which represents a complete list of the recognised setting
keywords.

The reader should also see the {\tt show} command, which can be used to display
the current values of settings, and the {\tt unset} command, which returns
settings to their default values. Section~\ref{config_files} describes how
commonly used settings can be saved into a configuration file.

\subsection{arrow}\indcmd{set arrow}

\begin{verbatim}
set arrow <arrow number> from [<co-ordinate>] <x>,
          [<co-ordinate>] <y> to [<co-ordinate>] <x>,
          [<co-ordinate>] <y> [with <modifier> ]
\end{verbatim}

\begin{verbatim}
<co-ordinate> = ( first | second | screen | graph |
                  axis<axisnumber>                  )
\end{verbatim}

The {\tt set arrow} command causes an arrow to be added to a plot. An example of
its syntax would be:

\begin{verbatim}
set arrow 1 from 0,0 to 1,1
\end{verbatim}

\noindent which would cause an arrow to be drawn between the points 0,0 and 1,1, as
measured on the $x$ and $y$ axes.  The tag `1' immediately following the {\tt
arrow} keyword is an identification number, and allows the arrow to be removed
later with the {\tt unset arrow} command.  By default the co-ordinates are
measured relative to the first $x$- and $y$-axes, but can be specified in a range
of coordinate systems. These are specified as follows:

\begin{verbatim}
set arrow 1 from first 0, second 0 to axis3 1, axis4 1
\end{verbatim}

As can be seen, the name of the desired coordinate system precedes the position
value in that coordinate system. The coordinate system {\tt first}, the default,
measures the graph using the $x$- and $y$-axes. {\tt second} uses the $x2$- and
$y2$-axes.  {\tt screen} and {\tt graph} both measure in centimetres from the
origin of the graph.  The syntax {\tt axisn} may also be
used, to use the $n$ th $x$- or $y$-axis; for example, {\tt axis3} above.

The {\tt set arrow} command can be followed by the keyword `{\tt with}', to
specify the style of the arrow. For example, the specifiers `{\tt nohead}', `{\tt
head}' and `{\tt twohead}', after the keyword `{\tt with}', can be used to make
arrows with no arrow heads, normal arrow heads, or two arrow heads. `{\tt twoway}'
is an alias for `{\tt twohead}'.  Normal line type modifiers can also be used
here.  For example:

\begin{verbatim}
set arrow 2 from first 0, second 2.5 to axis3 0,
             axis4 2.5 with colour blue nohead
\end{verbatim}



\subsection{autoscale}\indcmd{set autoscale}

\begin{verbatim}
set autoscale <axis>[<axis>... ] 
\end{verbatim}

The {\tt autoscale} setting causes PyXPlot to choose the scaling for an axis
automatically based on the data and/or functions to be plotted against it. As
an example of the syntax:

\begin{verbatim}
set autoscale x1
\end{verbatim}

\noindent would cause the size of the first $x$-axis to be scaled to fit the
data.  Multiple axes can be specified, viz.:

\begin{verbatim}
set autoscale x1y3
\end{verbatim}


\subsection{axescolour}\indcmd{set axescolour}

\begin{verbatim}
set axescolour <colour>
\end{verbatim}

The {\tt axescolour} setting changes the colour of the plot's axes.  For example:

\begin{verbatim}
set axescolour blue
\end{verbatim}

\noindent changes the axes to be blue. Any of the recognised colour names listed in
Section~\ref{colour_names} can be used.
 

\subsection{axis}\indcmd{set axis}

\begin{verbatim}
set axis <axis>, ...
\end{verbatim}

The command:

\begin{verbatim}
set axis x2
\end{verbatim}

\noindent may be used to add a second $x$-axis to a plot, with default settings. In
general, there is no practical reason to use this command, as a second $x$-axis
would implicitly be created anyway by any of the following statements:

\begin{verbatim}
set x2label 'foo' \\
set x2ticdir outwards \\
plot sin(x) axes x2y1
\end{verbatim}

Of more practical use is the `{\tt unset x2}' command, which is used to remove an
axis once it has been added to a plot. After executing:

\begin{verbatim}
set x2label 'foo'
\end{verbatim}

\noindent for example, the only way to tell PyXPlot to subsequently produce a plot
without a second $x$-axis would be to delete this axis with the following
command:

\begin{verbatim}
unset axis x2
\end{verbatim}

Note that in this case, the {\tt unset x2label} command would be sufficient to
remove the label `foo' placed on the new axis, but not sufficient to delete the
new axis that the {\tt set x2label} command implicitly created. Multiple axes
can be deleted in a single {\tt unset axis} statement, for example:

\begin{verbatim}
unset axis x2x4x5
\end{verbatim}

In the special cases of {\tt unset axis x1} or {\tt unset axis y1}, these axes
cannot be deleted; a plot must have at least one $x$- and one $y$-axis. Instead,
the {\tt unset axis} command restores these axes to their default
configurations, removing any set titles or ranges that they might have been
given.

\subsection{backup}\indcmd{set backup}

\begin{verbatim}
set backup
\end{verbatim}

The setting {\tt backup} changes PyXPlot's behaviour when it detects that a file
which it is about to write is going to overwrite an existing file. Whereas by
default the existing file would be overwritten by the new one, when the
{\tt backup} setting is turned on, it is renamed, placing a tilde at the end of
its filename. For example, suppose that a plot were to be written with filename
`{\tt out.ps}', but such a file already existed.  With the backup setting turned on
the existing file would be renamed `{\tt out.ps~}' to save it from being overwritten.

The setting may be turned off via {\tt set nobackup}.


\subsection{bar}\indcmd{set bar}

\begin{verbatim}
set bar ( large | small | <barsize> )
\end{verbatim}

The {\tt bar} setting changes the size of the bar on the end of the errorbars,
relative to the current pointsize.  For example:

\begin{verbatim}
set bar 2
\end{verbatim}

\noindent sets the bars to be twice the size of the points.  The options `{\tt large}' and
`{\tt small}' are equivalent to 1 (the default) and 0 (no bar) respectively.

\subsection{binorigin}\indcmd{set binorigin}

\begin{verbatim}
set binorigin <bin origin>
\end{verbatim}

The {\tt binorigin} setting changes the position on the $x$ axis where
the bins used by the histogram command originate from.

\subsection{binwidth}\indcmd{set binwidth}

\begin{verbatim}
set binwidth <bin width>
\end{verbatim}

The {\tt binwidth} setting changes the width of the bins used by the histogram
command.

\subsection{boxfrom}\indcmd{set boxfrom}

\begin{verbatim}
set boxfrom <value>
\end{verbatim}

The `{\tt boxfrom}' setting alters PyXPlot's behaviour when plotting bar charts.
It changes the horizontal line (vertical point; $y$-axis value) from which the
boxes of bar charts appear to emanate.  The default value is zero (i.e.\ boxes
extend from the line of the $y$-axis). An example of its syntax would be:

\begin{verbatim}
set boxfrom 2
\end{verbatim}

\noindent which would make the boxes of a barchart emanate vertically from the line $y=2$.


\subsection{boxwidth}\indcmd{set boxwidth}

\begin{verbatim}
set boxwidth <width>
\end{verbatim}

The `{\tt boxwidth}' setting alters PyXPlot's behaviour when plotting bar charts.
It sets the default width of the boxes used, in graph $x$-axis units.  If the
specified width is negative then, as happens by default, the boxes have
automatically selected widths, such that the interfaces between them occur at
the horizontal midpoints between their specified $x$-positions.  For example:

\begin{verbatim}
set boxwidth 2
\end{verbatim}

\noindent would set all boxes to be two units wide.

\begin{verbatim}
set boxwidth -2
\end{verbatim}

\noindent would set all of the bars to have differing widths, centred upon their
specified $x$-positions, such that their interfaces occur at the horizontal
midpoints between them.


\subsection{data style}\indcmd{set data style}

See `{\tt set style data}'.

\subsection{display}\indcmd{set display}

\begin{verbatim}
set [no]display
\end{verbatim}

By default, whenever an item is added to a multiplot, or an existing item moved
or replotted, the whole multiplot is replotted to show the change. This can be
a time consuming process on large and complex multiplots. For this reason, the
`{\tt set nodisplay}' command is provided, which stops PyXPlot from producing any
output. The `{\tt set display}' command can subsequently be issued to return to
normal behaviour.

This can be especially useful in scripts which produce large multiplots. There
is no point in producing output at each step in the construction of a large
multiplot, and so a great speed increase can be achieved by wrapping the script
with:

\begin{verbatim}
set nodisplay 
[...prepare large multiplot...] 
set display 
refresh
\end{verbatim}


\subsection{dpi}\indcmd{set dpi}

\begin{verbatim}
set dpi <value>
\end{verbatim}

When PyXPlot is set to produce bitmapped graphics output, using the {\tt gif},
{\tt jpg} or {\tt png} terminals (see the `{\tt set terminal}' command), the `{\tt
dpi}' setting changes how many dots per inch these graphics files are produced
with. That is to say, it changes the image resolution of these file formats:

\begin{verbatim}
set dpi 100
\end{verbatim}

\noindent sets the output to a resolution of 100 dots per inch. Higher dpi
values yield better quality images, but larger file sizes.

\subsection{fontsize}\indcmd{set fontsize}

\begin{verbatim}
set fontsize <value>
\end{verbatim}

The {\tt fontsize} setting changes the size of the fount\footnote{This is not a
spelling mistake. `font', by contrast, \textit{would} be a spelling mistake. See the
Oxford English Dictionary.} used to render all text labels which appear on a
plot, including keys, axis labels, etc. The value specified should be an integer
in the range -4 to 5, corresponding to \LaTeX's tiny (-4) and Huge (5) sizes,
for example:

\begin{verbatim}
set fontsize 2
\end{verbatim}

The default value is zero, \LaTeX's normal fount size. As an alternative, fount
sizes can be specified directly in the \LaTeX\ text of labels, for example:

\begin{verbatim}
set xlabel '\Large This is a BIG label'
\end{verbatim}

\subsection{function style}\indcmd{set function style}

See `{\tt set style function}'.

\subsection{grid}\indcmd{set grid}

\begin{verbatim}
set [no]grid <axis> ...
\end{verbatim}

The {\tt grid} setting controls whether a grid is placed behind a plot or not.
Issuing the command:

\begin{verbatim}
set grid
\end{verbatim}

\noindent would cause a grid to be drawn with its gridlines connecting to the ticks of
the default axes (usually the first $x$- and $y$-axes). Conversely, issuing:

\begin{verbatim}
unset grid
\end{verbatim}

\noindent would remove from the plot all gridlines associated with the ticks of any axes.
One or more axes can be specified for the {\tt set grid} command; a grid will
then be drawn to connect with the ticks of these axes. An example of this syntax
would be:

\begin{verbatim}
set grid x1 y3
\end{verbatim}

\noindent which would cause gridlines to be drawn from ticks of the first $x$- and third
$y$-axes.

It is possible, though not always aesthetically very pleasing, to draw
gridlines from multiple parallel axes, for example:

\begin{verbatim}
set grid x1x2x3
\end{verbatim}


\subsection{gridmajcolour}\indcmd{set gridmajcolour}

\begin{verbatim}
set gridmajcolour <colour>
\end{verbatim}

The `{\tt gridmajcolour}' setting changes the colour that is used to plot the
gridlines (see the {\tt set grid} command) which are associated with the major
ticks of axes (i.e.\ major gridlines). For example:

\begin{verbatim}
set gridmajcolour purple
\end{verbatim}

\noindent would cause the major grid lines to be drawn in purple. Any of the recognised
colour names listed in Section~\ref{colour_names} can be used.

See also the {\tt set gridmincolour} command.


\subsection{gridmincolour}\indcmd{set gridmincolour}

\begin{verbatim}
set gridmincolour <colour>
\end{verbatim}

The {\tt gridmincolour} setting changes the colour that is used to plot the
gridlines (see the {\tt set grid} command) which are associated with the minor
ticks of axes (i.e.\ minor gridlines). For example:

\begin{verbatim}
set gridmincolour purple
\end{verbatim}

\noindent would cause the minor grid lines to be drawn in purple. Any of the recognised
colour names listed in Section~\ref{colour_names} can be used.

\begin{verbatim}
See also the set gridmajcolour command.
\end{verbatim}


\subsection{key}\indcmd{set key}

\begin{verbatim}
set key [ <position> ... ] [<xoffset>, <yoffset>]
\end{verbatim}

The setting `{\tt key}' determines whether a legend is placed on a plot, and if
so, where it should be located on the plot. Issuing the command:

\begin{verbatim}
set key
\end{verbatim}

\noindent simply causes a legend to be added to the plot in its default position, usually
the plot's upper-right corner. The converse action is achieved by:

\begin{verbatim}
set nokey
\end{verbatim}

\noindent or:

\begin{verbatim}
unset key
\end{verbatim}

\noindent both of which cause a plot to have no legend. A position for the key may also
be specified after the {\tt set key} command, for example:

\begin{verbatim}
set key bottom left
\end{verbatim}

Recognised positions are `{\tt top}', `{\tt bottom}', `{\tt left}', `{\tt right}', `{\tt
below}', `{\tt outside}', `{\tt xcentre}' and `{\tt ycentre}'. In addition, if none of
these quite achieved the desired result, a positional offset may be specified
after one of the position keywords above.  The first value is assumed to be an
$x$-offset, and the second a $y$-offset, in units approximately equal to the
size of the plot. For example:

\begin{verbatim}
set key bottom left 0.0, -0.5
\end{verbatim}

\noindent would display a key below the bottom left corner of the graph.


\subsection{keycolumns}\indcmd{set keycolumns}

\begin{verbatim}
set keycolumns <value>
\end{verbatim}

The `{\tt keycolumns}' settings sets how many columns the legend of a plot should
be arranged into. By default, all of the entries in the legends of plots are
arranged in a single vertical list. However, for plots with very large number
of datasets, it may be preferably to split this list into several columns. The
{\tt set keycolumns} command can be followed by any positive integer, for
example:

\begin{verbatim}
set keycolumns 3
\end{verbatim}


\subsection{label}\indcmd{set label}

\begin{verbatim}
set label <label number> '<text>' [<co-ordinate>] <x>,
                                  [<co-ordinate>] <y>
                                  [rotate <angle>]
\end{verbatim}

\begin{verbatim}
<co-ordinate> = ( first | second | screen | graph |
                  axis<axisnumber>                  )
\end{verbatim}

The {\tt set label} command can be used to place text labels onto a plot.  For
example:

\begin{verbatim}
set label 1 'Hello' 0, 0
\end{verbatim}

\noindent would place the word `Hello' at plot co-ordinates (0,0), as measured on the $x$-
and $y$-axes.  The tag `{\tt 1}' immediately following the `{\tt label}' keyword is an
identification number, and allows the label to be removed later with the {\tt
unset label} command.  By default the position coordinates of the label are
measured relative to the first $x$- and $y$-axes, but can be specified in a
range of coordinate systems. These are specified as follows:

\begin{verbatim}
set label 1 'Hello' first 0, second 0
\end{verbatim}

As can be seen, the name of the desired coordinate system precedes the position
value in that coordinate system. Following \gnuplot's nomenclature, the
coordinate system {\tt first} the default, measures the graph using the $x$- and
$y$-axes. {\tt second} uses the $x2$- and $y2$-axes.  {\tt screen} and {\tt
graph} both measure in centimetres from the origin of the graph.  The syntax
{\tt axisn} may also be used, to use the $n$ th $x$- or $y$-axis; for example,
{\tt axis3}:

\begin{verbatim}
set label 1 'Hello' axis3 1, axis4 1
\end{verbatim}

A rotation angle may optionally be specified after the keyword `{\tt rotate}'
to produce text rotated to any arbitrary angle, measured in degrees
counter-clockwise. The following example would produce upward-running text:

\begin{verbatim}
set label 1 'Hello' 1.2, 2.5 rotate 90
\end{verbatim}


\subsection{linestyle}\indcmd{set linestyle}

\begin{verbatim}
set linestyle <style number> <style specifier> ...
\end{verbatim}

At times, the string of style keywords following the `{\tt with}' modifier in plot
commands can grow rather unwieldily long. For clarity, frequently used plot
styles can be stored as {\tt linestyles}; this is true of styles involving
points as well as lines. The syntax for setting a linestyle is:

\begin{verbatim}
set linestyle 2 points pointtype 3
\end{verbatim}

\noindent where the `{\tt 2}' is the identification number of the linestyle. In a subsequent
plot statement, this linestyle can be recalled as follows:

\begin{verbatim}
plot sin(x) with linestyle 2
\end{verbatim}


\subsection{linewidth}\indcmd{set linewidth}

\begin{verbatim}
set linewidth <value>
\end{verbatim}

Sets the default linewidth, in units of pt, of the lines used to plot datasets
onto graphs with the `{\tt lines}' plot style (see the {\tt plot} command for
details of plot styles), for example in the following statement:

\begin{verbatim}
plot sin(x) with lines
\end{verbatim}

The linewidths of individual datasets can be set as follows; the {\tt set
linewidth} setting only affects plot statements where no linewidth is manually
specified:

\begin{verbatim}
plot sin(x) with lines linewidth 5.0
\end{verbatim}


\subsection{logscale}\indcmd{set logscale}

\begin{verbatim}
set logscale [<axis> ... ] <base>
\end{verbatim}

The `{\tt logscale}' setting causes an axis to be laid out with logarithmically,
rather than linearly, spaced intervals.  For example, issuing the command:

\begin{verbatim}
set log
\end{verbatim}

\noindent would cause all of the axes of a plot to be scaled logarithmically. Alternatively
only one, or a selection of axes, can be set to scale logarithmically as
follows:

\begin{verbatim}
set log x1 y2
\end{verbatim}

This would cause the first $x$- and second $y$-axes to be scaled logarithmically.
Linear scaling can be restored to all axes via:

\begin{verbatim}
set nolog
\end{verbatim}

\noindent or:

\begin{verbatim}
unset log
\end{verbatim}

\noindent and to only one, or a selection of axes, via:

\begin{verbatim}
set nolog x1 y2
\end{verbatim}

\noindent or:

\begin{verbatim}
unset log x1y2
\end{verbatim}

Optionally, a base may be specified at the end of the {\tt set logscale}
command, to produce axes labelled in logarithms to arbitrary bases.

\subsection{multiplot}\indcmd{set multiplot}

\begin{verbatim}
set multiplot
\end{verbatim}

Issuing the command:

\begin{verbatim}
set multiplot
\end{verbatim}

\noindent causes PyXPlot to enter multiplot mode, which allows many graphs to
be plotted together and displayed side-by-side. See Section~\ref{multiplot} for
a full discussion of multiplot mode.

\subsection{mxtics}\indcmd{set mxtics}

See {\tt set xtics}.

\subsection{mytics}\indcmd{set mytics}

See {\tt set xtics}.

\subsection{noarrow}\indcmd{set noarrow}

\begin{verbatim}
set noarrow [<arrow number>]
\end{verbatim}

Issuing the command:

\begin{verbatim}
set noarrow
\end{verbatim}

\noindent removes all arrows, as set using the {\tt set arrow} command, from the current
plot. Alternatively, individual arrows can be removed using the syntax:

\begin{verbatim}
set noarrow 2
\end{verbatim}

\noindent where the tag `{\tt 2}' here is the identification number given to
the arrow to be removed when it was initially set using the {\tt set arrow}
command.

\subsection{noaxis}\indcmd{set noaxis}

\begin{verbatim}
set noaxis <axis specification>, ...
\end{verbatim}

The {\tt set noaxis} command is equivalent to the {\tt unset axis} command. It
should be followed by a comma-separated lists of axes, which are to be removed
from the current axis configuration.


\subsection{nobackup}\indcmd{set nobackup}

See {\tt backup}.


\subsection{nodisplay}\indcmd{set nodisplay}

See {\tt display}.


\subsection{nogrid}\indcmd{set nogrid}

\begin{verbatim}
set norgrid [<axis> ... ]
\end{verbatim}

Issuing the command {\tt set nogrid} removes gridlines from the current plot. On
its own, the command removes all gridlines from the plot, but alternatively,
those gridlines connected to the ticks of certain axes can selectively be
removed.  The syntax for doing this is as follows:

\begin{verbatim}
set nogrid x1 y2
\end{verbatim}


\subsection{nokey}\indcmd{set nokey}

\begin{verbatim}
set nokey
\end{verbatim}

Issuing the command {\tt set nokey} causes plots to be generated with no legend.
See the command {\tt set key} for more details.


\subsection{nolabel}\indcmd{set nolabel}

\begin{verbatim}
set nolabel [<label number> ... ]
\end{verbatim}

Issuing the command:

\begin{verbatim}
set nolabel
\end{verbatim}

\noindent removes all text labels, as set using the {\tt set label} command,
from the current plot. Alternatively, individual labels can be removed using
the syntax:

\begin{verbatim}
set nolabel 2
\end{verbatim}

\noindent where the tag `{\tt 2}' here is the identification number given to
the label to be removed when it was initially set using the {\tt set label}
command.

\subsection{nolinestyle}\indcmd{set nolinestyle}

\begin{verbatim}
set nolinestyle <style number>
\end{verbatim}

The {\tt nolinestyle} setting deletes a line style. For example, the command:

\begin{verbatim}
set nolinestyle 3
\end{verbatim}

\noindent would delete the third linestyle, if defined. See the command {\tt set
linestyle} for more details.


\subsection{nologscale}\indcmd{set nologscale}

\begin{verbatim}
set nologscale [<axis> ... ]
\end{verbatim}

The {\tt logscale} setting causes an axis to be laid out with logarithmically,
rather than linearly, spaced intervals. Conversely, the {\tt nologscale} setting
is used to restore linear scaling. For example, issuing the command:

\begin{verbatim}
set nolog 
\end{verbatim}

\noindent would cause all of the axes of a plot to be scaled linearly. Alternatively only one,
or a selection of axes, can be set to scale linearly as follows:

\begin{verbatim}
set nologscale x1 y2
\end{verbatim}

This would cause the first $x$- and second $y$-axes to be scaled linearly.


\subsection{nomultiplot}\indcmd{set nomultiplot}

\begin{verbatim}
set nomultiplot
\end{verbatim}

Issuing the command {\tt set nomultiplot} places PyXPlot into single plotting
mode.  See above for a detailed discussion of PyXPlot's multiplot and
single plot modes. Broadly speaking, single plot mode is used to produce single
graphs on their own; multiplot mode is used to produce galleries of many plots
side-by-side.


\subsection{notitle}\indcmd{set notitle}

\begin{verbatim}
set notitle
\end{verbatim}

Issuing the command {\tt set notitle} will cause graphs to be produced with no
title at the top.


\subsection{noxtics}\indcmd{set noxtics}

\begin{verbatim}
set no<axis specification>tics
\end{verbatim}

This command causes graphs to be produced with no tick marks along their $x$-axes.

\subsection{noytics}\indcmd{set noytics}

See {\tt set noxtics}.


\subsection{origin}\indcmd{set origin}

\begin{verbatim}
set origin <x>, <y>
\end{verbatim}

The `{\tt origin}' setting controls the default location of graphs on a multiplot.
For example, the command:

\begin{verbatim}
set origin 3,5
\end{verbatim}

\noindent would cause the next graph to be plotted at position $(3,5)$ centimetres on the
multiplot page. The {\tt set origin} command is of little use outside multiplot
mode.


\subsection{output}\indcmd{set output}

\begin{verbatim}
set output '<filename>'
\end{verbatim}

The {\tt output} setting controls the name of the file that is produced for
non-interactive terminals ({\tt postscript}, {\tt eps}, {\tt jpeg}, {\tt gif}
and {\tt png}).  For example:

\begin{verbatim}
set output 'myplot.eps'
\end{verbatim}

\noindent causes the output to be written to the file `{\tt myplot.eps}'.


\subsection{palette}\indcmd{set palette}

\begin{verbatim}
set palette <colour>, [<colour> ... ]
\end{verbatim}

PyXPlot has a palette of colours which it assigns sequentially to datasets when
colours are not manually assigned. This is also the palette to which is
referred if the user issues a command such as:

\begin{verbatim}
plot sin(x) with colour 5
\end{verbatim}

\noindent requesting the fifth colour from the palette. By default, this palette contains
a range of distinctive colours, however the user can choose to substitute his
own list of colours for these using the {\tt set palette} command. It should be
followed by a comma-separated list of colour names, for example:

\begin{verbatim}
set palette red,green,blue
\end{verbatim}

If, after issuing this command, the following plot statement were to be
executed:

\begin{verbatim}
plot sin(x), cos(x), tan(x), exp(x)
\end{verbatim}

\noindent the first function would be plotted in red, the second in green, and the third
in blue. Upon reaching the fourth, the palette would cycle back to red.

Any of the recognised colour names listed in Section~\ref{colour_names} can be used.

\subsection{papersize}\indcmd{set papersize}

\begin{verbatim}
set papersize ( size | <height>,<width> )
\end{verbatim}

The {\tt papersize} option sets the size of output produced by the postscript
terminal. This can take the form of either a recognised papersize name -- a
list of these is given below -- or a height, width pair of values, both measured
in millimetres. For example:

\begin{verbatim}
set papersize a4
set papersize letter
set papersize 200,100
\end{verbatim}

A list of recognised papersizes can be found in Figure~\ref{paper_sizes}.

\subsection{pointlinewidth}\indcmd{set pointlinewidth}

\begin{verbatim}
set pointlinewidth <value>
\end{verbatim}

The `{\tt pointlinewidth}' setting changes the width of the lines that are used to
plot \datapoint s.  For instance:

\begin{verbatim}
set pointlinewidth 20
\end{verbatim}

\noindent would cause points to be plotted with lines 20 times the default thickness.

Note that `{\tt pointlinewidth}' can be abbreviated as `{\tt plw}'.

\subsection{pointsize}\indcmd{set pointsize}

\begin{verbatim}
set pointsize <value>
\end{verbatim}

The `{\tt pointsize}' setting changes the size at which points are plotted
relative to their default size. It should be followed by a single value, the
relative size, which can be any positive number. For example:

\begin{verbatim}
set pointsize 1.5
\end{verbatim}

\noindent would cause points to be plotted 1.5 times the default size.

\subsection{preamble}\indcmd{set preamble}

The {\tt preamble} setting changes the preamble that is prepended to each item of
text rendered using \LaTeX{}.  This allows, for example, different packages to
be loaded by default and user-defined macros to be set up.

\subsection{samples}\indcmd{set samples}

The {\tt samples} setting determines the number of values along the $x$-axis at
which functions are evaluated when they are plotted. For example:

\begin{verbatim}
set samples 100
\end{verbatim}

\noindent causes 100 points to be evaluated.  Increasing this value will cause functions
to be plotted more smoothly, but also more slowly, and the resulting postscript
files generated will be correspondingly larger.

When functions are plotted with the {\tt points} plot style, this also affects
the number of points plotted.


\subsection{size}\indcmd{set size}

\begin{verbatim}
set size (<width>|ratio <ratio>|noratio|square)
\end{verbatim}

The setting {\tt size} is deprecated; use {\tt set width} instead.  It sets the
width of the plot in centimetres. However, the command {\tt set size}, when
followed by the keyword {\tt ratio}, is still used to set the aspect ratio of
plots. See the `{\tt ratio}' setting below for details.

\subsubsection{noratio}\index{set size command!noratio modifier@{\tt noratio} modifier}

\begin{verbatim}
set size noratio
\end{verbatim}

Running:

\begin{verbatim}
set size noratio
\end{verbatim}

\noindent resets PyXPlot to produce plots with its default aspect ratio, which is the
golden section. Other aspect ratios can be set with the {\tt set size ratio}
command.


\subsubsection{ratio}\index{set size command!ratio modifier@{\tt ratio} modifier}

\begin{verbatim}
set size ratio <ratio>
\end{verbatim}

The command:

\begin{verbatim}
set size ratio x
\end{verbatim}

\noindent sets the aspect ratio of plots produced by PyXPlot.  The height of resulting
plots will equal the plot width, as set by the {\tt set width} command,
multiplied by this aspect ratio. The value $x$ in the above statement can be
substituted with any positive value, for example:

\begin{verbatim}
set size ratio 2.0
\end{verbatim}

\noindent would cause PyXPlot to produce plots that are twice as high as they are wide.

The default aspect ratio which PyXPlot uses is a golden ratio of
$2/(1+\sqrt{5})$, which matches that of a sheet of A4 paper.


\subsubsection{square}\index{set size command!square modifier@{\tt square} modifier}

\begin{verbatim}
set size square
\end{verbatim}

The command:

\begin{verbatim}
set size square
\end{verbatim}

\noindent sets PyXPlot to produce square plots, i.e.\ with unit aspect ratio. Other aspect
ratios can be set with the {\tt set size ratio} command.

\subsection{style}\indcmd{set style}

\begin{verbatim}
set style { data | function } <style modifier> ...
\end{verbatim}

The {\tt set style data} command affects the default style that data from a file
is plotted with.  Likewise the {\tt set style function} command changes the
default style that functions are plotted with.  Any valid style modifier can be
used.  For example:

\begin{verbatim}
set style data points
set style function lines linestyle 1
\end{verbatim}

\noindent would cause \datafile s to be plotted by default using points and
functions using lines with the first defined linestyle.
 
\subsection{terminal}\indcmd{set terminal}

\begin{verbatim}
set terminal <terminal type> [<option> ... ]
\end{verbatim}

Syntax:

\begin{verbatim}
set terminal { X11_singlewindow | X11_multiwindow | X11_persist | 
               postscript | eps | pdf | gif | png | jpg } 
             { colour | color | monochrome } 
             { portrait | landscape } 
             { invert | noinvert } 
             { transparent | solid }
             { enlarge | noenlarge }
\end{verbatim}

The {\tt set terminal} command controls the graphic format in which PyXPlot
should output plots, for example setting whether it should output plots to files
or display them in a window on the screen. Various options can also be set
within many of the graphic formats which PyXPlot supports using this command.

The following graphic formats are supported:  {\tt X11\_singlewindow},
\newline\noindent % WHY IS THIS NECESSARY?
{\tt X11\_multiwindow}, {\tt X11\_persist}, {\tt postscript}, {\tt eps}, {\tt
pdf}, {\tt gif}, {\tt jpeg}, {\tt png}. To select one of these formats, simply
type the name of the desired format after the {\tt set terminal} command. To
obtain more details on each, see the subtopics below.

The following settings, which can also be typed following the {\tt set terminal}
command, are used to change the options within some of these graphic formats:
{\tt colour}, {\tt monochrome}, {\tt enhanced}, {\tt noenhanced}, {\tt
portrait}, {\tt landscape}, {\tt invert}, {\tt noinvert}, {\tt transparent},
{\tt solid}, {\tt enlarge}, {\tt noenlarge}. Details of each of these can be
found below.

\subsubsection{colour}\index{set terminal command!colour modifier@{\tt colour} modifier}

The {\tt colour} terminal option causes plots to be produced in colour.

\subsubsection{color}\index{set terminal command!color modifier@{\tt color} modifier}

The {\tt color} terminal option is provided for the convenience of users unable
to spell {\tt colour}.

\subsubsection{enlarge}\index{set terminal command!enlarge modifier@{\tt enlarge} modifier}

The {\tt enlarge} terminal option causes the complete plot to be enlarged or
shrunk to fit the current paper size.

\subsubsection{eps}\index{set terminal command!eps modifier@{\tt eps} modifier}

\begin{verbatim}
set terminal eps [<option> ... ]
\end{verbatim}

Sends output to eps files.  The filename to which output is to be sent should
be set using the {\tt set output} command; the default is
`{\tt pyxplot.eps}'.  This terminal produces encapsulated postscript
suitable for including in, for example, \LaTeX documents.


\subsubsection{gif}\index{set terminal command!gif modifier@{\tt gif} modifier}

\begin{verbatim}
set terminal gif [<option> ... ]
\end{verbatim}

The {\tt gif} terminal renders output as gif files. The filename to which output
is to be sent should be set using the {\tt set output} command; the default is
{\tt pyxplot.gif}. The number of dots per inch used can be changed using the dpi
option; the filename using {\tt set output}. Transparent gifs can be produced
with the {\tt transparent} option. Also of relevance is the {\tt invert} option
for producing gifs with inverted colours.


\subsubsection{invert}\index{set terminal command!invert modifier@{\tt invert} modifier}

The {\tt invert} terminal option causes the bitmap terminals ({\tt gif}, {\tt
jpeg}, {\tt png}) to produce output with inverted colours. Useful for producing
plots for slideshows, where bright colours on a dark background may be desired.


\subsubsection{jpeg}\index{set terminal command!jpeg modifier@{\tt jpeg} modifier}

\begin{verbatim}
set terminal jpeg [<option> ... ]
\end{verbatim}

The {\tt jpeg} terminal renders output as jpeg files. The filename to which
output is to be sent should be set using the {\tt set output} command; the
default is {\tt pyxplot.jpg}.  The number of dots per inch used can be changed
using the dpi option. Of relevance is the {\tt invert} option for producing
jpegs with inverted colours.

\subsubsection{landscape}\index{set terminal command!landscape modifier@{\tt landscape} modifier}

The {\tt landscape} terminal option causes PyXPlot's output to be displayed in
rotated orientation.  Useful for printing as you get more on your sheet of
paper that way around; probably less useful for plotting things on screen.

\subsubsection{monochrome}\index{set terminal command!monochrome modifier@{\tt monochrome} modifier}

The {\tt monochrome} terminal option causes plots to be rendered in black and
white; by default, different dash styles are used to differentiate between
lines on plots with several datasets.

\subsubsection{noenlarge}\index{set terminal command!noenlarge modifier@{\tt noenlarge} modifier}

The {\tt noenlarge} terminal option causes the output not to be scaled (the
opposite of {\tt enlarge} above).

\subsubsection{noinvert}\index{set terminal command!noinvert modifier@{\tt noinvert} modifier}

The {\tt noinvert} terminal option causes the bitmap terminals ({\tt gif}, {\tt
jpeg}, {\tt png}) to produce normal output without inverted colours. The
converse of {\tt inverse}.


\subsubsection{pdf}\index{set terminal command!pdf modifier@{\tt pdf}
modifier}

\begin{verbatim}
set terminal pdf [<option> ... ]
\end{verbatim}

The {\tt pdf} terminal options causes pdf format output files to be produced.

\subsubsection{png}\index{set terminal command!png modifier@{\tt png} modifier}

\begin{verbatim}
set terminal png [<option> ... ]
\end{verbatim}

The {\tt png} terminal renders output as png files. The filename to which output
is to be sent should be set using the {\tt set output} command; the default is
{\tt pyxplot.png}. The number of dots per inch used can be changed using the dpi
option; the filename using {\tt set output}. Transparent pngs can be produced
with the {\tt transparent} option. Also of relevance is the {\tt invert} option
for producing pngs with inverted colours.


\subsubsection{portrait}\index{set terminal command!portrait modifier@{\tt portrait} modifier}

The {\tt portrait} terminal option causes PyXPlot's output to be displayed in
upright (normal) orientation.
 

\subsubsection{postscript}\index{set terminal command!postscript modifier@{\tt postscript} modifier}

\begin{verbatim}
set terminal postscript [<option> ... ]
\end{verbatim}

Sends output to postscript files. The filename to which output is to be sent
should be set using the {\tt set output} command; the default is {\tt
pyxplot.ps}.  This terminal produces non-encapsulated postscript suitable for
sending directly to a printer.

\subsubsection{solid}\index{set terminal command!solid modifier@{\tt solid} modifier}

The solid option causes the {\tt gif} and {\tt png} terminals to produce output
with a non-transparent background. The converse of {\tt transparent}.


\subsubsection{transparent}\index{set terminal command!transparent modifier@{\tt transparent} modifier}

The {\tt transparent} terminal option causes the {\tt gif} and {\tt png}
terminals to produce output with a transparent background.


\subsubsection{X11\_multiwindow}\index{set terminal command!X11\_multiwindow modifier@{\tt X11\_multiwindow} modifier}

Displays plots on the screen (in X11 windows, using \ghostview). Each time a new
plot is generated it appears in a new window, and the old plots remain visible.
As many plots as may be desired can be left on the desktop simultaneously.

\subsubsection{X11\_persist}\index{set terminal command!X11\_persist
modifier@{\tt X11\_persist} modifier}

Displays plots on the screen in X11 windows, using \ghostview.  Each time a new
plot is generated it appears in a new window, and the old plots remain visible.
When PyXPlot is exited the windows remain in place until they are closed
manually.

\subsubsection{X11\_singlewindow}\index{set terminal command!X11\_singlewindow modifier@{\tt X11\_singlewindow} modifier}

Displays plots on the screen (in X11 windows, using \ghostview). Each time a new
plot is generated it replaces the old one, preventing the desktop from becoming
flooded with old plots. This terminal is the default when running
interactively.

\subsection{textcolour}\indcmd{set textcolour}

\begin{verbatim}
set textcolour <colour>
\end{verbatim}

The `{\tt textcolour}' setting changes the colour of all text displayed on a plot.
For example:

\begin{verbatim}
set textcolour red
\end{verbatim}

\noindent causes all text labels, including the labels on graph axes and
legends, etc. to be rendered in red. Any of the recognised colour names listed
in Section~\ref{colour_names} can be used.

\subsection{texthalign}\indcmd{set texthalign}

\begin{verbatim}
set texthalign ( left | centre | right )
\end{verbatim}

The `{\tt texthalign}' setting controls how text labels, placed on plots using the
{\tt set label} command, and upon multiplots using the {\tt text} command, are
justified horizontally with respect to their specified positions. Three options
are available:

\begin{verbatim}
set texthalign left
set texthalign centre
set texthalign right
\end{verbatim}

\subsection{textvalign}\indcmd{set textvalign}

\begin{verbatim}
set textvalign ( bottom | center | top )
\end{verbatim}

The `{\tt textvalign}' setting controls how text labels, placed on plots using the
{\tt set label} command, and upon multiplots using the {\tt text} command, are
justified vertically with respect to their specified positions. Three options
are available:

\begin{verbatim}
set textvalign bottom 
set textvalign centre
set textvalign top
\end{verbatim}

\subsection{title}\indcmd{set title}

\begin{verbatim}
set title '<title>'
\end{verbatim}

The `{\tt title}' setting can be used to set a title for a plot, to be displayed
above it.  For example, the command:

\begin{verbatim}
set title 'foo'
\end{verbatim}

\noindent would cause a title `foo' to be displayed above a graph. The easiest
way to remove a title, having set one, is via:

\begin{verbatim}
set title ''
\end{verbatim}
   
\subsection{width}\indcmd{set width}

\begin{verbatim}
set width <value>
\end{verbatim}

The {\tt width} setting controls the size of a graph.  For example:

\begin{verbatim}
set width 10
\end{verbatim}

\noindent sets output to be 10 centimetres in width.  For the bitmap terminals ({\tt gif},
{\tt jpg} and {\tt png}) this setting, in conjunction with the {\tt dpi}
setting, controls the number of pixels across the final image.

\subsection{xlabel}\indcmd{set xlabel}

\begin{verbatim}
set xlabel '<text>'
\end{verbatim}

The {\tt xlabel} setting controls the label placed on its $x$-axis (abscissa).
For example:

\begin{verbatim}
set xlabel '$x$'
\end{verbatim}

\noindent sets the label on the $x$-axis to `$x$'.  Labels can be placed on higher axes by
inserting their number after the `{\tt x}', for example:

\begin{verbatim}
set x10label 'foo'
\end{verbatim}

\noindent would label the tenth $x$ axis.

Similarly, labels can be placed on $y$-axes as follows:

\begin{verbatim}
set ylabel '$y$' 
set y2label 'foo'
\end{verbatim}


\subsection{xrange}\indcmd{set xrange}

\begin{verbatim}
set x[<axisnumber>]range '<text>'
\end{verbatim}

The {\tt xrange} setting controls the range of values along the $x$-axes of
plots.  For function plots, this is also the domain across which the function
will be evaluated.  For example:

\begin{verbatim}
set xrange [0:10]
\end{verbatim}

\noindent sets the first $x$ axis to be between 0 and 10.  Higher numbered axes may be
referred to be inserting their number after the $x$; $y$-axes similarly be
replacing the $x$ with a $y$.  Hence:

\begin{verbatim}
set y23range [-5:5]
\end{verbatim}

\noindent sets the range of the 23rd $y$-axis to be between -5 and 5.  The
following command:

\begin{verbatim}
set xrange [:10]
\end{verbatim}

\noindent would set the $x$-axis to have an upper limit of 10, but an
automatically-scaling lower-limit.

\subsection{xticdir}\indcmd{set xticdir}

\begin{verbatim}
set (x|y)[<axisnumber>]ticdir (inward|outward|both)
\end{verbatim}

The `{\tt xticdir}' setting can be used to set whether the ticks along the
$x$-axis of a plot point inwards, towards the graph, as by default, or outwards,
towards the numeric labels along the axis. They can also be set to point in both
directions simultaneously. The syntax for this is as follows:

\begin{verbatim}
set xticdir inward 
set xticdir outward 
set xticdir both
\end{verbatim}

The same setting can also be made on higher numbered axes, by inserting their
numbers after the `{\tt x}', for example:

\begin{verbatim}
set x10ticdir outward
\end{verbatim}

Similarly, the `{\tt x}' can be substituted with a `{\tt y}' to set the directions of ticks
on vertical axes:

\begin{verbatim}
set yticdir inward
set y10ticdir both
\end{verbatim}

\subsection{xtics}\indcmd{set xtics}

\begin{verbatim}
set [m]x[<axisnumber>]tics 
         [axis|border|inward|outward|both] 
         [auto 
          | [<minimum>,] <increment[, <maximum>] 
          | ( '<label>' <position> ... ) 
         ] 
\end{verbatim}

The {\tt xtics} option specifies the positions of tick marks on the $x$-axis
(similarly, {\tt ytics} acts on the $y$-axis).  One can specify:

\begin{itemize}
\item The axis to modify; if none is specified, then the command acts upon all axes.

\item {\tt mxtics} to alter the placement of minor tic marks.

\item The keywords {\tt inward}, {\tt outward} and {\tt both}, which alter the
directions of the tics.  {\tt axis} is an alias for {\tt inward}, {\tt border}
for {\tt outward}.

\item The {\tt autofreq} keyword restores automatic placement of the tics

\item If {\tt minimum}, {\tt increment}, {\tt maximum} are specified, then ticks
are placed at evenly spaced intervals between the specified limits. In the case
of logarithmic axes, increment is applied multiplicatively. 

\item The final form allows ticks to be placed on an axis manually with
individual labels.
\end{itemize}
   
Two examples:

\begin{verbatim}
set xtics 2 1 5
\end{verbatim}

\noindent will set tick marks on the $x$-axis at positions 2, 3, 4 and 5.

\begin{verbatim}
set x2tics ("a" 2, "b" 3)
\end{verbatim}

\noindent will set tick marks on the second $x$-axis at positions 2 and 3 reading `a' and
`b' respectively.


\subsection{ylabel}\indcmd{set ylabel}

See {\tt xlabel}.


\subsection{yrange}\indcmd{set yrange}

See {\tt xrange}.
   

\subsection{yticdir}\indcmd{set yticdir}

See {\tt xticdir}.


\subsection{ytics}\indcmd{set ytics}

See {\tt xtics}.

\section{show}\indcmd{show}

\begin{verbatim}
show ( all | settings | axes | variables | functions |
       <parameter> ...                                 )
\end{verbatim}

The {\tt show} command displays the values of PyXPlot's internal parameters. For
example:

\begin{verbatim}
show pointsize
\end{verbatim}

\noindent will display the current default point size.

Details of the various settings that can be shown can be found under the {\tt
set} command; any keyword which can follow the {\tt set} command can also follow
the {\tt show} command.

In addition, {\tt show all} shows the configuration state of all aspects of
PyXPlot. The command {\tt show settings} shows all of PyXPlot's settings, as
distinct from variables, functions and axes. {\tt show axes} shows the
configuration of all of PyXPlot's axes. {\tt show variables} lists all of the
currently defined variables. And finally, {\tt show functions} lists all of the
current user-defined functions.


\section{spline}\indcmd{spline}

\begin{verbatim}
spline [<range specification>] <function name> '<filename>' 
       [index <index specification>] [every <every specification>]
       [using <using specification>]
\end{verbatim}

The {\tt spline} command fits a spline to a \datafile. A special function is
created that represents the spline fit and can be used in the same way as any
other user-defined function. For example:

\begin{verbatim}
spline f() 'data.1'
\end{verbatim}

\noindent would create a function $f(x)$ that is a fit to the data in the file {\tt
data.1}. By default, the {\tt spline} command uses the first two columns of a
\datafile\ in a manner analogous to the plot command. The {\tt index}, {\tt
every} and {\tt using} modifiers can be used in the same way as in the {\tt
plot} command to select which parts of the \datafile\ should be used; see the
{\tt datafile} section for more details.

Note that trying to generate splines of multi-valued functions will not, in
general, produce useful results.

\section{tabulate}\indcmd{tabulate}

\begin{verbatim}
tabulate [<range specification>] ( <expression> | <filename> )
         [index <index specification>] [every <every specification>]
         [using <using specification>] [select <select specifier>]
         [with <output format>]
\end{verbatim}

The {\tt tabulate} commands produces a text file containing the values of a
function at a set of points.  For example, to produce a \datafile\ called {\tt
sine.dat} with the principal values of the sine function:

\begin{verbatim}
set output 'sine.dat'
tabulate [-pi:pi] sin(x)
\end{verbatim}

The format statement used to print the output file is chosen automatically with
integers and small numbers treated differently.  If desired, however, a format
statement may be specified using the {\tt with format} specifier; it should be a
standard Python format string.  If there are not enough columns available in the
supplied format statement it will be repeated for each line of the \datafile.

The {\tt index}, {\tt every}, {\tt using} and {\tt select} modifiers work in the
same way as for the {\tt plot} command.  For example multiple functions may be
tabulated into the same file with the {\tt using} modifier:

\begin{verbatim}
tabulate [0:2*pi] sin(x):cos(x):tan(x) u 1:2:3:4
\end{verbatim}

The {\tt samples} setting can be used to control the number of points that are
inserted into the \datafile.  If the $x$-axis is set to be logarithmic then the
points at which the functions are evaluated are spaced logarithmically.

The {\tt tabulate} command can also be used to select portions of \datafile s.
For example, to print the third, sixth and ninth columns of the \datafile\ {\tt
data.dat}, but only when the arcsine of the value in the fourth column is
positive:

\begin{verbatim}
set output 'filtered.dat'
tabulate 'data.dat' u 3:6:9 select (asin($4)>0)
\end{verbatim}


The format used in each column of the output file is chosen automatically with
integers and small numbers treated intelligently to produce output which
preserves accuracy, but is also easily human-readable. If desired, however, a
format statement may be specified using the {\tt with format} specifier. The
syntax for this is similar to that expected by the Python string substitution
operator ({\tt \%})\index{\% operator@{\tt \%} operator}\index{string
operators!substitution}\footnote{Note that this operator can also be used
within PyXPlot; see Section~\ref{string_subs_op} for details.}.  For example,
to tabulate the values of $x^2$ to very many significant figures one could use:

\begin{verbatim}
tabulate x**2 with format "%27.20e"
\end{verbatim}

If there are not enough columns present in the supplied format statement it
will be repeated in a cyclic fashion; e.g. in the example above the single
supplied format is used for both columns.

\section{text}\indcmd{text}

\begin{verbatim}
text '<text string>' [at <x>, <y>] [rotate <angle>]
\end{verbatim}

The {\tt text} command is primarily part of the multiplot environment; it can be
used to add blocks of text to a multiplot. It can, however, also be used in
single plot mode, in a way that is described below. As always in PyXPlot, the
text is rendered using \LaTeX. An example would be:

\begin{verbatim}
text 'Hello World!' at 0,2
\end{verbatim}

\noindent which would render the text `Hello World!' at position $(0,2)$,
measured in centimetres. The alignment of the text item with respect to this
position can be set using the {\tt set texthalign} and {\tt set textvalign}
commands.

A rotation angle may optionally be specified after the keyword `{\tt rotate}'
to produce text rotated to any arbitrary angle, measured in degrees
counter-clockwise. The following example would produce upward-running text:

\begin{verbatim}
text 'Hello' at 1.5, 3.6 rotate 90
\end{verbatim}

Outside of multiplot mode, the text command can be used to produce images
consisting simply of one single text item. This can be useful for importing
\LaTeX ed equations as gif images into slideshow programs such as Microsoft
Powerpoint which are incapable of producing such neat mathematical notation
by themselves.

\section{undelete}\indcmd{undelete}

\begin{verbatim}
undelete <item number>, ...
\end{verbatim}

The {\tt undelete} command is part of the multiplot environment; it can be used
to reverse the effect of deleting a multiplot item with the {\tt delete}
command. The desired item to be undeleted should be identified using the
reference number which it was given when it was created; it would have been
displayed on the terminal at that time. For example:

\begin{verbatim}
undelete 1
\end{verbatim}

\noindent will cause the previously item numbered {\tt 1} to reappear.
  
\section{unset}\indcmd{unset}

\begin{verbatim}
unset <setting>
\end{verbatim}

The {\tt unset} command causes a setting that has been changed using the set
command to be returned to its default value.  For example:

\begin{verbatim}
unset linewidth
\end{verbatim}

\noindent returns the linewidth to its default value.

The list of keywords which can follow the {\tt unset} command are essentially
the same as those which can follow the {\tt set} command.
