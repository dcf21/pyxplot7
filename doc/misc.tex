\chapter{Other features}

This Chapter describes a few additional features of PyXPlot.

\section{The Commandline Environment}

PyXPlot uses the GNU Readline command-line environment, which means that the up
and down arrow keys can be used to repeat previously executed commands. Each
user's command history is stored in his homespace in a history file called
`\texttt{.pyxplot\_history}', allowing PyXPlot to remember command histories
between sessions. Additionally, a \texttt{save} command is provided, allowing
the user to save his command history from the present session to a text file;
this has the following syntax:

\begin{verbatim}
save 'output_filename'
\end{verbatim}

From the shell command line, the PyXPlot accepts the following switches which
modify its behaviour:\index{command line syntax}

\begin{longtable}{p{3.5cm}p{8.5cm}}
\texttt{-h --help} & Display a short help message listing the available command-line switches.\\
\texttt{-v --version} & Display the current version number of PyXPlot.\\
\texttt{-q --quiet} & Turn off the display of the welcome message on startup. \\
\texttt{-V --verbose} & Display the welcome message on startup, as happens by default. \\
\texttt{-c --colour} & Use colour highlighting\footnote{This will only function on terminals which support colour output.} to display output in green, warning messages in amber, and error messages in red.\footnote{The author apologies to those members of the population who are red/green colourblind, but draws their attention to the following sentence.} These colours can be changed in the \texttt{terminal} section of the configuration file; see Section~\ref{configuration} for more details. \\
\texttt{-m --monochrome} & Do not use colour highlighting, as happens by default. \\
\end{longtable}

\section{Variables}

PyXPlot recognises two types of variables; numeric variables and string
variables.  The former can be assigned a value using any valid mathematical
expression, for example

\begin{verbatim}
a = 5.2 * sqrt(64)
\end{verbatim}

would assign the value 41.6 to the variable ``a''.  Numerical variables can
subsequently be used in mathematical expressions themselves, for example:

\begin{verbatim}
a=2*pi
plot [0:1] sin(x) with lines
\end{verbatim}

String variables can be assigned to in an analogous manner by enclosing the
string in quotation marks.  They can then be used wherever a string would be
required, for example

\begin{verbatim}
plotname = "A very exciting graph"
set title plotname
\end{verbatim}

String variables can also be modified using the search-and-replace string
operator, =~.  This takes a regular expression in standard form and applies it
to the relvent string.  For example:

\begin{verbatim}
twister="seven silver soda syphons"
twister =~ s/s/th/
print twister
\end{verbatim}

Doesn't do what you expect it to. XXX FIXME

\section{Tabulating Functions and Slicing Data Files}

PyXPlot includes a command that can be used to tabulate datafiles, that is, to
produce a text file containing the values of the function at a set of points.
For example, to produce a data file called {\tt sine.dat} with the principal
values of the sine function:

\begin{verbatim}
set output 'sine.dat'
tabulate [-pi:pi] sin(x)
\end{verbatim}

Multiple functions may be tabulated into the same file with the {\tt using}
modifier:

\begin{verbatim}
tabulate [0:2*pi] sin(x):cos(x):tan(x) u 1:2:3:4
\end{verbatim}

The {\tt samples} setting can be used to control the number of points that are
inserted into the data file.  If the $x$-axis is set to be logrithmic then the
points at which the functions are evaluated are spaced logrithmically.

The {\tt tabulate} command can also be used to select portions of data files.
For example, to print the third, sixth and nineth columns of the datafile {\tt
plot.1}, but only when the arcsine of the value in the fourth column is
positive:

\begin{verbatim}
set output 'filtered.dat'
tabulate 'plot.1' u 3:6:9 select (asin($4)>0)
\end{verbatim}

\noindent The {\tt select}, {\tt using} and {\tt every} modifiers operate in the
same manner as with the {\tt plot} command.  The format statement used to print
the output file is chosen automatically with integers and small numbers treated
differently.  If desired, however, a format statement may be specified using the
{\tt with format} specifier; it should be a standard python format string.  If
there are not enough columns available in the supplied format statement it will
be repeated for each line of the data file.

\section{Script Watching: pyxplot\_watch}

PyXPlot includes a simple tool for watching command script files, and executing
them whenever they are modified. This may be useful when developing a command
script, if one wants to make small modifications to it, and see the results in
a semi-live fashion. This tool is invoked by calling the
\texttt{pyxplot\_watch}\index{pyxplot\_watch}\index{watching scripts} command
from a shell prompt. The command-line syntax of \texttt{pyxplot\_watch} is
similar to that of PyXPlot itself, for example:

\begin{verbatim}
pyxplot_watch script
\end{verbatim}

\noindent would set \texttt{pyxplot\_watch} to watch the command script file
\texttt{script}. One difference, however, is that if multiple script files are
specified on the command line, they are watched and executed independently,
\textit{not} sequentially, as PyXPlot itself would do. Wildcard characters can
also be used to set \texttt{pyxplot\_watch} to watch multiple
files.\footnote{Note that \texttt{pyxplot\_watch *.script} and
\texttt{pyxplot\_watch $\backslash$*.script} will behave differently in most
UNIX shells.  In the first case, the wildcard is expanded by your shell, and a
list of files passed to \texttt{pyxplot\_watch}. Any files matching the
wildcard, created after running \texttt{pyxplot\_watch}, will not be picked up.
In the latter case, the wildcard is expanded by \texttt{pyxplot\_watch} itself,
which \textit{will} pick up any newly created files.}

This is especially useful when combined with GhostView's watch facility. For
example, suppose that a script \texttt{foo} produces postscript output
\texttt{foo.ps}. The following two commands could be used to give a live view
of the result of executing this script:

\begin{verbatim}
gv --watch foo.ps &
pyxplot_watch foo
\end{verbatim}
