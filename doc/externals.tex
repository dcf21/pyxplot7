% EXTERNALS.TEX
%
% The documentation in this file is part of PyXPlot
% <http://www.pyxplot.org.uk>
%
% Copyright (C) 2006-7 Dominic Ford <coders@pyxplot.org.uk>
%               2008   Ross Church
%
% $Id$
%
% PyXPlot is free software; you can redistribute it and/or modify it under the
% terms of the GNU General Public License as published by the Free Software
% Foundation; either version 2 of the License, or (at your option) any later
% version.
%
% You should have received a copy of the GNU General Public License along with
% PyXPlot; if not, write to the Free Software Foundation, Inc., 51 Franklin
% Street, Fifth Floor, Boston, MA  02110-1301, USA

% ----------------------------------------------------------------------------

% LaTeX source for the PyXPlot Users' Guide

\chapter{PyXPlot and the Outside World}
\label{gnuplot_ext_first}

This chapter describes PyXPlot as a UNIX programme, and how it can be interfaced
with other programs. 

\section{Command Line Switches}

From the shell command line, the PyXPlot accepts the following switches which
modify its behaviour:\index{command line syntax}

\begin{longtable}{p{3.5cm}p{8.5cm}}
{\tt -h --help} & Display a short help message listing the available command-line switches.\\
{\tt -v --version} & Display the current version number of PyXPlot.\\
{\tt -q --quiet} & Turn off the display of the welcome message on startup. \\
{\tt -V --verbose} & Display the welcome message on startup, as happens by default. \\
{\tt -c --colour} & Use colour highlighting\footnote{This will only function on terminals which support colour output.} to display output in green, warning messages in amber, and error messages in red.\footnote{The author apologies to those members of the population who are red/green colourblind, but draws their attention to the following sentence.} These colours can be changed in the {\tt terminal} section of the configuration file; see Section~\ref{configuration} for more details. \\
{\tt -m --monochrome} & Do not use colour highlighting, as happens by default. \\
\end{longtable}

\section{Command Histories}

When PyXPlot is used interactively, its command-line environment is based upon
the GNU Readline Library.  This means that the up and down arrow keys can be
used to repeat or modify previously executed commands. Each user's command
history is stored in his homespace in a history file called {\tt
.pyxplot\_history}, which allows PyXPlot to remember command histories between
sessions. Additionally, a \indcmdt{save} is provided, allowing the user to save
his command history from the present session to a text file; this has the
following syntax:

\begin{verbatim}
save 'output_filename.ppl'
\end{verbatim}

The related \indcmdt{history} outputs the history to the terminal. This outputs
not only the history of the present session, but also commands entered in
previous sessions, which can be up to several hundred lines long. It can
optionally be followed by a number, to display the last $n$ commands, e.g.:

\begin{verbatim}
history 20
\end{verbatim}

\section{Reading data from a pipe}

PyXPlot usually reads data from a file, but it possible to read data via a pipe
from standard input.  To do this one uses the magic filename `{\tt -}':

\begin{verbatim}
plot '-' with lines
\end{verbatim}

This facility should be used with caution; it is generally preferable to write
data to a file in order that it can be perused at a later date.

\section{Formatting and Terminals}
\label{set_terminal2}

In this section, we describe the commands used to control the format of the
graphic output produced by PyXPlot. This continues the discussion from
Section~\ref{directing_output} of how the \indcmdt{set terminal} can be used to
produce plots in various graphic formats, such as postscript files, jpeg
images, etc.

Many of these {\it terminals} -- the word we use to describe an output format
-- accept additional parameters which configure the exact appearance of the
output produced.  For example, the default terminal, {\tt X11}, which displays
plots on the screen, has such settings. By default, each time a new plot is
generated, if the previous plot is still open on the display, the old plot is
replaced with the new one. This way, only one plot window is open at any one
time.  This behaviour has the advantage that the desktop does not become
flooded with plot windows.

If this is not desired, however -- for example if you want to compare two plots
-- old graphs can be kept visible when plotting further graphs by using the the
{\tt X11\_multiwindow} terminal:\index{X11 terminal}\index{multiple windows}

\begin{verbatim} 
set terminal X11_singlewindow
plot sin(x)
plot cos(x)  <-- first plot window disappears
\end{verbatim}

\noindent c.f.:

\begin{verbatim} 
set terminal X11_multiwindow
plot sin(x)
plot cos(x)  <-- first plot window remains
\end{verbatim}

As an additional option, the {\tt X11\_persist} terminal keeps plot windows
open after PyXPlot exits; the above two terminals close all plot windows upon
exit.

If the {\tt enlarge} modifier is used with the \indcmdt{set terminal} then the
whole plot is enlarged, or, in the case of large plots, shrunk, to the current
paper size, minus a small margin. The aspect ratio of the plot is preserved.
This is most useful with the {\tt postscript} terminal, when preparing a plot
to send directly to a printer.

As there are many changes to the options accepted by the {\tt set terminal}
command in comparison to those understood by \gnuplot, the syntax of PyXPlot's
command is given below, followed by a list of the recognised settings:

\begin{verbatim} 
set terminal { X11_singlewindow | X11_multiwindow | X11_persist |
               postscript | eps | pdf | gif | png | jpg }
             { colour | color | monochrome }
             { portrait | landscape }
             { invert | noinvert }
             { transparent | solid }
             { enlarge | noenlarge }
\end{verbatim}

\begin{longtable}{p{3cm}p{9cm}}
{\tt x11\_singlewindow} & Displays plots on the screen (in X11 windows, using \ghostview). Each time a new plot is generated, it replaces the old one, to prevent the desktop from becoming flooded with old plots.\footnote{The author is aware of a bug, that this terminal can occasionally go blank when a new plot is generated. This is a known bug in \ghostview, and can be worked around by selecting File $\to$ Reload within the \ghostview\ window.} {\bf [default when running interactively; see below]}\\
{\tt x11\_multiwindow} & As above, but each new plot appears in a new window, and the old plots remain visible. As many plots as may be desired can be left on the desktop simultaneously.\\
{\tt x11\_persist} & As above, but plot windows remain open after PyXPlot closes.\\
{\tt postscript} & Sends output to a postscript file. The filename for this file should be set using {\tt set output}. {\bf [default when running non-interactively; see below]}\index{postscript output}\\
{\tt eps} & As above, but produces encapsulated postscript.\index{encapsulated postscript}\index{postscript!encapsulated}\\
{\tt pdf} & As above, but produces pdf output.\index{pdf output}\\
{\tt gif} & Sends output to a gif image file; as above, the filename should be set using {\tt set output}.\index{gif output}\\
{\tt png} & As above, but produces a png image.\index{png output}\\
{\tt jpg} & As above, but produces a jpeg image.\index{jpeg output}\\
{\tt colour} & Allows datasets to be plotted in colour. Automatically they will be displayed in a series of different colours, or alternatively colours may be specified using the {\tt with colour} plot modifier (see below). {\bf [default]}\index{colour output}\\
{\tt color} & Equivalent to the above; provided for users of nationalities which can't spell. \smiley \\
{\tt monochrome} & Opposite to the above; all datasets will be plotted in black.\index{monochrome output}\\
{\tt portrait} & Sets plots to be displayed in upright (normal) orientation. {\bf [default]}\index{portrait orientation}\\
{\tt landscape} & Opposite of the above; produces side-ways plots. Not very useful when displayed on the screen, but you fit more on a sheet of paper that way around.\index{landscape orientation}\\
{\tt invert} & Modifier for the gif, png and jpg terminals; produces output with inverted colours.\footnote{This terminal setting is useful for producing plots to embed in talk slideshows, which often contain bright text on a dark background. It only works when producing bitmapped output, though a similar effect can be achieved in postscript using the {\tt set textcolour} and {\tt set axescolour} commands (see Section~\ref{set_colours}).}\index{colours!inverting}\\
{\tt noinvert} & Modifier for the gif, png and jpg terminals; opposite to the above. {\bf [default]}\\
{\tt transparent} & Modifier for the gif and png terminals; produces output with a transparent background.\index{transparent terminal}\index{gif output!transparency}\index{png output!transparency}\\
{\tt solid} & Modifier for the gif and png terminals; opposite to the above. {\bf [default]}\\
{\tt enlarge} & Enlarge or shrink contents to fit the current paper
size.\index{enlarging output}\\
{\tt noenlarge} & Do not enlarge output; opposite to the above. {\bf [default]}\\
\end{longtable}
\label{terminals}

The default terminal is normally {\tt x11\_singlewindow}, matching
approximately the behaviour of \gnuplot. However, there is an exception to this.
When PyXPlot is used non-interactively -- i.e.\ one or more command scripts are
specified on the command line, and PyXPlot exits as soon as it finishes
executing them -- the {\tt x11\_singlewindow} is not a very sensible
terminal to use: any plot window would close as soon as PyXPlot exited. The
default terminal in this case changes to {\tt postscript}.

This rule does not apply when the special `--' filename is specified in a list
of command scripts on the command line, to produce an interactive terminal
between running a series of scripts. In this case, PyXPlot detects that the
session will be interactive, and defaults to the usual
{\tt x11\_singlewindow} terminal.

An additional exception is on machines where the {\tt DISPLAY} environment
variable\index{display environment variable@{\tt DISPLAY} environment
variable} is not set. In this case, PyXPlot detects that it has access to no
X-terminal on which to display plots, and defaults to the {\tt postscript}
terminal.

The {\tt gif}, {\tt png} and {\tt jpg} terminals result in some loss of image
quality, since the plot has to be sampled into a bitmapped graphic format.  By
default, this sampling is performed at $300\,\mathrm{dpi}$, though this may be
changed using the command {\tt set dpi <value>}. Alternatively, it may be
changed using the {\tt DPI} option in the {\tt settings} section of a
configuration file (see Section~\ref{configuration}).\indcmd{set dpi}
\index{bitmap output!resolution}\index{image resolution}

\section{Paper Sizes}

By default, when the postscript terminal produces printable, i.e.\ not
encapsulated, output, the paper size for this output is read from the user's
system locale settings. It may be changed, however, with \indcmdt{set
papersize}, which may be followed either by the name of a recognised paper
size, or by the dimensions of a user-defined size, specified as a {\tt height},
{\tt width} pair, both being measured in millimetres. For example:

\begin{verbatim}
set papersize a4
set papersize 100,100
\end{verbatim}

\noindent A list of recognised paper size names is given in
Figure~\ref{paper_sizes}.\footnote{For everything that you ever wanted to know
about international paper sizes, see Marcus Kuhn's excellent treatise:
\url{http://www.cl.cam.ac.uk/~mgk25/iso-paper.html}. If you still want to know
more, then Wikipedia has a good article on the Swedish extensions to this
system and the Japanese B-series:
\url{http://en.wikipedia.org/wiki/Paper_size}.}\index{Kuhn, Marcus}\index{paper
sizes}

\begin{figure}
\tiny \center
\begin{tabular}{|rrr|rrr|}
\hline
{\bf Name} & {\bf $h$/mm} & {\bf $w$/mm} & {\bf Name} & {\bf $h$/mm} & {\bf $w$/mm} \\
\hline
                       2a0 &   1681 &   1189 &           medium &    584 &    457 \\
                       4a0 &   2378 &   1681 &          monarch &    267 &    184 \\
                        a0 &   1189 &    840 &             post &    489 &    394 \\
                        a1 &    840 &    594 &        quad\_demy &   1143 &   889 \\
                       a10 &     37 &     26 &           quarto &    254 &    203 \\
                        a2 &    594 &    420 &            royal &    635 &    508 \\
                        a3 &    420 &    297 &        statement &    216 &    140 \\
                        a4 &    297 &    210 &       swedish\_d0 &   1542 &   1090 \\
                        a5 &    210 &    148 &       swedish\_d1 &   1090 &    771 \\
                        a6 &    148 &    105 &      swedish\_d10 &     48 &     34 \\
                        a7 &    105 &     74 &       swedish\_d2 &    771 &    545 \\
                        a8 &     74 &     52 &       swedish\_d3 &    545 &    385 \\
                        a9 &     52 &     37 &       swedish\_d4 &    385 &    272 \\
                        b0 &   1414 &    999 &       swedish\_d5 &    272 &    192 \\
                        b1 &    999 &    707 &       swedish\_d6 &    192 &    136 \\
                       b10 &     44 &     31 &       swedish\_d7 &    136 &     96 \\
                        b2 &    707 &    499 &       swedish\_d8 &     96 &     68 \\
                        b3 &    499 &    353 &       swedish\_d9 &     68 &     48 \\
                        b4 &    353 &    249 &       swedish\_e0 &   1241 &    878 \\
                        b5 &    249 &    176 &       swedish\_e1 &    878 &    620 \\
                        b6 &    176 &    124 &      swedish\_e10 &     38 &     27 \\
                        b7 &    124 &     88 &       swedish\_e2 &    620 &    439 \\
                        b8 &     88 &     62 &       swedish\_e3 &    439 &    310 \\
                        b9 &     62 &     44 &       swedish\_e4 &    310 &    219 \\
                        c0 &   1296 &    917 &       swedish\_e5 &    219 &    155 \\
                        c1 &    917 &    648 &       swedish\_e6 &    155 &    109 \\
                       c10 &     40 &     28 &       swedish\_e7 &    109 &     77 \\
                        c2 &    648 &    458 &       swedish\_e8 &     77 &     54 \\
                        c3 &    458 &    324 &       swedish\_e9 &     54 &     38 \\
                        c4 &    324 &    229 &       swedish\_f0 &   1476 &   1044 \\
                        c5 &    229 &    162 &       swedish\_f1 &   1044 &    738 \\
                        c6 &    162 &    114 &      swedish\_f10 &     46 &     32 \\
                        c7 &    114 &     81 &       swedish\_f2 &    738 &    522 \\
                        c8 &     81 &     57 &       swedish\_f3 &    522 &    369 \\
                        c9 &     57 &     40 &       swedish\_f4 &    369 &    261 \\
                     crown &    508 &    381 &       swedish\_f5 &    261 &    184 \\
                      demy &    572 &    445 &       swedish\_f6 &    184 &    130 \\
               double\_demy &    889 &    597 &       swedish\_f7 &    130 &     92 \\
                  elephant &    711 &    584 &       swedish\_f8 &     92 &     65 \\
               envelope\_dl &    110 &    220 &       swedish\_f9 &     65 &     46 \\
                 executive &    267 &    184 &       swedish\_g0 &   1354 &    957 \\
                  foolscap &    330 &    203 &       swedish\_g1 &    957 &    677 \\
         government\_letter &    267 &    203 &      swedish\_g10 &     42 &     29 \\
international\_businesscard &     85 &     53 &       swedish\_g2 &    677 &    478 \\
               japanese\_b0 &   1435 &   1015 &       swedish\_g3 &    478 &    338 \\
               japanese\_b1 &   1015 &    717 &       swedish\_g4 &    338 &    239 \\
              japanese\_b10 &     44 &     31 &       swedish\_g5 &    239 &    169 \\
               japanese\_b2 &    717 &    507 &       swedish\_g6 &    169 &    119 \\
               japanese\_b3 &    507 &    358 &       swedish\_g7 &    119 &     84 \\
               japanese\_b4 &    358 &    253 &       swedish\_g8 &     84 &     59 \\
               japanese\_b5 &    253 &    179 &       swedish\_g9 &     59 &     42 \\
               japanese\_b6 &    179 &    126 &       swedish\_h0 &   1610 &   1138 \\
               japanese\_b7 &    126 &     89 &       swedish\_h1 &   1138 &    805 \\
               japanese\_b8 &     89 &     63 &      swedish\_h10 &     50 &     35 \\
               japanese\_b9 &     63 &     44 &       swedish\_h2 &    805 &    569 \\
            japanese\_kiku4 &    306 &    227 &       swedish\_h3 &    569 &    402 \\
            japanese\_kiku5 &    227 &    151 &       swedish\_h4 &    402 &    284 \\
         japanese\_shiroku4 &    379 &    264 &       swedish\_h5 &    284 &    201 \\
         japanese\_shiroku5 &    262 &    189 &       swedish\_h6 &    201 &    142 \\
         japanese\_shiroku6 &    188 &    127 &       swedish\_h7 &    142 &    100 \\
                large\_post &    533 &    419 &       swedish\_h8 &    100 &     71 \\
                    ledger &    432 &    279 &       swedish\_h9 &     71 &     50 \\
                     legal &    356 &    216 &          tabloid &    432 &    279 \\
                    letter &    279 &    216 &  us\_businesscard &     89 &     51 \\
\hline
\end{tabular}
\caption{A list of all of the named paper sizes recognised by the \indcmdt{set
papersize}, with their heights, $h$, and widths, $w$, measured in millimetres.}
\label{paper_sizes}
\end{figure}

\section{Script Watching: pyxplot\_watch}

PyXPlot includes a simple tool for watching command script files and executing
them whenever they are modified. This may be useful when developing a command
script, if one wants to make small modifications to it and see the results in a
semi-live fashion. This tool is invoked by calling the {\tt
pyxplot\_watch}\index{pyxplot\_watch}\index{watching scripts} command from a
shell prompt. The command-line syntax of {\tt pyxplot\_watch} is similar to
that of PyXPlot itself, for example:

\begin{verbatim}
pyxplot_watch script.ppl
\end{verbatim}

\noindent would set {\tt pyxplot\_watch} to watch the command script file
{\tt script.ppl}. One difference, however, is that if multiple script files are
specified on the command line, they are watched and executed independently,
\textit{not} sequentially, as PyXPlot itself would do. Wildcard characters can
also be used to set {\tt pyxplot\_watch} to watch multiple
files.\footnote{Note that {\tt pyxplot\_watch *.script} and
{\tt pyxplot\_watch $\backslash$*.script} will behave differently in most
UNIX shells.  In the first case, the wildcard is expanded by your shell, and a
list of files passed to {\tt pyxplot\_watch}. Any files matching the
wildcard, created after running {\tt pyxplot\_watch}, will not be picked up.
In the latter case, the wildcard is expanded by {\tt pyxplot\_watch} itself,
which {\it will} pick up any newly created files.}

This is especially useful when combined with \ghostview's\index{Ghostview}
watch facility. For example, suppose that a script {\tt foo.ppl} produces
postscript output {\tt foo.ps}. The following two commands could be used to
give a live view of the result of executing this script:

\begin{verbatim}
gv --watch foo.ps &
pyxplot_watch foo.ppl
\end{verbatim}

\section{Variables}

As has already been hinted at in Section~\ref{string_subs_op}, PyXPlot
recognises two types of variables: numeric variables and string variables.  The
former can be assigned using any valid mathematical expression. For example:

\begin{verbatim}
a = 5.2 * sqrt(64)
\end{verbatim}

\noindent would assign the value 41.6 to the variable {\tt a}.  Numerical variables can
subsequently be used in mathematical expressions themselves, for example:

\begin{verbatim}
a=2*pi
plot [0:1] sin(a*x)
\end{verbatim}

\noindent String variables can be assigned in an analogous manner, by enclosing
the string in quotation marks. They can then be used wherever a quoted string
could be used, for example as a filename or a plot title, as
in:\index{variables!string}

\begin{verbatim}
plotname = "The Growth of a Rabbit Population"
set title plotname
\end{verbatim}

String variables can be modified using the search-and-replace string
operator\index{string operators!search and replace}\footnote{Programmers with
experience of {\tt perl} will recognise this syntax.}, =$\sim$\index{=$\sim$
operator}, which takes a regular expression with a syntax similar to that
expected by the shell command {\tt sed}\index{sed shell command@{\tt sed} shell
command} and applies it to the relevant string.\footnote{Regular expression
syntax is a massive subject, and is beyond the scope of this manual. The
official GNU documentation for the {\tt sed} command is heavy reading, but
there are many more accessible tutorials on the web.}\index{regular
expressions} For example:

\begin{verbatim}
twister="seven silver soda syphons"
twister =~ s/s/th/
print twister
\end{verbatim}

Note that only the {\tt s} (substitute) command of {\tt sed} is implemented in
PyXPlot. Any character can be used in place of the {\tt /} characters in the
above example, for example:

\begin{verbatim}
twister =~ s@s@th@
\end{verbatim}

\noindent Flags can be passed, as in {\tt sed} or {\tt perl}, for example:

\begin{verbatim}
twister =~ s@s@th@g
\end{verbatim}

\noindent Table~\ref{re_flags} lists all of the regular expression flags
recognised by the =$\sim$ operator.

\begin{table}
\begin{tabular}{p{2cm}p{10.5cm}}
\hline
{\tt g} & Replace {\it all} matches of the pattern; by default, only the first match is replaced. \\
{\tt i} & Perform case-insensitive matching, such that expressions like {\tt [A-Z]} will match lowercase letters, too. \\
{\tt l} & Make {\tt $\backslash$w}, {\tt $\backslash$W}, {\tt $\backslash$b}, {\tt $\backslash$B}, {\tt $\backslash$s} and {\tt $\backslash$S} dependent on the current locale. \\
{\tt m} & When specified, the pattern character {\tt \^{}} matches the beginning of the string and the beginning of each line immediately following each newline. The pattern character {\tt \$} matches at the end of the string and the end of each line immediately preceding each newline. By default, {\tt \^{}} matches only the beginning of the string, and {\tt \$} only the end of the string and immediately before the newline, if present, at the end of the string. \\
{\tt s} & Make the {\tt .} special character match any character at all, including a newline; without this flag, {\tt .} will match anything except a newline. \\
{\tt u} & Make {\tt $\backslash$w}, {\tt $\backslash$W}, {\tt $\backslash$b}, {\tt $\backslash$B}, {\tt $\backslash$s} and {\tt $\backslash$S} dependent on the Unicode character properties database. \\
{\tt x} & This flag allows the user to write regular expressions that look nicer. Whitespace within the pattern is ignored, except when in a character class or preceded by an unescaped backslash. When a line contains a {\tt \#}, neither in a character class or preceded by an unescaped backslash, all characters from the leftmost such {\tt \#} through the end of the line are ignored. \\
\hline
\end{tabular}
\caption{A list of the flags accepted by the =$\sim$ operator. Most are rarely used, but the {\tt g} flag is very useful. This table is adapted from Guido van Rossum's\index{van Rossum, Guido} {\it Python Library Reference}\index{Python Library Reference}: \protect\url{http://docs.python.org/lib/node46.html}.}
\label{re_flags}
\end{table}

Strings may also be put together using the string substitution operator, {\tt
\%}\index{\% operator@{\tt \%} operator}, which works in a similar fashion to
Python string substitution operator\index{string operators!substitution}. This
is described in detail in Section~\ref{string_subs_op}.  For example, to
concatenate the two strings contained in variables {\tt a} and {\tt b} into
variable {\tt c} one would run:\index{string operators!concatenation}

\begin{verbatim}
c = "%s%s"%(a,b)
\end{verbatim}

One common practical application of these string operators is to label plots
with the title of the \datafile\ being plotted, as in:

\begin{verbatim}
filename="data_file.dat"
title="A plot of the data in {\tt %s}."%(filename)
title=~s/_/\_/g # Underscore is a reserved character in LaTeX
set title title
plot filename
\end{verbatim}

\section{The \indcmdt{exec}}

The \indcmdt{exec} can be used to execute PyXPlot commands contained within
string variables. For example:

\begin{verbatim}
terminal="eps"
exec "set terminal %s"%(terminal)
\end{verbatim}

It can also be used to write obfuscated PyXPlot scripts.

