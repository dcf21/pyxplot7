\chapter{Controlling the Appearance of your Plots}

\section{Formatting and Terminals}
\label{set_terminal2}

In this section I shall outline the new and modified commands for controlling
the graphic output format of PyXPlot.

The widths of plots may be set be means of two commands -- \texttt{set
size}\index{set size command@\texttt{set size} command} and \texttt{set
width}\index{set width command@\texttt{set width} command}. Both are
equivalent, and should be followed by the desired width measured in
centimetres, for example:

\begin{verbatim}
set width 20
\end{verbatim}

The \texttt{set size} command can also be used to set the aspect ratio of plots
by following it with the keyword \texttt{ratio}\index{set size ratio
command@\texttt{set size ratio} command}. The number which follows should be
the desired ratio of height to width. The following, for example would produce
plots three times as high as they are wide:

\begin{verbatim}
set size ratio 3.0
\end{verbatim}

The command \texttt{set size noratio} returns to PyXPlot's default aspect ratio
of the golden ratio, i.e. $\left((1+\sqrt{5})/2\right)^{-1}$, which matches
that of a sheet of A4 paper\footnote{Of less practical significance, it has
been in use since the time of the Pythagoreans, and is seen repeatedly in the
architecture of the Parthenon.}.  The special command \texttt{set size
square}\index{set size square command@\texttt{set size square} command} sets
the aspect ratio to unity.

If the \texttt{enlarge} modifier is used with the {\tt set terminal} command
then the whole plot is enlarged or, in the case of large plots, shrunk to the
current paper size (minus a small margin).  The aspect ratio of the plot is
preserved.  This is perhaps most useful when preparing a plot to send to a
printer with the postscript terminal.

In Section~\ref{directing_output} I described how the \texttt{set terminal}
command\index{set terminal command@\texttt{set terminal} command} can be used
to produce plots in various graphic formats. In addition, I here describe how
the way in which plots are displayed on the screen can be changed. The default
terminal, \texttt{X11}, is used to send output to screen.

By default, each time a new plot is generated, if the previous plot is still
open on the display, the X11 terminal will replace it with the new one, thus
keeping only one plot window open at a time. This has the advantage that the
desktop does not become flooded with plot windows.

If this behaviour is not desired, old plots can be kept visible when plotting
further graphs by using the the \texttt{X11\_multiwindow} terminal: \index{X11
terminal}\index{multiple windows}

\begin{verbatim} 
set terminal X11_singlewindow
plot sin(x)
plot cos(x)  <-- first plot window disappears
\end{verbatim}

\noindent c.f.:

\begin{verbatim} 
set terminal X11_multiwindow
plot sin(x)
plot cos(x)  <-- first plot window remains
\end{verbatim}

As an additional option, the \texttt{X11\_persist} terminal keeps plot windows
open after PyXPlot exits; the above two terminals close all plot windows upon
exit.

\index{set terminal command@\texttt{set terminal} command} As there are many
changes to the options accepted by the \texttt{set terminal} command in
comparison to those understood by gnuplot, the syntax of PyXPlot's command is
given below, followed by a list of the recognised settings:

\begin{verbatim} 
set terminal { X11_singlewindow | X11_multiwindow | X11_persist |
               postscript | eps | pdf | gif | png | jpg }
             { colour | color | monochrome }
             { portrait | landscape }
             { invert | noinvert }
             { transparent | solid }
             { enlarge | noenlarge }
\end{verbatim}

\begin{longtable}{p{3cm}p{9cm}}
\texttt{x11\_singlewindow} & Displays plots on the screen (in X11 windows, using ghostview). Each time a new plot is generated, it replaces the old one, preventing the desktop from becoming flooded with old plots.\footnote{The author is aware of a bug, that this terminal can occasionally go blank when a new plot is generated. This is a known bug in ghostview, and can be worked around by selecting File $\to$ Reload within the ghostview window.} \textbf{[default when running interactively; see below]}\\
\end{longtable} % WHY IS THIS NECESSARY??
\begin{longtable}{p{3cm}p{9cm}}
\texttt{x11\_multiwindow} & As above, but each new plot appears in a new window, and the old plots remain visible. As many plots as may be desired can be left on the desktop simultaneously.\\
\texttt{x11\_persist} & As above, but plot windows remain open after PyXPlot closes.\\
\texttt{postscript} & Sends output to a postscript file. The filename for this file should be set using \texttt{set output}. \textbf{[default when running non-interactively; see below]}\index{postscript output}\\
\texttt{eps} & As above, but produces encapsulated postscript.\index{encapsulated postscript}\index{postscript!encapsulated}\\
\texttt{pdf} & As above, but produces pdf output.\index{pdf output}\\
\texttt{gif} & Sends output to a gif image file; as above, the filename should be set using \texttt{set output}.\index{gif output}\\
\texttt{png} & As above, but produces a png image.\index{png output}\\
\texttt{jpg} & As above, but produces a jpeg image.\index{jpeg output}\\
\texttt{colour} & Allows datasets to be plotted in colour. Automatically they will be displayed in a series of different colours, or alternatively colours may be specified using the \texttt{with colour} plot modifier (see below). \textbf{[default]}\index{colour output}\\
\texttt{color} & Equivalent to the above; provided for users of nationalities which can't spell. \smiley \\
\texttt{monochrome} & Opposite to the above; all datasets will be plotted in black.\index{monochrome output}\\
\texttt{portrait} & Sets plots to be displayed in upright (normal) orientation. \textbf{[default]}\index{portrait orientation}\\
\texttt{landscape} & Opposite of the above; produces side-ways plots. Not very useful when displayed on the screen, but you fit more on a sheet of paper that way around.\index{landscape orientation}\\
\texttt{invert} & Modifier for the gif, png and jpg terminals; produces output with inverted colours.\footnote{This terminal setting is useful for producing plots to embed in talk slideshows, which often contain bright text on a dark background. It only works when producing bitmapped output, though a similar effect can be achieved in postscript using the \texttt{set textcolour} and \texttt{set axescolour} commands (see Section~\ref{set_colours}).}\index{colours!inverting}\\
\texttt{noinvert} & Modifier for the gif, png and jpg terminals; opposite to the above. \textbf{[default]}\\
\texttt{transparent} & Modifier for the gif and png terminals; produces output with a transparent background.\index{transparent terminal}\index{gif output!transparency}\index{png output!transparency}\\
\texttt{solid} & Modifier for the gif and png terminals; opposite to the above. \textbf{[default]}\\
\texttt{enlarge} & Enlarge or shrink contents to fit the current paper
size.\index{enlarging output}\\
\texttt{noenlarge} & Do not enlarge output; opposite to the above. \textbf{[default]}\\
\end{longtable}
\label{terminals}

The default terminal is normally \texttt{x11\_singlewindow}, matching
approximately the behaviour of gnuplot. However, there is an exception to this.
When PyXPlot is used non-interactively -- i.e. one or more command scripts are
specified on the command line, and PyXPlot exits as soon as it finishes
executing them -- the \texttt{x11\_singlewindow} is not a very sensible
terminal to use. Any plot window would close as soon as PyXPlot exited. The
default terminal in this case changes to \texttt{postscript}.

One exception to this is when the special `--' filename is specified in a list
of command scripts on the command line, to produce an interactive terminal
between running a series of scripts. In this case, PyXPlot detects that the
session will be interactive, and defaults to the usual
\texttt{x11\_singlewindow} terminal.

An additional exception is on machines where the \texttt{DISPLAY} environment
variable\index{display environment variable@\texttt{DISPLAY} environment
variable} is not set. In this case, PyXPlot detects that it has access to no
X-terminal on which to display plots, and defaults to the \texttt{postscript}
terminal.

The \texttt{gif}, \texttt{png} and \texttt{jpg} terminals result in some loss
of quality, since the plot has to be sampled into a bitmapped graphic format.
By default, this sampling is performed at 300 dpi, though this may be changed
using the command \texttt{set dpi <value>}. Alternatively, it may be changed
using the \texttt{DPI} option in the \texttt{settings} section of a
configuration file (see Section~\ref{configuration}).\index{set dpi
command@\texttt{set dpi} command}\index{resolution of bitmap
output}\index{image resolution}

\subsection{Paper Sizes}

By default, when the postscript terminal produces printable, i.e. not
encapsulated, output, the paper size for this output is read from your system
locale settings. It may be changed, however, with the \texttt{set
papersize}\index{set papersize command@\texttt{set papersize} command} command,
which may be followed either by the name of a recognised paper size, or by the
dimensions of a user-defined size, specified as a \texttt{height},
\texttt{width} pair, both being measured in millimetres. For example:

\begin{verbatim}
set papersize a4
set papersize 100,100
\end{verbatim}

A list of recognised paper size names is given in Figure~\ref{paper_sizes}.

\begin{figure}
\tiny \center
\begin{tabular}{|rrr|rrr|}
\hline
\textbf{Name} & \textbf{$h$/mm} & \textbf{$w$/mm} & \textbf{Name} & \textbf{$h$/mm} & \textbf{$w$/mm} \\
\hline
                       2a0 &   1681 &   1189 &           medium &    584 &    457 \\
                       4a0 &   2378 &   1681 &          monarch &    267 &    184 \\
                        a0 &   1189 &    840 &             post &    489 &    394 \\
                        a1 &    840 &    594 &        quad\_demy &   1143 &   889 \\
                       a10 &     37 &     26 &           quarto &    254 &    203 \\
                        a2 &    594 &    420 &            royal &    635 &    508 \\
                        a3 &    420 &    297 &        statement &    216 &    140 \\
                        a4 &    297 &    210 &       swedish\_d0 &   1542 &   1090 \\
                        a5 &    210 &    148 &       swedish\_d1 &   1090 &    771 \\
                        a6 &    148 &    105 &      swedish\_d10 &     48 &     34 \\
                        a7 &    105 &     74 &       swedish\_d2 &    771 &    545 \\
                        a8 &     74 &     52 &       swedish\_d3 &    545 &    385 \\
                        a9 &     52 &     37 &       swedish\_d4 &    385 &    272 \\
                        b0 &   1414 &    999 &       swedish\_d5 &    272 &    192 \\
                        b1 &    999 &    707 &       swedish\_d6 &    192 &    136 \\
                       b10 &     44 &     31 &       swedish\_d7 &    136 &     96 \\
                        b2 &    707 &    499 &       swedish\_d8 &     96 &     68 \\
                        b3 &    499 &    353 &       swedish\_d9 &     68 &     48 \\
                        b4 &    353 &    249 &       swedish\_e0 &   1241 &    878 \\
                        b5 &    249 &    176 &       swedish\_e1 &    878 &    620 \\
                        b6 &    176 &    124 &      swedish\_e10 &     38 &     27 \\
                        b7 &    124 &     88 &       swedish\_e2 &    620 &    439 \\
                        b8 &     88 &     62 &       swedish\_e3 &    439 &    310 \\
                        b9 &     62 &     44 &       swedish\_e4 &    310 &    219 \\
                        c0 &   1296 &    917 &       swedish\_e5 &    219 &    155 \\
                        c1 &    917 &    648 &       swedish\_e6 &    155 &    109 \\
                       c10 &     40 &     28 &       swedish\_e7 &    109 &     77 \\
                        c2 &    648 &    458 &       swedish\_e8 &     77 &     54 \\
                        c3 &    458 &    324 &       swedish\_e9 &     54 &     38 \\
                        c4 &    324 &    229 &       swedish\_f0 &   1476 &   1044 \\
                        c5 &    229 &    162 &       swedish\_f1 &   1044 &    738 \\
                        c6 &    162 &    114 &      swedish\_f10 &     46 &     32 \\
                        c7 &    114 &     81 &       swedish\_f2 &    738 &    522 \\
                        c8 &     81 &     57 &       swedish\_f3 &    522 &    369 \\
                        c9 &     57 &     40 &       swedish\_f4 &    369 &    261 \\
                     crown &    508 &    381 &       swedish\_f5 &    261 &    184 \\
                      demy &    572 &    445 &       swedish\_f6 &    184 &    130 \\
               double\_demy &    889 &    597 &       swedish\_f7 &    130 &     92 \\
                  elephant &    711 &    584 &       swedish\_f8 &     92 &     65 \\
               envelope\_dl &    110 &    220 &       swedish\_f9 &     65 &     46 \\
                 executive &    267 &    184 &       swedish\_g0 &   1354 &    957 \\
                  foolscap &    330 &    203 &       swedish\_g1 &    957 &    677 \\
         government\_letter &    267 &    203 &      swedish\_g10 &     42 &     29 \\
international\_businesscard &     85 &     53 &       swedish\_g2 &    677 &    478 \\
               japanese\_b0 &   1435 &   1015 &       swedish\_g3 &    478 &    338 \\
               japanese\_b1 &   1015 &    717 &       swedish\_g4 &    338 &    239 \\
              japanese\_b10 &     44 &     31 &       swedish\_g5 &    239 &    169 \\
               japanese\_b2 &    717 &    507 &       swedish\_g6 &    169 &    119 \\
               japanese\_b3 &    507 &    358 &       swedish\_g7 &    119 &     84 \\
               japanese\_b4 &    358 &    253 &       swedish\_g8 &     84 &     59 \\
               japanese\_b5 &    253 &    179 &       swedish\_g9 &     59 &     42 \\
               japanese\_b6 &    179 &    126 &       swedish\_h0 &   1610 &   1138 \\
               japanese\_b7 &    126 &     89 &       swedish\_h1 &   1138 &    805 \\
               japanese\_b8 &     89 &     63 &      swedish\_h10 &     50 &     35 \\
               japanese\_b9 &     63 &     44 &       swedish\_h2 &    805 &    569 \\
            japanese\_kiku4 &    306 &    227 &       swedish\_h3 &    569 &    402 \\
            japanese\_kiku5 &    227 &    151 &       swedish\_h4 &    402 &    284 \\
         japanese\_shiroku4 &    379 &    264 &       swedish\_h5 &    284 &    201 \\
         japanese\_shiroku5 &    262 &    189 &       swedish\_h6 &    201 &    142 \\
         japanese\_shiroku6 &    188 &    127 &       swedish\_h7 &    142 &    100 \\
                large\_post &    533 &    419 &       swedish\_h8 &    100 &     71 \\
                    ledger &    432 &    279 &       swedish\_h9 &     71 &     50 \\
                     legal &    356 &    216 &          tabloid &    432 &    279 \\
                    letter &    279 &    216 &  us\_businesscard &     89 &     51 \\
\hline
\end{tabular}
\caption{A list of all of the named paper sizes recognised by the \texttt{set papersize}\index{set
papersize command@\texttt{set papersize} command} command, with their heights, $h$, and widths, $w$, measured in millimetres.}
\label{paper_sizes}
\end{figure}
\section{Plotting}

In this section I outline some of the extensions of the \texttt{plot} command,
to give greater flexibility in the appearance of graphs.

TODO: The top half of this really needs breaking off and sticking in "plotting",
but at the moment it's a continuous narative, so I'm not sure how to achieve
this.

\subsection{Configuring Axes}
\label{axis_extensions}\label{ranges_multiaxes}\label{multiple_axes}

By default, plots have only one $x$-axis and one $y$-axis. Further parallel
axes can be added and configured via statements such as:\index{axes
modifier@\texttt{axes} modifier}\index{set axis command@\texttt{set axis}
command}

\begin{verbatim}
set x3label 'foo'
plot sin(x) axes x3y1
set axis x3
\end{verbatim}

\noindent In the top statement, a further $x$ axis, called $x3$, is implicitly
created by giving it a label. In the next, the \texttt{axes} modifier is used
to tell the \texttt{plot} command to plot data against the $x3$-axis, which
also implicitly created such an axis if it doesn't already exist. In the third,
an $x3$-axis is explicitly created.

Unlike gnuplot, which allowed only a maximum of two parallel axes to be added
to plots, PyXPlot allows an unlimited number of axes to be used. Odd-numbered
$x$-axes appear below the plot, and even numbered $x$-axes above it; a similar
rule applies for $y$-axes, to the left and to the right. This is illustrated in
Figure~\ref{fig:ex_multiaxes}.

\begin{figure}
\begin{center}
\includegraphics{examples/eps/ex_multiaxes.eps}
\end{center}
\caption{A plot demonstating the use of large numbers of axes. Odd-numbered
$x$-axes appear below the plot, and even numbered $x$-axes above it; a similar
rule applies for $y$-axes, to the left and to the right.}
\label{fig:ex_multiaxes}
\end{figure}

As discussed in the previous chapter, the ranges of axes can be set either
using the \texttt{set xrange} command\index{set xrange command@\texttt{set
xrange} command}, or within the \texttt{plot} command. The following two
statements would set equivalent ranges for the $x3$-axis:

\begin{verbatim}
set x3range [-2:2]
plot [:][:][:][:][-2:2] sin(x) axes x3y1
\end{verbatim}

\noindent As usual, the first two ranges specified in the \texttt{plot} command
apply to the $x$- and $y$-axes. The next pair apply to the $x2$- and $y2$-axes,
and so forth.

\index{axes!removal}\index{removing axes}\index{hidden
axes}\label{axis_removal} Having made axes with the above commands, they may
subsequently be removed using the \texttt{unset axis} command as follows:

\begin{verbatim}
unset axis x3
unset axis x3x5y3 y7
\end{verbatim}

\noindent The top statement, for example, would remove axis $x3$. The command
\texttt{unset axis} on its own, with no axes specified, returns all axes to
their default configuration.  The special case of \texttt{unset axis x1} does
not remove the first $x$-axis -- it cannot be removed -- but instead returns it
to its default configuration.

It should be noted, that if the following two commands are typed in succession,
the second may not entirely negate the first:

\begin{verbatim}
set x3label 'foo'
unset x3label 'foo'
\end{verbatim}

\noindent The first may have implicitly created an $x3$-axis, which would need
to be removed with the \texttt{unset axis x3} command.

A subtly different task is that of removing labels from axes, or setting axes
not to display. To achieve this, a number of special axis labels are used.
Labelling an axis ``\texttt{nolabels}''\index{nolabels
keyword@\texttt{nolabels} keyword} has the effect that no title or numerical
labels are placed upon it. Labelling it\label{nolabelstics}
``\texttt{nolabelstics}''\index{nolabelstics keyword@\texttt{nolabelstics}
keyword} is stronger still; this removes all tick marks from it as well
(similar in effect to \texttt{set noxtics} in gnuplot). Finally, labelling it
``\texttt{invisible}''\index{invisible keyword@\texttt{invisible} keyword}
makes an axis completely invisible.

Labels may be placed on such axes, by following the magic keywords above with a
colon and the desired title, for example:

\begin{verbatim}
set xlabel 'nolabels:Time'
\end{verbatim}

\noindent would produce an $x$-axis with no numeric labels, but a label of
`Time'.

In the unlikely event of wanting
to label a normal axis with one of these magic words\index{axes!reserved
labels}\index{magic axis labels}, this may be achieved by prefixing the magic
word with a space. There is one further magic axis label, \texttt{linkaxis},
which will be described in Section~\ref{linked_axes}.

The ticks of axes can be configured to point either inward, towards the plot,
as is the default, or outward towards the axis labels, or in both directions.
This is achieved using the \texttt{set xticdir} command, for example:

\begin{verbatim}
set xticdir inward
set y2ticdir outward
set x2ticdir both
\end{verbatim}

The position of ticks along each axis can be configured with the \texttt{set
xtics}\index{set xtics command@\texttt{set xtics} command} command. The
appearance of ticks along any axis can be turned off with the \texttt{set
noxtics}\index{set noxtics command@\texttt{set noxtics} command} command. The
syntax for these is given below:

\begin{verbatim}
set xtics { axis | border | inward | outward | both }
          {  autofreq
           | <increment>
           | <minimum>, <increment> { , <maximum> }
           | (     {"label"} <position>
               { , {"label"} <position> } .... )
          }
set noxtics
show xtics
\end{verbatim}

The keywords \texttt{inward}, \texttt{outward} and \texttt{both} alter the
directions of the ticks, and have the same effect as in the \texttt{set
xticdir} command. The keyword \texttt{axis} is an alias for \texttt{inward},
and \texttt{border} an alias for \texttt{outward}, both provided for gnuplot
compatibility. If the keyword \texttt{autofreq} is given, the automatic
placement of ticks on the axis is restored.

If \texttt{<minimum>, <increment>, <maximum>} are specified, then ticks are
placed at evenly spaced intervals between the specified limits. In the case of
logarithmic axes, \texttt{<increment>} is applied multiplicatively.

Alternatively, the final form allows ticks to be placed on an axis
individually, and each given its own textual label.

The following pair of examples would both place tick marks at $x=$2, 3, 4, 5.
In the second example, they would be labelled ``a'', ``b'', ``c'' and ``d'':

\begin{verbatim}
set xtics 2, 1, 5

set x2tics ("a" 2, "b" 3, "c" 4, "d" 5)
\end{verbatim}

The following example would place tick marks at intervals of two units along
the $x3$-axis:

\begin{verbatim}
set x3tics 2
\end{verbatim}

The following example would restore the automatic placement of ticks along the
$x4$-axis, placing those ticks facing outwards from the graph:

\begin{verbatim}
set x4tics border autofreq
\end{verbatim}

\begin{figure}
\begin{center}
\includegraphics{examples/eps/ex_axistics.eps}
\end{center}
\caption{A plot demonstrating the use of the {\tt set xtics} command. The commands used to create the axes in this plot are as given in the text.}
\label{fig:ex_axistics}
\end{figure}


All of the examples above are illustrated in Figure~\ref{fig:ex_axistics}.
Minor tick marks can be placed on axes with the \texttt{set mxtics} command,
which has the same syntax as above.

\subsection{Keys and Legends}

By default, plots are displayed with a legend in their top-right corners. The
textual description of each dataset is drawn by default from the command used
to plot it. Alternatively, the user may specify his own description for each
dataset by following the \texttt{plot} command with the \texttt{title}
modifier\index{title modifier@\texttt{title} modifier}, as follows:

\begin{verbatim}
plot sin(x) title 'A sine wave'
plot cos(x) title ''
\end{verbatim}

In the lower case, a blank title is specified, in which case, PyXPlot makes no
entry for this dataset in the legend. This is useful if it is desired to place
some but not all datasets into the legend of a plot.  Alternatively, the
production of the legend can be completely turned off for all datasets, by the
command \texttt{set nokey}. The opposite effect can be achieved by the
\texttt{set key}\index{set key@\texttt{set key} command} command.

This latter command can also be used to dictate where on the plot the legend
should be placed, using a syntax along the lines of:

\begin{verbatim}
set key top right
\end{verbatim}

The following recognised positioning keywords are self-explanatory:
\texttt{top}, \texttt{bottom}, \texttt{left}, \texttt{right}, \texttt{xcentre}
and \texttt{ycentre}. The word \texttt{outside} places the key outside the
plot, on its right side. The word \texttt{below} places the legend below the
plot.

In addition, two positional offset coordinates may be specified after such
keywords -- the first value is assumed to be an $x$-offset, and the second a
$y$-offset, in units approximately equal to the size of the plot. For example:

\begin{verbatim}
set key bottom left 0.0 -0.5
\end{verbatim}

\noindent would display a key below the bottom left corner of the graph.

By default, entries in the key are placed in a single vertical list. They can
instead be arranged into a number of columns by means of the \texttt{set
keycolumns} command.\index{set keycolumns command@\texttt{set keycolumns}
command}. This should be followed by the integer number of desired columns, for
example:

\begin{verbatim}
set keycolumns 2
\end{verbatim}

An example of a plot with a two-column legend is given in Figure~\ref{fig:ex_legends}.

\begin{figure}
\begin{center}
\includegraphics{examples/eps/ex_legends.eps}
\end{center}
\caption{An example of a plot with a two-column legend, positioned below the plot using {\tt set key below}.}
\label{fig:ex_legends}
\end{figure}

\subsection{The \texttt{linestyle} Keyword}

At times, the string of style keywords following the \texttt{with} modifier in
\texttt{plot} commands can grow rather unwieldily long. For clarity, frequently
used plot styles can be stored as ``linestyles''; this is true of styles
involving points as well as lines. The syntax for setting a linestyle is:

\begin{verbatim}
set linestyle 2 points pointtype 3
\end{verbatim}

\noindent where the ``2'' is the identification number of the linestyle. In a
subsequent \texttt{plot} statement, this linestyle can be recalled as follows:

\begin{verbatim}
plot sin(x) with linestyle 2
\end{verbatim}

\subsection{Colour Plotting}

\index{colours!setting for datasets} In the \texttt{with}
clause of the plot command, the modifier \texttt{colour}, (abbrev.
`\texttt{c}'), allows the colour of each dataset to be manually selected. It
should be followed either by an integer, to set a colour from the present
palette, or by a colour name. A list of valid colour names is given in
Section~\ref{colour_names}. For example:

\begin{verbatim}
plot sin(x) with c 5
plot sin(x) with colour blue
\end{verbatim}

\noindent The \texttt{colour} modifier can also be used when defining linestyles.

\index{colours!setting the palette}\index{set palette command@\texttt{set
palette} command} PyXPlot has a palette of colours which it assigns
sequentially to datasets when colours are not manually assigned. This is also
the palette to which integers passed to \texttt{set colour} refer -- the `5'
above, for example. It may be set using the \texttt{set palette} command, which
differs in syntax from gnuplot. It should be followed by a comma-separated list
of colours, for example:

\begin{verbatim}
set palette red,green,blue
\end{verbatim}

Another way of setting the palette, in a configuration file, is described in
Section~\ref{config_files}; a list of valid colour names is given in
Section~\ref{colour_names}.

\subsection{General Extensions Beyond Gnuplot}

\begin{longtable}{p{3cm}lp{9cm}}



datafile wildcards\index{globbing}\index{wildcards}\index{datafiles!globbing} & --- & PyXPlot allows the wildcards `*' and `?' to be used both in the filenames of datafiles following the \texttt{plot} command, and also when specifying command files on the command line and with the \texttt{load} command. For example, the following would plot all datafiles in the current directory with a `.dat' suffix, using the same plot options:
\\ & &
\begin{verbatim}
plot '*.dat' with linewidth 2
\end{verbatim}

In the legend, full filenames are displayed, allowing the datafiles to be distinguished.
\\ & &
As in gnuplot, a blank filename passed to the plot command causes the last used datafile to be used again.
\\

backing up overwritten files\index{set backup command@\texttt{set backup}
command}\index{overwriting files}\index{backup files}\label{filebackup} & ---
& By default, when plotting to a file, if the output filename matches that of
an existing file, that file is overwritten. This behaviour may be changed with
the \texttt{set backup} command, which has syntax:
\\ & &
\begin{verbatim}
set backup
set nobackup
\end{verbatim}
\\ & &
When this switch is turned on, pre-existing files will be renamed with a tilde at the end of their filenames, rather than being overwritten. \\

\end{longtable}
\section{Sundry Items (Arrows, Text Labels, and More)}

This section describes how to put arrows and text labels on plots; the syntax
is similar to that used by gnuplot, but slightly changed. It is now possible,
for example, to set the linestyles and colours with which arrows should be
drawn.  Also covered is how to put grids onto plots, and how to change the size
and colour of textual labels on plots.

\subsection{Arrows}

\label{set_arrow}\index{arrows}\index{set arrow command@\texttt{set arrow}
command} Arrows may be placed on plots using the \texttt{set arrow} command,
which has similar syntax to that used by gnuplot. A simple example would be:

\begin{verbatim}
set arrow 1 from 0,0 to 1,1
\end{verbatim}

\noindent The number `1' immediately following `set arrow' specifies an identification
number for the arrow, allowing it to be subsequently removed via:

\begin{verbatim}
unset arrow 1
\end{verbatim}

\noindent or equivalently, via:\index{set noarrow command@\texttt{set noarrow}
command}

\begin{verbatim}
set noarrow 1
\end{verbatim}

In PyXPlot, this syntax is extended; the \texttt{set arrow} command can be
followed by the keyword `\texttt{with}', to specify the style of the arrow. For
example, the specifiers `\texttt{nohead}', `\texttt{head}' and
`\texttt{twohead}', after the keyword `\texttt{with}', can be used to make
arrows with no arrow heads, normal arrow heads, or two arrow heads.
`\texttt{twoway}' is an alias for `\texttt{twohead}'.  For example:

\begin{verbatim}
set arrow 1 from 0,0 to 1,1 with nohead
\end{verbatim}

In addition, linestyles and colours can be specified after the keyword
`\texttt{with}':

\begin{verbatim}
set arrow 1 from 0,0 to 1,1 with nohead \
linetype 1 c blue
\end{verbatim}

As in gnuplot, the coordinates for the start and end points of the arrow can be
specified in a range of coordinate systems. `\texttt{first}', the default,
measures the graph using the $x$- and $y$-axes. `\texttt{second}' uses the $x2$-
and $y2$-axes. `\texttt{screen}' and `\texttt{graph}' both measure in centimetres
from the origin of the graph. In the following example, we use these
specifiers, and specify coordinates using variables rather than doing so
explicitly:

\begin{verbatim}
x0 = 0.0
y0 = 0.0
x1 = 1.0
y1 = 1.0
set arrow 1 from first  x0, first  x1 \
            to   screen x1, screen x1 \
            with nohead
\end{verbatim}

In addition to these four options, which are those available in gnuplot, the
syntax `\texttt{axis}\textit{n}' may also be used, to use the $n\,$th $x$- or
$y$-axis -- for example, `\texttt{axis3}'.\index{set arrow command@\texttt{set
arrow} command} This allows arrows to reference any arbitrary axis on plots
which make use of large numbers of parallel axes (see
Section~\ref{multiple_axes}).

\subsection{Text Labels}

Text labels may be placed on plots using the \texttt{set label}
command\index{set label command@\texttt{set label} command}. As with all
textual labels in PyXPlot, these are rendered in \LaTeX:

\begin{verbatim}
set label 1 'Hello World' at 0,0
\end{verbatim}

As in the previous section, the number `1' is a reference number, which allows
the label to be removed by either of the following two commands:

\begin{verbatim}
set nolabel 1
unset label 1
\end{verbatim}

\noindent The positional coordinates for the text label, placed after the
keyword `\texttt{at}', can be specified in any of the coordinate systems
described for arrows above. A rotation angle may optionally be specified after
the keyword `{\tt rotate}', to rotate text counter-clockwise by a given
angle, measured in degrees. For example, the following would produce
upward-running text:

\begin{verbatim}
set label 1 'Hello World' at axis3 3.0, axis4 2.7 rotate 90
\end{verbatim}

\index{fontsize}\index{text!size}\index{set fontsize command@\texttt{set
fontsize} command} The fontsize of these text labels can globally be set using
the \texttt{set fontsize x} command. This applies not only to the \texttt{set
label} command, but also to plot titles, axis labels, keys, etc. The value
given should be an integer in the range $-4 \leq x \leq 5$. The default is
zero, which corresponds to \LaTeX's \texttt{normalsize}; $-4$ corresponds to
\texttt{tiny} and 5 to \texttt{Huge}.

\index{text!colour}\index{colours!text}\index{set textcolour
command@\texttt{set textcolour} command} The \texttt{set textcolour} command
can be used to globally set the colour of all text output, and applies to all
of the text that the \texttt{set fontsize} command does. It is especially
useful when producing plots to be embedded in presentation slideshows, where
bright text on a dark background may be desired. It should be followed either
by an integer, to set a colour from the present palette, or by a colour name. A
list of the recognised colour names can be found in Section~\ref{colour_names}.
For example:

\begin{verbatim}
set textcolour 2
set textcolour blue
\end{verbatim}

\index{text!alignment}\index{alignment!text}\index{set texthalign command@\texttt{set texthalign} command}\index{set textvalign command@\texttt{set textvalign} command}By default, each label's specified position corresponds to its bottom left corner. This alignment may be changed with the \texttt{set texthalign} and \texttt{set textvalign} commands. The former takes the options \texttt{left}, \texttt{centre} or \texttt{right}, and the latter takes the options \texttt{bottom}, \texttt{centre} or \texttt{top}, for example:

\begin{verbatim}
set texthalign right
set textvalign top
\end{verbatim}

\subsection{Gridlines}

Gridlines may be placed on a plot and subsequently removed via the statements:

\begin{verbatim}
set grid
set nogrid
\end{verbatim}

\noindent respectively. The following commands are also valid:

\begin{verbatim}
unset grid
unset nogrid
\end{verbatim}

\noindent By default, gridlines are drawn from the major and minor ticks of the
$x$- and $y$-axes. However, the axes which should be used may be specified
after the \texttt{set grid} command\index{grid}\index{set grid
command@\texttt{set grid} command}:

\begin{verbatim}
set grid x2y2
set grid x x2y2
\end{verbatim}

\noindent The top example would connect the gridlines to the ticks of the $x2$-
and $y2$- axes, whilst the lower would draw gridlines from both the $x$- and
the $x2$-axes.

If one of the specified axes does not exist, then no gridlines will be drawn in
that direction.  Gridlines can subsequently be removed selectively from some
axes via:

\begin{verbatim}
unset grid x2x3
\end{verbatim}

The colours of gridlines can be controlled via the \texttt{set
gridmajcolour}\index{grid!colour}\index{colours!grid}\index{set gridmajcolour
command@\texttt{set gridmajcolour} command}\index{set gridmincolour
command@\texttt{set gridmincolour} command} and \texttt{set gridmincolour}
commands, which control the gridlines emanating from major and minor axis ticks
respectively. An example would be:

\begin{verbatim}
set gridmincolour blue
\end{verbatim}

\noindent Any of the colour names listed in Section~\ref{colour_names} can be
used.

A related command is \texttt{set
axescolour}\index{axes!colour}\index{colours!axes}\index{set
axescolourcommand@\texttt{set axescolour} command}, which has a syntax similar
to that above, and sets the colour of the graph's axes.
\label{set_colours}

TODO: There should be a section on advanced LaTeX trickery.


