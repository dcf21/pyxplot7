% INTRODUCTION.TEX
%
% The documentation in this file is part of PyXPlot <http://www.pyxplot.org.uk>
%
% Copyright (C) 2006-8 Dominic Ford <coders@pyxplot.org.uk>
%               2008   Ross Church
%
% $Id$
%
% PyXPlot is free software; you can redistribute it and/or modify it under the
% terms of the GNU General Public License as published by the Free Software
% Foundation; either version 2 of the License, or (at your option) any later
% version.
%
% You should have received a copy of the GNU General Public License along with
% PyXPlot; if not, write to the Free Software Foundation, Inc., 51 Franklin
% Street, Fifth Floor, Boston, MA  02110-1301, USA

% ----------------------------------------------------------------------------

% LaTeX source for the PyXPlot Users' Guide

\chapter{Introduction} \pagenumbering{arabic}

\label{introduction}

\section{Overview}

{\sc PyXPlot} is a stand-alone command-line graphing package that is simple to
use yet produces high-quality attractive output suitable for use in
publications. For ease of use, its interface is based heavily upon that of the
popular {\sc \gnuplot}\index{gnuplot} plotting package, so users do not need to
learn a whole new scripting language. However, it uses the PyX graphics library
to produce its output, allowing the quality of its output to reach modern
standards.  The command-line interface has also been extended by addition of
tools to carry out some commonly-required data-processing.  An attempt has been
made to rectify frequently-noted flaws in \gnuplot; for example, all text is
now rendered automatically in the \LaTeX\ typesetting environment, making it
straightforward to label graphs with mathematical expressions. The multiplot
environment has been re-designed from scratch, making it easy to produce
galleries of plots with basic vector graphics around them.  For some samples of
the results of which PyXPlot is capable, the reader is referred to the project
website\footnote{\url{http://www.pyxplot.org.uk/}}.

The ability to represent data visually, usually as some form of graph, is a
requirement for any scientific or mathematical work.  Historically, a very
widely-used open-source plotting programme has been \gnuplot, the principal
attraction of which is its easy-to-use command-line interface, which allows
\datafile s to be turned into graphs within seconds. One of its main rivals has
been {\sc Pgplot}\index{Pgplot}, which is somewhat more flexible, but much less
easy to use; it can only be called from within a programme, and so code must be
written to produce each plot.  This is potentially time consuming, and it is
necessary for the user to have some programming experience.

Alongside these, several commercial packages have existed, including {\sc
Maple}\index{Maple}, {\sc Mathematica}\index{Mathematica} and {\sc
SuperMongo}\index{SuperMongo}.  These can typically produce prettier results
than their free counterparts, but carry with them considerable price tags and
licensing restrictions. Moreover, none of these are as easy to use as \gnuplot.

Both \gnuplot\ and Pgplot were developed in the mid-1980s. At that time, the
quality of their output seemed state-of-the-art. But this is less true today.
In most journal articles today, the text and equations are rendered with a high
degree of professionalism by the \LaTeX\ typesetting system. But all too often,
the graphs are less neat.  The same is often true of the slides used in
presentations: graphs are often rendered with less professionalism than the
text around them.

In the early 2000s, several new free graphing packages emerged, among them {\sc
MatPlotLib}\index{MatPlotLib} and {\sc PyX}\index{PyX}.  These have
significantly improved upon the quality of the plots produced by \gnuplot\ and
Pgplot. However, both are libraries which can be called from within the Python
programming language, rather than stand-alone plotting packages. They are not
as easy to use as \gnuplot.  For visualising data at speed, \gnuplot\ remains the
best option.

The command-line interface of \gnuplot\ is very flexible: it can be controlled
interactively, by typing commands into a terminal, it can read a list of
commands in from a file, or it can receive commands through a UNIX pipe from
another process. These modes of use are all possible in PyXPlot too.

We argue that \gnuplot's interface brings another distinct advantage to PyXPlot
in comparison with plotting packages which insist upon being called from within
a programming language. PyXPlot requires that data be written to a file on disk
before it can be plotted. When plotting is done from within a programming
language, this can tempt the user into writing programmes which both perform
calculations and plot the results immediately.  This sounds neat, but it can be
a dangerous temptation. Remembering to store a copy of the data used to produce
a graph becomes a secondary priority.  Months later, when the need arises to
replot the same data in a different form, or to compare it with newer data,
remembering how to use a hurriedly written programme can prove tricky --
especially if the programme was originally written by someone else. But a simple
\datafile\ is quite straightforward to plot.

The similarity of PyXPlot's interface to that of \gnuplot\ is such that simple
scripts written for \gnuplot\ should work with PyXPlot with minimal
modification; \gnuplot\ users should be able to get started very quickly.
However, PyXPlot is still a work in progress, and a small number of \gnuplot's
features are still missing.  A detailed list of which features are supported
can be found in Section~\ref{missing_features}. The new features which have
been added to the interface are described in
Chapters~\ref{gnuplot_ext_first}\,--\,\ref{gnuplot_ext_last}.

A brief overview of \gnuplot's interface is provided for novice users in
Chapter~\ref{gnuplot_intro}. Past \gnuplot\ users may skip over this chapter,
though their attention is drawn to one of the key changes to the interface --
namely that all textual labels on plots are now rendered using the \LaTeX\
typesetting environment. This does unfortunately introduce some incompatibility
with \gnuplot, since some strings which were valid before are no longer valid
(see Section~\ref{sec:latex_incompatibility} for more details). For
example:\index{latex}

\begin{dontdo}
set xlabel 'x\^{}2'
\end{dontdo}

\noindent would have been valid in \gnuplot, but now needs to be written in
\LaTeX\ mathmode as:

\begin{dodo}
set xlabel '\$x\^{}2\$'
\end{dodo}

\noindent The nuisance of this incompatibility is surely far outweighed by the
power that \LaTeX\ brings, however. For users with no prior knowledge of
\LaTeX\ we recommend Tobias Oetiker's\index{Tobias Oetiker} excellent
introduction, {\it The Not So Short Guide to \LaTeX $2\epsilon$}\index{Not So
Short Guide to \LaTeX $2\epsilon$, The}\footnote{Download from:\\
\url{http://www.ctan.org/tex-archive/info/lshort/english/lshort.pdf}}.

\section{System Requirements}

PyXPlot works on most UNIX-like operating systems. We have tested it under
Linux, Solaris\index{Solaris} and MacOS X\index{MacOS X}, and believe that it
should work on other similar systems. It requires that the following software
packages (not included) be installed:\index{system requirements}

\vspace{0.5cm}
\begin{tabular}{ll}
Python       & (Version 2.3 or later)\index{python} \\
Latex        & (Used for all textual labels)\index{latex} \\
\imagemagick & (needed for the gif, png and jpg terminals)\index{ImageMagick} \\
\end{tabular}
\vspace{0.5cm}

\noindent The following package is not {\it required} for installation, but
many PyXPlot features are disabled when it is not present, including the {\tt
fit} and {\tt spline} commands and the integration of functions. It is very
strongly recommended:\indcmd{fit}\indcmd{spline}

\vspace{0.5cm}
\begin{tabular}{ll} 
Scipy        & (Python Scientific Library)\index{scipy} \\
\end{tabular}
\vspace{0.5cm}

\noindent The following package is not {\it required} for installation, but it
is not possible to use the X11 terminal, i.e.\ to display plots on the screen,
without it:

\vspace{0.5cm}
\begin{tabular}{ll}
\ghostview   & (used for the X11 terminal)\index{Ghostview} \\
\end{tabular}
\vspace{0.5cm}

Debian and Ubuntu users can find the above software in the packages {\tt
texlive}, {\tt gv}, {\tt imagemagick}, {\tt python2.3}, {\tt
python2.3-scipy}.\index{Debian Linux}\index{Ubuntu
Linux}\index{installation!under Debian}\index{installation!under Ubuntu}

\section{Installation}
\index{installation}

\subsection{Installation within Linux Distributions}

PyXPlot is available as a user-installable package within some Linux
distributions. Gentoo\index{Gentoo Linux}\index{installation!under
Gentoo}\footnote{See \url{http://gentoo-portage.com/sci-visualization/pyxplot}}
and Ubuntu\index{Ubuntu Linux}\index{installation!under Ubuntu}\footnote{Note
that there is an error in the packaging of PyXPlot in {\it Hardy Herron} and
{\it Intrepid Ibex}, which means that the {\tt tetex-extra} package, upon which
PyXPlot depends, is not automatically installed with PyXPlot.} already have
such packages. Debian will have such a package in their next release, {\it
Lenny}, which is scheduled for late 2008. Alternatively, and to ensure that
they are using the latest version, Debian and Ubuntu users can download the
package from the PyXPlot website and install it manually by typing:

\begin{verbatim}
dpkg -i pyxplot_0.7.0.deb
\end{verbatim}

\subsection{Installation as User}

The following steps describe the installation of PyXPlot from a {\tt .tar.gz}
archive by a user without superuser (i.e.\ root) access to his machine. It is
assumed that the packages listed above have already been installed; if they are
not, you will need to contact your system
administrator.\index{installation!user-level}

\begin{itemize}
\item Unpack the distributed .tar.gz:

\begin{verbatim}
tar xvfz pyxplot_0.7.0.tar.gz
cd pyxplot
\end{verbatim}

\item Run the installation script:

\begin{verbatim}
./configure
make
\end{verbatim}

\item Finally, start PyXPlot:

\begin{verbatim}
./pyxplot
\end{verbatim}

\end{itemize}

\subsection{System-wide Installation}

Having completed the steps described above, PyXPlot may be installed
system-wide by a superuser with the following additional
step:\index{installation!system-wide}

\begin{verbatim}
make install
\end{verbatim}

By default, the PyXPlot executable installs to {\tt /usr/local/bin/pyxplot}.
If desired, this installation path may be modified in the file {\tt
Makefile.skel}, by changing the variable {\tt USRDIR} in the first line to an
alternative desired installation location.

PyXPlot may now be started by any system user, simply by typing:

\begin{verbatim}
pyxplot
\end{verbatim}

\section{Credits}

We would like to express our gratitude to several people who have contributed
to PyXPlot -- first and foremost to J\"org Lehmann\index{Lehmann, J\"org},
Andr\'e Wobst\index{Wobst, Andr\'e} and Michael Schindler\index{Schindler,
Michael} for writing the PyX\index{PyX} graphics library for Python, upon which
this software is heavily built. We would also like to think all of the users
who have got in touch with us by email since PyXPlot was first released on the
web.  Your feedback and suggestions have been gratefully received.

\section{Legal Blurb}

This manual and the software which it describes are both copyright \copyright\
Dominic Ford 2006-8, Ross Church 2008. They are distributed under the GNU
General Public License (GPL) Version~2, a copy of which is provided in the {\tt
COPYING} file in this distribution.\index{General Public
License}\index{license} Alternatively, it may be downloaded from the web, from
the following location:\\ \url{http://www.gnu.org/copyleft/gpl.html}.

