% INTRODUCTION.TEX
%
% The documentation in this file is part of PyXPlot
% <http://www.pyxplot.org.uk>
%
% Copyright (C) 2006-7 Dominic Ford <coders@pyxplot.org.uk>
%               2008   Ross Church
%
% $Id$
%
% PyXPlot is free software; you can redistribute it and/or modify it under the
% terms of the GNU General Public License as published by the Free Software
% Foundation; either version 2 of the License, or (at your option) any later
% version.
%
% You should have received a copy of the GNU General Public License along with
% PyXPlot; if not, write to the Free Software Foundation, Inc., 51 Franklin
% Street, Fifth Floor, Boston, MA  02110-1301, USA

% ----------------------------------------------------------------------------

% LaTeX source for the PyXPlot Users' Guide

\chapter{Introduction}
\pagenumbering{arabic}

\label{introduction}

\section{Overview}

The ability to represent data visually, usually as some form of graph, is a
cornerstone requirement for any scientific or mathematical computing project.
Historically, the most widely used open-source programme has been {\sc
\gnuplot}\index{gnuplot}, the principal attraction of which is its easy-to-use
command-line interface, which allows data files to be turned into graphs within
seconds. Its main rival has been {\sc pgplot}\index{pgplot}, which is somewhat
more flexible, but much less easy to use; it can only be called from within a
program, and so C or FORTRAN code must be written to produce each plot. This is
potentially time consuming, and it is assumed that the user has some
programming experience.

Alongside these, several commercial packages have existed, including {\sc
Maple}\index{Maple}, {\sc Mathematica}\index{Mathematica} and {\sc
SuperMongo}\index{SuperMongo}.  These can typically produce prettier results
than their free counterparts, but carry with them considerable price tags and
licensing restrictions.

Both \gnuplot\ and pgplot were developed in the mid 1980s. At that time, the
quality of their output seemed state-of-the-art. But this is less true today.
In most journal articles today, the text and equations are rendered with a high
degree of professionalism by the \LaTeX\ typesetting system. But all to often,
the graphs are less neat.  The same is often true of the slides used in
presentations: graph axes are often poorly labelled or illegible.

In the early 2000s, several new free graphing packages emerged, among them {\sc
MatPlotLib}\index{MatPlotLib} and {\sc PyX}\index{PyX}.  These have
significantly improved upon the quality of the plots produced by \gnuplot\ and
pgplot. However, both are libraries which can be called from within the python
programming language, rather than stand-alone plotting packages. They are not
as easy to use as \gnuplot.  For visualising data at speed, \gnuplot\ remains the
best option.

{\sc PyXPlot} aims to fill this gap. Like \gnuplot, it is a stand-alone
command-line graphing package. For ease of use, its interface is based heavily
upon that of \gnuplot, so users don't need to learn a whole new scripting
language. However, it uses the PyX graphics library to produce its output,
allowing it to bring the quality of \gnuplot's output up-to-date.  The
command-line interface has also been extended into a more fully-featured vector
graphics and data processing package.  Where we are aware of
frequently-repeated complaints about \gnuplot's interface, we have tried to put
them right.  For example, all text is now rendered automatically in the \LaTeX\
typesetting environment, making it straightforward to label graphs with
mathematical expressions. The ``multiplot'' environment has been essentially
re-designed from scratch, making it easy to produce galleries of plots with
other basic vector graphics around them.  For some samples of the results of
which PyXPlot is capable, the reader is referred to the project
website\footnote{\url{http://www.pyxplot.org.uk/}}.

The command-line interface of \gnuplot\ is very flexible: it can be controlled
interactively, by typing commands into a terminal, or it can read a list of
commands in from a file, or it can receive commands through a UNIX pipe from
another process. These modes of use are all possible in PyXPlot too.  But we
also argue that PyXPlot had another distinct advantage over plotting packages
which insist upon being called from within a programming language. PyXPlot
insists that data is written to a file on disk before it can be plotted. When
plotting is done from within programmes, this can tempt the user into writing
programmes which both perform calculations and plot immediately the results.
This sounds neat, but it can be a dangerous temptation. Remembering to store a
copy of the data used to produce a graph becomes a secondary priority.  Months
later, when the need arises to replot the same data in a different form, or to
compare it with newer data, remembering how to use a hurriedly written program
can prove tricky -- especially if the program was originally written by someone
else. But a simple datafile is quite straightforward to plot.

The similarity of the interface to that of \gnuplot\ is such that simple scripts
written for \gnuplot\ should work with PyXPlot with minimal modification; \gnuplot\
users should be able to get started very quickly.  However, PyXPlot is still
work in progress, and some of \gnuplot's features are still missing.  A detailed
list of which features are supported can be found in
Section~\ref{missing_features}. The new features which have been added to the
interface are described in the remainder of this manual.

A brief overview of \gnuplot's interface is provided for novice users in
Chapter~\ref{gnuplot_intro}. However, the attention of past \gnuplot\ users is
drawn to one of the key changes to the interface -- namely that all textual
labels on plots are now rendered using the \LaTeX\ typesetting environment. This
does unfortunately introduce some incompatibility with \gnuplot, since some
strings which were valid before are no longer valid (see
Section~\ref{sec:latex_incompatibility} for more details). For example:

\begin{dontdo}
set xlabel 'x^2'
\end{dontdo}

\noindent would have been valid in \gnuplot, but now needs to be written in
\LaTeX\ mathmode as:

\begin{dodo}
set xlabel '$x^2$'
\end{dodo}

\noindent The nuisance of this incompatibility is surely far outweighed by the
power that \LaTeX\ brings, however. For users with no prior knowledge of
\LaTeX\ we recommend Tobias Oetiker's\index{Tobias Oetiker} excellent
introduction, {\it The Not So Short Guide to \LaTeX $2\epsilon$}\index{The Not
So Short Guide to \LaTeX $2\epsilon$}\footnote{Download from:\\
\url{http://www.ctan.org/tex-archive/info/lshort/english/lshort.pdf}}.

\section{System Requirements}

PyXPlot works on most UNIX-like operating systems. We have tested it under
Linux, SunOS\index{SunOS} and MacOS X\index{MacOS X}, and believe that it
should work on other similar systems. It requires that the following software
packages (not included) be installed:\index{system requirements}

\vspace{0.5cm}
\begin{tabular}{ll}
python       & (Version 2.3 or later)\index{python} \\
latex        & (Used for all textual labels) \\
\imagemagick & (needed for the gif, png and jpg terminals)\index{ImageMagick} \\
\end{tabular}
\vspace{0.5cm}

The following package is not required for installation, but many PyXPlot
features are disabled when it is not present, including the {\tt fit} and {\tt
spline} commands and the integration of functions. It is very strongly
recommended:\indcmd{fit}\indcmd{spline}

\vspace{0.5cm}
\begin{tabular}{ll} 
scipy        & (Python Scientific Library)\index{scipy} \\
\end{tabular}
\vspace{0.5cm}

The following package is not required for installation, but it is not possible
to use the X11 terminal, i.e.\ to display plots on screen, without it:

\vspace{0.5cm}
\begin{tabular}{ll}
\ghostview   & (used for the X11 terminal)\index{Ghostview} \\
\end{tabular}
\vspace{0.5cm}

Debian/Ubuntu users can find the above software in the packages {\tt
tetex-extra}, {\tt gv}, {\tt imagemagick}, {\tt python2.3}, {\tt
python2.3-scipy}.\index{Debian Linux}\index{Ubuntu
Linux}\index{installation!under Debian}\index{installation!under Ubuntu}

\section{Installation}
\index{installation}

\subsection{Installation within Linux Distributions}

PyXPlot is available as a user-installable package within some Linux
distributions. Gentoo has such a package already.\index{Gentoo
Linux}\index{installation!under Gentoo}\footnote{See
\url{http://gentoo-portage.com/sci-visualization/pyxplot}}. Debian will have
such a package in their next release, Lenny, which is scheduled for late 2008.
Ubuntu will also have such a package in their next release, Hardy Heron, which
is scheduled for 24th April 2008. In the meantime, Debian/Ubuntu users can
download the package from the PyXPlot website, for manual installation.

\subsection{Installation as User}

The following steps describe the installation of PyXPlot from a {\tt .tar.gz}
archive by a user without superuser (i.e.\ root) access to his machine. It is
assumed that the packages listed above have already been installed; if they are
not, you need to contact your system administrator.

\begin{itemize}
\item Unpack the distributed .tar.gz:

\begin{verbatim}
tar xvfz pyxplot_0.7.0.tar.gz
cd pyxplot
\end{verbatim}

\item Run the installation script:

\begin{verbatim}
./configure
make
\end{verbatim}

\item Finally, start PyXPlot:

\begin{verbatim}
./pyxplot
\end{verbatim}

\end{itemize}

\subsection{System-wide Installation}

Having completed the steps described above, PyXPlot may be installed
system-wide by a superuser with the following additional step:

\begin{verbatim}
make install
\end{verbatim}

By default, the PyXPlot executable installs to {\tt /usr/local/bin/pyxplot}.
If desired, this installation path may be modified in the file {\tt
Makefile.skel}, by changing the variable {\tt USRDIR} in the first line to an
alternative desired installation location.

PyXPlot may now be started by any system user, simply by typing:

\begin{verbatim}
pyxplot
\end{verbatim}

\section{Credits}

We would like to express his gratitude to several people who have contributed
to PyXPlot -- first and foremost, to J\"org Lehmann, Andr\'e Wobst and Michael
Schindler, for writing the PyX graphics library for Python, upon which this
software is heavily built. We would also like to think all of the users who
have got in touch with us by email since PyXPlot was first released on the web.
Your feedback and suggestions have been gratefully received.

\section{Legal Blurb}

This manual and the software which it describes are both copyright \copyright
Dominic Ford 2006-8, Ross Church 2008. They are distributed under the GNU
General Public License (GPL) Version~2, a copy of which is provided in the {\tt
COPYING} file in this distribution.\index{General Public License}
Alternatively, it may be downloaded from the web, from the following
location:\\ \url{http://www.gnu.org/copyleft/gpl.html}.

