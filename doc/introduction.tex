% INTRODUCTION.TEX
%
% The documentation in this file is part of PyXPlot
% <http://www.pyxplot.org.uk>
%
% Copyright (C) 2006-7 Dominic Ford <coders@pyxplot.org.uk>
%
% $Id$
%
% PyXPlot is free software; you can redistribute it and/or modify it under the
% terms of the GNU General Public License as published by the Free Software
% Foundation; either version 2 of the License, or (at your option) any later
% version.
%
% You should have received a copy of the GNU General Public License along with
% PyXPlot; if not, write to the Free Software Foundation, Inc., 51 Franklin
% Street, Fifth Floor, Boston, MA  02110-1301, USA

% ----------------------------------------------------------------------------

% LaTeX source for the PyXPlot Users' Guide

\chapter{Introduction}
\pagenumbering{arabic}

\label{introduction}

\section{Overview}

PyXPlot is a command-line graphing package, which, for ease of use, has an
interface based heavily upon that of gnuplot -- perhaps UNIX's most widely-used
plotting package. Despite the shared interface, however, PyXPlot is intended to
significantly improve upon the quality of gnuplot's output, producing
publication-quality figures. The command-line interface has also been extended,
providing a wealth of new features, and short-cuts for some operations which
were felt to be excessively cumbersome in the original.

The motivation behind PyXPlot's creation was the apparent lack of a free
plotting package which combined both high-quality output and a simple
interface.  Some -- pgplot for one -- provided very attractive output, but
required a program to be written each time a plot was to be produced -- a
potentially time consuming task. Others, gnuplot being the prime example, were
quick and simple to use, but produced less attractive results.

PyXPlot attempts to fill that gap, offering the best of both worlds. Though the
interface is based upon that of gnuplot, text is now rendered with all of the
beauty and flexibility of the \LaTeX\ typesetting environment; the
``multiplot'' environment is made massively more flexible, making it easy to
produce galleries of plots; and the range of possible output formats is
extended -- to name but a few of the enhancements. A number of examples of the
results of which PyXPlot is capable can be seen on the project
website\footnote{\url{http://www.pyxplot.org.uk/}}.

As well as the ease of use and flexibility of gnuplot's command-line interface
-- it can be used either interactively, read a list of commands from a file, or
receive instructions through a UNIX pipe from another process -- I believe it
to bring another distinct advantage. It insists upon data being written to a
datafile on disk before being plotted. Packages which allow, or more often
require, plotting to be done from within a programming language can encourage
the calculation of data and its plotting to occur in the same program. I
believe this to be a dangerous temptation, as the storage of raw datapoints to
disk can then often be seen as a secondary priority. Months later, when the
need arises to replot the same data in a different form, or to compare it with
newer data, remembering how to use a hurriedly written program can prove
tricky, but remembering how to plot a simple datafile less so.

The similarity of the interface to that of gnuplot is such that simple scripts
written for gnuplot should work with PyXPlot with minimal modification; gnuplot
users should be able to get started very quickly.  However, as PyXPlot remains
work in progress, some gnuplot features may be found to be missing.  A detailed
list of which features are supported can be found in
Section~\ref{missing_features}.  Furthermore many new features have been added
to the interface; these are described in the remainder of this manual.

A brief overview of gnuplot's interface is provided for novice users in
Chapter~\ref{gnuplot_intro}. However, the attention of past gnuplot users is
drawn to one of the key changes to the interface -- namely that all textual
labels on plots are now printed using the \LaTeX\ typesetting environment. This
does unfortunately introduce some incompatibility with the original, since some
strings which were valid before are no longer valid (see
Section~\ref{sec:latex_incompatibility} for more details). For example:

\begin{verbatim}set xlabel 'x^2'\end{verbatim}

\noindent would have been valid in gnuplot, but now needs to be written in
\LaTeX\ mathmode as:

\begin{verbatim}set xlabel '$x^2$'\end{verbatim}

\noindent It is the view of the author, however, that the nuisance of this
incompatibility is far outweighed by the power that \LaTeX\ brings. For users
with no prior knowledge of \LaTeX\ the author recommends Tobias Oetiker's
excellent introduction, \textit{The Not So Short Guide to \LaTeX
$2\epsilon$}\footnote{Download from:\\
\url{http://www.ctan.org/tex-archive/info/lshort/english/lshort.pdf}}.

\section{System Requirements}

PyXPlot works on many UNIX-like operating systems. The authors have tested it
under Linux, SunOS and MacOS X, and believe that it should work on other
similar systems. It requires that the following software packages (not
included) be installed:\index{system requirements}

\vspace{0.5cm}
\begin{tabular}{ll}
python  & (Version 2.3 or later) \\
latex   & (Used for all textual labels) \\
convert & (ImageMagick; needed for the gif, png and jpg terminals) \\
\end{tabular}
\vspace{0.5cm}

The following package is not required for installation, but many PyXPlot
features are disabled when it is not present, including the \texttt{fit} and
\texttt{spline} commands and the integration of functions. It is very strongly
recommended:

\vspace{0.5cm}
\begin{tabular}{ll} 
scipy   & (Python Scientific Library) \\
\end{tabular}
\vspace{0.5cm}

The following package is not required for installation, but it is not possible
to use the X11 terminal, i.e. to display plots on screen, without it:

\vspace{0.5cm}
\begin{tabular}{ll}
gv      & (Ghostview; used for the X11 terminal) \\
\end{tabular}
\vspace{0.5cm}

Debian/Ubuntu users can find the above software in the packages \texttt{tetex-extra},
\texttt{gv}, \texttt{imagemagick}, \texttt{python2.3},
\texttt{python2.3-scipy}.\index{Debian Linux}\index{installation!under Debian}

\section{Installation}
\index{installation}

\subsection{Installation as User}

The following steps describe the installation of PyXPlot from a
\texttt{.tar.gz} archive by a user without superuser (i.e. root) access to his
machine. It is assumed that the packages listed above have already been
installed; if they are not, you need to contact your system administrator.

\begin{itemize}
\item Unpack the distributed .tar.gz:

\begin{verbatim}
tar xvfz pyxplot_0.6.3.1.tar.gz
cd pyxplot
\end{verbatim}

\item Run the installation script:

\begin{verbatim}
./configure
make
\end{verbatim}

\item Finally, start PyXPlot:

\begin{verbatim}
./pyxplot
\end{verbatim}

\end{itemize}

\subsection{System-wide Installation}

Having completed the steps described above, PyXPlot may be installed
system-wide by a superuser with the following additional step:

\begin{verbatim}
make install
\end{verbatim}

By default, the PyXPlot executable installs to \texttt{/usr/local/bin/pyxplot}.
If desired, this installation path may be modified in the file
\texttt{Makefile.skel}, by changing the variable \texttt{USRDIR} in the first
line to an alternative desired installation location.

PyXPlot may now be started by any system user, simply by typing:

\begin{verbatim}
pyxplot
\end{verbatim}

\section{Credits}

Before proceeding any further, the author would like to express his gratitude
to those people who have contributed to PyXPlot -- first and foremost, to
J\"org Lehmann and Andr\'e Wobst, for writing the PyX graphics library for
Python, upon which this software is heavily built. Thanks must also go to Ross
Church for his many useful comments and suggestions during its development.

\section{Legal Blurb}

This manual and the software which it describes are both copyright (C)
Dominic Ford 2006-7. They are both distributed under the GNU General Public
License (GPL) Version~2, a copy of which is provided in the \texttt{COPYING}
file in this distribution.\index{General Public License} Alternatively, it may
be downloaded from:\\ \url{http://www.gnu.org/copyleft/gpl.html}.

